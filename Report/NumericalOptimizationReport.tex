\documentclass[11pt,twocolumn]{scrartcl}
%
% \usepackage{mathpazo}
% \usepackage{multicol}
\usepackage[dvipsnames]{xcolor}
\usepackage[top = 2.0cm, bottom = 2.0cm, right = 1.8cm, left = 1.8cm]{geometry}
\usepackage{amsmath,amssymb} %packages for Mathematics
\usepackage{physics}
\usepackage{cases}
\usepackage{amsfonts}

% \definecolor{UniTnRed}{RGB}{177,10,37}
\definecolor{UniTnRed}{RGB}{125,02,12}
\definecolor{UniTnGray}{RGB}{218,218,218}
\definecolor{UniTnOrange}{RGB}{206,176,123}

\definecolor{customgreen}{RGB}{0,128,0} % Define the custom green color

\makeatletter
\renewcommand\maketitle
{
    \noindent
    {\color{UniTnOrange}\rule{\textwidth}{2pt}}
    \vspace{0.5cm}\\
    {\Large\bfseries\textcolor{UniTnRed}{\@title}}%
    \medskip\par\noindent
    {\Large\bfseries\textcolor{UniTnRed}{\@subtitle}}\\
    \;
    \medskip\par\noindent
    {\large\bfseries\textcolor{UniTnRed}{\@author}}%
    {\large\quad University of Trento}
    \hfill
    {\large\@date}\\
    \;\\%
    {\color{UniTnOrange}\rule{\textwidth}{2pt}}
    \bigskip\par\noindent
}

% avoid indent
\setlength{\parindent}{0pt}

% create a command \rm to remove the math font
\newcommand{\rm}{\mathrm}
% add package 

\usepackage{cancel}



\definecolor{darkolivegreen   }{RGB}{ 85,107, 47}
\definecolor{darksalmon       }{RGB}{233,150,122}
\definecolor{darkseagreen     }{RGB}{143,188,143}
\definecolor{darkslateblue    }{RGB}{ 72, 61,139}
\definecolor{deepskyblue      }{RGB}{  0,191,255}
\definecolor{dodgerblue       }{RGB}{ 30,144,255}
\definecolor{green            }{RGB}{  0,128,  0}
\definecolor{indigo           }{RGB}{ 75,  0,130}
\definecolor{lawngreen        }{RGB}{124,252,  0}
\definecolor{lightgreen       }{RGB}{144,238,144}
\definecolor{lightseagreen    }{RGB}{ 32,178,170}
\definecolor{lightskyblue     }{RGB}{135,206,250}
\definecolor{lightsteelblue   }{RGB}{176,196,222}
\definecolor{lime             }{RGB}{  0,255,  0}
\definecolor{limegreen        }{RGB}{ 50,205, 50}
\definecolor{midnightblue     }{RGB}{ 25, 25,112}
\definecolor{orange           }{RGB}{255,165,  0}
\definecolor{orchid           }{RGB}{218,112,214}
\definecolor{skyblue          }{RGB}{135,206,235}
\definecolor{slateblue        }{RGB}{106, 90,205}
\definecolor{springgreen      }{RGB}{  0,255,127}
\definecolor{steelblue        }{RGB}{ 70,130,180}
\definecolor{turquoise        }{RGB}{ 64,224,208}
\definecolor{violet           }{RGB}{238,130,238} 
\usepackage{siunitx}
\usepackage{graphicx}
\usepackage{subcaption}
\usepackage{tikz}
\usetikzlibrary{calc}

\makeatother

\title{Report of Course Numerical Optimization}
\subtitle{Duboids: Extended Dubins path with Clothoids Junctions}
\author{Mattia Piazza}
\date{\today}
 
\begin{document}
%
\twocolumn[\maketitle ]
%
\section*{Introduction}
%
The Dubins path is a well-known problem in robotics and trajectory planning. The problem consists in finding the shortest path to connect two points in a plane with a prescribed heading (initial and final) with a maxim curvature constraint. This problem is suitable for mobile robots that are not omnidirectional. The solution can be derived analytically and is composed of a sequence of circular arcs and straight lines.\cite{shkel2001classification,chen2019shortest,jha2020shortest} The Dubins path is a special case of the Reeds-Shepp path\cite{duits2018optimal} where also backward motion is allowed.\\
However, the Dubins path is not suitable for vehicles that have a limited steering rate. In this case, the curvature of the path cannot change instantaneously.\\
Duboids are the proposed suboptimal solution to account for limits in curvature rate and therefore in the steering rate of a vehicle. Duboids are a combination of Dubins path and clothoids. 
%
\subsection*{Dubins}
The problem can be stated as follows:
%
\begin{equation}
  \begin{split}
    \min_{\kappa}  \quad & \int_0^T v \differential t = \min_{\kappa}  \quad v T \\ % \min_{|\kappa|\le\kappa_{max}}
    \text{s.t.} \quad 
      &\dot{x}(t)      = v \cos(\theta(t)) \\
      &\dot{y}(t)      = v \sin(\theta(t)) \\
      &\dot{\theta}(t) = v \kappa(t)       \\
      &x(0)      = x_0,      \; x(T)      = x_T\\
      &y(0)      = y_0,      \; y(T)      = y_T\\
      &\theta(0) = \theta_0, \; \theta(T) = \theta_T\\
      &-\kappa_{max} \le \kappa(t) \le \kappa_{max}   
  \end{split}
\end{equation}
%
Where $v$ is a fixed velocity, $x$, $y$ and $\theta$ are the position and heading of the vehicle, $\kappa$ is the curvature of the path and $\kappa_{max}$ is the maximum curvature allowed. 
The analytic solution to this problem is at most a sequence of 3 arcs, either left or right circular arcs at maximum curvature or straight lines.\cite{shkel2001classification}\\
%
This problem and its solution can be applied in vehicles moving slowly and where the curvature change is almost instantaneous. However, real vehicles have a physical limit in the maximum curvature and maximum curvature rate. For this reason, we are looking to an extension of the Dubins path to incorporate this limitation as we will see in the next section exploiting clothoids.\cite{bertolazzi2015g1,bertolazzi2018clothoids}
%
\section*{Problem}
%
The problem we want to solve is to minimize the time while connecting two points in the Cartesian space with a prescribed heading and curvature both at the initial and final time. This problem can be formulated in the following way:
%
\begin{equation}
  \begin{split}
    \min_{J} \quad & v T \\ 
    \text{s.t.} \quad
      &\dot{x}(t)      = v \cos(\theta(t)) \\
      &\dot{y}(t)      = v \sin(\theta(t)) \\
      &\dot{\theta}(t) = v \kappa(t)       \\
      &\dot{\kappa}(t) = J(t)              \\
      &x(0)      = x_0,      \; x(T)      = x_T     \\
      &y(0)      = y_0,      \; y(T)      = y_T     \\
      &\theta(0) = \theta_0, \; \theta(T) = \theta_T\\
      &\kappa(0) = \kappa_0, \; \kappa(T) = \kappa_T\\
      &-\kappa_{max} \le \kappa(t) \le \kappa_{max} \\
      &-J_{max} \le J(t) \le J_{max}
  \end{split}
\end{equation}
%
Where $v$ is a fixed velocity, $x$, $y$ and $\theta$ are the position and heading of the vehicle, $\kappa$ is the curvature of the path, $\kappa_{max}$ is the maximum curvature allowed, $J$ is the controlled curvature rate (Jerk) and $J_{max}$ is the maximum curvature rate allowed.\\
%
This problem can be translated into a BVP (Boundary Value Problem) and solved in a semi-analytical fashion. The solution is composed of several arcs either at the maximum rate (positive or negative or at zero rate). 
%
\subsection*{Analytic solution}
%
The Hamiltonian function of the problem is:
%
\begin{equation}
  \begin{split}
    H = &\lambda_1(t) v \cos(\theta(t)) + \lambda_2(t) v \sin(\theta(t)) \\
        &+ \lambda_3(t) v \kappa(t) + \lambda_4(t) J(t)\\
        &+ \mu_1(t) (\kappa(t)-\kappa_{max}) \\ 
        &+ \mu_2(t) (-\kappa(t)+\kappa_{max})
  \end{split}
\end{equation}
%
The costate equations are:
%
\begin{equation}
  \begin{split}
    &\dot{\lambda_1}(t) = 0 \\
    &\dot{\lambda_2}(t) = 0 \\
    &\dot{\lambda_3}(t) = \lambda_1(t) v \sin(\theta(t)) - \lambda_2(t) v \cos(\theta(t)) \\
    &\dot{\lambda_4}(t) = -\lambda_3(t) v -\mu_1(t) + \mu_2(t) \\
  \end{split}
\end{equation}
%
Which yields that $\lambda_1$ and $\lambda_2$ are constant.\\
Moreover, the control is
%
\begin{equation}
  J(t) = \underset{ J \in [-J_{max},J_{max}]}{ \textrm{argmin} } \; H 
\end{equation}
%
However, $H$ is linear in $J$, thus the second derivative concerning the control is null, and the problem became singular. In the case of a singular arc, the control is either at the maximum or minimum of the control set or at zero.
%
\begin{equation}
  J(t) = \begin{cases}
    +J_{max} & \text{if } \lambda_4(t) > 0 \\
    -J_{max} & \text{if } \lambda_4(t) < 0 \\
    0 & \text{if } \lambda_4(t) = 0
  \end{cases}
\end{equation}
%
\subsection*{Physical interpretation}
%
The physical interpretation is that the vehicle is either changing the curvature at the maximum rate or keeping the curvature constant.
The only case when the curvature is kept constant is when the vehicle is travelling straight or when it is travelling at maximum curvature.
Thus, the problem can be treated as a mixed integer optimization that in general is NP-hard.\\
%
Figure \ref{fig:possiblecombination} illustrates all the possible combinations of manoeuvres connecting point $P_0$ and $P_T$. All intermediate points (at most $7$) are switching points between clothoid and circular arcs or straight lines. From the starting point (and configuration) the vehicle can either steer toward maximum, minimum or zero curvature. If point $P_0$ already satisfy a bound the arc of clothoid $L_1$ is not necessary and will have zero length.
There are some useless connections such as the repetition of two straight lines or two arcs with the same curvature which are accounted as special cases connecting directly the point to the final configuration.\\
%
Figure \ref{fig:possiblecombination} represents a graph connecting the initial and final configuration with all the possible combinations of manoeuvres. The problem could be solved with a graph search algorithm. However, the number of possible paths inside the graph is not high and is known a priori. Thus, a naive exploration of all the possible combinations is feasible.\\
%
From figure \ref{fig:possiblecombination}, we can count $12$ possible $7$-arcs connections with a strange familiar resemblance to the Dubins path. In addition, $3$ are the $3$-arcs connections and $6$ are the $5$-arcs connections. The total number of possible combination is $21$ as shown in table \ref{tab:possiblecombination}.\\
%
However, we are neglecting the possibility that for some configurations of initial and final points, the vehicle could perform manoeuvres not reaching the maximum curvature values. This will need a separate analysis when it comes to naive exploration.  
%
\begin{figure}[ht]
  \centering
  \resizebox{1.1\linewidth}{!}{%
    
\tikzset{every picture/.style={line width=0.75pt}} %set default line width to 0.75pt 

% \begin{tikzpicture}[x=0.75pt,y=0.75pt,yscale=-1,xscale=1]
\begin{tikzpicture}[
    every node/.style={minimum size=0.05cm, font=\footnotesize, inner sep=0.01cm,scale = 1}
]

% Level 0
\node[draw,circle,font=\footnotesize, inner sep=0.01cm] (P0) at (0,0) {$P_0$};

% Level 1
\node[draw,circle] (P11) at ( 3.0,-1.) {$P_1$};
\node[draw,circle] (P12) at ( 0.0,-1.) {$P_1$};
\node[draw,circle] (P13) at (-3.0,-1.) {$P_1$};
\draw[red] (P0) -- node[midway, right, shift={(+0.2cm,0)}] {$L_1$} (P11);
\draw[red] (P0) -- (P12);
\draw[red] (P0) -- (P13);

% Level 2
\node[draw,circle] (P21) at ( 3.0,-2) {$P_2$};
\node[draw,circle] (P22) at ( 0.0,-2) {$P_2$};
\node[draw,circle] (P23) at (-3.0,-2) {$P_2$};

\draw[green] (P11) -- node[midway, right, shift={(+0.2cm,0)}] {$L_2$} (P21);
\draw[green] (P12) -- (P22);
\draw[green] (P13) -- (P23);

% Level 3
\node[draw,circle] (P31) at ( 4.0,-3) {$P_3$};
\node[draw,circle] (P32) at ( 3.0,-3) {$P_3$};
\node[draw,circle] (P33) at ( 2.0,-3) {$P_3$};
%
\node[draw,circle] (P34) at ( 1.0,-3) {$P_3$};
\node[draw,circle] (P35) at ( 0.0,-3) {$P_3$};
\node[draw,circle] (P36) at (-1.0,-3) {$P_3$};
%
\node[draw,circle] (P37) at (-2.0,-3) {$P_3$};
\node[draw,circle] (P38) at (-3.0,-3) {$P_3$};
\node[draw,circle] (P39) at (-4.0,-3) {$P_3$};




\draw[red] (P21) -- node[midway, right, shift={(+0.2cm,0)}] {$L_3$} (P31);
\draw[red] (P21) -- (P32);
\draw[red] (P21) -- (P33);
%
\draw[red] (P22) -- (P34);
\draw[red] (P22) -- (P35);
\draw[red] (P22) -- (P36);
%
\draw[red] (P23) -- (P37);
\draw[red] (P23) -- (P38);
\draw[red] (P23) -- (P39);
%

% layer 4

\node[draw,circle] (P41) at ( 4.0,-4) {$P_4$};
\node[draw,circle] (P42) at ( 3.0,-4) {$P_4$};
\node[draw,circle] (P43) at ( 2.0,-4) {$P_4$};
%
\node[draw,circle] (P44) at ( 1.0,-4) {$P_4$};
\node[draw,circle] (P45) at ( 0.0,-4) {$P_4$};
\node[draw,circle] (P46) at (-1.0,-4) {$P_4$};
%
\node[draw,circle] (P47) at (-2.0,-4) {$P_4$};
\node[draw,circle] (P48) at (-3.0,-4) {$P_4$};
\node[draw,circle] (P49) at (-4.0,-4) {$P_4$};
%

\draw[green] (P31) -- node[midway, right, shift={(+0.2cm,0)}] {$L_4$} (P41);
\draw[green] (P32) -- (P42);
\draw[green] (P33) -- (P43);
%
\draw[green] (P34) -- (P44);
\draw[green] (P35) -- (P45);
\draw[green] (P36) -- (P46);
%
\draw[green] (P37) -- (P47);
\draw[green] (P38) -- (P48);
\draw[green] (P39) -- (P49);
%

% layer 5

\node[draw,circle] (P51) at ( 4.5,-5) {$P_5$};
\node[draw,circle] (P52) at ( 4.0,-5) {$P_5$};
\node[draw,circle] (P53) at ( 3.5,-5) {$P_5$};
%
\node[draw,circle] (P54) at ( 3.0,-5) {$P_5$};
\node[draw,circle] (P55) at ( 2.5,-5) {$P_5$};
\node[draw,circle] (P56) at ( 2.0,-5) {$P_5$};
%
\node[draw,circle] (P57) at ( 1.5,-5) {$P_5$};
\node[draw,circle] (P58) at ( 1.0,-5) {$P_5$};
\node[draw,circle] (P59) at ( 0.5,-5) {$P_5$};
%
\node[draw,circle] (P510) at (-0.5,-5) {$P_5$};
\node[draw,circle] (P511) at (-1.0,-5) {$P_5$};
\node[draw,circle] (P512) at (-1.5,-5) {$P_5$};
%
\node[draw,circle] (P513) at (-2.0,-5) {$P_5$};
\node[draw,circle] (P514) at (-2.5,-5) {$P_5$};
\node[draw,circle] (P515) at (-3.0,-5) {$P_5$};
%
\node[draw,circle] (P516) at (-3.5,-5) {$P_5$};
\node[draw,circle] (P517) at (-4.0,-5) {$P_5$};
\node[draw,circle] (P518) at (-4.5,-5) {$P_5$};
%




%
\draw[red] (P42) -- node[midway, right, shift={(+0.2cm,0)}] {$L_5$} (P51);
\draw[red] (P42) -- (P52);
\draw[red] (P42) -- (P53);
%
\draw[red] (P43) -- (P54);
\draw[red] (P43) -- (P55);
\draw[red] (P43) -- (P56);
%
\draw[red] (P44) -- (P57);
\draw[red] (P44) -- (P58);
\draw[red] (P44) -- (P59);
%
\draw[red] (P46) -- (P510);
\draw[red] (P46) -- (P511);
\draw[red] (P46) -- (P512);
%
\draw[red] (P47) -- (P513);
\draw[red] (P47) -- (P514);
\draw[red] (P47) -- (P515);
%
\draw[red] (P48) -- (P516);
\draw[red] (P48) -- (P517);
\draw[red] (P48) -- (P518);

% layer 6
\node[draw,circle] (P61) at ( 4.5,-6) {$P_6$};
\node[draw,circle] (P62) at ( 4.0,-6) {$P_6$};
\node[draw,circle] (P63) at ( 3.5,-6) {$P_6$};
%
\node[draw,circle] (P64) at ( 3.0,-6) {$P_6$};
\node[draw,circle] (P65) at ( 2.5,-6) {$P_6$};
\node[draw,circle] (P66) at ( 2.0,-6) {$P_6$};
%
\node[draw,circle] (P67) at ( 1.5,-6) {$P_6$};
\node[draw,circle] (P68) at ( 1.0,-6) {$P_6$};
\node[draw,circle] (P69) at ( 0.5,-6) {$P_6$};
%
\node[draw,circle] (P610) at (-0.5,-6) {$P_6$};
\node[draw,circle] (P611) at (-1.0,-6) {$P_6$};
\node[draw,circle] (P612) at (-1.5,-6) {$P_6$};
%
\node[draw,circle] (P613) at (-2.0,-6) {$P_6$};
\node[draw,circle] (P614) at (-2.5,-6) {$P_6$};
\node[draw,circle] (P615) at (-3.0,-6) {$P_6$};
%
\node[draw,circle] (P616) at (-3.5,-6) {$P_6$};
\node[draw,circle] (P617) at (-4.0,-6) {$P_6$};
\node[draw,circle] (P618) at (-4.5,-6) {$P_6$};
%

%
\draw[green] (P51) -- node[midway, right, shift={(+0.2cm,0)}] {$L_6$} (P61);
\draw[green] (P52) -- (P62);
\draw[green] (P53) -- (P63);
%
\draw[green] (P54) -- (P64);
\draw[green] (P55) -- (P65);
\draw[green] (P56) -- (P66);
%
\draw[green] (P57) -- (P67);
\draw[green] (P58) -- (P68);
\draw[green] (P59) -- (P69);
%
\draw[green] (P510) -- (P610);
\draw[green] (P511) -- (P611);
\draw[green] (P512) -- (P612);
%
\draw[green] (P513) -- (P613);
\draw[green] (P514) -- (P614);
\draw[green] (P515) -- (P615);
%
\draw[green] (P516) -- (P616);
\draw[green] (P517) -- (P617);
\draw[green] (P518) -- (P618);
%

% % Level 7

\node[draw,circle,font=\footnotesize, inner sep=0.01cm] (P7) at (0,-9) {$P_T$};

\draw[orange] (P61) -- node[midway, right, shift={(+0.2cm,0)}] {$L_7$} (P7);
\draw[orange] (P62) -- (P7);
\draw[orange] (P63) -- (P7);
%   
\draw[orange] (P64) -- (P7);
\draw[orange] (P65) -- (P7);
\draw[orange] (P66) -- (P7);
%
\draw[orange] (P67) -- (P7);
\draw[orange] (P68) -- (P7);
\draw[orange] (P69) -- (P7);
%
\draw[orange] (P610) -- (P7);
\draw[orange] (P611) -- (P7);
\draw[orange] (P612) -- (P7);
%
\draw[orange] (P613) -- (P7);
\draw[orange] (P614) -- (P7);
\draw[orange] (P615) -- (P7);
%
\draw[orange] (P616) -- (P7);
\draw[orange] (P617) -- (P7);
\draw[orange] (P618) -- (P7);
%

% curved connection between P41 and P7
\draw[blue] (P11) to[out=0,in=0] (P7);

% curved large connction between P49 and P7
\draw[blue] (P13) to[out=180,in=180] (P7);

% connect P45 and P7
\draw[blue] (P12) to[out=180,in=180] (P7);


\end{tikzpicture}
%
  }%
  % 
\tikzset{every picture/.style={line width=0.75pt}} %set default line width to 0.75pt 

% \begin{tikzpicture}[x=0.75pt,y=0.75pt,yscale=-1,xscale=1]
\begin{tikzpicture}[
    every node/.style={minimum size=0.05cm, font=\footnotesize, inner sep=0.01cm,scale = 1}
]

% Level 0
\node[draw,circle,font=\footnotesize, inner sep=0.01cm] (P0) at (0,0) {$P_0$};

% Level 1
\node[draw,circle] (P11) at ( 3.0,-1.) {$P_1$};
\node[draw,circle] (P12) at ( 0.0,-1.) {$P_1$};
\node[draw,circle] (P13) at (-3.0,-1.) {$P_1$};
\draw[red] (P0) -- node[midway, right, shift={(+0.2cm,0)}] {$L_1$} (P11);
\draw[red] (P0) -- (P12);
\draw[red] (P0) -- (P13);

% Level 2
\node[draw,circle] (P21) at ( 3.0,-2) {$P_2$};
\node[draw,circle] (P22) at ( 0.0,-2) {$P_2$};
\node[draw,circle] (P23) at (-3.0,-2) {$P_2$};

\draw[green] (P11) -- node[midway, right, shift={(+0.2cm,0)}] {$L_2$} (P21);
\draw[green] (P12) -- (P22);
\draw[green] (P13) -- (P23);

% Level 3
\node[draw,circle] (P31) at ( 4.0,-3) {$P_3$};
\node[draw,circle] (P32) at ( 3.0,-3) {$P_3$};
\node[draw,circle] (P33) at ( 2.0,-3) {$P_3$};
%
\node[draw,circle] (P34) at ( 1.0,-3) {$P_3$};
\node[draw,circle] (P35) at ( 0.0,-3) {$P_3$};
\node[draw,circle] (P36) at (-1.0,-3) {$P_3$};
%
\node[draw,circle] (P37) at (-2.0,-3) {$P_3$};
\node[draw,circle] (P38) at (-3.0,-3) {$P_3$};
\node[draw,circle] (P39) at (-4.0,-3) {$P_3$};




\draw[red] (P21) -- node[midway, right, shift={(+0.2cm,0)}] {$L_3$} (P31);
\draw[red] (P21) -- (P32);
\draw[red] (P21) -- (P33);
%
\draw[red] (P22) -- (P34);
\draw[red] (P22) -- (P35);
\draw[red] (P22) -- (P36);
%
\draw[red] (P23) -- (P37);
\draw[red] (P23) -- (P38);
\draw[red] (P23) -- (P39);
%

% layer 4

\node[draw,circle] (P41) at ( 4.0,-4) {$P_4$};
\node[draw,circle] (P42) at ( 3.0,-4) {$P_4$};
\node[draw,circle] (P43) at ( 2.0,-4) {$P_4$};
%
\node[draw,circle] (P44) at ( 1.0,-4) {$P_4$};
\node[draw,circle] (P45) at ( 0.0,-4) {$P_4$};
\node[draw,circle] (P46) at (-1.0,-4) {$P_4$};
%
\node[draw,circle] (P47) at (-2.0,-4) {$P_4$};
\node[draw,circle] (P48) at (-3.0,-4) {$P_4$};
\node[draw,circle] (P49) at (-4.0,-4) {$P_4$};
%

\draw[green] (P31) -- node[midway, right, shift={(+0.2cm,0)}] {$L_4$} (P41);
\draw[green] (P32) -- (P42);
\draw[green] (P33) -- (P43);
%
\draw[green] (P34) -- (P44);
\draw[green] (P35) -- (P45);
\draw[green] (P36) -- (P46);
%
\draw[green] (P37) -- (P47);
\draw[green] (P38) -- (P48);
\draw[green] (P39) -- (P49);
%

% layer 5

\node[draw,circle] (P51) at ( 4.5,-5) {$P_5$};
\node[draw,circle] (P52) at ( 4.0,-5) {$P_5$};
\node[draw,circle] (P53) at ( 3.5,-5) {$P_5$};
%
\node[draw,circle] (P54) at ( 3.0,-5) {$P_5$};
\node[draw,circle] (P55) at ( 2.5,-5) {$P_5$};
\node[draw,circle] (P56) at ( 2.0,-5) {$P_5$};
%
\node[draw,circle] (P57) at ( 1.5,-5) {$P_5$};
\node[draw,circle] (P58) at ( 1.0,-5) {$P_5$};
\node[draw,circle] (P59) at ( 0.5,-5) {$P_5$};
%
\node[draw,circle] (P510) at (-0.5,-5) {$P_5$};
\node[draw,circle] (P511) at (-1.0,-5) {$P_5$};
\node[draw,circle] (P512) at (-1.5,-5) {$P_5$};
%
\node[draw,circle] (P513) at (-2.0,-5) {$P_5$};
\node[draw,circle] (P514) at (-2.5,-5) {$P_5$};
\node[draw,circle] (P515) at (-3.0,-5) {$P_5$};
%
\node[draw,circle] (P516) at (-3.5,-5) {$P_5$};
\node[draw,circle] (P517) at (-4.0,-5) {$P_5$};
\node[draw,circle] (P518) at (-4.5,-5) {$P_5$};
%




%
\draw[red] (P42) -- node[midway, right, shift={(+0.2cm,0)}] {$L_5$} (P51);
\draw[red] (P42) -- (P52);
\draw[red] (P42) -- (P53);
%
\draw[red] (P43) -- (P54);
\draw[red] (P43) -- (P55);
\draw[red] (P43) -- (P56);
%
\draw[red] (P44) -- (P57);
\draw[red] (P44) -- (P58);
\draw[red] (P44) -- (P59);
%
\draw[red] (P46) -- (P510);
\draw[red] (P46) -- (P511);
\draw[red] (P46) -- (P512);
%
\draw[red] (P47) -- (P513);
\draw[red] (P47) -- (P514);
\draw[red] (P47) -- (P515);
%
\draw[red] (P48) -- (P516);
\draw[red] (P48) -- (P517);
\draw[red] (P48) -- (P518);

% layer 6
\node[draw,circle] (P61) at ( 4.5,-6) {$P_6$};
\node[draw,circle] (P62) at ( 4.0,-6) {$P_6$};
\node[draw,circle] (P63) at ( 3.5,-6) {$P_6$};
%
\node[draw,circle] (P64) at ( 3.0,-6) {$P_6$};
\node[draw,circle] (P65) at ( 2.5,-6) {$P_6$};
\node[draw,circle] (P66) at ( 2.0,-6) {$P_6$};
%
\node[draw,circle] (P67) at ( 1.5,-6) {$P_6$};
\node[draw,circle] (P68) at ( 1.0,-6) {$P_6$};
\node[draw,circle] (P69) at ( 0.5,-6) {$P_6$};
%
\node[draw,circle] (P610) at (-0.5,-6) {$P_6$};
\node[draw,circle] (P611) at (-1.0,-6) {$P_6$};
\node[draw,circle] (P612) at (-1.5,-6) {$P_6$};
%
\node[draw,circle] (P613) at (-2.0,-6) {$P_6$};
\node[draw,circle] (P614) at (-2.5,-6) {$P_6$};
\node[draw,circle] (P615) at (-3.0,-6) {$P_6$};
%
\node[draw,circle] (P616) at (-3.5,-6) {$P_6$};
\node[draw,circle] (P617) at (-4.0,-6) {$P_6$};
\node[draw,circle] (P618) at (-4.5,-6) {$P_6$};
%

%
\draw[green] (P51) -- node[midway, right, shift={(+0.2cm,0)}] {$L_6$} (P61);
\draw[green] (P52) -- (P62);
\draw[green] (P53) -- (P63);
%
\draw[green] (P54) -- (P64);
\draw[green] (P55) -- (P65);
\draw[green] (P56) -- (P66);
%
\draw[green] (P57) -- (P67);
\draw[green] (P58) -- (P68);
\draw[green] (P59) -- (P69);
%
\draw[green] (P510) -- (P610);
\draw[green] (P511) -- (P611);
\draw[green] (P512) -- (P612);
%
\draw[green] (P513) -- (P613);
\draw[green] (P514) -- (P614);
\draw[green] (P515) -- (P615);
%
\draw[green] (P516) -- (P616);
\draw[green] (P517) -- (P617);
\draw[green] (P518) -- (P618);
%

% % Level 7

\node[draw,circle,font=\footnotesize, inner sep=0.01cm] (P7) at (0,-9) {$P_T$};

\draw[orange] (P61) -- node[midway, right, shift={(+0.2cm,0)}] {$L_7$} (P7);
\draw[orange] (P62) -- (P7);
\draw[orange] (P63) -- (P7);
%   
\draw[orange] (P64) -- (P7);
\draw[orange] (P65) -- (P7);
\draw[orange] (P66) -- (P7);
%
\draw[orange] (P67) -- (P7);
\draw[orange] (P68) -- (P7);
\draw[orange] (P69) -- (P7);
%
\draw[orange] (P610) -- (P7);
\draw[orange] (P611) -- (P7);
\draw[orange] (P612) -- (P7);
%
\draw[orange] (P613) -- (P7);
\draw[orange] (P614) -- (P7);
\draw[orange] (P615) -- (P7);
%
\draw[orange] (P616) -- (P7);
\draw[orange] (P617) -- (P7);
\draw[orange] (P618) -- (P7);
%

% curved connection between P41 and P7
\draw[blue] (P11) to[out=0,in=0] (P7);

% curved large connction between P49 and P7
\draw[blue] (P13) to[out=180,in=180] (P7);

% connect P45 and P7
\draw[blue] (P12) to[out=180,in=180] (P7);


\end{tikzpicture}

  \caption{Possible Combination}
  \label{fig:possiblecombination}
\end{figure}
%
\begin{table}[ht]
  \centering
  \caption{Possible combination. $-$ = none,\\ $L$ = Left, $R$ = Right, $S$ = Straight}
  \label{tab:possiblecombination}
  \begin{tabular}{c|c|c|c}
  \hline
  \textbf{\#} & \textbf{$L_2$} & \textbf{$L_4$} & \textbf{$L_6$} \\
  \hline
  1 & $L$ & $S$ & $L$  \\
  2 & $L$ & $S$ & $R$  \\
  3 & $L$ & $R$ & $L$  \\
  4 & $L$ & $R$ & $S$  \\
  5 & $S$ & $L$ & $S$  \\
  6 & $S$ & $L$ & $R$  \\
  7 & $S$ & $R$ & $L$  \\
  8 & $S$ & $R$ & $S$  \\
  9 & $R$ & $L$ & $S$  \\
  10 & $R$ & $L$ & $R$  \\
  11 & $R$ & $S$ & $L$  \\
  12 & $R$ & $S$ & $R$  \\
  13 & $L$ & $-$ & $-$  \\
  14 & $S$ & $-$ & $-$  \\
  15 & $R$ & $-$ & $-$  \\
  16 & $L$ & $S$ & $-$  \\
  17 & $L$ & $R$ & $-$  \\
  18 & $S$ & $L$ & $-$  \\
  19 & $S$ & $R$ & $-$  \\
  20 & $R$ & $L$ & $-$  \\
  21 & $R$ & $S$ & $-$  \\
  \hline
  \end{tabular}
\end{table}
%
\section*{Numeric solution}
%
The problem was tackled with two numerical approaches. The first is to write the problem as an Optimal Control Problem (OCP) and solve it with the indirect direct method and constraints formulated as penalization with PINS (PINS Is Not a Solver)\cite{biral2016notes}. The second approach was a naive exploration of all the possible manoeuvre combinations to solve the problem.
%
\subsection*{PINS}
%
The problem was developed and solved using PINS to analyze the numerical solution of the optimal control problem and to have a reference solution to compare the naive exploration of all the possible combinations of manoeuvres.\\
%
\begin{figure}[htb!]
  \centering
  
%
\begin{tikzpicture}[scale = 0.7]

\begin{axis}[%
width=\linewidth,
height=0.776\linewidth,
at={(0\linewidth,0\linewidth)},
scale only axis,
xmin=0,
xmax=20,
xlabel style={font=\color{white!15!black}},
xlabel={$x(m)$},
ymin=-2.88709677419355,
ymax=12.8870967741935,
ylabel style={font=\color{white!15!black}},
ylabel={$y(m)$},
axis background/.style={fill=white},
title style={font=\bfseries},
title={PINS Traj.: $L_{tot}$ = 22.6899, $k_{max}$ = 0.2, $J_{max}$ = 0.2},
xmajorgrids,
ymajorgrids
]
\addplot [color=dodgerblue, line width=2.0pt, forget plot]
  table[row sep=crcr]{%
0	0\\
0.0226898947273693	5.84073128786943e-07\\
0.0453797892743192	3.50443875559533e-06\\
0.0680696827387528	1.10973892321468e-05\\
0.0907595727752182	2.56992162637816e-05\\
0.113449454873232	4.96462096750791e-05\\
0.136139321635602	8.52746550606898e-05\\
0.158829162056759	0.000134920829949825\\
0.181518960801085	0.000200920998112846\\
0.204208697481263	0.000285611401638402\\
0.22689834593663	0.000391328250409564\\
0.249587873511564	0.000520407708607431\\
0.272277240333901	0.000675185877870678\\
0.294966398593393	0.000857998776739559\\
0.31765529182023	0.00107118231601292\\
0.340343854163643	0.00131707226964679\\
0.363032009670599	0.00159800424082322\\
0.385719671564624	0.00191631362281816\\
0.408406741524775	0.00227433555429708\\
0.431093108964803	0.00267440486866745\\
0.453778650312523	0.00311885603711712\\
0.476463228289459	0.00361002310496794\\
0.49914669119079	0.00415023962097398\\
0.521828872165656	0.00474183855919445\\
0.544509588497874	0.0053871522330709\\
0.567188640887142	0.0060885122013395\\
0.589865812730775	0.0068482491654088\\
0.612540869406069	0.00766869285783432\\
0.635213557553359	0.00855217192152134\\
0.657883604359866	0.00950101377928797\\
0.680550716844421	0.0105175444934209\\
0.703214581143176	0.011604088614857\\
0.725874861796404	0.0127629690216228\\
0.748531201036502	0.0139965067461675\\
0.77118321807733	0.0153070207912196\\
0.793830508405019	0.0166968279338022\\
0.816472643070384	0.0181682425170355\\
0.83910916798309	0.019723576229353\\
0.861739603207755	0.0213651378707466\\
0.884363442262134	0.0230952331056394\\
0.906980151417596	0.0249161642019347\\
0.929589169002096	0.0268302297556795\\
0.952189904705895	0.028839724400425\\
0.974781738890476	0.0309469384988594\\
0.997364021902623	0.0331541577991449\\
1.0199361252502	0.0354631561851104\\
1.04249752021524	0.03787454174361\\
1.06504774238755	0.04038826469171\\
1.08758632758746	0.043004273152405\\
1.11011281187881	0.0457225131256218\\
1.13262673157488	0.0485429285204215\\
1.15512762325193	0.0514654611237824\\
1.17761502375459	0.0544900506353008\\
1.20008847020996	0.0576166346335199\\
1.22254750003233	0.0608451486133749\\
1.24499165093802	0.0641755259498068\\
1.2674204609493	0.0676076979382275\\
1.28983346841005	0.071141593755124\\
1.31223021198881	0.0747771405018551\\
1.33461023069532	0.0785142631619125\\
1.35697306388264	0.0823528846484036\\
1.37931825126467	0.086292925757589\\
1.40164533291717	0.0903343052204551\\
1.42395384929648	0.0944769396520958\\
1.44624334123922	0.0987207436078339\\
1.4685133499824	0.103065629527954\\
1.49076341716163	0.107511507798898\\
1.51299308483279	0.112058286692782\\
1.53520189546857	0.116705872434272\\
1.55738939198191	0.121454169134236\\
1.57955511772059	0.126303078862988\\
1.60169861649281	0.131252501577325\\
1.62381943255944	0.136302335200958\\
1.64591711066206	0.141452475544044\\
1.6679911960126	0.146702816391744\\
1.69004123432416	0.152053249415242\\
1.71206677179766	0.157503664269524\\
1.73406735515584	0.16305394849468\\
1.75604253162647	0.168703987624193\\
1.7779918489801	0.174453665075135\\
1.79991485550917	0.180302862268472\\
1.82181110007022	0.186251458506485\\
1.84368013205816	0.192299331106909\\
1.8655215014537	0.198446355265598\\
1.88733475879199	0.204692404206631\\
1.90911945521622	0.211037349027873\\
1.93087514243946	0.217481058869628\\
1.95260137280573	0.224023400740231\\
1.97429769924377	0.23066423970637\\
1.99596367533691	0.237403438695264\\
2.01759885526721	0.244240858710435\\
2.03920279389597	0.251176358606245\\
2.06077504669627	0.258209795333784\\
2.08231516984629	0.265341023682543\\
2.10382272014776	0.27256989656219\\
2.12529725513499	0.279896264704887\\
2.14673833297603	0.287319976990172\\
2.16814551260103	0.294840880099741\\
2.18951835358201	0.302458818894573\\
2.21085641628528	0.310173636011755\\
2.23215926172458	0.317985172305566\\
2.25342645174355	0.32589326637288\\
2.27465754883538	0.333897755073426\\
2.29585211636349	0.341998472966115\\
2.31700971833836	0.350195252928594\\
2.33812991968731	0.358487925480896\\
2.35921228597608	0.366876319531398\\
2.3802563837427	0.375360261555561\\
2.40126178014615	0.383939576505709\\
2.42222804338596	0.392614086799645\\
2.44315474225261	0.401383613447328\\
2.46404144666415	0.410247974783907\\
2.4848877270811	0.419206987891273\\
2.50569315520768	0.428260466976701\\
2.52645730321334	0.437408225209128\\
2.54717974467531	0.446650072585914\\
2.56786005351152	0.455985818379312\\
2.58849780529718	0.465415268220842\\
2.60909257573944	0.474938227504104\\
2.62964394262413	0.484554497168135\\
2.65015148349084	0.494263878748878\\
2.67061477880055	0.504066167672785\\
2.69103340797513	0.513961161645056\\
2.71140695572758	0.523948647567879\\
2.73173500343751	0.534028419360091\\
2.75201723620171	0.544200064122829\\
2.77225381014814	0.554462247028568\\
2.79244531313149	0.564812828188752\\
2.81259243317009	0.575249536349406\\
2.83269588588832	0.585770111993644\\
2.85275641365277	0.59637230746873\\
2.87277478479302	0.607053886938605\\
2.89275179284727	0.617812626283\\
2.91268825582066	0.628646312968096\\
2.93258501545245	0.639552745897123\\
2.95244293649027	0.650529735244648\\
2.97226290597087	0.661575102276509\\
2.99204583250656	0.672686679156659\\
3.01179264557745	0.683862308741769\\
3.03150429482892	0.695099844364282\\
3.05118174937442	0.706397149604485\\
3.07082599710326	0.717752098052109\\
3.09043804399342	0.729162573057913\\
3.11001891342913	0.740626467475691\\
3.12956964552308	0.752141683395121\\
3.14909129644322	0.763706131865857\\
3.16858493774398	0.775317732613274\\
3.18805165570169	0.78697441374623\\
3.2074925506541	0.798674111457259\\
3.22690873634394	0.810414769715558\\
3.24630133926612	0.822194339953172\\
3.26567149801869	0.834010780744734\\
3.28502036265711	0.845862057481189\\
3.30434909405186	0.857746142037853\\
3.32365886324898	0.869661012437245\\
3.34295085083352	0.881604652507096\\
3.36222624629556	0.893575051533958\\
3.38148624739853	0.905570203912886\\
3.40073205954966	0.917588108793667\\
3.41996489517215	0.929626769724133\\
3.43918597307875	0.941684194291128\\
3.45839651784642	0.953758393759812\\
3.47759775919152	0.96584738271205\\
3.49679093134502	0.977949178684816\\
3.515977272427	0.990061801809732\\
3.53515802381963	1.00218327445518\\
3.55433442953743	1.01431162087282\\
3.57350773559324	1.02644486685115\\
3.59267918935769	1.03858103937949\\
3.61185003890915	1.05071816632765\\
3.63102153236911	1.06285427614895\\
3.6501949172159	1.0749873976186\\
3.66937143956421	1.08711555962713\\
3.68855234338978	1.09923679106269\\
3.70773886966057	1.11134912084439\\
3.72693225529789	1.12345057823006\\
3.74613373179666	1.13553919367383\\
3.7653445230544	1.14761300095786\\
3.78456584082437	1.15967004314348\\
3.80379886660889	1.17170840026779\\
3.82304419457644	1.18372708049907\\
3.84230127765504	1.19572691693063\\
3.86156895790171	1.20770973049808\\
3.88084603960678	1.21967741381677\\
3.900131324638	1.23163187341747\\
3.91942361867966	1.24357501846768\\
3.93872173466671	1.2555087539445\\
3.95802449626706	1.26743497370099\\
3.9773307429012	1.27935555103525\\
3.99663933834376	1.29127232346388\\
4.01594918713202	1.30318706487335\\
4.03525926896255	1.31510142859025\\
4.05456871454781	1.32701682343349\\
4.07387693788523	1.33893419875831\\
4.09318373421829	1.35085388576904\\
4.11248921988148	1.36277569546341\\
4.13179367213018	1.37469917844692\\
4.15109740275648	1.38662382967819\\
4.17040069201192	1.39854919537492\\
4.18970375883603	1.41047492110815\\
4.2090067537649	1.42240076321076\\
4.22830976621681	1.43432657695078\\
4.24761283883121	1.4462522933118\\
4.26691598304353	1.45817789378299\\
4.28621919246008	1.47010338871197\\
4.30552245272314	1.48202880133783\\
4.32482574787919	1.49395415748339\\
4.34412906388306	1.50587947988302\\
4.36343239005166	1.51780478582915\\
4.38273571922934	1.52973008690452\\
4.40203904727073	1.54165538981919\\
4.42134237226511	1.55358069766602\\
4.44064569376182	1.56550601117448\\
4.45994901212818	1.57743132974998\\
4.47925232808417	1.5893566522271\\
4.49855564240607	1.60128197734931\\
4.51785895576644	1.61320730402793\\
4.53716226867056	1.62513263144505\\
4.55646558145271	1.63705795905961\\
4.57576889430336	1.64898328656331\\
4.59507220730751	1.66090861381852\\
4.61437552048217	1.67283394079774\\
4.6336788338069	1.68475926753404\\
4.65298214724576	1.6966845940856\\
4.67228546076105	1.70860992051346\\
4.69158877432059	1.72053524686966\\
4.71089208790054	1.73246057319286\\
4.73019540148528	1.74438589950828\\
4.74949871506608	1.75631122583007\\
4.7688020286392	1.76823655216433\\
4.78810534220401	1.78016187851201\\
4.80740865576164	1.79208720487132\\
4.82671196931387	1.80401253123937\\
4.84601528286253	1.8159378576132\\
4.86531859640916	1.82786318399031\\
4.88462190995488	1.8397885103689\\
4.90392522350042	1.85171383674778\\
4.92322853704618	1.86363916312631\\
4.94253185059232	1.87556448950421\\
4.96183516413887	1.88748981588146\\
4.98113847768576	1.89941514225816\\
5.0004417912329	1.91134046863444\\
5.01974510478021	1.92326579501046\\
5.03904841832762	1.93519112138632\\
5.05835173187506	1.94711644776212\\
5.07765504542251	1.95904177413791\\
5.09695835896995	1.97096710051371\\
5.11626167251737	1.98289242688955\\
5.13556498606477	1.99481775326542\\
5.15486829961215	2.00674307964132\\
5.17417161315953	2.01866840601723\\
5.19347492670689	2.03059373239316\\
5.21277824025425	2.04251905876909\\
5.23208155380161	2.05444438514503\\
5.25138486734896	2.06636971152097\\
5.27068818089632	2.07829503789691\\
5.28999149444368	2.09022036427284\\
5.30929480799104	2.10214569064877\\
5.3285981215384	2.11407101702471\\
5.34790143508576	2.12599634340064\\
5.36720474863313	2.13792166977657\\
5.38650806218049	2.1498469961525\\
5.40581137572785	2.16177232252843\\
5.42511468927521	2.17369764890436\\
5.44441800282257	2.18562297528029\\
5.46372131636994	2.19754830165622\\
5.4830246299173	2.20947362803215\\
5.50232794346466	2.22139895440809\\
5.52163125701202	2.23332428078402\\
5.54093457055939	2.24524960715995\\
5.56023788410675	2.25717493353588\\
5.57954119765411	2.26910025991181\\
5.59884451120147	2.28102558628774\\
5.61814782474883	2.29295091266367\\
5.6374511382962	2.3048762390396\\
5.65675445184356	2.31680156541553\\
5.67605776539092	2.32872689179146\\
5.69536107893828	2.3406522181674\\
5.71466439248564	2.35257754454333\\
5.73396770603301	2.36450287091926\\
5.75327101958037	2.37642819729519\\
5.77257433312773	2.38835352367112\\
5.79187764667509	2.40027885004705\\
5.81118096022245	2.41220417642298\\
5.83048427376981	2.42412950279891\\
5.84978758731718	2.43605482917484\\
5.86909090086454	2.44798015555077\\
5.8883942144119	2.4599054819267\\
5.90769752795926	2.47183080830264\\
5.92700084150662	2.48375613467857\\
5.94630415505399	2.4956814610545\\
5.96560746860135	2.50760678743043\\
5.98491078214871	2.51953211380636\\
6.00421409569607	2.53145744018229\\
6.02351740924343	2.54338276655822\\
6.0428207227908	2.55530809293415\\
6.06212403633816	2.56723341931008\\
6.08142734988552	2.57915874568601\\
6.10073066343288	2.59108407206194\\
6.12003397698024	2.60300939843788\\
6.13933729052761	2.61493472481381\\
6.15864060407497	2.62686005118974\\
6.17794391762233	2.63878537756567\\
6.19724723116969	2.6507107039416\\
6.21655054471705	2.66263603031753\\
6.23585385826442	2.67456135669346\\
6.25515717181178	2.68648668306939\\
6.27446048535914	2.69841200944532\\
6.2937637989065	2.71033733582125\\
6.31306711245386	2.72226266219719\\
6.33237042600123	2.73418798857312\\
6.35167373954859	2.74611331494905\\
6.37097705309595	2.75803864132498\\
6.39028036664331	2.76996396770091\\
6.40958368019067	2.78188929407684\\
6.42888699373804	2.79381462045277\\
6.4481903072854	2.8057399468287\\
6.46749362083276	2.81766527320463\\
6.48679693438012	2.82959059958056\\
6.50610024792748	2.8415159259565\\
6.52540356147485	2.85344125233243\\
6.54470687502221	2.86536657870836\\
6.56401018856957	2.87729190508429\\
6.58331350211693	2.88921723146022\\
6.60261681566429	2.90114255783615\\
6.62192012921166	2.91306788421208\\
6.64122344275902	2.92499321058801\\
6.66052675630638	2.93691853696394\\
6.67983006985374	2.94884386333987\\
6.6991333834011	2.96076918971581\\
6.71843669694847	2.97269451609174\\
6.73774001049583	2.98461984246767\\
6.75704332404319	2.9965451688436\\
6.77634663759055	3.00847049521953\\
6.79564995113791	3.02039582159546\\
6.81495326468528	3.03232114797139\\
6.83425657823264	3.04424647434732\\
6.85355989178	3.05617180072325\\
6.87286320532736	3.06809712709918\\
6.89216651887472	3.08002245347511\\
6.91146983242209	3.09194777985105\\
6.93077314596945	3.10387310622698\\
6.95007645951681	3.11579843260291\\
6.96937977306417	3.12772375897884\\
6.98868308661153	3.13964908535477\\
7.00798640015889	3.1515744117307\\
7.02728971370626	3.16349973810663\\
7.04659302725362	3.17542506448256\\
7.06589634080098	3.18735039085849\\
7.08519965434834	3.19927571723442\\
7.1045029678957	3.21120104361035\\
7.12380628144307	3.22312636998629\\
7.14310959499043	3.23505169636222\\
7.16241290853779	3.24697702273815\\
7.18171622208515	3.25890234911408\\
7.20101953563251	3.27082767549001\\
7.22032284917988	3.28275300186594\\
7.23962616272724	3.29467832824187\\
7.2589294762746	3.3066036546178\\
7.27823278982196	3.31852898099373\\
7.29753610336932	3.33045430736966\\
7.31683941691669	3.3423796337456\\
7.33614273046405	3.35430496012153\\
7.35544604401141	3.36623028649746\\
7.37474935755877	3.37815561287339\\
7.39405267110613	3.39008093924932\\
7.4133559846535	3.40200626562525\\
7.43265929820086	3.41393159200118\\
7.45196261174822	3.42585691837711\\
7.47126592529558	3.43778224475304\\
7.49056923884294	3.44970757112897\\
7.50987255239031	3.46163289750491\\
7.52917586593767	3.47355822388084\\
7.54847917948503	3.48548355025677\\
7.56778249303239	3.4974088766327\\
7.58708580657975	3.50933420300863\\
7.60638912012712	3.52125952938456\\
7.62569243367448	3.53318485576049\\
7.64499574722184	3.54511018213642\\
7.6642990607692	3.55703550851235\\
7.68360237431656	3.56896083488828\\
7.70290568786393	3.58088616126422\\
7.72220900141129	3.59281148764015\\
7.74151231495865	3.60473681401608\\
7.76081562850601	3.61666214039201\\
7.78011894205337	3.62858746676794\\
7.79942225560074	3.64051279314387\\
7.8187255691481	3.6524381195198\\
7.83802888269546	3.66436344589573\\
7.85733219624282	3.67628877227166\\
7.87663550979018	3.68821409864759\\
7.89593882333755	3.70013942502352\\
7.91524213688491	3.71206475139946\\
7.93454545043227	3.72399007777539\\
7.95384876397963	3.73591540415132\\
7.97315207752699	3.74784073052725\\
7.99245539107436	3.75976605690318\\
8.01175870462172	3.77169138327911\\
8.03106201816908	3.78361670965504\\
8.05036533171644	3.79554203603097\\
8.0696686452638	3.8074673624069\\
8.08897195881116	3.81939268878283\\
8.10827527235853	3.83131801515877\\
8.12757858590589	3.8432433415347\\
8.14688189945325	3.85516866791063\\
8.16618521300061	3.86709399428656\\
8.18548852654797	3.87901932066249\\
8.20479184009534	3.89094464703842\\
8.2240951536427	3.90286997341435\\
8.24339846719006	3.91479529979028\\
8.26270178073742	3.92672062616621\\
8.28200509428478	3.93864595254214\\
8.30130840783215	3.95057127891807\\
8.32061172137951	3.96249660529401\\
8.33991503492687	3.97442193166994\\
8.35921834847423	3.98634725804587\\
8.37852166202159	3.9982725844218\\
8.39782497556896	4.01019791079773\\
8.41712828911632	4.02212323717366\\
8.43643160266368	4.03404856354959\\
8.45573491621104	4.04597388992552\\
8.4750382297584	4.05789921630145\\
8.49434154330577	4.06982454267738\\
8.51364485685313	4.08174986905332\\
8.53294817040049	4.09367519542925\\
8.55225148394785	4.10560052180518\\
8.57155479749521	4.11752584818111\\
8.59085811104258	4.12945117455704\\
8.61016142458994	4.14137650093297\\
8.6294647381373	4.1533018273089\\
8.64876805168466	4.16522715368483\\
8.66807136523202	4.17715248006076\\
8.68737467877939	4.18907780643669\\
8.70667799232675	4.20100313281262\\
8.72598130587411	4.21292845918856\\
8.74528461942147	4.22485378556449\\
8.76458793296883	4.23677911194042\\
8.7838912465162	4.24870443831635\\
8.80319456006356	4.26062976469228\\
8.82249787361092	4.27255509106821\\
8.84180118715828	4.28448041744414\\
8.86110450070564	4.29640574382007\\
8.88040781425301	4.308331070196\\
8.89971112780037	4.32025639657193\\
8.91901444134773	4.33218172294787\\
8.93831775489509	4.3441070493238\\
8.95762106844245	4.35603237569973\\
8.97692438198982	4.36795770207566\\
8.99622769553718	4.37988302845159\\
9.01553100908454	4.39180835482752\\
9.0348343226319	4.40373368120345\\
9.05413763617926	4.41565900757938\\
9.07344094972663	4.42758433395531\\
9.09274426327399	4.43950966033124\\
9.11204757682135	4.45143498670718\\
9.13135089036871	4.46336031308311\\
9.15065420391607	4.47528563945904\\
9.16995751746344	4.48721096583497\\
9.1892608310108	4.4991362922109\\
9.20856414455816	4.51106161858683\\
9.22786745810552	4.52298694496276\\
9.24717077165288	4.53491227133869\\
9.26647408520025	4.54683759771462\\
9.28577739874761	4.55876292409055\\
9.30508071229497	4.57068825046649\\
9.32438402584233	4.58261357684242\\
9.34368733938969	4.59453890321835\\
9.36299065293706	4.60646422959428\\
9.38229396648442	4.61838955597021\\
9.40159728003178	4.63031488234614\\
9.42090059357914	4.64224020872207\\
9.4402039071265	4.654165535098\\
9.45950722067386	4.66609086147393\\
9.47881053422123	4.67801618784986\\
9.49811384776859	4.68994151422579\\
9.51741716131595	4.70186684060173\\
9.53672047486331	4.71379216697766\\
9.55602378841068	4.72571749335359\\
9.57532710195804	4.73764281972952\\
9.5946304155054	4.74956814610545\\
9.61393372905276	4.76149347248138\\
9.63323704260012	4.77341879885731\\
9.65254035614748	4.78534412523324\\
9.67184366969485	4.79726945160917\\
9.69114698324221	4.8091947779851\\
9.71045029678957	4.82112010436103\\
9.72975361033693	4.83304543073697\\
9.74905692388429	4.8449707571129\\
9.76836023743166	4.85689608348883\\
9.78766355097902	4.86882140986476\\
9.80696686452638	4.88074673624069\\
9.82627017807374	4.89267206261662\\
9.8455734916211	4.90459738899255\\
9.86487680516847	4.91652271536848\\
9.88418011871583	4.92844804174441\\
9.90348343226319	4.94037336812034\\
9.92278674581055	4.95229869449628\\
9.94209005935791	4.96422402087221\\
9.96139337290528	4.97614934724814\\
9.98069668645264	4.98807467362407\\
10	5\\
10.0193033135474	5.01192532637593\\
10.0386066270947	5.02385065275186\\
10.0579099406421	5.03577597912779\\
10.0772132541894	5.04770130550372\\
10.0965165677368	5.05962663187966\\
10.1158198812842	5.07155195825559\\
10.1351231948315	5.08347728463152\\
10.1544265083789	5.09540261100745\\
10.1737298219263	5.10732793738338\\
10.1930331354736	5.11925326375931\\
10.212336449021	5.13117859013524\\
10.2316397625683	5.14310391651117\\
10.2509430761157	5.1550292428871\\
10.2702463896631	5.16695456926303\\
10.2895497032104	5.17887989563897\\
10.3088530167578	5.1908052220149\\
10.3281563303052	5.20273054839083\\
10.3474596438525	5.21465587476676\\
10.3667629573999	5.22658120114269\\
10.3860662709472	5.23850652751862\\
10.4053695844946	5.25043185389455\\
10.424672898042	5.26235718027048\\
10.4439762115893	5.27428250664641\\
10.4632795251367	5.28620783302234\\
10.482582838684	5.29813315939827\\
10.5018861522314	5.31005848577421\\
10.5211894657788	5.32198381215014\\
10.5404927793261	5.33390913852607\\
10.5597960928735	5.345834464902\\
10.5790994064209	5.35775979127793\\
10.5984027199682	5.36968511765386\\
10.6177060335156	5.38161044402979\\
10.6370093470629	5.39353577040572\\
10.6563126606103	5.40546109678165\\
10.6756159741577	5.41738642315758\\
10.694919287705	5.42931174953352\\
10.7142226012524	5.44123707590945\\
10.7335259147998	5.45316240228538\\
10.7528292283471	5.46508772866131\\
10.7721325418945	5.47701305503724\\
10.7914358554418	5.48893838141317\\
10.8107391689892	5.5008637077891\\
10.8300424825366	5.51278903416503\\
10.8493457960839	5.52471436054096\\
10.8686491096313	5.53663968691689\\
10.8879524231787	5.54856501329282\\
10.907255736726	5.56049033966876\\
10.9265590502734	5.57241566604469\\
10.9458623638207	5.58434099242062\\
10.9651656773681	5.59626631879655\\
10.9844689909155	5.60819164517248\\
11.0037723044628	5.62011697154841\\
11.0230756180102	5.63204229792434\\
11.0423789315575	5.64396762430027\\
11.0616822451049	5.6558929506762\\
11.0809855586523	5.66781827705213\\
11.1002888721996	5.67974360342807\\
11.119592185747	5.691668929804\\
11.1388954992944	5.70359425617993\\
11.1581988128417	5.71551958255586\\
11.1775021263891	5.72744490893179\\
11.1968054399364	5.73937023530772\\
11.2161087534838	5.75129556168365\\
11.2354120670312	5.76322088805958\\
11.2547153805785	5.77514621443551\\
11.2740186941259	5.78707154081144\\
11.2933220076733	5.79899686718737\\
11.3126253212206	5.8109221935633\\
11.331928634768	5.82284751993924\\
11.3512319483153	5.83477284631517\\
11.3705352618627	5.8466981726911\\
11.3898385754101	5.85862349906703\\
11.4091418889574	5.87054882544296\\
11.4284452025048	5.88247415181889\\
11.4477485160521	5.89439947819482\\
11.4670518295995	5.90632480457075\\
11.4863551431469	5.91825013094668\\
11.5056584566942	5.93017545732261\\
11.5249617702416	5.94210078369855\\
11.544265083789	5.95402611007448\\
11.5635683973363	5.96595143645041\\
11.5828717108837	5.97787676282634\\
11.602175024431	5.98980208920227\\
11.6214783379784	6.0017274155782\\
11.6407816515258	6.01365274195413\\
11.6600849650731	6.02557806833006\\
11.6793882786205	6.03750339470599\\
11.6986915921679	6.04942872108192\\
11.7179949057152	6.06135404745786\\
11.7372982192626	6.07327937383379\\
11.7566015328099	6.08520470020972\\
11.7759048463573	6.09713002658565\\
11.7952081599047	6.10905535296158\\
11.814511473452	6.12098067933751\\
11.8338147869994	6.13290600571344\\
11.8531181005467	6.14483133208937\\
11.8724214140941	6.1567566584653\\
11.8917247276415	6.16868198484123\\
11.9110280411888	6.18060731121717\\
11.9303313547362	6.1925326375931\\
11.9496346682836	6.20445796396903\\
11.9689379818309	6.21638329034496\\
11.9882412953783	6.22830861672089\\
12.0075446089256	6.24023394309682\\
12.026847922473	6.25215926947275\\
12.0461512360204	6.26408459584868\\
12.0654545495677	6.27600992222461\\
12.0847578631151	6.28793524860054\\
12.1040611766625	6.29986057497648\\
12.1233644902098	6.31178590135241\\
12.1426678037572	6.32371122772834\\
12.1619711173045	6.33563655410427\\
12.1812744308519	6.3475618804802\\
12.2005777443993	6.35948720685613\\
12.2198810579466	6.37141253323206\\
12.239184371494	6.38333785960799\\
12.2584876850414	6.39526318598392\\
12.2777909985887	6.40718851235985\\
12.2970943121361	6.41911383873579\\
12.3163976256834	6.43103916511172\\
12.3357009392308	6.44296449148765\\
12.3550042527782	6.45488981786358\\
12.3743075663255	6.46681514423951\\
12.3936108798729	6.47874047061544\\
12.4129141934202	6.49066579699137\\
12.4322175069676	6.5025911233673\\
12.451520820515	6.51451644974323\\
12.4708241340623	6.52644177611916\\
12.4901274476097	6.53836710249509\\
12.5094307611571	6.55029242887103\\
12.5287340747044	6.56221775524696\\
12.5480373882518	6.57414308162289\\
12.5673407017991	6.58606840799882\\
12.5866440153465	6.59799373437475\\
12.6059473288939	6.60991906075068\\
12.6252506424412	6.62184438712661\\
12.6445539559886	6.63376971350254\\
12.663857269536	6.64569503987847\\
12.6831605830833	6.6576203662544\\
12.7024638966307	6.66954569263034\\
12.721767210178	6.68147101900627\\
12.7410705237254	6.6933963453822\\
12.7603738372728	6.70532167175813\\
12.7796771508201	6.71724699813406\\
12.7989804643675	6.72917232450999\\
12.8182837779148	6.74109765088592\\
12.8375870914622	6.75302297726185\\
12.8568904050096	6.76494830363778\\
12.8761937185569	6.77687363001371\\
12.8954970321043	6.78879895638964\\
12.9148003456517	6.80072428276558\\
12.934103659199	6.81264960914151\\
12.9534069727464	6.82457493551744\\
12.9727102862937	6.83650026189337\\
12.9920135998411	6.8484255882693\\
13.0113169133885	6.86035091464523\\
13.0306202269358	6.87227624102116\\
13.0499235404832	6.88420156739709\\
13.0692268540306	6.89612689377302\\
13.0885301675779	6.90805222014895\\
13.1078334811253	6.91997754652489\\
13.1271367946726	6.93190287290082\\
13.14644010822	6.94382819927675\\
13.1657434217674	6.95575352565268\\
13.1850467353147	6.96767885202861\\
13.2043500488621	6.97960417840454\\
13.2236533624094	6.99152950478047\\
13.2429566759568	7.0034548311564\\
13.2622599895042	7.01538015753233\\
13.2815633030515	7.02730548390826\\
13.3008666165989	7.0392308102842\\
13.3201699301463	7.05115613666013\\
13.3394732436936	7.06308146303606\\
13.358776557241	7.07500678941199\\
13.3780798707883	7.08693211578792\\
13.3973831843357	7.09885744216385\\
13.4166864978831	7.11078276853978\\
13.4359898114304	7.12270809491571\\
13.4552931249778	7.13463342129164\\
13.4745964385252	7.14655874766757\\
13.4938997520725	7.15848407404351\\
13.5132030656199	7.17040940041944\\
13.5325063791672	7.18233472679537\\
13.5518096927146	7.1942600531713\\
13.571113006262	7.20618537954723\\
13.5904163198093	7.21811070592316\\
13.6097196333567	7.23003603229909\\
13.629022946904	7.24196135867502\\
13.6483262604514	7.25388668505095\\
13.6676295739988	7.26581201142688\\
13.6869328875461	7.27773733780281\\
13.7062362010935	7.28966266417874\\
13.7255395146409	7.30158799055468\\
13.7448428281882	7.31351331693061\\
13.7641461417356	7.32543864330654\\
13.7834494552829	7.33736396968247\\
13.8027527688303	7.3492892960584\\
13.8220560823777	7.36121462243433\\
13.841359395925	7.37313994881026\\
13.8606627094724	7.38506527518619\\
13.8799660230198	7.39699060156212\\
13.8992693365671	7.40891592793805\\
13.9185726501145	7.42084125431399\\
13.9378759636618	7.43276658068992\\
13.9571792772092	7.44469190706585\\
13.9764825907566	7.45661723344178\\
13.9957859043039	7.46854255981771\\
14.0150892178513	7.48046788619364\\
14.0343925313987	7.49239321256957\\
14.053695844946	7.5043185389455\\
14.0729991584934	7.51624386532143\\
14.0923024720407	7.52816919169736\\
14.1116057855881	7.5400945180733\\
14.1309090991355	7.55201984444923\\
14.1502124126828	7.56394517082516\\
14.1695157262302	7.57587049720109\\
14.1888190397775	7.58779582357702\\
14.2081223533249	7.59972114995295\\
14.2274256668723	7.61164647632888\\
14.2467289804196	7.62357180270481\\
14.266032293967	7.63549712908074\\
14.2853356075144	7.64742245545667\\
14.3046389210617	7.65934778183261\\
14.3239422346091	7.67127310820854\\
14.3432455481564	7.68319843458447\\
14.3625488617038	7.6951237609604\\
14.3818521752512	7.70704908733633\\
14.4011554887985	7.71897441371226\\
14.4204588023459	7.73089974008819\\
14.4397621158933	7.74282506646412\\
14.4590654294406	7.75475039284005\\
14.478368742988	7.76667571921598\\
14.4976720565353	7.77860104559191\\
14.5169753700827	7.79052637196785\\
14.5362786836301	7.80245169834378\\
14.5555819971774	7.81437702471971\\
14.5748853107248	7.82630235109564\\
14.5941886242721	7.83822767747157\\
14.6134919378195	7.8501530038475\\
14.6327952513669	7.86207833022343\\
14.6520985649142	7.87400365659936\\
14.6714018784616	7.88592898297529\\
14.690705192009	7.89785430935123\\
14.7100085055563	7.90977963572716\\
14.7293118191037	7.92170496210309\\
14.748615132651	7.93363028847903\\
14.7679184461984	7.94555561485497\\
14.7872217597458	7.95748094123091\\
14.8065250732931	7.96940626760684\\
14.8258283868405	7.98133159398277\\
14.8451317003878	7.99325692035868\\
14.8644350139352	8.00518224673458\\
14.8837383274826	8.01710757311045\\
14.9030416410301	8.02903289948629\\
14.9223449545775	8.0409582258621\\
14.9416482681249	8.05288355223788\\
14.9609515816724	8.06480887861368\\
14.9802548952198	8.07673420498954\\
14.9995582087671	8.08865953136556\\
15.0188615223142	8.10058485774184\\
15.0381648358611	8.11251018411854\\
15.0574681494077	8.12443551049579\\
15.0767714629538	8.13636083687369\\
15.0960747764996	8.14828616325222\\
15.1153780900451	8.1602114896311\\
15.1346814035908	8.17213681600969\\
15.1539847171375	8.18406214238681\\
15.1732880306861	8.19598746876063\\
15.1925913442384	8.20791279512868\\
15.211894657796	8.21983812148799\\
15.2311979713608	8.23176344783567\\
15.2505012849339	8.24368877416993\\
15.2698045985147	8.25561410049172\\
15.2891079120995	8.26753942680715\\
15.3084112256794	8.27946475313034\\
15.327714539239	8.29139007948655\\
15.3470178527542	8.3033154059144\\
15.3663211661931	8.31524073246596\\
15.3856244795178	8.32716605920226\\
15.4049277926925	8.33909138618148\\
15.4242311056966	8.35101671343669\\
15.4435344185473	8.36294204094039\\
15.4628377313294	8.37486736855495\\
15.4821410442336	8.38679269597207\\
15.5014443575939	8.39871802265069\\
15.5207476719158	8.4106433477729\\
15.5400509878718	8.42256867025002\\
15.5593543062382	8.43449398882552\\
15.5786576277349	8.44641930233398\\
15.5979609527293	8.45834461018081\\
15.6172642807707	8.47026991309548\\
15.6365676099483	8.48219521417085\\
15.6558709361169	8.49412052011698\\
15.6751742521208	8.50604584251661\\
15.6944775472769	8.51797119866217\\
15.7137808075399	8.52989661128803\\
15.7330840169565	8.54182210621701\\
15.7523871611688	8.5537477066882\\
15.7716902337832	8.56567342304922\\
15.7909932462351	8.57759923678924\\
15.810296241164	8.58952507889185\\
15.8295993079881	8.60145080462508\\
15.8489025972435	8.61337617032181\\
15.8682063278698	8.62530082155308\\
15.8875107801185	8.63722430453659\\
15.9068162657817	8.64914611423096\\
15.9261230621148	8.66106580124169\\
15.9454312854522	8.67298317656651\\
15.9647407310374	8.68489857140975\\
15.984050812868	8.69681293512665\\
16.0033606616562	8.70872767653612\\
16.0226692570988	8.72064444896475\\
16.0419755037329	8.73256502629901\\
16.0612782653333	8.7444912460555\\
16.0805763813203	8.75642498153232\\
16.099868675362	8.76836812658253\\
16.1191539603932	8.78032258618323\\
16.1384310420983	8.79229026950192\\
16.157698722345	8.80427308306937\\
16.1769558054236	8.81627291950093\\
16.1962011333911	8.82829159973221\\
16.2154341591756	8.84032995685652\\
16.2346554769456	8.85238699904214\\
16.2538662682033	8.86446080632617\\
16.2730677447021	8.87654942176994\\
16.2922611303394	8.88865087915561\\
16.3114476566102	8.90076320893731\\
16.3306285604358	8.91288444037287\\
16.3498050827841	8.9250126023814\\
16.3689784676309	8.93714572385105\\
16.3881499610908	8.94928183367235\\
16.4073208106423	8.96141896062051\\
16.4264922644068	8.97355513314885\\
16.4456655704626	8.98568837912718\\
16.4648419761804	8.99781672554482\\
16.484022727573	9.00993819819027\\
16.503209068655	9.02205082131518\\
16.5224022408085	9.03415261728795\\
16.5416034821536	9.04624160624019\\
16.5608140269213	9.05831580570887\\
16.5800351048279	9.07037323027587\\
16.5992679404503	9.08241189120633\\
16.6185137526015	9.09442979608711\\
16.6377737537044	9.10642494846604\\
16.6570491491665	9.1183953474929\\
16.676341136751	9.13033898756275\\
16.6956509059481	9.14225385796215\\
16.7149796373429	9.15413794251881\\
16.7343285019813	9.16598921925526\\
16.7536986607339	9.17780566004683\\
16.7730912636561	9.18958523028444\\
16.7925074493459	9.20132588854274\\
16.8119483442983	9.21302558625377\\
16.831415062256	9.22468226738673\\
16.8509087035568	9.23629386813414\\
16.8704303544769	9.24785831660488\\
16.8899810865709	9.25937353252431\\
16.9095619560066	9.27083742694209\\
16.9291740028967	9.28224790194789\\
16.9488182506256	9.29360285039552\\
16.9684957051711	9.30490015563572\\
16.9882073544225	9.31613769125823\\
17.0079541674934	9.32731332084334\\
17.0277370940291	9.33842489772349\\
17.0475570635097	9.34947026475535\\
17.0674149845476	9.36044725410288\\
17.0873117441793	9.3713536870319\\
17.1072482071527	9.382187373717\\
17.127225215207	9.3929461130614\\
17.1472435863472	9.40362769253127\\
17.1673041141117	9.41422988800636\\
17.1874075668299	9.42475046365059\\
17.2075546868685	9.43518717181125\\
17.2277461898519	9.44553775297143\\
17.2479827637983	9.45579993587717\\
17.2682649965625	9.46597158063991\\
17.2885930442724	9.47605135243212\\
17.3089665920249	9.48603883835494\\
17.3293852211995	9.49593383232721\\
17.3498485165092	9.50573612125112\\
17.3703560573759	9.51544550283186\\
17.3909074242606	9.5250617724959\\
17.4115021947028	9.53458473177916\\
17.4321399464885	9.54401418162069\\
17.4528202553247	9.55334992741409\\
17.4735426967867	9.56259177479087\\
17.4943068447923	9.5717395330233\\
17.5151122729189	9.58079301210873\\
17.5359585533359	9.58975202521609\\
17.5568452577474	9.59861638655267\\
17.577771956614	9.60738591320036\\
17.5987382198538	9.61606042349429\\
17.6197436162573	9.62463973844444\\
17.6407877140239	9.6331236804686\\
17.6618700803127	9.6415120745191\\
17.6829902816616	9.64980474707141\\
17.7041478836365	9.65800152703389\\
17.7253424511646	9.66610224492658\\
17.7465735482565	9.67410673362712\\
17.7678407382754	9.68201482769443\\
17.7891435837147	9.68982636398825\\
17.810481646418	9.69754118110543\\
17.831854487399	9.70515911990026\\
17.853261667024	9.71268002300983\\
17.874702744865	9.72010373529511\\
17.8961772798522	9.72743010343781\\
17.9176848301537	9.73465897631746\\
17.9392249533037	9.74179020466622\\
17.960797206104	9.74882364139375\\
17.9824011447328	9.75575914128956\\
18.0040363246631	9.76259656130474\\
18.0257023007562	9.76933576029363\\
18.0473986271943	9.77597659925977\\
18.0691248575605	9.78251894113037\\
18.0908805447838	9.78896265097213\\
18.112665241208	9.79530759579337\\
18.1344784985463	9.8015536447344\\
18.1563198679418	9.80770066889309\\
18.1781888999298	9.81374854149352\\
18.2000851444908	9.81969713773153\\
18.2220081510199	9.82554633492487\\
18.2439574683735	9.83129601237581\\
18.2659326448442	9.83694605150532\\
18.2879332282023	9.84249633573048\\
18.3099587656758	9.84794675058476\\
18.3320088039874	9.85329718360826\\
18.3540828893379	9.85854752445596\\
18.3761805674406	9.86369766479904\\
18.3983013835072	9.86874749842267\\
18.4204448822794	9.87369692113701\\
18.4426106080181	9.87854583086576\\
18.4647981045314	9.88329412756573\\
18.4870069151672	9.88794171330722\\
18.5092365828384	9.8924884922011\\
18.5314866500176	9.89693437047205\\
18.5537566587608	9.90127925639217\\
18.5760461507035	9.9055230603479\\
18.5983546670828	9.90966569477954\\
18.6206817487353	9.91370707424241\\
18.6430269361174	9.9176471153516\\
18.6653897693047	9.92148573683809\\
18.6877697880112	9.92522285949815\\
18.7101665315899	9.92885840624488\\
18.7325795390507	9.93239230206177\\
18.755008349062	9.93582447405019\\
18.7774524999677	9.93915485138663\\
18.79991152979	9.94238336536648\\
18.8223849762454	9.9455099493647\\
18.8448723767481	9.94853453887622\\
18.8673732684251	9.95145707147958\\
18.8898871881212	9.95427748687438\\
18.9124136724125	9.95699572684759\\
18.9349522576124	9.95961173530829\\
18.9575024797848	9.96212545825639\\
18.9800638747498	9.96453684381489\\
19.0026359780974	9.96684584220085\\
19.0252182611095	9.96905306150114\\
19.0478100952941	9.97116027559957\\
19.0704108309979	9.97316977024432\\
19.0930198485824	9.97508383579807\\
19.1156365577379	9.97690476689436\\
19.1382603967922	9.97863486212925\\
19.1608908320169	9.98027642377065\\
19.1835273569296	9.98183175748296\\
19.206169491595	9.9833031720662\\
19.2288167819227	9.98469297920878\\
19.2514687989635	9.98600349325383\\
19.2741251382036	9.98723703097838\\
19.2967854188568	9.98839591138514\\
19.3194492831556	9.98948245550658\\
19.3421163956401	9.99049898622071\\
19.3647864424466	9.99144782807848\\
19.3874591305939	9.99233130714217\\
19.4101341872692	9.99315175083459\\
19.4328113591129	9.99391148779866\\
19.4554904115021	9.99461284776693\\
19.4781711278343	9.99525816144081\\
19.5008533088092	9.99584976037903\\
19.5235367717105	9.99638997689503\\
19.5462213496875	9.99688114396288\\
19.5689068910352	9.99732559513133\\
19.5915932584752	9.9977256644457\\
19.6142803284354	9.99808368637718\\
19.6369679903294	9.99840199575918\\
19.6596561458364	9.99868292773035\\
19.6823447081798	9.99892881768399\\
19.7050336014066	9.99914200122326\\
19.7277227596661	9.99932481412213\\
19.7504121264884	9.99947959229139\\
19.7731016540634	9.99960867174959\\
19.7957913025187	9.99971438859836\\
19.8184810391989	9.99979907900189\\
19.8411708379432	9.99986507917005\\
19.8638606783644	9.99991472534494\\
19.8865505451268	9.99995035379033\\
19.9092404272248	9.99997430078374\\
19.9319303172612	9.99998890261077\\
19.9546202107257	9.99999649556124\\
19.9773101052726	9.99999941592687\\
20	10\\
};
\addplot [color=blue, line width=2.0pt, only marks, mark size=2.5pt, mark=*, mark options={solid, fill=blue, blue}, forget plot]
  table[row sep=crcr]{%
0	0\\
};
\addplot [color=blue, line width=2.0pt, only marks, mark size=2.5pt, mark=*, mark options={solid, fill=blue, blue}, forget plot]
  table[row sep=crcr]{%
20	10\\
};
\end{axis}
\end{tikzpicture}%%
  \caption{PINS solutions trajectory}
  \label{fig:PINS_sol_traj}
\end{figure}
%
The problem solved with PINS does not present exactly $7$ segments or at least not always. In fact, for some boundary conditions, the solution shows unexpected chatter and Fuller's problem. \cite{zhu2015planar}
%
\subsubsection*{Numerical Results}
%
\begin{figure}[htb!]
  \centering
  
%
\begin{tikzpicture}[scale=0.7]

\begin{axis}[%
width=0.985\linewidth,
height=\linewidth,
at={(0\linewidth,0\linewidth)},
scale only axis,
xmin=0,
xmax=25,
xlabel style={font=\color{white!15!black}},
xlabel={Time(s)},
ymin=-0.3,
ymax=0.3,
ylabel style={font=\color{white!15!black}},
ylabel={$\kappa(m^{-1})$},
axis background/.style={fill=white},
title style={font=\bfseries},
title={Curvature - PINS},
xmajorgrids,
xminorgrids,
ymajorgrids,
yminorgrids
]
\addplot [color=dodgerblue, line width=2.0pt, forget plot]
  table[row sep=crcr]{%
0	0\\
0.0226898947348868	0.00453797663558565\\
0.0453797894697736	0.00907595319824747\\
0.0680696842046603	0.0136139296841995\\
0.0907595789395471	0.0181519060893712\\
0.113449473674434	0.0226898824093786\\
0.136139368409321	0.0272278586394918\\
0.158829263144207	0.0317658347745978\\
0.181519157879094	0.0363038108091588\\
0.204209052613981	0.040841786737164\\
0.226898947348868	0.0453797625520748\\
0.249588842083755	0.0499177382467621\\
0.272278736818641	0.0544557138134328\\
0.294968631553528	0.0589936892435466\\
0.317658526288415	0.063531664527717\\
0.340348421023302	0.0680696396555971\\
0.363038315758189	0.0726076146157449\\
0.385728210493075	0.0771455893954636\\
0.408418105227962	0.081683563980613\\
0.431107999962849	0.0862215383553826\\
0.453797894697736	0.0907595125020193\\
0.476487789432622	0.0952974864004973\\
0.499177684167509	0.0998354600281139\\
0.521867578902396	0.104373433358992\\
0.544557473637283	0.108911406363463\\
0.56724736837217	0.113449379007286\\
0.589937263107056	0.117987351250665\\
0.612627157841943	0.122525323046978\\
0.63531705257683	0.127063294341123\\
0.658006947311717	0.13160126506734\\
0.680696842046603	0.136139235146264\\
0.70338673678149	0.140677204480912\\
0.726076631516377	0.145215172951047\\
0.748766526251264	0.149753140405113\\
0.771456420986151	0.15429110664832\\
0.794146315721037	0.158829071424493\\
0.816836210455924	0.163367034387256\\
0.839526105190811	0.16790499505207\\
0.862215999925698	0.172442952711501\\
0.884905894660584	0.176980906273776\\
0.907595789395471	0.181518853922845\\
0.930285684130358	0.186056792295903\\
0.952975578865245	0.190594714030596\\
0.975665473600132	0.195132597256713\\
0.998355368335018	0.199670294517643\\
1.02104526306991	0.200249061267801\\
1.04373515780479	0.199671940815581\\
1.06642505253968	0.200246284825373\\
1.08911494727457	0.199673683497262\\
1.11180484200945	0.200243369576474\\
1.13449473674434	0.199675530408991\\
1.15718463147923	0.200240305046191\\
1.17987452621411	0.199677490219173\\
1.202564420949	0.200237079702711\\
1.22525431568389	0.199679572524839\\
1.24794421041877	0.200233680822864\\
1.27063410515366	0.199681787974779\\
1.29332399988855	0.200230094336846\\
1.31601389462343	0.199684148412325\\
1.33870378935832	0.200226304648294\\
1.36139368409321	0.199686667041471\\
1.38408357882809	0.200222294425064\\
1.40677347356298	0.19968935862087\\
1.42946336829787	0.200218044355011\\
1.45215326303275	0.1996922396912\\
1.47484315776764	0.200213532859821\\
1.49753305250253	0.199695328842712\\
1.52022294723741	0.200208735758259\\
1.5429128419723	0.199698647031374\\
1.56560273670719	0.20020362586817\\
1.58829263144207	0.199702217954094\\
1.61098252617696	0.20019817253388\\
1.63367242091185	0.199706068496114\\
1.65636231564673	0.200192341062226\\
1.67905221038162	0.199710229267143\\
1.70174210511651	0.200186092046\\
1.7244319998514	0.199714735247191\\
1.74712189458628	0.20017938054773\\
1.76981178932117	0.199719626568941\\
1.79250168405606	0.200172155109015\\
1.81519157879094	0.199724949471239\\
1.83788147352583	0.200164356540286\\
1.86057136826072	0.199730757468633\\
1.8832612629956	0.200155916431974\\
1.90595115773049	0.199737112795924\\
1.92864105246538	0.200146755308981\\
1.95133094720026	0.199744088205838\\
1.97402084193515	0.200136780324083\\
1.99671073667004	0.199751769224457\\
2.01940063140492	0.200125882348946\\
2.04209052613981	0.199760257006324\\
2.0647804208747	0.200113932268892\\
2.08747031560958	0.199769671984136\\
2.11016021034447	0.200100776211375\\
2.13285010507936	0.19978015858493\\
2.15553999981424	0.200086229325721\\
2.17822989454913	0.199791891398142\\
2.20091978928402	0.20007006756203\\
2.2236096840189	0.199805083352245\\
2.24629957875379	0.20005201663503\\
2.26898947348868	0.199819996721205\\
2.29167936822357	0.200031736942116\\
2.31436926295845	0.199836958202123\\
2.33705915769334	0.200008802521548\\
2.35974905242823	0.199856379993922\\
2.38243894716311	0.19998267097367\\
2.405128841898	0.199878789977698\\
2.42781873663289	0.19995263920015\\
2.45050863136777	0.199904876178517\\
2.47319852610266	0.199917775944643\\
2.49588842083755	0.199935554576566\\
2.51857831557243	0.199876814402028\\
2.54126821030732	0.199972077076439\\
2.56395810504221	0.199827971552597\\
2.58664799977709	0.200016213091294\\
2.60933789451198	0.199768621453809\\
2.63202778924687	0.200070577680793\\
2.65471768398175	0.199694643355023\\
2.67740757871664	0.200139285649348\\
2.70009747345153	0.199598922626462\\
2.72278736818641	0.200229454092338\\
2.7454772629213	0.199467034358503\\
2.76816715765619	0.200355524608985\\
2.79085705239107	0.199258129914598\\
2.81354694712596	0.200558389004522\\
2.83623684186085	0.197268048834495\\
2.85892673659573	0.192734070262348\\
2.88161663133062	0.188198201478266\\
2.90430652606551	0.183661706939255\\
2.92699642080039	0.179124903657045\\
2.94968631553528	0.174587918847457\\
2.97237621027017	0.17005081644574\\
2.99506610500506	0.165513633286261\\
3.01775599973994	0.160976392679937\\
3.04044589447483	0.156439110454436\\
3.06313578920972	0.151901797979063\\
3.0858256839446	0.147364463819374\\
3.10851557867949	0.142827114707214\\
3.13120547341438	0.138289756142881\\
3.15389536814926	0.133752392787913\\
3.17658526288415	0.12921502873323\\
3.19927515761904	0.124677667690495\\
3.22196505235392	0.120140313135074\\
3.24465494708881	0.115602968418238\\
3.2673448418237	0.111065636860056\\
3.29003473655858	0.106528321830843\\
3.31272463129347	0.101991026826872\\
3.33541452602836	0.0974537555448233\\
3.35810442076324	0.09291651195884\\
3.38079431549813	0.088379300403792\\
3.40348421023302	0.0838421256685015\\
3.4261741049679	0.0793049931030902\\
3.44886399970279	0.074767908745363\\
3.47155389443768	0.0702308794722715\\
3.49424378917256	0.065693913184118\\
3.51693368390745	0.0611570190314259\\
3.53962357864234	0.0566202076975596\\
3.56231347337722	0.0520834917546004\\
3.58500336811211	0.0475468861162201\\
3.607693262847	0.0430104086201833\\
3.63038315758189	0.0384740807859442\\
3.65307305231677	0.0339379288115953\\
3.67576294705166	0.0294019849023938\\
3.69845284178655	0.0248662890654489\\
3.72114273652143	0.020330891570591\\
3.74383263125632	0.0157958563807526\\
3.76652252599121	0.0112612660223449\\
3.78921242072609	0.00672722864409\\
3.81190231546098	0.00219388848974527\\
3.83459221019587	-0.00233855814151093\\
3.85728210493075	-0.00686983780018687\\
3.87997199966564	-0.0113995589463032\\
3.90266189440053	-0.0159271421232403\\
3.92535178913541	-0.020451694149974\\
3.9480416838703	-0.0249717650730355\\
3.97073157860519	-0.0294848332420847\\
3.99342147334007	-0.0339860646218868\\
4.01611136807496	-0.0384646843210499\\
4.03880126280985	-0.0428891819719231\\
4.06149115754473	-0.0470763723582762\\
4.08418105227962	-0.04325881355353\\
4.10687094701451	-0.038965163166384\\
4.12956084174939	-0.0346046458399141\\
4.15225073648428	-0.0302262125817832\\
4.17494063121917	-0.0258487991385783\\
4.19763052595405	-0.0214855438051406\\
4.22032042068894	-0.0171503333717187\\
4.24301031542383	-0.0128622705521624\\
4.26570021015872	-0.00865207586433721\\
4.2883901048936	-0.00457496297556489\\
4.31107999962849	-0.000740859510602822\\
4.33376989436338	0.002608545349586\\
4.35645978909826	0.00493751386817181\\
4.37914968383315	0.00559825179007668\\
4.40183957856804	0.00497315559335526\\
4.42452947330292	0.00383827247065169\\
4.44721936803781	0.00263051199329922\\
4.4699092627727	0.00156944281610252\\
4.49259915750758	0.000755481465515783\\
4.51528905224247	0.000207695706160313\\
4.53797894697736	-0.000108099246139354\\
4.56066884171224	-0.000250526366334616\\
4.58335873644713	-0.000280186430720808\\
4.60604863118202	-0.000247490352294107\\
4.6287385259169	-0.00018871723356181\\
4.65142842065179	-0.000126721829835868\\
4.67411831538668	-7.36534583813826e-05\\
4.69680821012156	-3.41602872781512e-05\\
4.71949810485645	-8.28952612098367e-06\\
4.74218799959134	6.25644959997739e-06\\
4.76487789432622	1.25743029570151e-05\\
4.78756778906111	1.36384660294265e-05\\
4.810257683796	1.18542926575374e-05\\
4.83294757853088	8.92827251191867e-06\\
4.85563747326577	5.92128533648886e-06\\
4.87832736800066	3.38567196833524e-06\\
4.90101726273554	1.52148586235998e-06\\
4.92370715747043	3.15851422208873e-07\\
4.94639705220532	-3.50003387664252e-07\\
4.96908694694021	-6.28365460088053e-07\\
4.99177684167509	-6.625810486514e-07\\
5.01446673640998	-5.67027523596258e-07\\
5.03715663114487	-4.21883420283539e-07\\
5.05984652587975	-2.76301978072128e-07\\
5.08253642061464	-1.55309738294027e-07\\
5.10522631534953	-6.74204026734105e-08\\
5.12791621008441	-1.13141668610824e-08\\
5.1506061048193	1.90956935435734e-08\\
5.17329599955419	3.12784174637766e-08\\
5.19598589428907	3.21320890564181e-08\\
5.21867578902396	2.70892238847562e-08\\
5.24136568375885	1.99125056045943e-08\\
5.26405557849373	1.28756810081052e-08\\
5.28674547322862	7.1091463482607e-09\\
5.30943536796351	2.97023917940279e-09\\
5.33212526269839	3.62963350867509e-10\\
5.35481515743328	-1.02244580054972e-09\\
5.37750505216817	-1.55147707964838e-09\\
5.40019494690305	-1.55561722469009e-09\\
5.42288484163794	-1.29259752831406e-09\\
5.44557473637283	-9.38783614967051e-10\\
5.46826463110771	-5.99179175865494e-10\\
5.4909545258426	-3.2466628561333e-10\\
5.51364442057749	-1.29981057782806e-10\\
5.53633431531237	-8.99909231855955e-12\\
5.55902421004726	5.39496923915491e-11\\
5.58171410478215	7.6710402347671e-11\\
5.60440399951703	7.51910527004553e-11\\
5.62709389425192	6.16051380427475e-11\\
5.64978378898681	4.42090854098905e-11\\
5.6724736837217	2.78435663082925e-11\\
5.69516357845658	1.47903545517196e-11\\
5.71785347319147	5.64329909286012e-12\\
5.74054336792636	3.81583981930169e-14\\
5.76323326266124	-2.81362387093757e-12\\
5.78592315739613	-3.78164927448819e-12\\
5.80861305213102	-3.62847729255478e-12\\
5.8313029468659	-2.93231587288952e-12\\
5.85399284160079	-2.07917967945071e-12\\
5.87668273633568	-1.29169210338571e-12\\
5.89937263107056	-6.7174546544416e-13\\
5.92206252580545	-2.42654547829117e-13\\
5.94475242054034	1.63512011867791e-14\\
5.96744231527522	1.44947634075923e-13\\
5.99013221001011	1.85512650252169e-13\\
6.012822104745	1.74504426147318e-13\\
6.03551199947988	1.39203918397405e-13\\
6.05820189421477	9.76202586465724e-14\\
6.08089178894966	5.99413639574207e-14\\
6.10358168368454	3.06535310113588e-14\\
6.12627157841943	1.05734689911055e-14\\
6.14896147315432	-1.48920936017738e-15\\
6.17165136788921	-7.39999747321932e-15\\
6.19434126262409	-9.10153437906908e-15\\
6.21703115735898	-8.40810767311586e-15\\
6.23972105209387	-6.68072633974358e-15\\
6.26241094682875	-4.73651589226194e-15\\
6.28510084156364	-2.99451239686421e-15\\
6.30779073629853	-1.61762216609788e-15\\
6.33048063103341	-6.42972893071952e-16\\
6.3531705257683	-3.86239539576361e-17\\
6.37586042050319	2.74748588393181e-16\\
6.39855031523807	3.87003145659414e-16\\
6.42124020997296	3.78031394288882e-16\\
6.44393010470784	3.09049716106673e-16\\
6.46661999944273	2.21361211344023e-16\\
6.48930989417762	1.39121039378221e-16\\
6.51199978891251	7.36627039020509e-17\\
6.53468968364739	2.78794098471739e-17\\
6.55737957838228	-1.13330527018921e-19\\
6.58006947311717	-1.43041684546633e-17\\
6.60275936785205	-1.90701272685456e-17\\
6.62544926258694	-1.82394133919216e-17\\
6.64813915732183	-1.47091982452668e-17\\
6.67082905205671	-1.04102651586011e-17\\
6.6935189467916	-6.45376260262927e-18\\
6.71620884152649	-3.3453908030087e-18\\
6.73889873626137	-1.19731644452818e-18\\
6.76158863099626	9.71824597117122e-20\\
6.78427852573115	7.37755444004493e-19\\
6.80696842046603	9.37199560572685e-19\\
6.82965831520092	8.78719368503985e-19\\
6.85234820993581	6.992739207143e-19\\
6.87503810467069	4.89004624703309e-19\\
6.89772799940558	2.98923859844128e-19\\
6.92041789414047	1.51486702111852e-19\\
6.94310778887535	5.08244989034035e-20\\
6.96579768361024	-8.93730526171574e-21\\
6.98848757834513	-3.77514420234124e-20\\
7.01117747308002	-4.5944648298099e-20\\
7.0338673678149	-4.22735507207235e-20\\
7.05655726254979	-3.32053519556919e-20\\
7.07924715728468	-2.29428631511805e-20\\
7.10193705201956	-1.3823127956111e-20\\
7.12462694675445	-6.83775870343362e-21\\
7.14731684148934	-2.12637803588706e-21\\
7.17000673622422	6.27688643790039e-22\\
7.19269663095911	1.91877902179882e-21\\
7.215386525694	2.24716992756936e-21\\
7.23807642042888	2.03089175602299e-21\\
7.26076631516377	1.57497931434855e-21\\
7.28345620989866	1.07512005453434e-21\\
7.30614610463354	6.38141972127317e-22\\
7.32883599936843	3.07558182786377e-22\\
7.35152589410332	8.73249459209829e-23\\
7.3742157888382	-3.93562589400523e-23\\
7.39690568357309	-9.69569091302811e-23\\
7.41959557830798	-1.09672926328464e-22\\
7.44228547304286	-9.74366698079371e-23\\
7.46497536777775	-7.46191183109182e-23\\
7.48766526251264	-5.03188648713862e-23\\
7.51035515724752	-2.94073883108748e-23\\
7.53304505198241	-1.37800902924211e-23\\
7.5557349467173	-3.49857115927358e-24\\
7.57842484145219	2.31695671722794e-24\\
7.60111473618707	4.87425259796606e-24\\
7.62380463092196	5.3417264592248e-24\\
7.64649452565685	4.66865703107413e-24\\
7.66918442039173	3.53133268122462e-24\\
7.69187431512662	2.35214077613255e-24\\
7.71456420986151	1.35263692769554e-24\\
7.73725410459639	6.14718374856268e-25\\
7.75994399933128	1.35367593117362e-25\\
7.78263389406617	-1.31032873195791e-25\\
7.80532378880105	-2.43939719763683e-25\\
7.82801368353594	-2.59684311619351e-25\\
7.85070357827083	-2.23421203389889e-25\\
7.87339347300571	-1.66932943706379e-25\\
7.8960833677406	-1.0980189251572e-25\\
7.91877326247549	-6.20841030407183e-26\\
7.94146315721037	-2.72836397476831e-26\\
7.96415305194526	-4.97104050992824e-27\\
7.98684294668015	7.20358007477194e-27\\
8.00953284141503	1.21565017400532e-26\\
8.03222273614992	1.26075926095999e-26\\
8.05491263088481	1.06866079438117e-26\\
8.07760252561969	7.87881637693585e-27\\
8.10029242035458	5.11126067172678e-27\\
8.12298231508947	2.84547746975786e-27\\
8.14567220982435	1.21099842691487e-27\\
8.16836210455924	1.67405489578608e-28\\
8.19105199929413	-3.89438138927259e-28\\
8.21374189402902	-6.03573998269679e-28\\
8.2364317887639	-6.09760111649257e-28\\
8.25912168349879	-5.1086560641427e-28\\
8.28181157823368	-4.08746412631173e-28\\
8.30450147296856	-2.82479626130215e-28\\
8.32719136770345	-1.19656645359517e-28\\
8.34988126243834	-1.89461327422045e-29\\
8.37257115717322	9.67112944766431e-30\\
8.39526105190811	1.87982142289526e-29\\
8.417950946643	2.85751782928411e-29\\
8.44064084137788	2.9709377601609e-29\\
8.46333073611277	4.02074055566369e-29\\
8.48602063084766	3.72340304646906e-29\\
8.50871052558254	7.92993369407315e-30\\
8.53140042031743	-6.33108582456477e-30\\
8.55409031505232	1.21584162425687e-31\\
8.5767802097872	2.9400741013573e-30\\
8.59947010452209	-1.60334621075622e-30\\
8.62215999925698	-1.49735637417642e-30\\
8.64484989399186	-6.34974698447593e-30\\
8.66753978872675	-6.88558281680119e-30\\
8.69022968346164	4.94568611216898e-30\\
8.71291957819652	1.14833258640879e-29\\
8.73560947293141	-4.68830011662108e-30\\
8.7582993676663	-1.47978423164076e-29\\
8.78098926240119	-3.09683191515979e-30\\
8.80367915713607	4.44906911552755e-30\\
8.82636905187096	7.95302482809041e-30\\
8.84905894660585	7.76472976109683e-30\\
8.87174884134073	4.73467377365224e-30\\
8.89443873607562	3.38452372028165e-30\\
8.91712863081051	-1.34827725427853e-31\\
8.93981852554539	-4.56236847366553e-30\\
8.96250842028028	-4.76914316482804e-30\\
8.98519831501517	-4.57075846864939e-30\\
9.00788820975005	2.33976032790214e-30\\
9.03057810448494	7.3638111495146e-30\\
9.05326799921983	3.01219772133579e-30\\
9.07595789395471	-5.0557212509037e-32\\
9.0986477886896	-6.28422757686366e-30\\
9.12133768342449	-1.1050756294584e-29\\
9.14402757815937	-1.08621549954415e-29\\
9.16671747289426	-1.39223434836053e-29\\
9.18940736762915	-1.79802771080588e-29\\
9.21209726236403	-5.51208826407843e-30\\
9.23478715709892	2.41620051385267e-29\\
9.25747705183381	3.9051074673357e-29\\
9.28016694656869	2.55308352235782e-29\\
9.30285684130358	1.11076756394383e-29\\
9.32554673603847	6.21105666511657e-30\\
9.34823663077335	-7.1954185770996e-30\\
9.37092652550824	-3.85736287255308e-29\\
9.39361642024313	-4.29442138844859e-29\\
9.41630631497801	-1.40939702406733e-29\\
9.4389962097129	6.89287976890388e-30\\
9.46168610444779	7.85612489489829e-30\\
9.48437599918267	1.58348567147069e-30\\
9.50706589391756	4.39917182597324e-30\\
9.52975578865245	7.41107027914788e-30\\
9.55244568338733	-1.44806969202043e-29\\
9.57513557812222	-1.9517862300708e-29\\
9.59782547285711	2.10240337706403e-29\\
9.620515367592	3.02657026756171e-29\\
9.64320526232688	2.80787603229938e-29\\
9.66589515706177	3.13493709472883e-29\\
9.68858505179666	1.26809701021891e-29\\
9.71127494653154	-1.2011828200441e-29\\
9.73396484126643	-3.75793001992211e-30\\
9.75665473600132	6.53370816670864e-30\\
9.7793446307362	-2.78398188972998e-29\\
9.80203452547109	-5.67422153087117e-29\\
9.82472442020598	-3.67168814558724e-29\\
9.84741431494086	-7.30033846301741e-30\\
9.87010420967575	-3.65132019601879e-30\\
9.89279410441064	-2.80602274135571e-30\\
9.91548399914552	-4.3617511007798e-29\\
9.93817389388041	-5.3279738783434e-29\\
9.9608637886153	1.59351669323936e-29\\
9.98355368335018	6.27209023650701e-29\\
10.0062435780851	4.49310308223467e-29\\
10.02893347282	1.55493378908353e-29\\
10.0516233675548	1.6310563825716e-29\\
10.0743132622897	2.34972671631267e-29\\
10.0970031570246	5.6712328983136e-29\\
10.1196930517595	5.96205511004899e-29\\
10.1423829464944	1.28206054172528e-29\\
10.1650728412293	-3.34496785684237e-29\\
10.1877627359642	-2.96442812951254e-29\\
10.2104526306991	-5.53343672881688e-30\\
10.2331425254339	-1.49666953546291e-29\\
10.2558324201688	-1.49611762265738e-29\\
10.2785223149037	-1.0262100200568e-29\\
10.3012122096386	-1.47145766941487e-29\\
10.3239021043735	-2.99645298506318e-29\\
10.3465919991084	-4.54470937063981e-29\\
10.3692818938433	-5.31425299891094e-29\\
10.3919717885781	-6.0874829711217e-29\\
10.414661683313	-7.46469890353026e-29\\
10.4373515780479	-2.84874115907281e-29\\
10.4600414727828	2.87539005133117e-29\\
10.4827313675177	4.69272079677521e-29\\
10.5054212622526	1.04133996575071e-28\\
10.5281111569875	2.28902790224338e-28\\
10.5508010517224	2.2238069358742e-28\\
10.5734909464572	5.8545962326068e-29\\
10.5961808411921	-8.44563796971048e-29\\
10.618870735927	-1.18099588283234e-28\\
10.6415606306619	-6.99357183701091e-29\\
10.6642505253968	-5.69854708356485e-29\\
10.6869404201317	-7.11123361791233e-29\\
10.7096303148666	-1.00082382017533e-28\\
10.7323202096014	-7.38051487972138e-29\\
10.7550101043363	-8.19050114368389e-30\\
10.7776999990712	7.87877372713701e-30\\
10.8003898938061	1.90774235690685e-30\\
10.823079788541	1.42115287251478e-29\\
10.8457696832759	2.97701819557032e-29\\
10.8684595780108	3.33930908024112e-29\\
10.8911494727457	5.66691103671311e-31\\
10.9138393674805	-1.07918890119005e-29\\
10.9365292622154	3.56261681551856e-29\\
10.9592191569503	5.84466330943936e-29\\
10.9819090516852	2.22484981269775e-29\\
11.0045989464201	-1.04411814789306e-29\\
11.027288841155	-8.24150856692116e-30\\
11.0499787358899	-1.32703634528222e-29\\
11.0726686306247	-5.62107151365056e-31\\
11.0953585253596	2.50496280741187e-30\\
11.1180484200945	-2.81943901712955e-29\\
11.1407383148294	-2.27356008194122e-29\\
11.1634282095643	3.11808873857503e-29\\
11.1861181042992	4.11315132508554e-29\\
11.2088079990341	-3.69546834244544e-30\\
11.231497893769	-2.21035354077362e-29\\
11.2541877885038	-2.81358905202044e-29\\
11.2768776832387	-2.2900259874241e-29\\
11.2995675779736	6.80913352595182e-30\\
11.3222574727085	2.30688158888696e-29\\
11.3449473674434	1.78944833736411e-29\\
11.3676372621783	7.08431918071444e-30\\
11.3903271569132	-4.55923089517615e-30\\
11.4130170516481	-1.18213381459871e-29\\
11.4357069463829	1.19641209183411e-29\\
11.4583968411178	4.11252556470149e-29\\
11.4810867358527	4.96038478842577e-29\\
11.5037766305876	5.54359268877376e-29\\
11.5264665253225	1.2983574307383e-29\\
11.5491564200574	-2.58785888703454e-29\\
11.5718463147923	-1.79994521415415e-29\\
11.5945362095271	-2.0703982098832e-29\\
11.617226104262	-6.06997498296898e-29\\
11.6399159989969	-7.94478982353305e-29\\
11.6626058937318	-5.54083603790521e-29\\
11.6852957884667	-3.77963885950739e-29\\
11.7079856832016	-5.24061475605884e-30\\
11.7306755779365	2.30049524814198e-29\\
11.7533654726714	3.21509197673589e-29\\
11.7760553674062	5.68784166265694e-29\\
11.7987452621411	7.10805173284572e-29\\
11.821435156876	5.56717914826048e-29\\
11.8441250516109	3.95135177451249e-29\\
11.8668149463458	2.52707071545163e-29\\
11.8895048410807	-1.5338691535465e-29\\
11.9121947358156	-4.58637727724392e-29\\
11.9348846305504	-4.32533017675264e-29\\
11.9575745252853	-4.05848498781016e-29\\
11.9802644200202	-6.16351248857649e-29\\
12.0029543147551	-6.66414656136288e-29\\
12.02564420949	-3.17679685598836e-29\\
12.0483341042249	-7.59893388193887e-30\\
12.0710239989598	2.64480332995601e-29\\
12.0937138936947	4.42515720771318e-29\\
12.1164037884295	2.89425300939787e-29\\
12.1390936831644	2.79994572280602e-29\\
12.1617835778993	-3.85725221058047e-30\\
12.1844734726342	-2.09410807556355e-29\\
12.2071633673691	2.69450491675362e-29\\
12.229853262104	3.56031376212484e-29\\
12.2525431568389	-2.99861097894819e-30\\
12.2752330515737	-2.21625861199862e-29\\
12.2979229463086	-2.16498012522984e-29\\
12.3206128410435	-2.1564695567851e-29\\
12.3433027357784	-1.81631814366426e-29\\
12.3659926305133	-4.4132996110751e-30\\
12.3886825252482	2.39881170544281e-29\\
12.4113724199831	4.84915789569545e-29\\
12.434062314718	4.22926966572435e-29\\
12.4567522094528	2.90497152836423e-29\\
12.4794421041877	6.26780867846487e-30\\
12.5021319989226	-1.28725272636333e-29\\
12.5248218936575	-1.41582402337387e-29\\
12.5475117883924	-1.82351449748904e-29\\
12.5702016831273	-2.21035706317431e-29\\
12.5928915778622	-3.24203929592361e-29\\
12.6155814725971	-1.11372934567692e-29\\
12.6382713673319	4.87822751546652e-29\\
12.6609612620668	7.37928335023673e-29\\
12.6836511568017	1.51906950032928e-29\\
12.7063410515366	-3.03838134232855e-29\\
12.7290309462715	-1.93096216889199e-29\\
12.7517208410064	-2.35972276985909e-29\\
12.7744107357413	-1.45902385432938e-29\\
12.7971006304761	2.65862394891604e-30\\
12.819790525211	-5.16827696956181e-30\\
12.8424804199459	-1.10581938136098e-29\\
12.8651703146808	-2.78286491983091e-29\\
12.8878602094157	-3.25816214660394e-29\\
12.9105501041506	-2.02509980933335e-30\\
12.9332399988855	2.73421209475845e-29\\
12.9559298936204	2.16233612108007e-29\\
12.9786197883552	1.41645344134762e-29\\
13.0013096830901	3.29459971608279e-29\\
13.023999577825	3.19494842882693e-29\\
13.0466894725599	-1.71557860581768e-29\\
13.0693793672948	-3.87374698212998e-29\\
13.0920692620297	-1.22001346590771e-29\\
13.1147591567646	-8.77022416266479e-30\\
13.1374490514994	-6.49639703026635e-29\\
13.1601389462343	-6.95256745976614e-29\\
13.1828288409692	2.42119481159204e-29\\
13.2055187357041	9.38329752067422e-29\\
13.228208630439	6.72281762980211e-29\\
13.2508985251739	1.65975745468626e-29\\
13.2735884199088	9.85067937025218e-30\\
13.2962783146437	-1.68915713804758e-29\\
13.3189682093785	-4.96250065331129e-29\\
13.3416581041134	-3.63734974774869e-29\\
13.3643479988483	7.56481313452124e-30\\
13.3870378935832	2.92467092146082e-29\\
13.4097277883181	2.15087198429134e-29\\
13.432417683053	1.06837534936583e-29\\
13.4551075777879	1.15384108742308e-29\\
13.4777974725227	1.11731405204767e-29\\
13.5004873672576	-8.80055440878394e-30\\
13.5231772619925	-2.53559597356713e-29\\
13.5458671567274	-2.44960003402919e-29\\
13.5685570514623	-4.08725970138832e-29\\
13.5912469461972	-2.95109168880933e-29\\
13.6139368409321	5.73128963098388e-30\\
13.636626735667	2.32180828492825e-29\\
13.6593166304018	4.50351739962224e-29\\
13.6820065251367	4.40008441282623e-29\\
13.7046964198716	3.12701548995037e-29\\
13.7273863146065	3.5755794231229e-29\\
13.7500762093414	3.31801027929425e-29\\
13.7727661040763	-8.35727179888607e-30\\
13.7954559988112	-3.92264424907045e-29\\
13.818145893546	-3.74672686642818e-29\\
13.8408357882809	-4.37042312820417e-29\\
13.8635256830158	-5.01754962963544e-29\\
13.8862155777507	-3.48763370527546e-29\\
13.9089054724856	-4.25261383656259e-30\\
13.9315953672205	-3.38225804992426e-30\\
13.9542852619554	-7.01432759928555e-30\\
13.9769751566903	2.45751073544838e-30\\
13.9996650514251	5.84693493868354e-30\\
14.02235494616	1.87509859848192e-29\\
14.0450448408949	3.78363358239725e-29\\
14.0677347356298	1.81311489154499e-29\\
14.0904246303647	-3.02636160577072e-29\\
14.1131145250996	-3.80673682189641e-29\\
14.1358044198345	2.38498011807371e-29\\
14.1584943145694	4.61976504696768e-29\\
14.1811842093042	2.62259108572773e-30\\
14.2038741040391	-5.26127057369701e-30\\
14.226563998774	4.47045346812575e-29\\
14.2492538935089	2.45277755878174e-29\\
14.2719437882438	-9.48323200954609e-29\\
14.2946336829787	-1.42405359773726e-28\\
14.3173235777136	-3.38953211095621e-29\\
14.3400134724484	8.32062008237732e-29\\
14.3627033671833	1.3618320760547e-28\\
14.3853932619182	2.34436748092088e-28\\
14.4080831566531	3.73339304468552e-28\\
14.430773051388	5.2214162414315e-28\\
14.4534629461229	6.4184931711814e-28\\
14.4761528408578	6.28977126724481e-28\\
14.4988427355927	3.95184280207906e-28\\
14.5215326303275	-1.5550457749407e-28\\
14.5442225250624	-1.19742792991813e-27\\
14.5669124197973	-2.88067746604332e-27\\
14.5896023145322	-5.17352742547236e-27\\
14.6122922092671	-7.89931710417055e-27\\
14.634982104002	-1.06594308735515e-26\\
14.6576719987369	-1.25682462940796e-26\\
14.6803618934717	-1.21333241715778e-26\\
14.7030517882066	-7.18576456808638e-27\\
14.7257416829415	4.9669607001381e-27\\
14.7484315776764	2.72702227732728e-26\\
14.7711214724113	6.20857968969668e-26\\
14.7938113671462	1.09801720469848e-25\\
14.8165012618811	1.66918917447708e-25\\
14.839191156616	2.234582827932e-25\\
14.8618810513508	2.59766385716223e-25\\
14.8845709460857	2.43981686434509e-25\\
14.9072608408206	1.31065005702114e-25\\
14.9299507355555	-1.35252060916983e-25\\
14.9526406302904	-6.1449343654993e-25\\
14.9753305250253	-1.35234348706727e-24\\
14.9980204197602	-2.35183459378686e-24\\
15.020710314495	-3.53125897205138e-24\\
15.0434002092299	-4.66892928682122e-24\\
15.0660901039648	-5.34239010514594e-24\\
15.0887799986997	-4.8756354024792e-24\\
15.1114698934346	-2.31951305243149e-24\\
15.1341597881695	3.49459308717984e-24\\
15.1568496829044	1.37748025479012e-23\\
15.1795395776393	2.94012815007188e-23\\
15.2022294723741	5.03132861206256e-23\\
15.224919367109	7.46163968443603e-23\\
15.2476092618439	9.74403716335376e-23\\
15.2702991565788	1.09688159913336e-22\\
15.2929890513137	9.6989834256294e-23\\
15.3156789460486	3.94129370241396e-23\\
15.3383688407835	-8.72405512483424e-23\\
15.3610587355184	-3.07447516694623e-22\\
15.3837486302532	-6.38016770391364e-22\\
15.4064385249881	-1.07500819209502e-21\\
15.429128419723	-1.57493119690354e-21\\
15.4518183144579	-2.03098549265271e-21\\
15.4745082091928	-2.24751170568965e-21\\
15.4971981039277	-1.91949465993574e-21\\
15.5198879986626	-6.28900473406248e-22\\
15.5425778933974	2.12459563988696e-21\\
15.5652677881323	6.8354507250681e-21\\
15.5879576828672	1.38205595212146e-20\\
15.6106475776021	2.29406439959686e-20\\
15.633337472337	3.32045658534836e-20\\
15.6560273670719	4.22758514177275e-20\\
15.6787172618068	4.59522573460851e-20\\
15.7014071565417	3.77669535962968e-20\\
15.7240970512765	8.96317638942372e-21\\
15.7467869460114	-5.07869061042885e-20\\
15.7694768407463	-1.51438639640265e-19\\
15.7921667354812	-2.98871308979232e-19\\
15.8148566302161	-4.8896087497424e-19\\
15.837546524951	-6.99262196526622e-19\\
15.8602364196859	-8.78774667931086e-19\\
15.8829263144207	-9.37368324925657e-19\\
15.9056162091556	-7.38091065122103e-19\\
15.9283061038905	-9.77340961507224e-20\\
15.9509959986254	1.19652447723119e-18\\
15.9736858933603	3.3443912914297e-18\\
15.9963757880952	6.45268981221297e-18\\
16.0190656828301	1.04094081821528e-17\\
16.041755577565	1.47090541067137e-17\\
16.0644454722998	1.8240721861386e-17\\
16.0871353670347	1.90738579538532e-17\\
16.1098252617696	1.4311418194178e-17\\
16.1325151565045	1.25078557070412e-19\\
16.1552050512394	-2.78627447207178e-17\\
16.1778949459743	-7.36419474889518e-17\\
16.2005848407092	-1.39099192040288e-16\\
16.223274735444	-2.21344549771727e-16\\
16.2459646301789	-3.09048863810339e-16\\
16.2686545249138	-3.78061970998958e-16\\
16.2913444196487	-3.87085369230136e-16\\
16.3140343143836	-2.74904946905184e-16\\
16.3367242091185	3.83740502595534e-17\\
16.3594141038534	6.42622613546421e-16\\
16.3821039985883	1.61719175404497e-15\\
16.4047938933231	2.99406863259848e-15\\
16.427483788058	4.73619480977607e-15\\
16.4501736827929	6.68075472612886e-15\\
16.4728635775278	8.4088149520194e-15\\
16.4955534722627	9.10334156206402e-15\\
16.5182433669976	7.40336464627907e-15\\
16.5409332617325	1.49451911823052e-15\\
16.5636231564674	-1.05661151517969e-14\\
16.5863130512022	-3.06446194090235e-14\\
16.6090029459371	-5.99323755818221e-14\\
16.631692840672	-9.76141366913288e-14\\
16.6543827354069	-1.39205498352889e-13\\
16.6770726301418	-1.74520647747134e-13\\
16.6997625248767	-1.85552268533853e-13\\
16.7224524196116	-1.45020040746633e-13\\
16.7451423143464	-1.64638890693533e-14\\
16.7678322090813	2.42500342102427e-13\\
16.7905221038162	6.71561244554668e-13\\
16.8132119985511	1.29151060355068e-12\\
16.835901893286	2.07906447457821e-12\\
16.8585917880209	2.93233494481441e-12\\
16.8812816827558	3.62863768592595e-12\\
16.9039715774907	3.78189580397689e-12\\
16.9266614722255	2.81390151467395e-12\\
16.9493513669604	-3.79170303643265e-14\\
16.9720412616953	-5.64316636530211e-12\\
16.9947311564302	-1.47903281990019e-11\\
17.0174210511651	-2.78435984466381e-11\\
17.0401109459	-4.42091417631808e-11\\
17.0628008406349	-6.16051970842391e-11\\
17.0854907353697	-7.51911030442283e-11\\
17.1081806301046	-7.67104396961475e-11\\
17.1308705248395	-5.39497167778854e-11\\
17.1535604195744	8.99907866841621e-12\\
17.1762503143093	1.2998105190913e-10\\
17.1989402090442	3.24666284688322e-10\\
17.2216301037791	5.99179177610259e-10\\
17.244319998514	9.38783617769722e-10\\
17.2670098932488	1.29259753117608e-09\\
17.2896997879837	1.55561722709441e-09\\
17.3123896827186	1.55147708141052e-09\\
17.3350795774535	1.02244580168545e-09\\
17.3577694721884	-3.62963350243463e-10\\
17.3804593669233	-2.97023917914506e-09\\
17.4031492616582	-7.10914634823293e-09\\
17.425839156393	-1.28756810081989e-08\\
17.4485290511279	-1.99125056047335e-08\\
17.4712189458628	-2.7089223884894e-08\\
17.4939088405977	-3.21320890565307e-08\\
17.5165987353326	-3.12784174638553e-08\\
17.5392886300675	-1.9095693543619e-08\\
17.5619785248024	1.13141668610644e-08\\
17.5846684195373	6.74204026734126e-08\\
17.6073583142721	1.55309738294041e-07\\
17.630048209007	2.76301978072145e-07\\
17.6527381037419	4.21883420283549e-07\\
17.6754279984768	5.67027523596249e-07\\
17.6981178932117	6.62581048651358e-07\\
17.7208077879466	6.28365460087962e-07\\
17.7434976826815	3.500033876641e-07\\
17.7661875774163	-3.15851422209093e-07\\
17.7888774721512	-1.52148586236026e-06\\
17.8115673668861	-3.38567196833554e-06\\
17.834257261621	-5.9212853364891e-06\\
17.8569471563559	-8.92827251191871e-06\\
17.8796370510908	-1.18542926575371e-05\\
17.9023269458257	-1.36384660294254e-05\\
17.9250168405606	-1.25743029570131e-05\\
17.9477067352954	-6.2564495999741e-06\\
17.9703966300303	8.28952612098834e-06\\
17.9930865247652	3.4160287278157e-05\\
18.0157764195001	7.36534583813887e-05\\
18.038466314235	0.000126721829835873\\
18.0611562089699	0.000188717233561811\\
18.0838461037048	0.0002474903522941\\
18.1065359984397	0.000280186430720785\\
18.1292258931745	0.00025052636633457\\
18.1519157879094	0.000108099246139278\\
18.1746056826443	-0.000207695706160422\\
18.1972955773792	-0.000755481465515921\\
18.2199854721141	-0.00156944281610266\\
18.242675366849	-0.00263051199329934\\
18.2653652615839	-0.0038382724706518\\
18.2880551563187	-0.00497315559335537\\
18.3107450510536	-0.00559825179007676\\
18.3334349457885	-0.00493751386817177\\
18.3561248405234	-0.00260854534958584\\
18.3788147352583	0.000740859510603055\\
18.4015046299932	0.00457496297556516\\
18.4241945247281	0.0086520758643375\\
18.446884419463	0.0128622705521622\\
18.4695743141978	0.0171503333717186\\
18.4922642089327	0.0214855438051404\\
18.5149541036676	0.0258487991385782\\
18.5376439984025	0.030226212581783\\
18.5603338931374	0.034604645839914\\
18.5830237878723	0.0389651631663838\\
18.6057136826072	0.0432588135535299\\
18.628403577342	0.0470763723582763\\
18.6510934720769	0.0428891819719233\\
18.6737833668118	0.0384646843210501\\
18.6964732615467	0.033986064621887\\
18.7191631562816	0.0294848332420849\\
18.7418530510165	0.0249717650730357\\
18.7645429457514	0.0204516941499742\\
18.7872328404863	0.0159271421232405\\
18.8099227352211	0.0113995589463034\\
18.832612629956	0.006869837800187\\
18.8553025246909	0.00233855814151105\\
18.8779924194258	-0.00219388848974515\\
18.9006823141607	-0.00672722864408988\\
18.9233722088956	-0.0112612660223447\\
18.9460621036305	-0.0157958563807524\\
18.9687519983653	-0.0203308915705909\\
18.9914418931002	-0.0248662890654488\\
19.0141317878351	-0.0294019849023937\\
19.03682168257	-0.0339379288115952\\
19.0595115773049	-0.0384740807859441\\
19.0822014720398	-0.0430104086201832\\
19.1048913667747	-0.0475468861162199\\
19.1275812615096	-0.0520834917546002\\
19.1502711562444	-0.0566202076975594\\
19.1729610509793	-0.0611570190314258\\
19.1956509457142	-0.0656939131841179\\
19.2183408404491	-0.0702308794722714\\
19.241030735184	-0.0747679087453629\\
19.2637206299189	-0.0793049931030901\\
19.2864105246538	-0.0838421256685014\\
19.3091004193886	-0.0883793004037919\\
19.3317903141235	-0.0929165119588398\\
19.3544802088584	-0.0974537555448231\\
19.3771701035933	-0.101991026826872\\
19.3998599983282	-0.106528321830843\\
19.4225498930631	-0.111065636860055\\
19.445239787798	-0.115602968418238\\
19.4679296825329	-0.120140313135074\\
19.4906195772677	-0.124677667690495\\
19.5133094720026	-0.12921502873323\\
19.5359993667375	-0.133752392787913\\
19.5586892614724	-0.138289756142881\\
19.5813791562073	-0.142827114707215\\
19.6040690509422	-0.147364463819374\\
19.6267589456771	-0.151901797979063\\
19.649448840412	-0.156439110454436\\
19.6721387351468	-0.160976392679937\\
19.6948286298817	-0.165513633286261\\
19.7175185246166	-0.17005081644574\\
19.7402084193515	-0.174587918847457\\
19.7628983140864	-0.179124903657045\\
19.7855882088213	-0.183661706939255\\
19.8082781035562	-0.188198201478266\\
19.830967998291	-0.192734070262348\\
19.8536578930259	-0.197268048834495\\
19.8763477877608	-0.200558389004522\\
19.8990376824957	-0.199258129914598\\
19.9217275772306	-0.200355524608985\\
19.9444174719655	-0.199467034358503\\
19.9671073667004	-0.200229454092338\\
19.9897972614353	-0.199598922626462\\
20.0124871561701	-0.200139285649349\\
20.035177050905	-0.199694643355023\\
20.0578669456399	-0.200070577680794\\
20.0805568403748	-0.199768621453808\\
20.1032467351097	-0.200016213091294\\
20.1259366298446	-0.199827971552596\\
20.1486265245795	-0.199972077076439\\
20.1713164193143	-0.199876814402027\\
20.1940063140492	-0.199935554576566\\
20.2166962087841	-0.199917775944643\\
20.239386103519	-0.199904876178518\\
20.2620759982539	-0.19995263920015\\
20.2847658929888	-0.199878789977698\\
20.3074557877237	-0.199982670973669\\
20.3301456824586	-0.199856379993922\\
20.3528355771934	-0.200008802521547\\
20.3755254719283	-0.199836958202124\\
20.3982153666632	-0.200031736942115\\
20.4209052613981	-0.199819996721205\\
20.443595156133	-0.20005201663503\\
20.4662850508679	-0.199805083352246\\
20.4889749456028	-0.20007006756203\\
20.5116648403377	-0.199791891398143\\
20.5343547350725	-0.200086229325721\\
20.5570446298074	-0.19978015858493\\
20.5797345245423	-0.200100776211375\\
20.6024244192772	-0.199769671984136\\
20.6251143140121	-0.200113932268891\\
20.647804208747	-0.199760257006324\\
20.6704941034819	-0.200125882348946\\
20.6931839982167	-0.199751769224458\\
20.7158738929516	-0.200136780324082\\
20.7385637876865	-0.199744088205838\\
20.7612536824214	-0.20014675530898\\
20.7839435771563	-0.199737112795925\\
20.8066334718912	-0.200155916431974\\
20.8293233666261	-0.199730757468633\\
20.852013261361	-0.200164356540286\\
20.8747031560958	-0.199724949471239\\
20.8973930508307	-0.200172155109014\\
20.9200829455656	-0.199719626568941\\
20.9427728403005	-0.20017938054773\\
20.9654627350354	-0.199714735247191\\
20.9881526297703	-0.200186092045999\\
21.0108425245052	-0.199710229267143\\
21.03353241924	-0.200192341062225\\
21.0562223139749	-0.199706068496114\\
21.0789122087098	-0.20019817253388\\
21.1016021034447	-0.199702217954095\\
21.1242919981796	-0.20020362586817\\
21.1469818929145	-0.199698647031375\\
21.1696717876494	-0.200208735758258\\
21.1923616823843	-0.199695328842712\\
21.2150515771191	-0.20021353285982\\
21.237741471854	-0.199692239691201\\
21.2604313665889	-0.20021804435501\\
21.2831212613238	-0.199689358620871\\
21.3058111560587	-0.200222294425063\\
21.3285010507936	-0.199686667041471\\
21.3511909455285	-0.200226304648294\\
21.3738808402633	-0.199684148412325\\
21.3965707349982	-0.200230094336846\\
21.4192606297331	-0.19968178797478\\
21.441950524468	-0.200233680822864\\
21.4646404192029	-0.199679572524839\\
21.4873303139378	-0.200237079702711\\
21.5100202086727	-0.199677490219173\\
21.5327101034076	-0.200240305046191\\
21.5553999981424	-0.199675530408992\\
21.5780898928773	-0.200243369576474\\
21.6007797876122	-0.199673683497262\\
21.6234696823471	-0.200246284825373\\
21.646159577082	-0.199671940815581\\
21.6688494718169	-0.200249061267801\\
21.6915393665518	-0.199670294517643\\
21.7142292612867	-0.195132597256714\\
21.7369191560215	-0.190594714030596\\
21.7596090507564	-0.186056792295903\\
21.7822989454913	-0.181518853922845\\
21.8049888402262	-0.176980906273777\\
21.8276787349611	-0.172442952711501\\
21.850368629696	-0.16790499505207\\
21.8730585244309	-0.163367034387256\\
21.8957484191657	-0.158829071424493\\
21.9184383139006	-0.15429110664832\\
21.9411282086355	-0.149753140405113\\
21.9638181033704	-0.145215172951048\\
21.9865079981053	-0.140677204480912\\
22.0091978928402	-0.136139235146264\\
22.0318877875751	-0.13160126506734\\
22.05457768231	-0.127063294341123\\
22.0772675770448	-0.122525323046978\\
22.0999574717797	-0.117987351250665\\
22.1226473665146	-0.113449379007286\\
22.1453372612495	-0.108911406363463\\
22.1680271559844	-0.104373433358993\\
22.1907170507193	-0.099835460028114\\
22.2134069454542	-0.0952974864004974\\
22.236096840189	-0.0907595125020194\\
22.2587867349239	-0.0862215383553826\\
22.2814766296588	-0.0816835639806131\\
22.3041665243937	-0.0771455893954637\\
22.3268564191286	-0.0726076146157449\\
22.3495463138635	-0.0680696396555972\\
22.3722362085984	-0.063531664527717\\
22.3949261033333	-0.0589936892435467\\
22.4176159980681	-0.0544557138134329\\
22.440305892803	-0.0499177382467621\\
22.4629957875379	-0.0453797625520749\\
22.4856856822728	-0.040841786737164\\
22.5083755770077	-0.0363038108091589\\
22.5310654717426	-0.0317658347745979\\
22.5537553664775	-0.0272278586394918\\
22.5764452612123	-0.0226898824093786\\
22.5991351559472	-0.0181519060893712\\
22.6218250506821	-0.0136139296841995\\
22.644514945417	-0.00907595319824748\\
22.6672048401519	-0.00453797663558565\\
22.6898947348868	0\\
};
\end{axis}
\end{tikzpicture}%%
  \caption{PINS solutions curvature}
  \label{fig:PINS_sol_curv}
\end{figure}
%
From figure \ref{fig:PINS_sol_traj} we can see that the solution is the one expected for the trajectory. However, in figure \ref{fig:PINS_sol_curv} we can see that the curvature is not quite what we expect. In particular, there are ringing phenomena when the constraint on maximum curvature is active. 
%
\begin{figure}[htb!]
  \centering
  
%
\begin{tikzpicture}[scale=0.7]

\begin{axis}[%
width=0.985\linewidth,
height=\linewidth,
at={(0\linewidth,0\linewidth)},
scale only axis,
xmin=0,
xmax=25,
xlabel style={font=\color{white!15!black}},
xlabel={Time(s)},
ymin=-0.2,
ymax=0.2,
ylabel style={font=\color{white!15!black}},
ylabel={$J(m^{-1}s^{-1})$},
axis background/.style={fill=white},
title style={font=\bfseries},
title={Jerk - PINS},
xmajorgrids,
xminorgrids,
ymajorgrids,
yminorgrids
]
\addplot [color=dodgerblue, line width=2.0pt, forget plot]
  table[row sep=crcr]{%
0	0.199999898131228\\
0.0226898947348868	0.199999896524261\\
0.0453797894697736	0.199999893226898\\
0.0680696842046603	0.199999889756408\\
0.0907595789395471	0.199999886099613\\
0.113449473674434	0.199999882241981\\
0.136139368409321	0.19999987816745\\
0.158829263144207	0.199999873858214\\
0.181519157879094	0.19999986929449\\
0.204209052613981	0.199999864454226\\
0.226898947348868	0.199999859312775\\
0.249588842083755	0.199999853842497\\
0.272278736818641	0.199999848012292\\
0.294968631553528	0.199999841787046\\
0.317658526288415	0.199999835126952\\
0.340348421023302	0.199999827986711\\
0.363038315758189	0.199999820314543\\
0.385728210493075	0.199999812050988\\
0.408418105227962	0.19999980312743\\
0.431107999962849	0.19999979346426\\
0.453797894697736	0.199999782968584\\
0.476487789432622	0.199999771531331\\
0.499177684167509	0.199999759023571\\
0.521867578902396	0.199999745291776\\
0.544557473637283	0.199999730151659\\
0.56724736837217	0.199999713380058\\
0.589937263107056	0.199999694704107\\
0.612627157841943	0.199999673786576\\
0.63531705257683	0.199999650205677\\
0.658006947311717	0.199999623426764\\
0.680696842046603	0.199999592761827\\
0.70338673678149	0.199999557310172\\
0.726076631516377	0.1999995158692\\
0.748766526251264	0.199999466796065\\
0.771456420986151	0.199999407785389\\
0.794146315721037	0.199999335496733\\
0.816836210455924	0.199999244897822\\
0.839526105190811	0.199999128032314\\
0.862215999925698	0.199998971519069\\
0.884905894660584	0.199998750928312\\
0.907595789395471	0.199998416215031\\
0.930285684130358	0.199997845159587\\
0.952975578865245	0.199996629928305\\
0.975665473600132	0.199991683370232\\
0.998355368335018	0.112747636577203\\
1.02104526306991	3.62782189582828e-05\\
1.04373515780479	-6.11823558688857e-05\\
1.06642505253968	3.84021543823943e-05\\
1.08911494727457	-6.42411287734179e-05\\
1.11180484200945	4.0698992893265e-05\\
1.13449473674434	-6.75307294045802e-05\\
1.15718463147923	4.31868504521732e-05\\
1.17987452621411	-7.10744478431613e-05\\
1.202564420949	4.58861905281577e-05\\
1.22525431568389	-7.4898537136639e-05\\
1.24794421041877	4.88201899297533e-05\\
1.27063410515366	-7.90326720340573e-05\\
1.29332399988855	5.20151717930406e-05\\
1.31601389462343	-8.35104921330279e-05\\
1.33870378935832	5.5501120116197e-05\\
1.36139368409321	-8.8370247584392e-05\\
1.38408357882809	5.9312293667256e-05\\
1.40677347356298	-9.36555700785408e-05\\
1.42946336829787	6.34879615698664e-05\\
1.45215326303275	-9.94163975397769e-05\\
1.47484315776764	6.80732887335625e-05\\
1.49753305250253	-0.000105710088519034\\
1.52022294723741	7.31204067209848e-05\\
1.5429128419723	-0.000112602772032223\\
1.56560273670719	7.86897154294419e-05\\
1.58829263144207	-0.000120170991404303\\
1.61098252617696	8.48514738578859e-05\\
1.63367242091185	-0.000128503717674651\\
1.65636231564673	9.16877552268811e-05\\
1.67905221038162	-0.000137704830696699\\
1.70174210511651	9.92948645309855e-05\\
1.7244319998514	-0.000147896196689583\\
1.74712189458628	0.000107786347344684\\
1.76981178932117	-0.000159221512486184\\
1.79250168405606	0.000117296760513932\\
1.81519157879094	-0.000171851143865635\\
1.83788147352583	0.000127986433207933\\
1.86057136826072	-0.000185988265059209\\
1.8832612629956	0.000140047527014465\\
1.90595115773049	-0.000201876718690558\\
1.92864105246538	0.000153711817413124\\
1.95133094720026	-0.000219811176179913\\
1.97402084193515	0.000169260781278847\\
1.99671073667004	-0.000240150411975178\\
2.01940063140492	0.000187038811010073\\
2.04209052613981	-0.000263334850029395\\
2.0647804208747	0.00020747072478874\\
2.08747031560958	-0.000289910060630291\\
2.11016021034447	0.000231085267611791\\
2.13285010507936	-0.000320558685349746\\
2.15553999981424	0.000258547105429315\\
2.17822989454913	-0.000356144527773333\\
2.20091978928402	0.000290701086477188\\
2.2236096840189	-0.00039777458668845\\
2.24629957875379	0.000328634599985151\\
2.26898947348868	-0.000446888210609692\\
2.29167936822357	0.000373767289725165\\
2.31436926295845	-0.000505388430310216\\
2.33705915769334	0.000427983294439524\\
2.35974905242823	-0.000575841099826929\\
2.38243894716311	0.000493831814521724\\
2.405128841898	-0.000661787413960149\\
2.42781873663289	0.00057484182108787\\
2.45050863136777	-0.000768255117854965\\
2.47319852610266	0.000676036588246119\\
2.49588842083755	-0.000902638445309897\\
2.51857831557243	0.000804818627397095\\
2.54126821030732	-0.00107631282563775\\
2.56395810504221	0.000972591882220471\\
2.58664799977709	-0.00130785310997043\\
2.60933789451198	0.00119799122329337\\
2.63202778924687	-0.00163019925059329\\
2.65471768398175	0.00151406538809304\\
2.67740757871664	-0.00210932509117226\\
2.70009747345153	0.00198697358544525\\
2.72278736818641	-0.00290632172383593\\
2.7454772629213	0.0027781203509268\\
2.76816715765619	-0.0046034687764253\\
2.79085705239107	0.00447036881192074\\
2.81354694712596	-0.0438539072868409\\
2.83623684186085	-0.172418577379807\\
2.85892673659573	-0.199865346714962\\
2.88161663133062	-0.199920789168414\\
2.90430652606551	-0.199941381994826\\
2.92699642080039	-0.19995218571567\\
2.94968631553528	-0.19995877718537\\
2.97237621027017	-0.199963148071449\\
2.99506610500506	-0.199966193581546\\
3.01775599973994	-0.199968376624343\\
3.04044589447483	-0.199969960348061\\
3.06313578920972	-0.199971104782363\\
3.0858256839446	-0.199971912119455\\
3.10851557867949	-0.199972449906084\\
3.13120547341438	-0.199972763764059\\
3.15389536814926	-0.199972884750733\\
3.17658526288415	-0.1999728337978\\
3.19927515761904	-0.199972624469762\\
3.22196505235392	-0.199972264708311\\
3.24465494708881	-0.19997175793561\\
3.2673448418237	-0.199971103731953\\
3.29003473655858	-0.199970298214537\\
3.31272463129347	-0.199969334191479\\
3.33541452602836	-0.199968201132275\\
3.35810442076324	-0.199966884973663\\
3.38079431549813	-0.199965367763168\\
3.40348421023302	-0.199963627128459\\
3.4261741049679	-0.199961635546648\\
3.44886399970279	-0.199959359372145\\
3.47155389443768	-0.199956757562505\\
3.49424378917256	-0.199953780016751\\
3.51693368390745	-0.199950365406658\\
3.53962357864234	-0.199946438333946\\
3.56231347337722	-0.19994190557854\\
3.58500336811211	-0.199936651104572\\
3.607693262847	-0.199930529345427\\
3.63038315758189	-0.199923356070899\\
3.65307305231677	-0.199914895806053\\
3.67576294705166	-0.199904844252061\\
3.69845284178655	-0.199892803333626\\
3.72114273652143	-0.199878245154442\\
3.74383263125632	-0.1998604589007\\
3.76652252599121	-0.199838470883673\\
3.78921242072609	-0.199810921085017\\
3.81190231546098	-0.199775866999988\\
3.83459221019587	-0.19973046141982\\
3.85728210493075	-0.199670401971072\\
3.87997199966564	-0.199588945406774\\
3.90266189440053	-0.199475037443712\\
3.92535178913541	-0.199309495603094\\
3.9480416838703	-0.199056434541802\\
3.97073157860519	-0.198641281816777\\
3.99342147334007	-0.197882167015042\\
4.01611136807496	-0.196191244033128\\
4.03880126280985	-0.189769237315707\\
4.06149115754473	-0.00814529079851927\\
4.08418105227962	0.178740564614009\\
4.10687094701451	0.190705329723492\\
4.12956084174939	0.192573625543627\\
4.15225073648428	0.192945952452421\\
4.17494063121917	0.192611488038405\\
4.19763052595405	0.191681492322776\\
4.22032042068894	0.190024531927852\\
4.24301031542383	0.187269654766513\\
4.26570021015872	0.182621111147231\\
4.2883901048936	0.174333474133983\\
4.31107999962849	0.158297524274229\\
4.33376989436338	0.125130007104966\\
4.35645978909826	0.0658818931383993\\
4.37914968383315	0.000785409663638562\\
4.40183957856804	-0.038783329318821\\
4.42452947330292	-0.0516230601205503\\
4.44721936803781	-0.0499964781912525\\
4.4699092627727	-0.0413186255311376\\
4.49259915750758	-0.0300077881773612\\
4.51528905224247	-0.0190300731172485\\
4.53797894697736	-0.0100974922503803\\
4.56066884171224	-0.00379215475858646\\
4.58335873644713	6.69023385957103e-05\\
4.60604863118202	0.00201563731845703\\
4.6287385259169	0.00266128432655426\\
4.65142842065179	0.00253557313784078\\
4.67411831538668	0.0020397085054652\\
4.69680821012156	0.00144037539671567\\
4.71949810485645	0.000890632974510585\\
4.74218799959134	0.000459760376188958\\
4.76487789432622	0.000162671896800361\\
4.78756778906111	-1.58663208421736e-05\\
4.810257683796	-0.000103794961866123\\
4.83294757853088	-0.000130741182151151\\
4.85563747326577	-0.000122138084119479\\
4.87832736800066	-9.69550437658923e-05\\
4.90101726273554	-6.76473069177877e-05\\
4.92370715747043	-4.12405890792153e-05\\
4.94639705220532	-2.08069912471905e-05\\
4.96908694694021	-6.88803682518952e-06\\
4.99177684167509	1.35165758167851e-06\\
5.01446673640998	5.30407106732356e-06\\
5.03715663114487	6.40649833154857e-06\\
5.05984652587975	5.87428203401142e-06\\
5.08253642061464	4.60296484050127e-06\\
5.10522631534953	3.17312118710593e-06\\
5.12791621008441	1.90648958992219e-06\\
5.1506061048193	9.38580474315091e-07\\
5.17329599955419	2.87273159817719e-07\\
5.19598589428907	-9.23140813998437e-08\\
5.21867578902396	-2.69273691980501e-07\\
5.24136568375885	-3.13213063408288e-07\\
5.26405557849373	-2.82137916590857e-07\\
5.28674547322862	-2.18278708306926e-07\\
5.30943536796351	-1.48660517737455e-07\\
5.33212526269839	-8.79837704538498e-08\\
5.35481515743328	-4.2187071665263e-08\\
5.37750505216817	-1.17490942635487e-08\\
5.40019494690305	5.7047323127568e-09\\
5.42288484163794	1.35926943895122e-08\\
5.44557473637283	1.52803342754693e-08\\
5.46826463110771	1.35328377793108e-08\\
5.4909545258426	1.03393630416732e-08\\
5.51364442057749	6.95611850524404e-09\\
5.53633431531237	4.05314243021923e-09\\
5.55902421004726	1.88871512335507e-09\\
5.58171410478215	4.68079745567232e-10\\
5.60440399951703	-3.32863252152916e-10\\
5.62709389425192	-6.82726113376994e-10\\
5.64978378898681	-7.4397814817856e-10\\
5.6724736837217	-6.48278257830306e-10\\
5.69516357845658	-4.89210449735987e-10\\
5.71785347319147	-3.25082957102582e-10\\
5.74054336792636	-1.86358796781784e-10\\
5.76323326266124	-8.41742043608529e-11\\
5.78592315739613	-1.79563067863055e-11\\
5.80861305213102	1.8716115952111e-11\\
5.8313029468659	3.41406963585853e-11\\
5.85399284160079	3.61531815963269e-11\\
5.87668273633568	3.10145602359836e-11\\
5.89937263107056	2.31168449173907e-11\\
5.92206252580545	1.51630643216021e-11\\
5.94475242054034	8.54129528659918e-12\\
5.96744231527522	3.72768254418774e-12\\
5.99013221001011	6.51320608066758e-13\\
6.012822104745	-1.02047039873574e-12\\
6.03551199947988	-1.69423808261508e-12\\
6.05820189421477	-1.7466487915899e-12\\
6.08089178894966	-1.47569498267105e-12\\
6.10358168368454	-1.08788285585146e-12\\
6.12627157841943	-7.08305189316616e-13\\
6.14896147315432	-3.9606764760988e-13\\
6.17165136788921	-1.67747032496968e-13\\
6.19434126262409	-2.22149598240869e-14\\
6.21703115735898	5.33455105810472e-14\\
6.23972105209387	8.09080831743277e-14\\
6.26241094682875	8.12303006679807e-14\\
6.28510084156364	6.87286953642982e-14\\
6.30779073629853	5.18190924036472e-14\\
6.33048063103341	3.47951859316575e-14\\
6.3531705257683	2.02231321958073e-14\\
6.37586042050319	9.37922155633967e-15\\
6.39855031523807	2.27596485357213e-15\\
6.42124020997296	-1.71780059942023e-15\\
6.44393010470784	-3.4524219873081e-15\\
6.46661999944273	-3.74458935825689e-15\\
6.48930989417762	-3.25471997926193e-15\\
6.51199978891251	-2.45134741326075e-15\\
6.53468968364739	-1.6257465116318e-15\\
6.55737957838228	-9.29567518816602e-16\\
6.58006947311717	-4.17736551073102e-16\\
6.60275936785205	-8.67180078012366e-17\\
6.62544926258694	9.60984851237247e-17\\
6.64813915732183	1.72525001213048e-16\\
6.67082905205671	1.81918773513401e-16\\
6.6935189467916	1.55683277470868e-16\\
6.71620884152649	1.15832316974548e-16\\
6.73889873626137	7.58613758006399e-17\\
6.76158863099626	4.26417114566294e-17\\
6.78427852573115	1.85108196991634e-17\\
6.80696842046603	3.10631508313597e-18\\
6.82965831520092	-5.24298685909201e-18\\
6.85234820993581	-8.58784821071628e-18\\
6.87503810467069	-8.82221062609458e-18\\
6.89772799940558	-7.43762645298892e-18\\
6.92041789414047	-5.46717743382167e-18\\
6.94310778887535	-3.53514216896979e-18\\
6.96579768361024	-1.95188082540168e-18\\
6.98848757834513	-8.15502748443401e-19\\
7.01117747308002	-9.96502793456786e-20\\
7.0338673678149	2.80726210748343e-19\\
7.05655726254979	4.25975699654109e-19\\
7.07924715728468	4.27111368872501e-19\\
7.10193705201956	3.54895971002115e-19\\
7.12462694675445	2.57752406013582e-19\\
7.14731684148934	1.64510400653053e-19\\
7.17000673622422	8.91400578308161e-20\\
7.19269663095911	3.56872806749802e-20\\
7.215386525694	2.47054328665077e-21\\
7.23807642042888	-1.48125546873359e-20\\
7.26076631516377	-2.10616160333946e-20\\
7.28345620989866	-2.06443739199195e-20\\
7.30614610463354	-1.69141787724514e-20\\
7.32883599936843	-1.2137937012097e-20\\
7.35152589410332	-7.64469041791172e-21\\
7.3742157888382	-4.06087946207881e-21\\
7.39690568357309	-1.54951506408482e-21\\
7.41959557830798	-1.05721221552946e-23\\
7.44228547304286	7.72454178988522e-22\\
7.46497536777775	1.03829932855759e-21\\
7.48766526251264	9.96296600938572e-22\\
7.51035515724752	8.05177260756195e-22\\
7.53304505198241	5.7093295174625e-22\\
7.5557349467173	3.54718415350316e-22\\
7.57842484145219	1.84505566356067e-22\\
7.60111473618707	6.66545565181961e-23\\
7.62380463092196	-4.5305535634728e-24\\
7.64649452565685	-3.98942745031032e-23\\
7.66918442039173	-5.10473116338399e-23\\
7.69187431512662	-4.80102657809875e-23\\
7.71456420986151	-3.82862596229879e-23\\
7.73725410459639	-2.68240410279772e-23\\
7.75994399933128	-1.6433554601411e-23\\
7.78263389406617	-8.35850754957113e-24\\
7.80532378880105	-2.83499416649473e-24\\
7.82801368353594	4.52150982045896e-25\\
7.85070357827083	2.04389154283654e-24\\
7.87339347300571	2.50374257354913e-24\\
7.8960833677406	2.31047437396107e-24\\
7.91877326247549	1.81839214619981e-24\\
7.94146315721037	1.25855723876445e-24\\
7.96415305194526	7.59968704690137e-25\\
7.98684294668015	3.77426657331444e-25\\
8.00953284141503	1.19084125289459e-25\\
8.03222273614992	-3.23909346741401e-26\\
8.05491263088481	-1.04204455065038e-25\\
8.07760252561969	-1.22859698937082e-25\\
8.10029242035458	-1.1091587171268e-25\\
8.12298231508947	-8.59471207421496e-26\\
8.14567220982435	-5.90146409110829e-26\\
8.16836210455924	-3.52676066712045e-26\\
8.19105199929413	-1.69894901861956e-26\\
8.21374189402902	-4.85506819877968e-27\\
8.2364317887639	2.04294451205307e-27\\
8.25912168349879	4.42958641647895e-27\\
8.28181157823368	5.03276861687824e-27\\
8.30450147296856	6.37045192693708e-27\\
8.32719136770345	5.80728770378563e-27\\
8.34988126243834	2.84989807838264e-27\\
8.37257115717322	8.3174354513305e-28\\
8.39526105190811	4.16574185686823e-28\\
8.417950946643	2.40441031130505e-28\\
8.44064084137788	2.56330569173075e-28\\
8.46333073611277	1.65815067698634e-28\\
8.48602063084766	-7.11274164988816e-28\\
8.50871052558254	-9.60011423547044e-28\\
8.53140042031743	-1.72066676000082e-28\\
8.55409031505232	2.04301519116206e-28\\
8.5767802097872	-3.80109822753384e-29\\
8.59947010452209	-9.77842896004498e-29\\
8.62215999925698	-1.04592833706922e-28\\
8.64484989399186	-1.18736259149296e-28\\
8.66753978872675	2.48908891566866e-28\\
8.69022968346164	4.04781707786828e-28\\
8.71291957819652	-2.12296847150261e-28\\
8.73560947293141	-5.79138168936304e-28\\
8.7582993676663	3.50699776285085e-29\\
8.78098926240119	4.24129588451092e-28\\
8.80367915713607	2.4349731173906e-28\\
8.82636905187096	7.30646987200283e-29\\
8.84905894660585	-7.09203610722498e-29\\
8.87174884134073	-9.65232781401552e-29\\
8.89443873607562	-1.07305511021107e-28\\
8.91712863081051	-1.7511963556488e-28\\
8.93981852554539	-1.0212289421188e-28\\
8.96250842028028	-1.84883955779345e-31\\
8.98519831501517	1.56653514168939e-28\\
9.00788820975005	2.62993058311059e-28\\
9.03057810448494	1.48179927985006e-29\\
9.05326799921983	-1.63384811794784e-28\\
9.07595789395471	-2.04858272963999e-28\\
9.0986477886896	-2.424030435321e-28\\
9.12133768342449	-1.00880314169866e-28\\
9.14402757815937	-6.32789887869711e-29\\
9.16671747289426	-1.5685665790314e-28\\
9.18940736762915	1.85330415098912e-28\\
9.21209726236403	9.28657508969982e-28\\
9.23478715709892	9.82004620516921e-28\\
9.25747705183381	3.01638703277671e-29\\
9.28016694656869	-6.15767489456538e-28\\
9.30285684130358	-4.25735306052961e-28\\
9.32554673603847	-4.03331404365732e-28\\
9.34823663077335	-9.86886142795766e-28\\
9.37092652550824	-7.87769086747571e-28\\
9.39361642024313	5.39439666223818e-28\\
9.41630631497801	1.09822223143265e-27\\
9.4389962097129	4.83697597367806e-28\\
9.46168610444779	-1.169990905528e-28\\
9.48437599918267	-7.61782526831832e-29\\
9.50706589391756	1.28418061779014e-28\\
9.52975578865245	-4.16041347189758e-28\\
9.55244568338733	-5.93412461681912e-28\\
9.57513557812222	7.82390819916295e-28\\
9.59782547285711	1.09704266057831e-27\\
9.620515367592	1.55459658027762e-28\\
9.64320526232688	2.38799757400756e-29\\
9.66589515706177	-3.39309423880069e-28\\
9.68858505179666	-9.5551785617257e-28\\
9.71127494653154	-3.62251573094073e-28\\
9.73396484126643	4.08673918144348e-28\\
9.75665473600132	-5.30674319091606e-28\\
9.7793446307362	-1.3943635308763e-27\\
9.80203452547109	-1.95617094356586e-28\\
9.82472442020598	1.08951313841413e-27\\
9.84741431494086	7.28640693274228e-28\\
9.87010420967575	9.90378266217047e-29\\
9.89279410441064	-8.80704632585829e-28\\
9.91548399914552	-1.11225099613423e-27\\
9.93817389388041	1.31231719309459e-27\\
9.9608637886153	2.55621814256863e-27\\
9.98355368335018	6.38959859195952e-28\\
10.0062435780851	-1.0394839867136e-27\\
10.02893347282	-6.30687522595321e-28\\
10.0516233675548	1.75142488873988e-28\\
10.0743132622897	8.90303054057953e-28\\
10.0970031570246	7.96021408637163e-28\\
10.1196930517595	-9.67208620373254e-28\\
10.1423829464944	-2.05091805749683e-27\\
10.1650728412293	-9.35766498886839e-28\\
10.1877627359642	6.15169046966721e-28\\
10.2104526306991	3.23438828429065e-28\\
10.2331425254339	-2.07751944377882e-28\\
10.2558324201688	1.03671594976932e-28\\
10.2785223149037	5.43412685047188e-30\\
10.3012122096386	-4.34167497916355e-28\\
10.3239021043735	-6.77229166801511e-28\\
10.3465919991084	-5.10756008551158e-28\\
10.3692818938433	-3.39969316408726e-28\\
10.3919717885781	-4.73877452880565e-28\\
10.414661683313	7.13696967282211e-28\\
10.4373515780479	2.27856697346491e-27\\
10.4600414727828	1.66185476926261e-27\\
10.4827313675177	1.66109400115121e-27\\
10.5054212622526	4.01005787780869e-27\\
10.5281111569875	2.60571277200582e-27\\
10.5508010517224	-3.75402420083409e-27\\
10.5734909464572	-6.76153584821941e-27\\
10.5961808411921	-3.89260401322331e-27\\
10.618870735927	3.19980799750225e-28\\
10.6415606306619	1.346725451168e-27\\
10.6642505253968	-2.59282341936268e-29\\
10.6869404201317	-9.49693942732212e-28\\
10.7096303148666	-5.93394691682429e-29\\
10.7323202096014	2.02495167887601e-27\\
10.7550101043363	1.80000664345944e-27\\
10.7776999990712	2.22527332510552e-28\\
10.8003898938061	1.39550118499161e-28\\
10.823079788541	6.13983447793124e-28\\
10.8457696832759	4.22689534292334e-28\\
10.8684595780108	-6.43535177074509e-28\\
10.8911494727457	-9.73670885885275e-28\\
10.9138393674805	7.72579103189881e-28\\
10.9365292622154	1.52575679427523e-27\\
10.9592191569503	-2.94793567456155e-28\\
10.9819090516852	-1.5180285183826e-27\\
11.0045989464201	-6.71885150860206e-28\\
11.027288841155	-6.23445372258704e-29\\
11.0499787358899	1.69225144172751e-28\\
11.0726686306247	3.47628899220559e-28\\
11.0953585253596	-6.08911661838441e-28\\
11.1180484200945	-5.56207155690539e-28\\
11.1407383148294	1.30840795540929e-27\\
11.1634282095643	1.40739114959521e-27\\
11.1861181042992	-7.6854379748616e-28\\
11.2088079990341	-1.39346280353986e-27\\
11.231497893769	-5.38575045484556e-28\\
11.2541877885038	-1.75568127533079e-29\\
11.2768776832387	7.70056989123731e-28\\
11.2995675779736	1.01298565507316e-27\\
11.3222574727085	2.44279446361867e-28\\
11.3449473674434	-3.52238229725712e-28\\
11.3676372621783	-4.94795470212062e-28\\
11.3903271569132	-4.16609630577868e-28\\
11.4130170516481	3.64112570961262e-28\\
11.4357069463829	1.1667439274542e-27\\
11.4583968411178	8.29438113435786e-28\\
11.4810867358527	3.15353407495557e-28\\
11.5037766305876	-8.06973192356195e-28\\
11.5264665253225	-1.79186630674514e-27\\
11.5491564200574	-6.82749453246107e-28\\
11.5718463147923	1.14028884487494e-28\\
11.5945362095271	-9.40954072001621e-28\\
11.617226104262	-1.29449512267225e-27\\
11.6399159989969	1.16602335807484e-28\\
11.6626058937318	9.17842725294985e-28\\
11.6852957884667	1.1055085580863e-27\\
11.7079856832016	1.3398330355189e-27\\
11.7306755779365	8.23968884834138e-28\\
11.7533654726714	7.46443836363177e-28\\
11.7760553674062	8.57862013376471e-28\\
11.7987452621411	-2.65894830726756e-29\\
11.821435156876	-6.95618026266051e-28\\
11.8441250516109	-6.6992563613239e-28\\
11.8668149463458	-1.20873653054599e-27\\
11.8895048410807	-1.56753657868641e-27\\
11.9121947358156	-6.15133092466985e-28\\
11.9348846305504	1.16327619762976e-28\\
11.9575745252853	-4.05066293453573e-28\\
11.9802644200202	-5.74189877034958e-28\\
12.0029543147551	6.58159869731314e-28\\
12.02564420949	1.30107548804245e-27\\
12.0483341042249	1.28286187617072e-27\\
12.0710239989598	1.14259027123967e-27\\
12.0937138936947	5.49693337841959e-29\\
12.1164037884295	-3.58135527708551e-28\\
12.1390936831644	-7.22783924028436e-28\\
12.1617835778993	-1.07846551417512e-27\\
12.1844734726342	6.78766951940477e-28\\
12.2071633673691	1.24602205161189e-27\\
12.229853262104	-6.59845726398415e-28\\
12.2525431568389	-1.27293943881502e-27\\
12.2752330515737	-4.11002133135825e-28\\
12.2979229463086	1.31752605974825e-29\\
12.3206128410435	7.68319962783619e-29\\
12.3433027357784	3.7795230337497e-28\\
12.3659926305133	9.28856193110973e-28\\
12.3886825252482	1.16582468068236e-27\\
12.4113724199831	4.03364136693355e-28\\
12.434062314718	-4.28425603125903e-28\\
12.4567522094528	-7.93853131530124e-28\\
12.4794421041877	-9.23808661015427e-28\\
12.5021319989226	-4.50113346731146e-28\\
12.5248218936575	-1.18171938961367e-28\\
12.5475117883924	-1.75085219452773e-28\\
12.5702016831273	-3.12589550328551e-28\\
12.5928915778622	2.41655532189247e-28\\
12.6155814725971	1.78940160504668e-27\\
12.6382713673319	1.87154078834432e-27\\
12.6609612620668	-7.40232172600294e-28\\
12.6836511568017	-2.29566175037115e-27\\
12.7063410515366	-7.60257310475068e-28\\
12.7290309462715	1.49550841993105e-28\\
12.7517208410064	1.03997466730469e-28\\
12.7744107357413	5.78580287707655e-28\\
12.7971006304761	2.07624620647119e-28\\
12.819790525211	-3.0226710883526e-28\\
12.8424804199459	-4.99349434924519e-28\\
12.8651703146808	-4.74295449666649e-28\\
12.8878602094157	5.68613245907604e-28\\
12.9105501041506	1.32049405944307e-27\\
12.9332399988855	5.21123198156027e-28\\
12.9559298936204	-2.90384479259788e-28\\
12.9786197883552	2.49508340216742e-28\\
13.0013096830901	3.91913450514894e-28\\
13.023999577825	-1.10405499462197e-27\\
13.0466894725599	-1.55767479169638e-27\\
13.0693793672948	1.09203931023414e-28\\
13.0920692620297	6.60365462440047e-28\\
13.1147591567646	-1.16271662473652e-27\\
13.1374490514994	-1.33882177825935e-27\\
13.1601389462343	1.96510207430385e-27\\
13.1828288409692	3.59981065829203e-27\\
13.2055187357041	9.47915992662848e-28\\
13.228208630439	-1.70197794133282e-27\\
13.2508985251739	-1.26438437899699e-27\\
13.2735884199088	-7.3797490730181e-28\\
13.2962783146437	-1.3106205779773e-27\\
13.3189682093785	-4.29308428369704e-28\\
13.3416581041134	1.26024867756883e-27\\
13.3643479988483	1.4460227219824e-27\\
13.3870378935832	3.07271295686623e-28\\
13.4097277883181	-4.09057775229945e-28\\
13.432417683053	-2.19708136268098e-28\\
13.4551075777879	1.07842507103959e-29\\
13.4777974725227	-4.48194350846918e-28\\
13.5004873672576	-8.04964075040608e-28\\
13.5231772619925	-3.45868637006233e-28\\
13.5458671567274	-3.41928366341856e-28\\
13.5685570514623	-1.10509912155675e-28\\
13.5912469461972	1.02697450096728e-27\\
13.6139368409321	1.1619489722952e-27\\
13.636626735667	8.66109887783578e-28\\
13.6593166304018	4.57973946593558e-28\\
13.6820065251367	-3.03329285073627e-28\\
13.7046964198716	-1.81689910719927e-28\\
13.7273863146065	4.2088073024047e-29\\
13.7500762093414	-9.72086176371247e-28\\
13.7727661040763	-1.5955681180868e-27\\
13.7954559988112	-6.41474921005403e-28\\
13.818145893546	-9.86736351958442e-29\\
13.8408357882809	-2.80041573143335e-28\\
13.8635256830158	1.94533609179773e-28\\
13.8862155777507	1.01196772828369e-27\\
13.9089054724856	6.94011130740172e-28\\
13.9315953672205	-6.08577914329563e-29\\
13.9542852619554	1.28686555265117e-28\\
13.9769751566903	2.83413887293019e-28\\
13.9996650514251	3.59046955478878e-28\\
14.02235494616	7.04926163367263e-28\\
14.0450448408949	-1.365887935073e-29\\
14.0677347356298	-1.50066698584304e-27\\
14.0904246303647	-1.23840409555088e-27\\
14.1131145250996	1.19245633068708e-27\\
14.1358044198345	1.85688430187471e-27\\
14.1584943145694	-4.67767928036122e-28\\
14.1811842093042	-1.13396121147023e-27\\
14.2038741040391	9.27327871883865e-28\\
14.226563998774	6.56438615285902e-28\\
14.2492538935089	-3.07486783008907e-27\\
14.2719437882438	-3.67857888527652e-27\\
14.2946336829787	1.34282242596988e-27\\
14.3173235777136	4.97163083464767e-27\\
14.3400134724484	3.74789153282456e-27\\
14.3627033671833	3.3325528618627e-27\\
14.3853932619182	5.22602902380205e-27\\
14.4080831566531	6.33993412954704e-27\\
14.430773051388	5.91695148406241e-27\\
14.4534629461229	2.35425293571568e-27\\
14.4761528408578	-5.43557032309545e-27\\
14.4988427355927	-1.72870282869283e-26\\
14.5215326303275	-3.50951872790667e-26\\
14.5442225250624	-6.00525679028194e-26\\
14.5669124197973	-8.76182887142419e-26\\
14.5896023145322	-1.10591955069997e-25\\
14.6122922092671	-1.20888693230564e-25\\
14.634982104002	-1.02885651177797e-25\\
14.6576719987369	-3.24790686613493e-26\\
14.6803618934717	1.18609667186278e-25\\
14.7030517882066	3.76826007161316e-25\\
14.7257416829415	7.59280458194048e-25\\
14.7484315776764	1.25868446866361e-24\\
14.7711214724113	1.81868401464373e-24\\
14.7938113671462	2.3101279617123e-24\\
14.8165012618811	2.50456345547958e-24\\
14.839191156616	2.04600923347954e-24\\
14.8618810513508	4.52258679053125e-25\\
14.8845709460857	-2.83609469144482e-24\\
14.9072608408206	-8.35688644179563e-24\\
14.9299507355555	-1.64293058862391e-23\\
14.9526406302904	-2.68201205948953e-23\\
14.9753305250253	-3.82844693097163e-23\\
14.9980204197602	-4.80151078363956e-23\\
15.020710314495	-5.10600582353455e-23\\
15.0434002092299	-3.99105230380346e-23\\
15.0660901039648	-4.55502588427963e-24\\
15.0887799986997	6.66128487600832e-23\\
15.1114698934346	1.84448376412902e-22\\
15.1341597881695	3.54658225354983e-22\\
15.1568496829044	5.70886042360219e-22\\
15.1795395776393	8.05170848072398e-22\\
15.2022294723741	9.96371201187661e-22\\
15.224919367109	1.03850383757955e-21\\
15.2476092618439	7.72849840837983e-22\\
15.2702991565788	-9.92815044996335e-24\\
15.2929890513137	-1.54860178309124e-21\\
15.3156789460486	-4.0597452667194e-21\\
15.3383688407835	-7.64350072513665e-21\\
15.3610587355184	-1.21370377777949e-20\\
15.3837486302532	-1.69141524094476e-20\\
15.4064385249881	-2.06460725679707e-20\\
15.429128419723	-2.10661466641321e-20\\
15.4518183144579	-1.48211465201728e-20\\
15.4745082091928	2.45683891484853e-21\\
15.4971981039277	3.56681080100982e-20\\
15.5198879986626	8.91165504969204e-20\\
15.5425778933974	1.64486245654493e-19\\
15.5652677881323	2.57735084670635e-19\\
15.5879576828672	3.54897928330602e-19\\
15.6106475776021	4.27150644786072e-19\\
15.633337472337	4.26075300209085e-19\\
15.6560273670719	2.8091120830546e-19\\
15.6787172618068	-9.93591612943464e-20\\
15.7014071565417	-8.15100320844347e-19\\
15.7240970512765	-1.95139423816783e-18\\
15.7467869460114	-3.53465315515684e-18\\
15.7694768407463	-5.46684781427173e-18\\
15.7921667354812	-7.43772148962461e-18\\
15.8148566302161	-8.82311029261343e-18\\
15.837546524951	-8.59003088184205e-18\\
15.8602364196859	-5.24696414816187e-18\\
15.8829263144207	3.10013784666606e-18\\
15.9056162091556	1.85023826374116e-17\\
15.9283061038905	4.26316553020127e-17\\
15.9509959986254	7.58515063158929e-17\\
15.9736858933603	1.15826128688472e-16\\
15.9963757880952	1.55686418409433e-16\\
16.0190656828301	1.81939237510127e-16\\
16.041755577565	1.72572719502134e-16\\
16.0644454722998	9.61838716780914e-17\\
16.0871353670347	-8.65870845395877e-17\\
16.1098252617696	-4.17559878928132e-16\\
16.1325151565045	-9.29360039076141e-16\\
16.1552050512394	-1.62554800072743e-15\\
16.1778949459743	-2.45123321679714e-15\\
16.2005848407092	-3.25481021416276e-15\\
16.223274735444	-3.74505201006388e-15\\
16.2459646301789	-3.45346294150652e-15\\
16.2686545249138	-1.71963127928953e-15\\
16.2913444196487	2.27319309540835e-15\\
16.3140343143836	9.37552651655816e-15\\
16.3367242091185	2.02188589055211e-14\\
16.3594141038534	3.47912082059576e-14\\
16.3821039985883	5.18170323513357e-14\\
16.4047938933231	6.87311045770408e-14\\
16.427483788058	8.1240705093751e-14\\
16.4501736827929	8.09307443942545e-14\\
16.4728635775278	5.3384708572718e-14\\
16.4955534722627	-2.21563457540946e-14\\
16.5182433669976	-1.67669848907112e-13\\
16.5409332617325	-3.95979796469639e-13\\
16.5636231564674	-7.08225818206168e-13\\
16.5863130512022	-1.08784683681503e-12\\
16.6090029459371	-1.47575645600807e-12\\
16.631692840672	-1.7468816778859e-12\\
16.6543827354069	-1.69473045059037e-12\\
16.6770726301418	-1.02130862047817e-12\\
16.6997625248767	6.50082500275845e-13\\
16.7224524196116	3.72607236481619e-12\\
16.7451423143464	8.53949274284707e-12\\
16.7678322090813	1.51614880029864e-11\\
16.7905221038162	2.31162434578278e-11\\
16.8132119985511	3.10160810896014e-11\\
16.835901893286	3.61576014440667e-11\\
16.8585917880209	3.414676950804e-11\\
16.8812816827558	1.87211282619182e-11\\
16.9039715774907	-1.79537230289417e-11\\
16.9266614722255	-8.41743181044395e-11\\
16.9493513669604	-1.86361990189688e-10\\
16.9720412616953	-3.25087695227493e-10\\
16.9947311564302	-4.89214082760856e-10\\
17.0174210511651	-6.48280080359873e-10\\
17.0401109459	-7.4397874102278e-10\\
17.0628008406349	-6.8272598094982e-10\\
17.0854907353697	-3.32862774120396e-10\\
17.1081806301046	4.68080317571575e-10\\
17.1308705248395	1.88871564557726e-09\\
17.1535604195744	4.05314283816868e-09\\
17.1762503143093	6.9561187856582e-09\\
17.1989402090442	1.0339363209555e-08\\
17.2216301037791	1.35328378614548e-08\\
17.244319998514	1.52803343000894e-08\\
17.2670098932488	1.35926943807342e-08\\
17.2896997879837	5.70473228851963e-09\\
17.3123896827186	-1.1749094291504e-08\\
17.3350795774535	-4.21870716903423e-08\\
17.3577694721884	-8.79837704731957e-08\\
17.3804593669233	-1.48660517750593e-07\\
17.4031492616582	-2.1827870831467e-07\\
17.425839156393	-2.82137916594535e-07\\
17.4485290511279	-3.13213063409261e-07\\
17.4712189458628	-2.69273691979915e-07\\
17.4939088405977	-9.23140813985407e-08\\
17.5165987353326	2.87273159819194e-07\\
17.5392886300675	9.38580474316429e-07\\
17.5619785248024	1.90648958992324e-06\\
17.5846684195373	3.17312118710663e-06\\
17.6073583142721	4.6029648405016e-06\\
17.630048209007	5.87428203401133e-06\\
17.6527381037419	6.406498331548e-06\\
17.6754279984768	5.30407106732242e-06\\
17.6981178932117	1.35165758167673e-06\\
17.7208077879466	-6.88803682519195e-06\\
17.7434976826815	-2.08069912471933e-05\\
17.7661875774163	-4.12405890792181e-05\\
17.7888774721512	-6.76473069177894e-05\\
17.8115673668861	-9.69550437658914e-05\\
17.834257261621	-0.000122138084119473\\
17.8569471563559	-0.000130741182151138\\
17.8796370510908	-0.000103794961866099\\
17.9023269458257	-1.5866320842137e-05\\
17.9250168405606	0.00016267189680041\\
17.9477067352954	0.000459760376189016\\
17.9703966300303	0.00089063297451064\\
17.9930865247652	0.0014403753967157\\
18.0157764195001	0.00203970850546517\\
18.038466314235	0.00253557313784065\\
18.0611562089699	0.00266128432655396\\
18.0838461037048	0.00201563731845648\\
18.1065359984397	6.69023385948543e-05\\
18.1292258931745	-0.00379215475858763\\
18.1519157879094	-0.0100974922503817\\
18.1746056826443	-0.0190300731172499\\
18.1972955773792	-0.0300077881773619\\
18.2199854721141	-0.0413186255311372\\
18.242675366849	-0.0499964781912519\\
18.2653652615839	-0.05162306012055\\
18.2880551563187	-0.0387833293188201\\
18.3107450510536	0.000785409663641898\\
18.3334349457885	0.0658818931384045\\
18.3561248405234	0.12513000710497\\
18.3788147352583	0.158297524274231\\
18.4015046299932	0.174333474133985\\
18.4241945247281	0.182621111147232\\
18.446884419463	0.187269654766513\\
18.4695743141978	0.190024531927852\\
18.4922642089327	0.191681492322776\\
18.5149541036676	0.192611488038405\\
18.5376439984025	0.192945952452421\\
18.5603338931374	0.192573625543627\\
18.5830237878723	0.190705329723493\\
18.6057136826072	0.178740564614015\\
18.628403577342	-0.00814529079851153\\
18.6510934720769	-0.189769237315705\\
18.6737833668118	-0.196191244033128\\
18.6964732615467	-0.197882167015042\\
18.7191631562816	-0.198641281816777\\
18.7418530510165	-0.199056434541802\\
18.7645429457514	-0.199309495603094\\
18.7872328404863	-0.199475037443712\\
18.8099227352211	-0.199588945406774\\
18.832612629956	-0.199670401971072\\
18.8553025246909	-0.19973046141982\\
18.8779924194258	-0.199775866999988\\
18.9006823141607	-0.199810921085017\\
18.9233722088956	-0.199838470883673\\
18.9460621036305	-0.1998604589007\\
18.9687519983653	-0.199878245154442\\
18.9914418931002	-0.199892803333626\\
19.0141317878351	-0.199904844252061\\
19.03682168257	-0.199914895806053\\
19.0595115773049	-0.199923356070899\\
19.0822014720398	-0.199930529345427\\
19.1048913667747	-0.199936651104572\\
19.1275812615096	-0.19994190557854\\
19.1502711562444	-0.199946438333946\\
19.1729610509793	-0.199950365406658\\
19.1956509457142	-0.199953780016751\\
19.2183408404491	-0.199956757562505\\
19.241030735184	-0.199959359372145\\
19.2637206299189	-0.199961635546648\\
19.2864105246538	-0.199963627128459\\
19.3091004193886	-0.199965367763168\\
19.3317903141235	-0.199966884973663\\
19.3544802088584	-0.199968201132275\\
19.3771701035933	-0.199969334191479\\
19.3998599983282	-0.199970298214537\\
19.4225498930631	-0.199971103731953\\
19.445239787798	-0.19997175793561\\
19.4679296825329	-0.199972264708311\\
19.4906195772677	-0.199972624469762\\
19.5133094720026	-0.1999728337978\\
19.5359993667375	-0.199972884750733\\
19.5586892614724	-0.199972763764059\\
19.5813791562073	-0.199972449906084\\
19.6040690509422	-0.199971912119455\\
19.6267589456771	-0.199971104782363\\
19.649448840412	-0.199969960348061\\
19.6721387351468	-0.199968376624343\\
19.6948286298817	-0.199966193581546\\
19.7175185246166	-0.199963148071449\\
19.7402084193515	-0.19995877718537\\
19.7628983140864	-0.19995218571567\\
19.7855882088213	-0.199941381994826\\
19.8082781035562	-0.199920789168414\\
19.830967998291	-0.199865346714962\\
19.8536578930259	-0.172418577379815\\
19.8763477877608	-0.0438539072868324\\
19.8990376824957	0.00447036881192348\\
19.9217275772306	-0.00460346877642322\\
19.9444174719655	0.00277812035092775\\
19.9671073667004	-0.00290632172383559\\
19.9897972614353	0.00198697358544505\\
20.0124871561701	-0.00210932509117352\\
20.035177050905	0.00151406538809294\\
20.0578669456399	-0.0016301992505932\\
20.0805568403748	0.00119799122329332\\
20.1032467351097	-0.00130785310997184\\
20.1259366298446	0.000972591882218985\\
20.1486265245795	-0.00107631282563626\\
20.1713164193143	0.000804818627398551\\
20.1940063140492	-0.000902638445306955\\
20.2166962087841	0.000676036588249055\\
20.239386103519	-0.000768255117850565\\
20.2620759982539	0.00057484182108641\\
20.2847658929888	-0.000661787413961627\\
20.3074557877237	0.000493831814521737\\
20.3301456824586	-0.000575841099825481\\
20.3528355771934	0.000427983294439539\\
20.3755254719283	-0.000505388430311698\\
20.3982153666632	0.00037376728972518\\
20.4209052613981	-0.000446888210611172\\
20.443595156133	0.000328634599983707\\
20.4662850508679	-0.000397774586687016\\
20.4889749456028	0.000290701086478659\\
20.5116648403377	-0.000356144527770447\\
20.5343547350725	0.000258547105430786\\
20.5570446298074	-0.000320558685351219\\
20.5797345245423	0.000231085267610356\\
20.6024244192772	-0.000289910060630313\\
20.6251143140121	0.000207470724790204\\
20.647804208747	-0.00026333485002797\\
20.6704941034819	0.000187038811011532\\
20.6931839982167	-0.000240150411975196\\
20.7158738929516	0.000169260781280302\\
20.7385637876865	-0.00021981117618137\\
20.7612536824214	0.000153711817416011\\
20.7839435771563	-0.00020187671869057\\
20.8066334718912	0.000140047527014472\\
20.8293233666261	-0.000185988265059221\\
20.852013261361	0.00012798643320794\\
20.8747031560958	-0.000171851143864211\\
20.8973930508307	0.000117296760515374\\
20.9200829455656	-0.000159221512486195\\
20.9427728403005	0.00010778634734183\\
20.9654627350354	-0.000147896196689594\\
20.9881526297703	9.92948645295672e-05\\
21.0108425245052	-0.000137704830696707\\
21.03353241924	9.16877552240375e-05\\
21.0562223139749	-0.000128503717674663\\
21.0789122087098	8.48514738593223e-05\\
21.1016021034447	-0.000120170991404314\\
21.1242919981796	7.86897154322966e-05\\
21.1469818929145	-0.000112602772030812\\
21.1696717876494	7.31204067209945e-05\\
21.1923616823843	-0.000105710088519044\\
21.2150515771191	6.80732887335681e-05\\
21.237741471854	-9.94163975426246e-05\\
21.2604313665889	6.34879615698761e-05\\
21.2831212613238	-9.36555700785449e-05\\
21.3058111560587	5.93122936672616e-05\\
21.3285010507936	-8.8370247582982e-05\\
21.3511909455285	5.55011201176195e-05\\
21.3738808402633	-8.35104921330321e-05\\
21.3965707349982	5.20151717930462e-05\\
21.4192606297331	-7.90326720354784e-05\\
21.441950524468	4.88201899311744e-05\\
21.4646404192029	-7.48985371380573e-05\\
21.4873303139378	4.5886190529576e-05\\
21.5100202086727	-7.10744478417513e-05\\
21.5327101034076	4.31868504535901e-05\\
21.5553999981424	-6.75307294045829e-05\\
21.5780898928773	4.0698992893265e-05\\
21.6007797876122	-6.42411287720135e-05\\
21.6234696823471	3.84021543838042e-05\\
21.646159577082	-6.11823558674784e-05\\
21.6688494718169	3.62782189596955e-05\\
21.6915393665518	0.112747636577195\\
21.7142292612867	0.199991683370232\\
21.7369191560215	0.199996629928305\\
21.7596090507564	0.199997845159587\\
21.7822989454913	0.199998416215031\\
21.8049888402262	0.199998750928312\\
21.8276787349611	0.199998971519069\\
21.850368629696	0.199999128032314\\
21.8730585244309	0.199999244897822\\
21.8957484191657	0.199999335496733\\
21.9184383139006	0.199999407785389\\
21.9411282086355	0.199999466796065\\
21.9638181033704	0.1999995158692\\
21.9865079981053	0.199999557310172\\
22.0091978928402	0.199999592761827\\
22.0318877875751	0.199999623426764\\
22.05457768231	0.199999650205677\\
22.0772675770448	0.199999673786576\\
22.0999574717797	0.199999694704107\\
22.1226473665146	0.199999713380058\\
22.1453372612495	0.199999730151659\\
22.1680271559844	0.199999745291776\\
22.1907170507193	0.199999759023571\\
22.2134069454542	0.199999771531331\\
22.236096840189	0.199999782968584\\
22.2587867349239	0.19999979346426\\
22.2814766296588	0.19999980312743\\
22.3041665243937	0.199999812050988\\
22.3268564191286	0.199999820314543\\
22.3495463138635	0.199999827986711\\
22.3722362085984	0.199999835126952\\
22.3949261033333	0.199999841787046\\
22.4176159980681	0.199999848012292\\
22.440305892803	0.199999853842497\\
22.4629957875379	0.199999859312775\\
22.4856856822728	0.199999864454226\\
22.5083755770077	0.19999986929449\\
22.5310654717426	0.199999873858214\\
22.5537553664775	0.19999987816745\\
22.5764452612123	0.199999882241981\\
22.5991351559472	0.199999886099613\\
22.6218250506821	0.199999889756408\\
22.644514945417	0.199999893226898\\
22.6672048401519	0.199999896524261\\
22.6898947348868	0.199999898131228\\
};
\end{axis}
\end{tikzpicture}%%
  \caption{PINS solutions jerk}
  \label{fig:PINS_sol_jerk}
\end{figure}
%
Furthermore, the jerk (Figure \ref{fig:PINS_sol_jerk}) presents chattering (Fuller's phenomenon) at the transition between manoeuvres circular arcs and straight lines.\\
Due to this fact, we expect to find a sub-optimal solution to the problem with the proposed exploration approach.
%
\subsection*{Duboids MATLAB implementation and exploration}
%
The problem can be solved with a naive exploration of all the possible combinations of manoeuvres. All possible connections are a combination of the simple elementary Dubins connected by segments with linearly varying curvature also known as clothoids.\cite{bertolazzi2015g1}\\
An analysis was done by hand and the numerical solution using PINS suggests, that there is at most of $7$ connection between two points. With $3$ arcs at constant curvature and $4$ arcs that connect all the arcs. Moreover, from figure \ref{fig:possiblecombination} and table \ref{tab:possiblecombination}, we can see that there are $21$ possible combinations of manoeuvres.\\
As stated before, this approach does not account for a combination of manoeuvres not reaching the maximum curvature values. This is a limitation of the approach that can be overcome with a more complex exploration of the possible combination. However, the achieved algorithm yields a valid suboptimal solution.
%
\subsubsection*{Algorithm structure}
%
The algorithm was implemented in MATLAB. The implementation is divided into two main parts. The first part is a class that act as a collector for all the possible combination of manoeuvres generated. This class will determine if a connection of a certain type is suitable/feasible for the specific initial and final point. The second part is a class for the single Duboid manoeuvre. Given the topology (shape) of the manoeuvre as in table \ref{tab:possiblecombination} the class will generate the manoeuvre matching if possible the boundary conditions and compute the length of the manoeuvre.\\
The first class will store a list of all feasible candidate-to-be-best manoeuvres and select the best one according to the minimum-time/minimum length criteria.
%
\subsubsection*{Arch length computation}
The Duboids lack a closed-form solution at this stage of the development, therefore, the length of the manoeuvre is computed with a numerical optimization using \textit{fmincon} function of MATLAB.\\
The constraint on the final curvature is already satisfied by the definition of the problem. However, the coordinate and orientation of the final point depend on the choice of the manoeuvre and length of segment $2$, $4$ and $6$.
In general, $P_7$ the "final" point do not coincide with $P_T$. Thus, we should compute a suitable triplet of length for the segment $L_2$, $L_4$ and $L_6$. To satisfy this constraint we can minimize a function cost such as:
%
\begin{equation}
  \label{eq:costfunction}
  \begin{split}
    \min_{L_2,L_4,L_6} \quad & (x_7-x_T)^2 + (y_7-y_T)^2 + (\theta_7-\theta_T)^2\\
    \text{s.t.} \quad & [x_7,y_7,\theta_7] = \mathrm{Duboid}(L_1,L_2,L_3)\\
    & 0 \leq L_2 \leq L_{2,max}\\
    & 0 \leq L_4 \leq L_{4,max}\\
    & 0 \leq L_6 \leq L_{6,max}
  \end{split}
\end{equation}
%
where $L_{2,max}$, $L_{4,max}$ and $L_{6,max}$ are the maximum length of the segments $L_2$, $L_4$ and $L_6$ respectively. For circular arcs, the maximum length is a full circle with radius $R_{max}=1/\kappa_{max}$. For the straight line, the maximum length is not trivial, however, we can safely find a reasonable upper bound for our application (i.e. 10 times the distance between the initial and final point).
%
\subsubsection*{Duboids results}
%
In this section, we can see some of the results obtained with the Duboids algorithm. The results are shown in figures \ref{fig:DuboidsRes0}, \ref{fig:DuboidsRes1}, \ref{fig:DuboidsRes2}, \ref{fig:DuboidsRes3} and \ref{fig:DuboidsRes4}.\\
The first figure (Figure \ref{fig:DuboidsRes0}) shows results obtained with the initial condition set to zero for the position, the heading and the curvature. The final condition is set to $x = 40$, $y=20$, $\theta=0$ and $\kappa=0$.\\
%
\begin{figure}[htb!]
  \centering
  % This file was created by matlab2tikz.
%
%The latest updates can be retrieved from
%  http://www.mathworks.com/matlabcentral/fileexchange/22022-matlab2tikz-matlab2tikz
%where you can also make suggestions and rate matlab2tikz.
%
\begin{tikzpicture}

\begin{axis}[%
width=\linewidth,
height=0.776\linewidth,
at={(0\linewidth,0\linewidth)},
scale only axis,
xmin=0,
xmax=40,
xlabel style={font=\color{white!15!black}},
xlabel={x(m)},
ymin=-5.7741935483871,
ymax=25.7741935483871,
ylabel style={font=\color{white!15!black}},
ylabel={y(m)},
axis background/.style={fill=white},
title style={font=\bfseries},
title={$L_{tot}$ = 45.1441, $k_{max}$ = 0.15, $J_{max}$ = 0.1, Type = [LSR]},
axis x line*=bottom,
axis y line*=left,
xmajorgrids,
xminorgrids,
ymajorgrids,
yminorgrids
]
\addplot [color=red, line width=2.0pt, forget plot]
  table[row sep=crcr]{%
0	0\\
0.0149999999998102	5.62499999994915e-08\\
0.029999999993925	4.49999999934911e-07\\
0.044999999953868	1.51874999888789e-06\\
0.0599999998056	3.59999999166857e-06\\
0.0749999994067383	7.03124996027265e-06\\
0.089999998523775	1.21499998576497e-05\\
0.104999996809296	1.92937495812201e-05\\
0.1199999937792	2.87999989335771e-05\\
0.134999988789917	4.10062475678123e-05\\
0.149999981015626	5.62499949148997e-05\\
0.164999969425477	7.48687400905784e-05\\
0.179999952760806	9.71999817791672e-05\\
0.194999929512356	0.000123581218091752\\
0.209999897897498	0.000154349946396183\\
0.224999855837445	0.000189843663116309\\
0.239999800934476	0.00023039986349791\\
0.254999730449155	0.000276356041338733\\
0.269999641277546	0.000328049688680096\\
0.284999529928439	0.00038581829545748\\
0.29999939250057	0.000449999349107562\\
0.314999224659834	0.000520930334129102\\
0.329999021616518	0.000598948731595155\\
0.344998778102512	0.000684392018614011\\
0.359998488348539	0.000777597667736327\\
0.374998146061372	0.000878903146305879\\
0.389997744401064	0.000988645915751365\\
0.404997275958169	0.00110716343081671\\
0.419996732730967	0.0012347931387273\\
0.434996106102694	0.00137187247828961\\
0.44999538681877	0.0015187388789216\\
0.464994564964025	0.0016757297596114\\
0.479993629939938	0.00184318252780169\\
0.494992570441861	0.00202143457819716\\
0.509991374436264	0.0022108232914926\\
0.524990029137964	0.00241168603301898\\
0.539988520987371	0.00262436015130494\\
0.554986835627727	0.00284918297655125\\
0.569984957882355	0.00308649181901552\\
0.584982871731902	0.00333662396730478\\
0.599980560291598	0.00359991668657317\\
0.614978005788507	0.00387670721662243\\
0.629975189538788	0.00416733276990235\\
0.64497209192496	0.00447213052940889\\
0.65996869237317	0.00479143764647726\\
0.674964969330465	0.00512559123846742\\
0.689960900242074	0.00547492838633954\\
0.704956461528692	0.00583978613211674\\
0.719951628563768	0.00622050147623264\\
0.734946375650808	0.00661741137476123\\
0.749940676000675	0.00703085273652633\\
0.764934501708899	0.00746116242008827\\
0.779927823733006	0.00790867723060517\\
0.794920611869835	0.00837373391656631\\
0.809912834732883	0.00885666916639496\\
0.824904459729649	0.00935781960491831\\
0.839895453038988	0.0098775217897017\\
0.854885779588485	0.0104161122072449\\
0.869875403031825	0.0109739272690379\\
0.88486428572619	0.0115513033074728\\
0.899852388709659	0.0121485765716116\\
0.914839671678623	0.0127660832228037\\
0.929826092965216	0.0134041593301545\\
0.944811609514758	0.0140631408658399\\
0.959796176863217	0.014743363700265\\
0.974779749114681	0.0154451635970648\\
0.989762278918851	0.016168876207944\\
1.00474371744855	0.0169148370673536\\
1.01972401437727	0.0176833815870013\\
1.03470311785666	0.0184748450501945\\
1.0496809744942	0.0192895626060114\\
1.06465752933066	0.0201278692632996\\
1.07963272581783	0.0209900998844985\\
1.09460650579607	0.0218765891792838\\
1.10957880947201	0.0227876716980311\\
1.12454957539624	0.0237236818250961\\
1.13951874044096	0.0246849537719092\\
1.15448623977781	0.0256718215698825\\
1.16945200685555	0.0266846190631253\\
1.18441597337789	0.0277236799009672\\
1.19937806928134	0.0287893375302854\\
1.21433822271301	0.0298819251876342\\
1.22929636000855	0.0310017758911742\\
1.24425240567004	0.0321492224323983\\
1.25920628234399	0.0333245973676532\\
1.27415791079928	0.0345282330094523\\
1.28910720990525	0.0357604614175797\\
1.30405409660975	0.0370216143899805\\
1.31899848591725	0.0383120234534375\\
1.33394029086702	0.0396320198540298\\
1.34887942251132	0.0409819345473722\\
1.36381578989365	0.0423620981886327\\
1.37874930002704	0.0437728411223257\\
1.3936798578724	0.0452144933718782\\
1.40860736631693	0.0466873846289674\\
1.42353172615255	0.048191844242627\\
1.43845283605441	0.0497282012081196\\
1.45337059255943	0.0512967841555736\\
1.46828489004495	0.0528979213383813\\
1.48319562070737	0.0545319406213571\\
1.49810267454089	0.0561991694686528\\
};
\addplot [color=green, line width=2.0pt, forget plot]
  table[row sep=crcr]{%
1.49810267454089	0.0561991694686528\\
1.51762787324192	0.0584342197618744\\
1.53714639839389	0.0607268187281459\\
1.5566580803779	0.0630769464444424\\
1.57616274963453	0.065484582487803\\
1.59566023666531	0.0679497059355137\\
1.61515037203418	0.0704722953652834\\
1.63463298636896	0.0730523288554365\\
1.65410791036283	0.0756897839850968\\
1.6735749747758	0.0783846378343869\\
1.6930340104362	0.0811368669846272\\
1.7124848482421	0.0839464475185351\\
1.73192731916283	0.0868133550204395\\
1.75136125424043	0.0897375645764868\\
1.77078648459111	0.0927190507748613\\
1.79020284140672	0.0957577877060037\\
1.80961015595625	0.0988537489628384\\
1.82900825958723	0.102006907641002\\
1.84839698372726	0.105217236339076\\
1.86777615988545	0.108484707158826\\
1.88714561965387	0.111809291705445\\
1.90650519470903	0.1151909610878\\
1.92585471681335	0.118629685918678\\
1.94519401781659	0.122125436315051\\
1.96452292965737	0.125678181898328\\
1.98384128436455	0.129287891794619\\
2.00314891405877	0.132954534635011\\
2.02244565095384	0.13667807855583\\
2.04173132735826	0.140458491198927\\
2.06100577567663	0.144295739711953\\
2.08026882841111	0.148189790748647\\
2.09952031816293	0.152140610469127\\
2.11876007763376	0.156148164540181\\
2.13798793962724	0.16021241813557\\
2.15720373705039	0.164333335936323\\
2.17640730291506	0.168510882131053\\
2.19559847033942	0.172745020416261\\
2.21477707254936	0.177035713996657\\
2.23394294287998	0.181382925585473\\
2.25309591477701	0.185786617404795\\
2.27223582179828	0.190246751185887\\
2.29136249761514	0.194763288169521\\
2.31047577601394	0.199336189106319\\
2.32957549089746	0.203965414257093\\
2.34866147628631	0.208650923393186\\
2.36773356632046	0.213392675796829\\
2.38679159526059	0.218190630261485\\
2.40583539748961	0.223044745092218\\
2.42486480751404	0.227954978106047\\
2.44387965996548	0.232921286632318\\
2.46287978960202	0.237943627513069\\
2.48186503130972	0.243021957103413\\
2.50083522010401	0.248156231271909\\
2.51979019113111	0.25334640540095\\
2.53872977966951	0.258592434387152\\
2.55765382113136	0.263894272641743\\
2.57656215106395	0.269251874090957\\
2.59545460515105	0.274665192176444\\
2.61433101921444	0.280134179855662\\
2.63319122921527	0.285658789602295\\
2.65203507125551	0.291238973406664\\
2.67086238157936	0.296874682776139\\
2.68967299657471	0.302565868735569\\
2.7084667527745	0.308312481827701\\
2.72724348685821	0.314114472113612\\
2.74600303565321	0.319971789173144\\
2.76474523613624	0.325884382105339\\
2.78346992543479	0.331852199528886\\
2.80217694082852	0.337875189582563\\
2.82086611975069	0.343953299925691\\
2.83953729978955	0.350086477738586\\
2.85819031868976	0.356274669723022\\
2.87682501435382	0.362517822102689\\
2.89544122484346	0.368815880623664\\
2.91403878838102	0.375168790554881\\
2.93261754335093	0.381576496688609\\
2.95117732830103	0.388038943340928\\
2.96971798194403	0.394556074352215\\
2.98823934315889	0.401127833087631\\
3.00674125099223	0.407754162437616\\
3.02522354465972	0.414435004818381\\
3.04368606354748	0.421170302172414\\
3.06212864721347	0.427959995968978\\
3.08055113538889	0.434804027204622\\
3.09895336797958	0.441702336403699\\
3.1173351850674	0.448654863618875\\
3.13569642691163	0.455661548431653\\
3.15403693395035	0.462722329952901\\
3.17235654680181	0.469837146823377\\
3.19065510626587	0.477005937214264\\
3.20893245332531	0.484228638827706\\
3.22718842914726	0.491505188897352\\
3.2454228750846	0.498835524188899\\
3.26363563267725	0.506219581000645\\
3.28182654365366	0.513657295164035\\
3.29999544993211	0.521148602044228\\
3.31814219362208	0.528693436540652\\
3.33626661702568	0.536291733087573\\
3.35436856263897	0.543943425654664\\
3.37244787315336	0.551648447747578\\
3.39050439145695	0.559406732408526\\
};
\addplot [color=red, line width=2.0pt, forget plot]
  table[row sep=crcr]{%
3.39050439145695	0.559406732408526\\
3.40427067880942	0.565364016903493\\
3.41802366222663	0.571351951023899\\
3.43176340888294	0.577370195916311\\
3.44548998806308	0.583418413473378\\
3.45920347113764	0.589496266333757\\
3.47290393153872	0.595603417881799\\
3.48659144473557	0.601739532246993\\
3.50026608821038	0.607904274303182\\
3.51392794143403	0.614097309667537\\
3.52757708584206	0.620318304699316\\
3.54121360481052	0.626566926498376\\
3.5548375836321	0.632842842903483\\
3.56844910949213	0.639145722490382\\
3.58204827144481	0.645475234569665\\
3.5956351603894	0.65183104918441\\
3.60920986904655	0.658212837107616\\
3.62277249193466	0.664620269839427\\
3.63632312534636	0.671053019604147\\
3.64986186732497	0.677510759347049\\
3.66338881764113	0.683993162730991\\
3.67690407776946	0.690499904132824\\
3.69040775086528	0.697030658639608\\
3.7038999417414	0.703585102044636\\
3.71738075684502	0.710162910843269\\
3.73085030423463	0.716763762228575\\
3.74430869355709	0.723387334086795\\
3.75775603602464	0.730033304992618\\
3.77119244439211	0.736701354204279\\
3.78461803293411	0.743391161658482\\
3.79803291742234	0.750102407965148\\
3.81143721510295	0.756834774401988\\
3.82483104467396	0.763587942908913\\
3.83821452626277	0.770361596082274\\
3.85158778140374	0.777155417168939\\
3.86495093301579	0.783969090060211\\
3.87830410538015	0.790802299285583\\
3.89164742411808	0.797654730006346\\
3.90498101616876	0.804526068009035\\
3.91830500976713	0.811415999698725\\
3.93161953442191	0.818324212092192\\
3.94492472089362	0.825250392810906\\
3.95822070117264	0.832194230073906\\
3.97150760845743	0.839155412690515\\
3.98478557713269	0.846133630052928\\
3.9980547427477	0.853128572128662\\
4.01131524199464	0.860139929452874\\
4.02456721268697	0.867167393120547\\
4.03781079373795	0.874210654778549\\
4.05104612513914	0.881269406617571\\
4.06427334793898	0.888343341363935\\
4.07749260422144	0.89543215227129\\
4.09070403708472	0.902535533112181\\
4.10390779062002	0.909653178169512\\
4.11710400989032	0.916784782227892\\
4.13029284090928	0.923930040564869\\
4.14347443062012	0.931088648942058\\
4.15664892687464	0.938260303596166\\
4.16981647841222	0.945444701229908\\
4.18297723483889	0.95264153900283\\
4.19613134660647	0.959850514522026\\
4.20927896499173	0.967071325832766\\
4.22242024207562	0.974303671409022\\
4.23555533072256	0.981547250143916\\
4.24868438455971	0.988801761340066\\
4.26180755795638	0.996066904699852\\
4.2749250060034	1.0033423803156\\
4.2880368844926	1.01062788865968\\
4.30114334989627	1.01792313057453\\
4.31424455934671	1.02522780726259\\
4.32734067061578	1.03254162027618\\
4.34043184209456	1.03986427150729\\
4.35351823277293	1.04719546317733\\
4.36660000221932	1.05453489782674\\
4.37967731056037	1.06188227830464\\
4.39275031846074	1.06923730775832\\
4.40581918710285	1.07659968962273\\
4.41888407816673	1.08396912760989\\
4.43194515380987	1.09134532569824\\
4.44500257664708	1.09872798812195\\
4.45805650973041	1.10611681936016\\
4.4711071165291	1.11351152412617\\
4.48415456090951	1.12091180735663\\
4.49719900711515	1.12831737420064\\
4.51024061974666	1.13572793000877\\
4.52327956374183	1.14314318032214\\
4.53631600435571	1.15056283086141\\
4.54935010714064	1.15798658751568\\
4.56238203792636	1.16541415633145\\
4.57541196280009	1.17284524350153\\
4.58844004808674	1.18027955535382\\
4.60146646032895	1.18771679834024\\
4.61449136626734	1.19515667902546\\
4.6275149328206	1.20259890407572\\
4.64053732706573	1.21004318024759\\
4.65355871621821	1.21748921437673\\
4.6665792676122	1.2249367133666\\
4.67959914868075	1.23238538417719\\
4.692618526936	1.23983493381372\\
4.70563756994944	1.24728506931536\\
4.71865644533208	1.2547354977439\\
};
\addplot [color=green, line width=2.0pt, forget plot]
  table[row sep=crcr]{%
4.71865644533208	1.2547354977439\\
5.02428322855391	1.42964079600883\\
5.32991001177575	1.60454609427375\\
5.63553679499759	1.77945139253868\\
5.94116357821943	1.9543566908036\\
6.24679036144126	2.12926198906853\\
6.5524171446631	2.30416728733345\\
6.85804392788494	2.47907258559838\\
7.16367071110678	2.65397788386331\\
7.46929749432861	2.82888318212823\\
7.77492427755045	3.00378848039316\\
8.08055106077229	3.17869377865808\\
8.38617784399412	3.35359907692301\\
8.69180462721596	3.52850437518793\\
8.9974314104378	3.70340967345286\\
9.30305819365964	3.87831497171778\\
9.60868497688147	4.05322026998271\\
9.91431176010331	4.22812556824763\\
10.2199385433251	4.40303086651256\\
10.525565326547	4.57793616477749\\
10.8311921097688	4.75284146304241\\
11.1368188929907	4.92774676130734\\
11.4424456762125	5.10265205957226\\
11.7480724594343	5.27755735783719\\
12.0536992426562	5.45246265610211\\
12.359326025878	5.62736795436704\\
12.6649528090998	5.80227325263196\\
12.9705795923217	5.97717855089689\\
13.2762063755435	6.15208384916182\\
13.5818331587654	6.32698914742674\\
13.8874599419872	6.50189444569167\\
14.193086725209	6.67679974395659\\
14.4987135084309	6.85170504222152\\
14.8043402916527	7.02661034048644\\
15.1099670748745	7.20151563875137\\
15.4155938580964	7.37642093701629\\
15.7212206413182	7.55132623528122\\
16.0268474245401	7.72623153354614\\
16.3324742077619	7.90113683181107\\
16.6381009909837	8.076042130076\\
16.9437277742056	8.25094742834092\\
17.2493545574274	8.42585272660585\\
17.5549813406492	8.60075802487077\\
17.8606081238711	8.7756633231357\\
18.1662349070929	8.95056862140062\\
18.4718616903148	9.12547391966555\\
18.7774884735366	9.30037921793047\\
19.0831152567584	9.4752845161954\\
19.3887420399803	9.65018981446032\\
19.6943688232021	9.82509511272525\\
19.9999956064239	10.0000004109902\\
20.3056223896458	10.1749057092551\\
20.6112491728676	10.34981100752\\
20.9168759560895	10.524716305785\\
21.2225027393113	10.6996216040499\\
21.5281295225331	10.8745269023148\\
21.833756305755	11.0494322005797\\
22.1393830889768	11.2243374988447\\
22.4450098721986	11.3992427971096\\
22.7506366554205	11.5741480953745\\
23.0562634386423	11.7490533936394\\
23.3618902218642	11.9239586919044\\
23.667517005086	12.0988639901693\\
23.9731437883078	12.2737692884342\\
24.2787705715297	12.4486745866991\\
24.5843973547515	12.6235798849641\\
24.8900241379733	12.798485183229\\
25.1956509211952	12.9733904814939\\
25.501277704417	13.1482957797588\\
25.8069044876389	13.3232010780238\\
26.1125312708607	13.4981063762887\\
26.4181580540825	13.6730116745536\\
26.7237848373044	13.8479169728185\\
27.0294116205262	14.0228222710835\\
27.335038403748	14.1977275693484\\
27.6406651869699	14.3726328676133\\
27.9462919701917	14.5475381658782\\
28.2519187534136	14.7224434641432\\
28.5575455366354	14.8973487624081\\
28.8631723198572	15.072254060673\\
29.1687991030791	15.2471593589379\\
29.4744258863009	15.4220646572029\\
29.7800526695227	15.5969699554678\\
30.0856794527446	15.7718752537327\\
30.3913062359664	15.9467805519976\\
30.6969330191882	16.1216858502626\\
31.0025598024101	16.2965911485275\\
31.3081865856319	16.4714964467924\\
31.6138133688538	16.6464017450573\\
31.9194401520756	16.8213070433223\\
32.2250669352974	16.9962123415872\\
32.5306937185193	17.1711176398521\\
32.8363205017411	17.346022938117\\
33.1419472849629	17.520928236382\\
33.4475740681848	17.6958335346469\\
33.7532008514066	17.8707388329118\\
34.0588276346285	18.0456441311767\\
34.3644544178503	18.2205494294417\\
34.6700812010721	18.3954547277066\\
34.975707984294	18.5703600259715\\
35.2813347675158	18.7452653242364\\
};
\addplot [color=red, line width=2.0pt, forget plot]
  table[row sep=crcr]{%
35.2813347675158	18.7452653242364\\
35.2943536428985	18.752715752665\\
35.3073726859119	18.7601658881666\\
35.3203920641671	18.7676154378032\\
35.3334119452357	18.7750641086137\\
35.3464324966297	18.7825116076036\\
35.3594538857822	18.7899576417328\\
35.3724762800273	18.7974019179046\\
35.3854998465805	18.8048441429549\\
35.3985247525189	18.8122840236401\\
35.4115511647611	18.8197212666265\\
35.4245792500478	18.8271555784788\\
35.4376091749215	18.8345866656489\\
35.4506411057072	18.8420142344647\\
35.4636752084922	18.8494379911189\\
35.4767116491061	18.8568576416582\\
35.4897505931012	18.8642728919716\\
35.5027922057327	18.8716834477797\\
35.5158366519384	18.8790890146237\\
35.5288840963188	18.8864892978542\\
35.5419347031175	18.8938840026202\\
35.5549886362008	18.9012728338584\\
35.568046059038	18.9086554962821\\
35.5811071346812	18.9160316943705\\
35.594172025745	18.9234011323576\\
35.6072408943871	18.930763514222\\
35.6203139022875	18.9381185436757\\
35.6333912106286	18.9454659241536\\
35.646472980075	18.952805358803\\
35.6595593707533	18.9601365504731\\
35.6726505422321	18.9674592017042\\
35.6857466535012	18.9747730147178\\
35.6988478629516	18.9820776914058\\
35.7119543283553	18.9893729333207\\
35.7250662068445	18.9966584416647\\
35.7381836548915	19.0039339172805\\
35.7513068282882	19.0111990606403\\
35.7644358821253	19.0184535718364\\
35.7775709707723	19.0256971505713\\
35.7907122478562	19.0329294961476\\
35.8038598662414	19.0401503074583\\
35.817013978009	19.0473592829775\\
35.8301747344357	19.0545561207504\\
35.8433422859733	19.0617405183842\\
35.8565167822278	19.0689121730383\\
35.8696983719386	19.0760707814155\\
35.8828872029576	19.0832160397525\\
35.8960834222279	19.0903476438108\\
35.9092871757632	19.0974652888682\\
35.9224986086264	19.1045686697091\\
35.9357178649089	19.1116574806164\\
35.9489450877088	19.1187314153628\\
35.9621804191099	19.1257901672018\\
35.9754240001609	19.1328334288598\\
35.9886759708533	19.1398608925275\\
36.0019364701002	19.1468722498517\\
36.0152056357152	19.1538671919274\\
36.0284836043905	19.1608454092898\\
36.0417705116752	19.1678065919064\\
36.0550664919543	19.1747504291694\\
36.068371678426	19.1816766098882\\
36.0816862030808	19.1885848222816\\
36.0950101966791	19.1954747539713\\
36.1083437887298	19.202346091974\\
36.1216871074677	19.2091985226948\\
36.1350402798321	19.2160317319201\\
36.1484034314442	19.2228454048114\\
36.1617766865851	19.2296392258981\\
36.1751601681739	19.2364128790714\\
36.1885539977449	19.2431660475784\\
36.2019582954255	19.2498984140152\\
36.2153731799138	19.2566096603219\\
36.2287987684558	19.2632994677761\\
36.2422351768232	19.2699675169877\\
36.2556825192908	19.2766134878936\\
36.2691409086133	19.2832370597518\\
36.2826104560029	19.2898379111371\\
36.2960912711065	19.2964157199357\\
36.3095834619826	19.3029701633407\\
36.3230871350784	19.3095009178475\\
36.3366023952068	19.3160076592494\\
36.3501293455229	19.3224900626333\\
36.3636680875015	19.3289478023762\\
36.3772187209132	19.3353805521409\\
36.3907813438013	19.3417879848727\\
36.4043560524585	19.3481697727959\\
36.4179429414031	19.3545255874107\\
36.4315421033558	19.36085509949\\
36.4451536292158	19.3671579790769\\
36.4587776080374	19.373433895482\\
36.4724141270058	19.379682517281\\
36.4860632714139	19.3859035123128\\
36.4997251246375	19.3920965476772\\
36.5133997681123	19.3982612897334\\
36.5270872813092	19.4043974040985\\
36.5407877417102	19.4105045556466\\
36.5545012247848	19.416582408507\\
36.5682278039649	19.422630626064\\
36.5819675506213	19.4286488709564\\
36.5957205340385	19.4346368050769\\
36.6094868213909	19.4405940895718\\
};
\addplot [color=green, line width=2.0pt, forget plot]
  table[row sep=crcr]{%
36.6094868213909	19.4405940895718\\
36.6275434185365	19.448352407971\\
36.6456228080911	19.4560574633369\\
36.6637248329407	19.4637091887111\\
36.6818493357748	19.4713075175982\\
36.6999961590873	19.4788523839671\\
36.7181651451782	19.4863437222511\\
36.736356136155	19.493781467349\\
36.7545689739339	19.5011655546249\\
36.7728035002412	19.5084959199096\\
36.7910595566148	19.5157724995005\\
36.8093369844055	19.5229952301625\\
36.8276356247783	19.5301640491284\\
36.8459553187139	19.5372788940997\\
36.8642959070101	19.5443397032466\\
36.882657230283	19.5513464152092\\
36.9010391289686	19.5582989690976\\
36.9194414433241	19.5651973044925\\
36.9378640134292	19.5720413614458\\
36.9563066791877	19.5788310804812\\
36.9747692803287	19.5855664025945\\
36.9932516564079	19.5922472692541\\
37.0117536468095	19.5988736224019\\
37.0302750907469	19.6054454044534\\
37.0488158272646	19.6119625582983\\
37.0673756952396	19.6184250273011\\
37.0859545333823	19.6248327553015\\
37.1045521802384	19.6311856866148\\
37.1231684741904	19.6374837660327\\
37.1418032534583	19.6437269388235\\
37.1604563561017	19.6499151507325\\
37.1791276200211	19.6560483479829\\
37.1978168829589	19.6621264772757\\
37.2165239825013	19.6681494857907\\
37.2352487560793	19.6741173211866\\
37.2539910409704	19.6800299316014\\
37.2727506743001	19.6858872656534\\
37.2915274930428	19.691689272441\\
37.3103213340237	19.6974359015434\\
37.3291320339203	19.7031271030211\\
37.3479594292633	19.7087628274163\\
37.3668033564383	19.7143430257534\\
37.3856636516875	19.719867649539\\
37.4045401511106	19.7253366507631\\
37.4234326906666	19.7307499818987\\
37.4423411061751	19.7361075959027\\
37.4612652333177	19.7414094462165\\
37.4802049076395	19.7466554867655\\
37.4991599645505	19.7518456719606\\
37.518130239327	19.756979956698\\
37.537115567113	19.7620582963594\\
37.5561157829218	19.7670806468131\\
37.5751307216372	19.7720469644136\\
37.5941602180151	19.7769572060025\\
37.6132041066849	19.7818113289088\\
37.6322622221511	19.786609290949\\
37.6513343987941	19.7913510504277\\
37.6704204708727	19.796036566138\\
37.6895202725244	19.8006657973617\\
37.7086336377678	19.8052387038698\\
37.7277604005034	19.8097552459226\\
37.7469003945154	19.8142153842705\\
37.7660534534728	19.8186190801537\\
37.7852194109314	19.8229662953032\\
37.8043981003345	19.8272569919407\\
37.8235893550152	19.8314911327791\\
37.8427930081971	19.8356686810228\\
37.8620088929962	19.8397896003679\\
37.881236842422	19.8438538550027\\
37.9004766893794	19.847861409608\\
37.9197282666698	19.8518122293571\\
37.9389914069927	19.8557062799166\\
37.9582659429471	19.8595435274463\\
37.9775517070328	19.8633239385996\\
37.9968485316523	19.8670474805239\\
38.0161562491119	19.8707141208609\\
38.0354746916231	19.8743238277466\\
38.0548036913043	19.8778765698117\\
38.0741430801823	19.8813723161823\\
38.0934926901933	19.8848110364794\\
38.1128523531848	19.8881927008199\\
38.1322219009172	19.8915172798162\\
38.1516011650646	19.8947847445771\\
38.1709899772169	19.8979950667076\\
38.190388168881	19.9011482183092\\
38.2097955714822	19.9042441719803\\
38.2292120163658	19.9072829008164\\
38.2486373347986	19.9102643784104\\
38.2680713579702	19.9131885788523\\
38.2875139169946	19.9160554767304\\
38.3069648429115	19.9188650471306\\
38.3264239666881	19.9216172656371\\
38.3458911192201	19.9243121083326\\
38.3653661313337	19.9269495517981\\
38.3848488337866	19.9295295731139\\
38.4043390572698	19.9320521498588\\
38.4238366324089	19.9345172601111\\
38.4433413897655	19.9369248824485\\
38.4628531598389	19.9392749959483\\
38.4823717730674	19.9415675801873\\
38.50189705983	19.9438026152425\\
};
\addplot [color=red, line width=2.0pt, forget plot]
  table[row sep=crcr]{%
38.50189705983	19.9438026152425\\
38.5168041158082	19.9454698249139\\
38.5317148485725	19.9471038250163\\
38.5466291481177	19.948704943014\\
38.5615469066404	19.9502735067718\\
38.5764680185186	19.9518098445434\\
38.5913923802895	19.9533142849589\\
38.6063198906287	19.9547871570139\\
38.6212504503285	19.9562287900574\\
38.6361839622766	19.9576395137812\\
38.6511203314343	19.9590196582089\\
38.6660594648151	19.9603695536851\\
38.6810012714628	19.9616895308652\\
38.6959456624302	19.9629799207048\\
38.710892550757	19.9642410544501\\
38.7258418514481	19.965473263628\\
38.7407934814517	19.9666768800367\\
38.7557473596375	19.9678522357359\\
38.7707034067751	19.9689996630382\\
38.7856615455111	19.9701194945002\\
38.8006217003482	19.9712120629134\\
38.8155837976225	19.9722777012961\\
38.8305477654814	19.9733167428849\\
38.845513533862	19.9743295211268\\
38.8604810344683	19.9753163696711\\
38.8754502007495	19.9762776223622\\
38.8904209678778	19.9772136132315\\
38.9053932727257	19.9781246764905\\
38.9203670538443	19.9790111465236\\
38.9353422514406	19.9798733578813\\
38.9503188073554	19.9807116452733\\
38.9652966650409	19.9815263435622\\
38.9802757695384	19.9823177877569\\
38.9952560674557	19.9830863130065\\
39.010237506945	19.9838322545944\\
39.0252200376801	19.9845559479323\\
39.0402036108343	19.9852577285549\\
39.0551881790578	19.9859379321137\\
39.070173696455	19.9865968943726\\
39.0851601185624	19.9872349512021\\
39.1001474023257	19.9878524385742\\
39.1151355060774	19.9884496925583\\
39.1301243895145	19.9890270493156\\
39.1451140136754	19.9895848450953\\
39.1601043409177	19.9901234162299\\
39.1750953348955	19.9906430991309\\
39.1900869605369	19.9911442302848\\
39.2050791840212	19.9916271462492\\
39.2200719727563	19.992092183649\\
39.235065295356	19.9925396791728\\
39.2500591216178	19.9929699695689\\
39.2650534224995	19.9933833916426\\
39.280048170097	19.9937802822524\\
39.2950433376218	19.9941609783074\\
39.3100388993778	19.9945258167635\\
39.3250348307388	19.9948751346212\\
39.3400311081259	19.9952092689225\\
39.3550277089848	19.9955285567486\\
39.3700246117631	19.9958333352167\\
39.3850217958872	19.9961239414782\\
39.4000192417401	19.9964007127161\\
39.4150169306385	19.9966639861429\\
39.4300148448098	19.9969140989985\\
39.4450129673697	19.997151388548\\
39.4600112822992	19.99737619208\\
39.4750097744222	19.9975888469048\\
39.4900084293823	19.9977896903527\\
39.5050072336203	19.9979790597721\\
39.5200061743515	19.9981572925285\\
39.5350052395428	19.9983247260025\\
39.55000441789	19.9984816975888\\
39.565003698795	19.998628544695\\
39.580003072343	19.99876560474\\
39.59500252928	19.9988932151532\\
39.6100020609895	19.9990117133734\\
39.6250016594704	19.999121436848\\
39.6400013173135	19.9992227230316\\
39.6550010276794	19.9993159093857\\
39.6700007842753	19.9994013333776\\
39.6850005813324	19.99947933248\\
39.7000004135828	19.9995502441698\\
39.7150002762375	19.9996144059282\\
39.7300001649627	19.9996721552397\\
39.7450000758576	19.9997238295918\\
39.7600000054314	19.9997697664743\\
39.7749999505806	19.9998103033794\\
39.7899999085662	19.9998457778007\\
39.8049998769909	19.9998765272337\\
39.8199998537763	19.9999028891746\\
39.8349998371404	19.9999252011209\\
39.8499998255742	19.9999438005707\\
39.8649998178195	19.9999590250227\\
39.8799998128459	19.9999712119759\\
39.894999809828	19.9999806989299\\
39.9099998081227	19.9999878233842\\
39.9249998072463	19.9999929228387\\
39.9399998068518	19.9999963347933\\
39.9549998067062	19.9999983967479\\
39.9699998066675	19.9999994462025\\
39.9849998066622	19.9999998206571\\
39.999999806662	19.9999998576117\\
};
\addplot [color=red, line width=2.0pt, only marks, mark size=2.5pt, mark=*, mark options={solid, fill=red, red}, forget plot]
  table[row sep=crcr]{%
1.49810267454089	0.0561991694686528\\
};
\addplot [color=red, line width=2.0pt, only marks, mark size=2.5pt, mark=*, mark options={solid, fill=red, red}, forget plot]
  table[row sep=crcr]{%
3.39050439145695	0.559406732408526\\
};
\addplot [color=red, line width=2.0pt, only marks, mark size=2.5pt, mark=*, mark options={solid, fill=red, red}, forget plot]
  table[row sep=crcr]{%
4.71865644533208	1.2547354977439\\
};
\addplot [color=red, line width=2.0pt, only marks, mark size=2.5pt, mark=*, mark options={solid, fill=red, red}, forget plot]
  table[row sep=crcr]{%
35.2813347675158	18.7452653242364\\
};
\addplot [color=red, line width=2.0pt, only marks, mark size=2.5pt, mark=*, mark options={solid, fill=red, red}, forget plot]
  table[row sep=crcr]{%
36.6094868213909	19.4405940895718\\
};
\addplot [color=red, line width=2.0pt, only marks, mark size=2.5pt, mark=*, mark options={solid, fill=red, red}, forget plot]
  table[row sep=crcr]{%
38.50189705983	19.9438026152425\\
};
\addplot [color=red, line width=2.0pt, only marks, mark size=2.5pt, mark=*, mark options={solid, fill=red, red}, forget plot]
  table[row sep=crcr]{%
39.999999806662	19.9999998576117\\
};
\addplot [color=blue, line width=2.0pt, only marks, mark size=2.5pt, mark=*, mark options={solid, fill=blue, blue}, forget plot]
  table[row sep=crcr]{%
0	0\\
};
\addplot [color=blue, line width=2.0pt, only marks, mark size=2.5pt, mark=*, mark options={solid, fill=blue, blue}, forget plot]
  table[row sep=crcr]{%
40	20\\
};
\end{axis}
\end{tikzpicture}%%
  \caption{Duboids solution example 1}
  \label{fig:DuboidsRes0}
\end{figure}
%
The second example is a trivial connection with a straight line (Figure \ref{fig:DuboidsRes1}).\\
%
\begin{figure}[htb!]
  \centering
  % This file was created by matlab2tikz.
%
%The latest updates can be retrieved from
%  http://www.mathworks.com/matlabcentral/fileexchange/22022-matlab2tikz-matlab2tikz
%where you can also make suggestions and rate matlab2tikz.
%
\begin{tikzpicture}[scale = 0.7]

\begin{axis}[%
width=\linewidth,
height=0.776\linewidth,
at={(0\linewidth,0\linewidth)},
scale only axis,
xmin=0,
xmax=40,
xlabel style={font=\color{white!15!black}},
xlabel={x(m)},
ymin=-15.7741935483871,
ymax=15.7741935483871,
ylabel style={font=\color{white!15!black}},
ylabel={y(m)},
axis background/.style={fill=white},
title style={font=\bfseries},
title={$L_{tot}$ = 40, $k_{max}$ = 0.15, $J_{max}$ = 0.08, Type = [S00]},
axis x line*=bottom,
axis y line*=left,
xmajorgrids,
xminorgrids,
ymajorgrids,
yminorgrids
]
\addplot [color=red, line width=2.0pt, forget plot]
  table[row sep=crcr]{%
0	0\\
0	0\\
0	0\\
0	0\\
0	0\\
0	0\\
0	0\\
0	0\\
0	0\\
0	0\\
0	0\\
0	0\\
0	0\\
0	0\\
0	0\\
0	0\\
0	0\\
0	0\\
0	0\\
0	0\\
0	0\\
0	0\\
0	0\\
0	0\\
0	0\\
0	0\\
0	0\\
0	0\\
0	0\\
0	0\\
0	0\\
0	0\\
0	0\\
0	0\\
0	0\\
0	0\\
0	0\\
0	0\\
0	0\\
0	0\\
0	0\\
0	0\\
0	0\\
0	0\\
0	0\\
0	0\\
0	0\\
0	0\\
0	0\\
0	0\\
0	0\\
0	0\\
0	0\\
0	0\\
0	0\\
0	0\\
0	0\\
0	0\\
0	0\\
0	0\\
0	0\\
0	0\\
0	0\\
0	0\\
0	0\\
0	0\\
0	0\\
0	0\\
0	0\\
0	0\\
0	0\\
0	0\\
0	0\\
0	0\\
0	0\\
0	0\\
0	0\\
0	0\\
0	0\\
0	0\\
0	0\\
0	0\\
0	0\\
0	0\\
0	0\\
0	0\\
0	0\\
0	0\\
0	0\\
0	0\\
0	0\\
0	0\\
0	0\\
0	0\\
0	0\\
0	0\\
0	0\\
0	0\\
0	0\\
0	0\\
0	0\\
};
\addplot [color=green, line width=2.0pt, forget plot]
  table[row sep=crcr]{%
0	0\\
0.399999997019739	0\\
0.799999994039479	0\\
1.19999999105922	0\\
1.59999998807896	0\\
1.9999999850987	0\\
2.39999998211844	0\\
2.79999997913818	0\\
3.19999997615791	0\\
3.59999997317765	0\\
3.99999997019739	0\\
4.39999996721713	0\\
4.79999996423687	0\\
5.19999996125661	0\\
5.59999995827635	0\\
5.99999995529609	0\\
6.39999995231583	0\\
6.79999994933557	0\\
7.19999994635531	0\\
7.59999994337505	0\\
7.99999994039479	0\\
8.39999993741453	0\\
8.79999993443426	0\\
9.19999993145401	0\\
9.59999992847374	0\\
9.99999992549348	0\\
10.3999999225132	0\\
10.799999919533	0\\
11.1999999165527	0\\
11.5999999135724	0\\
11.9999999105922	0\\
12.3999999076119	0\\
12.7999999046317	0\\
13.1999999016514	0\\
13.5999998986711	0\\
13.9999998956909	0\\
14.3999998927106	0\\
14.7999998897304	0\\
15.1999998867501	0\\
15.5999998837698	0\\
15.9999998807896	0\\
16.3999998778093	0\\
16.7999998748291	0\\
17.1999998718488	0\\
17.5999998688685	0\\
17.9999998658883	0\\
18.399999862908	0\\
18.7999998599278	0\\
19.1999998569475	0\\
19.5999998539672	0\\
19.999999850987	0\\
20.3999998480067	0\\
20.7999998450264	0\\
21.1999998420462	0\\
21.5999998390659	0\\
21.9999998360857	0\\
22.3999998331054	0\\
22.7999998301251	0\\
23.1999998271449	0\\
23.5999998241646	0\\
23.9999998211844	0\\
24.3999998182041	0\\
24.7999998152238	0\\
25.1999998122436	0\\
25.5999998092633	0\\
25.9999998062831	0\\
26.3999998033028	0\\
26.7999998003225	0\\
27.1999997973423	0\\
27.599999794362	0\\
27.9999997913818	0\\
28.3999997884015	0\\
28.7999997854212	0\\
29.199999782441	0\\
29.5999997794607	0\\
29.9999997764805	0\\
30.3999997735002	0\\
30.7999997705199	0\\
31.1999997675397	0\\
31.5999997645594	0\\
31.9999997615791	0\\
32.3999997585989	0\\
32.7999997556186	0\\
33.1999997526384	0\\
33.5999997496581	0\\
33.9999997466778	0\\
34.3999997436976	0\\
34.7999997407173	0\\
35.1999997377371	0\\
35.5999997347568	0\\
35.9999997317765	0\\
36.3999997287963	0\\
36.799999725816	0\\
37.1999997228358	0\\
37.5999997198555	0\\
37.9999997168752	0\\
38.399999713895	0\\
38.7999997109147	0\\
39.1999997079345	0\\
39.5999997049542	0\\
39.9999997019739	0\\
};
\addplot [color=red, line width=2.0pt, forget plot]
  table[row sep=crcr]{%
39.9999997019739	0\\
39.9999997019739	0\\
39.9999997019739	0\\
39.9999997019739	0\\
39.9999997019739	0\\
39.9999997019739	0\\
39.9999997019739	0\\
39.9999997019739	0\\
39.9999997019739	0\\
39.9999997019739	0\\
39.9999997019739	0\\
39.9999997019739	0\\
39.9999997019739	0\\
39.9999997019739	0\\
39.9999997019739	0\\
39.9999997019739	0\\
39.9999997019739	0\\
39.9999997019739	0\\
39.9999997019739	0\\
39.9999997019739	0\\
39.9999997019739	0\\
39.9999997019739	0\\
39.9999997019739	0\\
39.9999997019739	0\\
39.9999997019739	0\\
39.9999997019739	0\\
39.9999997019739	0\\
39.9999997019739	0\\
39.9999997019739	0\\
39.9999997019739	0\\
39.9999997019739	0\\
39.9999997019739	0\\
39.9999997019739	0\\
39.9999997019739	0\\
39.9999997019739	0\\
39.9999997019739	0\\
39.9999997019739	0\\
39.9999997019739	0\\
39.9999997019739	0\\
39.9999997019739	0\\
39.9999997019739	0\\
39.9999997019739	0\\
39.9999997019739	0\\
39.9999997019739	0\\
39.9999997019739	0\\
39.9999997019739	0\\
39.9999997019739	0\\
39.9999997019739	0\\
39.9999997019739	0\\
39.9999997019739	0\\
39.9999997019739	0\\
39.9999997019739	0\\
39.9999997019739	0\\
39.9999997019739	0\\
39.9999997019739	0\\
39.9999997019739	0\\
39.9999997019739	0\\
39.9999997019739	0\\
39.9999997019739	0\\
39.9999997019739	0\\
39.9999997019739	0\\
39.9999997019739	0\\
39.9999997019739	0\\
39.9999997019739	0\\
39.9999997019739	0\\
39.9999997019739	0\\
39.9999997019739	0\\
39.9999997019739	0\\
39.9999997019739	0\\
39.9999997019739	0\\
39.9999997019739	0\\
39.9999997019739	0\\
39.9999997019739	0\\
39.9999997019739	0\\
39.9999997019739	0\\
39.9999997019739	0\\
39.9999997019739	0\\
39.9999997019739	0\\
39.9999997019739	0\\
39.9999997019739	0\\
39.9999997019739	0\\
39.9999997019739	0\\
39.9999997019739	0\\
39.9999997019739	0\\
39.9999997019739	0\\
39.9999997019739	0\\
39.9999997019739	0\\
39.9999997019739	0\\
39.9999997019739	0\\
39.9999997019739	0\\
39.9999997019739	0\\
39.9999997019739	0\\
39.9999997019739	0\\
39.9999997019739	0\\
39.9999997019739	0\\
39.9999997019739	0\\
39.9999997019739	0\\
39.9999997019739	0\\
39.9999997019739	0\\
39.9999997019739	0\\
39.9999997019739	0\\
};
\addplot [color=green, line width=2.0pt, forget plot]
  table[row sep=crcr]{%
39.9999997019739	0\\
39.9999997019739	0\\
39.9999997019739	0\\
39.9999997019739	0\\
39.9999997019739	0\\
39.9999997019739	0\\
39.9999997019739	0\\
39.9999997019739	0\\
39.9999997019739	0\\
39.9999997019739	0\\
39.9999997019739	0\\
39.9999997019739	0\\
39.9999997019739	0\\
39.9999997019739	0\\
39.9999997019739	0\\
39.9999997019739	0\\
39.9999997019739	0\\
39.9999997019739	0\\
39.9999997019739	0\\
39.9999997019739	0\\
39.9999997019739	0\\
39.9999997019739	0\\
39.9999997019739	0\\
39.9999997019739	0\\
39.9999997019739	0\\
39.9999997019739	0\\
39.9999997019739	0\\
39.9999997019739	0\\
39.9999997019739	0\\
39.9999997019739	0\\
39.9999997019739	0\\
39.9999997019739	0\\
39.9999997019739	0\\
39.9999997019739	0\\
39.9999997019739	0\\
39.9999997019739	0\\
39.9999997019739	0\\
39.9999997019739	0\\
39.9999997019739	0\\
39.9999997019739	0\\
39.9999997019739	0\\
39.9999997019739	0\\
39.9999997019739	0\\
39.9999997019739	0\\
39.9999997019739	0\\
39.9999997019739	0\\
39.9999997019739	0\\
39.9999997019739	0\\
39.9999997019739	0\\
39.9999997019739	0\\
39.9999997019739	0\\
39.9999997019739	0\\
39.9999997019739	0\\
39.9999997019739	0\\
39.9999997019739	0\\
39.9999997019739	0\\
39.9999997019739	0\\
39.9999997019739	0\\
39.9999997019739	0\\
39.9999997019739	0\\
39.9999997019739	0\\
39.9999997019739	0\\
39.9999997019739	0\\
39.9999997019739	0\\
39.9999997019739	0\\
39.9999997019739	0\\
39.9999997019739	0\\
39.9999997019739	0\\
39.9999997019739	0\\
39.9999997019739	0\\
39.9999997019739	0\\
39.9999997019739	0\\
39.9999997019739	0\\
39.9999997019739	0\\
39.9999997019739	0\\
39.9999997019739	0\\
39.9999997019739	0\\
39.9999997019739	0\\
39.9999997019739	0\\
39.9999997019739	0\\
39.9999997019739	0\\
39.9999997019739	0\\
39.9999997019739	0\\
39.9999997019739	0\\
39.9999997019739	0\\
39.9999997019739	0\\
39.9999997019739	0\\
39.9999997019739	0\\
39.9999997019739	0\\
39.9999997019739	0\\
39.9999997019739	0\\
39.9999997019739	0\\
39.9999997019739	0\\
39.9999997019739	0\\
39.9999997019739	0\\
39.9999997019739	0\\
39.9999997019739	0\\
39.9999997019739	0\\
39.9999997019739	0\\
39.9999997019739	0\\
39.9999997019739	0\\
};
\addplot [color=red, line width=2.0pt, forget plot]
  table[row sep=crcr]{%
39.9999997019739	0\\
39.9999997019739	0\\
39.9999997019739	0\\
39.9999997019739	0\\
39.9999997019739	0\\
39.9999997019739	0\\
39.9999997019739	0\\
39.9999997019739	0\\
39.9999997019739	0\\
39.9999997019739	0\\
39.9999997019739	0\\
39.9999997019739	0\\
39.9999997019739	0\\
39.9999997019739	0\\
39.9999997019739	0\\
39.9999997019739	0\\
39.9999997019739	0\\
39.9999997019739	0\\
39.9999997019739	0\\
39.9999997019739	0\\
39.9999997019739	0\\
39.9999997019739	0\\
39.9999997019739	0\\
39.9999997019739	0\\
39.9999997019739	0\\
39.9999997019739	0\\
39.9999997019739	0\\
39.9999997019739	0\\
39.9999997019739	0\\
39.9999997019739	0\\
39.9999997019739	0\\
39.9999997019739	0\\
39.9999997019739	0\\
39.9999997019739	0\\
39.9999997019739	0\\
39.9999997019739	0\\
39.9999997019739	0\\
39.9999997019739	0\\
39.9999997019739	0\\
39.9999997019739	0\\
39.9999997019739	0\\
39.9999997019739	0\\
39.9999997019739	0\\
39.9999997019739	0\\
39.9999997019739	0\\
39.9999997019739	0\\
39.9999997019739	0\\
39.9999997019739	0\\
39.9999997019739	0\\
39.9999997019739	0\\
39.9999997019739	0\\
39.9999997019739	0\\
39.9999997019739	0\\
39.9999997019739	0\\
39.9999997019739	0\\
39.9999997019739	0\\
39.9999997019739	0\\
39.9999997019739	0\\
39.9999997019739	0\\
39.9999997019739	0\\
39.9999997019739	0\\
39.9999997019739	0\\
39.9999997019739	0\\
39.9999997019739	0\\
39.9999997019739	0\\
39.9999997019739	0\\
39.9999997019739	0\\
39.9999997019739	0\\
39.9999997019739	0\\
39.9999997019739	0\\
39.9999997019739	0\\
39.9999997019739	0\\
39.9999997019739	0\\
39.9999997019739	0\\
39.9999997019739	0\\
39.9999997019739	0\\
39.9999997019739	0\\
39.9999997019739	0\\
39.9999997019739	0\\
39.9999997019739	0\\
39.9999997019739	0\\
39.9999997019739	0\\
39.9999997019739	0\\
39.9999997019739	0\\
39.9999997019739	0\\
39.9999997019739	0\\
39.9999997019739	0\\
39.9999997019739	0\\
39.9999997019739	0\\
39.9999997019739	0\\
39.9999997019739	0\\
39.9999997019739	0\\
39.9999997019739	0\\
39.9999997019739	0\\
39.9999997019739	0\\
39.9999997019739	0\\
39.9999997019739	0\\
39.9999997019739	0\\
39.9999997019739	0\\
39.9999997019739	0\\
39.9999997019739	0\\
};
\addplot [color=green, line width=2.0pt, forget plot]
  table[row sep=crcr]{%
39.9999997019739	0\\
39.9999997019739	0\\
39.9999997019739	0\\
39.9999997019739	0\\
39.9999997019739	0\\
39.9999997019739	0\\
39.9999997019739	0\\
39.9999997019739	0\\
39.9999997019739	0\\
39.9999997019739	0\\
39.9999997019739	0\\
39.9999997019739	0\\
39.9999997019739	0\\
39.9999997019739	0\\
39.9999997019739	0\\
39.9999997019739	0\\
39.9999997019739	0\\
39.9999997019739	0\\
39.9999997019739	0\\
39.9999997019739	0\\
39.9999997019739	0\\
39.9999997019739	0\\
39.9999997019739	0\\
39.9999997019739	0\\
39.9999997019739	0\\
39.9999997019739	0\\
39.9999997019739	0\\
39.9999997019739	0\\
39.9999997019739	0\\
39.9999997019739	0\\
39.9999997019739	0\\
39.9999997019739	0\\
39.9999997019739	0\\
39.9999997019739	0\\
39.9999997019739	0\\
39.9999997019739	0\\
39.9999997019739	0\\
39.9999997019739	0\\
39.9999997019739	0\\
39.9999997019739	0\\
39.9999997019739	0\\
39.9999997019739	0\\
39.9999997019739	0\\
39.9999997019739	0\\
39.9999997019739	0\\
39.9999997019739	0\\
39.9999997019739	0\\
39.9999997019739	0\\
39.9999997019739	0\\
39.9999997019739	0\\
39.9999997019739	0\\
39.9999997019739	0\\
39.9999997019739	0\\
39.9999997019739	0\\
39.9999997019739	0\\
39.9999997019739	0\\
39.9999997019739	0\\
39.9999997019739	0\\
39.9999997019739	0\\
39.9999997019739	0\\
39.9999997019739	0\\
39.9999997019739	0\\
39.9999997019739	0\\
39.9999997019739	0\\
39.9999997019739	0\\
39.9999997019739	0\\
39.9999997019739	0\\
39.9999997019739	0\\
39.9999997019739	0\\
39.9999997019739	0\\
39.9999997019739	0\\
39.9999997019739	0\\
39.9999997019739	0\\
39.9999997019739	0\\
39.9999997019739	0\\
39.9999997019739	0\\
39.9999997019739	0\\
39.9999997019739	0\\
39.9999997019739	0\\
39.9999997019739	0\\
39.9999997019739	0\\
39.9999997019739	0\\
39.9999997019739	0\\
39.9999997019739	0\\
39.9999997019739	0\\
39.9999997019739	0\\
39.9999997019739	0\\
39.9999997019739	0\\
39.9999997019739	0\\
39.9999997019739	0\\
39.9999997019739	0\\
39.9999997019739	0\\
39.9999997019739	0\\
39.9999997019739	0\\
39.9999997019739	0\\
39.9999997019739	0\\
39.9999997019739	0\\
39.9999997019739	0\\
39.9999997019739	0\\
39.9999997019739	0\\
39.9999997019739	0\\
};
\addplot [color=red, line width=2.0pt, forget plot]
  table[row sep=crcr]{%
39.9999997019739	0\\
39.9999997019739	0\\
39.9999997019739	0\\
39.9999997019739	0\\
39.9999997019739	0\\
39.9999997019739	0\\
39.9999997019739	0\\
39.9999997019739	0\\
39.9999997019739	0\\
39.9999997019739	0\\
39.9999997019739	0\\
39.9999997019739	0\\
39.9999997019739	0\\
39.9999997019739	0\\
39.9999997019739	0\\
39.9999997019739	0\\
39.9999997019739	0\\
39.9999997019739	0\\
39.9999997019739	0\\
39.9999997019739	0\\
39.9999997019739	0\\
39.9999997019739	0\\
39.9999997019739	0\\
39.9999997019739	0\\
39.9999997019739	0\\
39.9999997019739	0\\
39.9999997019739	0\\
39.9999997019739	0\\
39.9999997019739	0\\
39.9999997019739	0\\
39.9999997019739	0\\
39.9999997019739	0\\
39.9999997019739	0\\
39.9999997019739	0\\
39.9999997019739	0\\
39.9999997019739	0\\
39.9999997019739	0\\
39.9999997019739	0\\
39.9999997019739	0\\
39.9999997019739	0\\
39.9999997019739	0\\
39.9999997019739	0\\
39.9999997019739	0\\
39.9999997019739	0\\
39.9999997019739	0\\
39.9999997019739	0\\
39.9999997019739	0\\
39.9999997019739	0\\
39.9999997019739	0\\
39.9999997019739	0\\
39.9999997019739	0\\
39.9999997019739	0\\
39.9999997019739	0\\
39.9999997019739	0\\
39.9999997019739	0\\
39.9999997019739	0\\
39.9999997019739	0\\
39.9999997019739	0\\
39.9999997019739	0\\
39.9999997019739	0\\
39.9999997019739	0\\
39.9999997019739	0\\
39.9999997019739	0\\
39.9999997019739	0\\
39.9999997019739	0\\
39.9999997019739	0\\
39.9999997019739	0\\
39.9999997019739	0\\
39.9999997019739	0\\
39.9999997019739	0\\
39.9999997019739	0\\
39.9999997019739	0\\
39.9999997019739	0\\
39.9999997019739	0\\
39.9999997019739	0\\
39.9999997019739	0\\
39.9999997019739	0\\
39.9999997019739	0\\
39.9999997019739	0\\
39.9999997019739	0\\
39.9999997019739	0\\
39.9999997019739	0\\
39.9999997019739	0\\
39.9999997019739	0\\
39.9999997019739	0\\
39.9999997019739	0\\
39.9999997019739	0\\
39.9999997019739	0\\
39.9999997019739	0\\
39.9999997019739	0\\
39.9999997019739	0\\
39.9999997019739	0\\
39.9999997019739	0\\
39.9999997019739	0\\
39.9999997019739	0\\
39.9999997019739	0\\
39.9999997019739	0\\
39.9999997019739	0\\
39.9999997019739	0\\
39.9999997019739	0\\
39.9999997019739	0\\
};
\addplot [color=red, line width=2.0pt, only marks, mark size=2.5pt, mark=*, mark options={solid, fill=red, red}, forget plot]
  table[row sep=crcr]{%
0	0\\
};
\addplot [color=red, line width=2.0pt, only marks, mark size=2.5pt, mark=*, mark options={solid, fill=red, red}, forget plot]
  table[row sep=crcr]{%
39.9999997019739	0\\
};
\addplot [color=red, line width=2.0pt, only marks, mark size=2.5pt, mark=*, mark options={solid, fill=red, red}, forget plot]
  table[row sep=crcr]{%
39.9999997019739	0\\
};
\addplot [color=red, line width=2.0pt, only marks, mark size=2.5pt, mark=*, mark options={solid, fill=red, red}, forget plot]
  table[row sep=crcr]{%
39.9999997019739	0\\
};
\addplot [color=red, line width=2.0pt, only marks, mark size=2.5pt, mark=*, mark options={solid, fill=red, red}, forget plot]
  table[row sep=crcr]{%
39.9999997019739	0\\
};
\addplot [color=red, line width=2.0pt, only marks, mark size=2.5pt, mark=*, mark options={solid, fill=red, red}, forget plot]
  table[row sep=crcr]{%
39.9999997019739	0\\
};
\addplot [color=red, line width=2.0pt, only marks, mark size=2.5pt, mark=*, mark options={solid, fill=red, red}, forget plot]
  table[row sep=crcr]{%
39.9999997019739	0\\
};
\addplot [color=blue, line width=2.0pt, only marks, mark size=2.5pt, mark=*, mark options={solid, fill=blue, blue}, forget plot]
  table[row sep=crcr]{%
0	0\\
};
\addplot [color=blue, line width=2.0pt, only marks, mark size=2.5pt, mark=*, mark options={solid, fill=blue, blue}, forget plot]
  table[row sep=crcr]{%
40	0\\
};
\end{axis}
\end{tikzpicture}%%
  \caption{Duboids solution example 2}
  \label{fig:DuboidsRes1}
\end{figure}
%
The third example (Figure \ref{fig:DuboidsRes2}) shows a connection from $x=0$, $y=0$, $\theta=0$ and $\kappa=0$ to $x=-1$, $y=0$, $\theta=0$ and $\kappa=0$ obtaining a fancy shape given the constraints on curvature and jerk.\\
%
\begin{figure}[htb!]
  \centering
  % This file was created by matlab2tikz.
%
%The latest updates can be retrieved from
%  http://www.mathworks.com/matlabcentral/fileexchange/22022-matlab2tikz-matlab2tikz
%where you can also make suggestions and rate matlab2tikz.
%
\begin{tikzpicture}

\begin{axis}[%
width=\linewidth,
height=0.776\linewidth,
at={(0\linewidth,0\linewidth)},
scale only axis,
xmin=-13.1872964510542,
xmax=12.1872931956209,
xlabel style={font=\color{white!15!black}},
xlabel={x(m)},
ymin=-10.006591850697,
ymax=10.0065925641807,
ylabel style={font=\color{white!15!black}},
ylabel={y(m)},
axis background/.style={fill=white},
title style={font=\bfseries},
title={$L_{tot}$ = 61.891, $k_{max}$ = 0.2, $J_{max}$ = 0.2, Type = [RSL]},
axis x line*=bottom,
axis y line*=left,
xmajorgrids,
xminorgrids,
ymajorgrids,
yminorgrids
]
\addplot [color=red, line width=2.0pt, forget plot]
  table[row sep=crcr]{%
0	0\\
0.0099999999999	-3.33333333330952e-08\\
0.0199999999968	-2.6666666663619e-07\\
0.0299999999757	-8.99999999479286e-07\\
0.0399999998976	-2.13333332943238e-06\\
0.0499999996875	-4.16666664806548e-06\\
0.0599999992224	-7.19999993334857e-06\\
0.0699999983193	-1.14333331372517e-05\\
0.0799999967232001	-1.70666661673448e-05\\
0.0899999940951002	-2.42999988611979e-05\\
0.0999999900000005	-3.3333330952381e-05\\
0.109999983894901	-4.43666620268643e-05\\
0.119999975116802	-5.75999914686177e-05\\
0.129999962870705	-7.32333183932116e-05\\
0.13999994621761	-9.14666415682164e-05\\
0.149999924062518	-0.000112499959319203\\
0.159999895142432	-0.000136533269420143\\
0.169999858014355	-0.000163766568967009\\
0.179999811043292	-0.000194399854233374\\
0.189999752390249	-0.000228633120506817\\
0.199999680000237	-0.000266666361904917\\
0.209999591590268	-0.00030869957116966\\
0.219999484637359	-0.000354932739439041\\
0.229999356366534	-0.000405565855994663\\
0.239999203738823	-0.000460798907984147\\
0.249999023439266	-0.000520831880117134\\
0.259998811864914	-0.000585864754333691\\
0.26999856511283	-0.000656097509443925\\
0.279998278968097	-0.00073173012073759\\
0.289997948891816	-0.000812962559562502\\
0.299997570009112	-0.000899994792870563\\
0.309997137097141	-0.000993026782730174\\
0.319996644573089	-0.00109225848580387\\
0.329996086482187	-0.00119788985278996\\
0.33999545648571	-0.00131012082782695\\
0.349994747848989	-0.0014291513478596\\
0.359993953429418	-0.0015551813419654\\
0.369993065664467	-0.00168841073064023\\
0.379992076559689	-0.00182903942504202\\
0.389990977676733	-0.0019772673261913\\
0.399989760121362	-0.00213329432412727\\
0.409988414531465	-0.00229732029701837\\
0.419986931065072	-0.00246954511022602\\
0.42998529938838	-0.00265016861532041\\
0.439983508663765	-0.00283939064904713\\
0.449981547537812	-0.0030374110322434\\
0.459979404129336	-0.00324442956870276\\
0.46997706601741	-0.00346064604398701\\
0.479974520229399	-0.00368626022418416\\
0.489971753228985	-0.00392147185461128\\
0.499968750904212	-0.00416648065846096\\
0.509965498555515	-0.0044214863353903\\
0.519961980883772	-0.00468668856005106\\
0.529958181978341	-0.00496228698056003\\
0.539954085305113	-0.00524848121690818\\
0.549949673694569	-0.00554547085930752\\
0.559944929329829	-0.00585345546647454\\
0.56993983373472	-0.00617263456384888\\
0.579934367761841	-0.00650320764174615\\
0.589928511580634	-0.00684537415344374\\
0.59992224466546	-0.00719933351319829\\
0.609915545783684	-0.00756528509419381\\
0.619908392983756	-0.00794342822641914\\
0.629900763583314	-0.00833396219447361\\
0.639892634157276	-0.0087370862352997\\
0.649883980525953	-0.00915299953584159\\
0.659874777743159	-0.00958190123062827\\
0.669865000084332	-0.0100239903992802\\
0.679854621034666	-0.0104794660639382\\
0.689843613277245	-0.0109485271866136\\
0.69983194868119	-0.0114313726664579\\
0.709819598289813	-0.0119282013369519\\
0.719806532308778	-0.012439211963012\\
0.729792720094278	-0.0129646032380125\\
0.739778130141214	-0.0135045737807243\\
0.749762730071391	-0.0140593221321665\\
0.75974648662172	-0.0146290467523714\\
0.769729365632437	-0.0152139460170618\\
0.779711332035326	-0.0158142182142379\\
0.789692349841963	-0.0164300615406744\\
0.799672382131964	-0.0170616740983251\\
0.809651391041255	-0.0177092538906361\\
0.819629337750348	-0.0183729988187629\\
0.829606182472637	-0.0190531066776944\\
0.839581884442705	-0.0197497751522788\\
0.849556401904651	-0.0204632018131534\\
0.859529692100426	-0.0211935841125752\\
0.869501711258192	-0.0219411193801515\\
0.879472414580694	-0.02270600481847\\
0.889441756233656	-0.023488437498627\\
0.899409689334183	-0.0242886143556521\\
0.909376165939199	-0.0251067321838288\\
0.919341137033889	-0.0259429876319097\\
0.92930455252017	-0.0267975771982251\\
0.939266361205181	-0.0276706972256841\\
0.949226510789796	-0.0285625438966671\\
0.959184947857156	-0.0294733132278078\\
0.969141617861228	-0.0304032010646653\\
0.979096465115384	-0.0313524030762829\\
0.989049432781012	-0.032321114749635\\
0.99900046285614	-0.0333095313839588\\
};
\addplot [color=green, line width=2.0pt, forget plot]
  table[row sep=crcr]{%
0.99900046285614	-0.0333095313839588\\
1.28107399012502	-0.0697411757808274\\
1.56061964966624	-0.122152660505431\\
1.83673291291116	-0.190374397214547\\
2.10852035752876	-0.274185640181141\\
2.37510255828077	-0.373315200563437\\
2.63561693258636	-0.487442323893249\\
2.88922053158886	-0.616197727944283\\
3.13509276769348	-0.759164797622177\\
3.3724380697504	-0.915880933009952\\
3.60048845729199	-1.085839046207\\
3.81850602549457	-1.26848920211825\\
4.02578533282409	-1.46324039788445\\
4.22165568364011	-1.66946247519573\\
4.40548329837206	-1.88648815930094\\
4.57667336424587	-2.11361521811491\\
4.73467195992542	-2.35010873443759\\
4.87896784784103	-2.59520348393258\\
5.00909412840582	-2.84810641117096\\
5.12462975076717	-3.10799919572826\\
5.22520087520498	-3.37404090003171\\
5.31048208276842	-3.64537069039003\\
5.3801974282372	-3.92111062240123\\
5.43412133300011	-4.20036848172571\\
5.47207931496193	-4.48224067103277\\
5.49394855311677	-4.76581513377915\\
5.49965828496124	-5.0501743053591\\
5.4891900354613	-5.33439808207698\\
5.46257767683215	-5.61756679833561\\
5.41990731893756	-5.89876420240705\\
5.36131703066334	-6.17708042115708\\
5.28699639316655	-6.45161490413042\\
5.19718588644594	-6.72147933747049\\
5.09217611121848	-6.98580051824503\\
4.97230684861986	-7.24372317987723\\
4.83796596077142	-7.49441275953993\\
4.68958813577096	-7.73705809855863\\
4.52765348116829	-7.97087406708534\\
4.35268597047662	-8.19510410455073\\
4.16525174774649	-8.40902266767447\\
3.96595729568801	-8.61193757811251\\
3.75544747326893	-8.80319226214524\\
3.53440342913834	-8.98216787515931\\
3.30354039762743	-9.14828530404912\\
3.06360538445891	-9.3010070410586\\
2.81537474965347	-9.43983892300024\\
2.55965169545419	-9.56433173022364\\
2.29726366739737	-9.67408264015983\\
2.02905967693905	-9.76873653073811\\
1.75590755430056	-9.84798712945796\\
1.4786911404219	-9.91157800439788\\
1.19830742710918	-9.95930339395464\\
0.915663654629652	-9.99100887262803\\
0.631674376145634	-10.006591850697\\
0.347258498486133	-10.00600190617\\
0.0633363088311863	-9.9892409479364\\
-0.219173503070174	-9.95636320958917\\
-0.499356817513137	-9.90747507394129\\
-0.776307042663017	-9.84273472880163\\
-1.04912804802311	-9.76235165512517\\
-1.31693706405276	-9.66658594919346\\
-1.5788675385532	-9.55574748101862\\
-1.8340719405788	-9.43019489169396\\
-2.08172450280107	-9.29033443293554\\
-2.32102389345193	-9.13661865256964\\
-2.55119580920055	-8.96954493021936\\
-2.77149548057412	-8.78965386792869\\
-2.98121008181551	-8.59752754093123\\
-3.17966103738041	-8.39378761422381\\
-3.36620621761076	-8.17909333103909\\
-3.54024201647978	-7.95413937972596\\
-3.70120530468581	-7.71965364593972\\
-3.84857525177518	-7.47639485741555\\
-3.9818750113983	-7.2251501289459\\
-4.10067326424594	-6.96673241550559\\
-4.20458561367327	-6.70197788176563\\
-4.29327582949579	-6.4317431965072\\
-4.3664569359325	-6.15690276069022\\
-4.42389214017623	-5.87834587814583\\
-4.46539559858638	-5.59697387804743\\
-4.49083301802497	-5.31369719847131\\
-4.50012209039024	-5.02943244048334\\
-4.49323275894174	-4.74509940228418\\
-4.47018731555525	-4.46161810300945\\
-4.4310603285927	-4.17990580581495\\
-4.37597840162064	-3.90087404987951\\
-4.30511976375786	-3.62542570092893\\
-4.21871369297771	-3.35445202982469\\
-4.11703977423119	-3.08882982867031\\
-4.00042699479123	-2.82941857376673\\
-3.86925267974538	-2.57705764459657\\
-3.72394127108152	-2.33256360783597\\
-3.56496295431671	-2.09672757518193\\
-3.39283213711354	-1.8703126435446\\
-3.20810578480622	-1.65405142588733\\
-3.0113816182226	-1.44864368070388\\
-2.80329617963311	-1.25475404780313\\
-2.58452277308483	-1.07300989772773\\
-2.35576928578519	-0.903999301765161\\
-2.11777589758448	-0.748269129119909\\
-1.87131268596889	-0.606323277403495\\
};
\addplot [color=red, line width=2.0pt, forget plot]
  table[row sep=crcr]{%
-1.87131268596889	-0.606323277403495\\
-1.86250395115063	-0.601589763074723\\
-1.8536858612179	-0.596873699292566\\
-1.84485854471502	-0.592174927998872\\
-1.83602212918422	-0.587493290515461\\
-1.82717674117394	-0.582828627553709\\
-1.81832250624725	-0.578180779224062\\
-1.80945954899028	-0.573549585045433\\
-1.80058799302073	-0.568934883954531\\
-1.79170796099651	-0.564336514315062\\
-1.7828195746243	-0.559754313926874\\
-1.77392295466834	-0.555188120034982\\
-1.76501822095915	-0.55063776933851\\
-1.75610549240238	-0.546103097999556\\
-1.74718488698767	-0.541583941651943\\
-1.73825652179764	-0.537080135409902\\
-1.72932051301684	-0.532591513876653\\
-1.72037697594087	-0.528117911152909\\
-1.71142602498543	-0.523659160845286\\
-1.70246777369552	-0.519215096074628\\
-1.69350233475463	-0.514785549484254\\
-1.68452981999405	-0.510370353248108\\
-1.67555034040217	-0.505969339078844\\
-1.66656400613383	-0.501582338235807\\
-1.65757092651978	-0.497209181532955\\
-1.64857121007609	-0.492849699346676\\
-1.63956496451375	-0.488503721623552\\
-1.63055229674813	-0.484171077888023\\
-1.62153331290866	-0.479851597249976\\
-1.61250811834844	-0.475545108412271\\
-1.60347681765395	-0.471251439678173\\
-1.59443951465478	-0.466970418958725\\
-1.58539631243342	-0.462701873780025\\
-1.57634731333504	-0.458445631290458\\
-1.5672926189774	-0.454201518267829\\
-1.55823233026071	-0.44996936112644\\
-1.5491665473776	-0.445748985924096\\
-1.54009536982303	-0.441540218369023\\
-1.53101889640438	-0.437342883826752\\
-1.52193722525146	-0.433156807326893\\
-1.51285045382658	-0.428981813569882\\
-1.50375867893469	-0.424817726933629\\
-1.4946619967335	-0.420664371480128\\
-1.48556050274373	-0.416521570961983\\
-1.47645429185923	-0.412389148828877\\
-1.46734345835732	-0.408266928233992\\
-1.45822809590901	-0.40415473204034\\
-1.44910829758934	-0.400052382827068\\
-1.4399841558877	-0.395959702895672\\
-1.43085576271821	-0.39187651427618\\
-1.42172320943012	-0.387802638733255\\
-1.41258658681821	-0.383737897772258\\
-1.40344598513327	-0.379682112645254\\
-1.39430149409257	-0.375635104356948\\
-1.38515320289034	-0.371596693670597\\
-1.37600120020832	-0.367566701113839\\
-1.36684557422633	-0.363544946984499\\
-1.35768641263277	-0.359531251356318\\
-1.34852380263534	-0.355525434084661\\
-1.33935783097152	-0.35152731481215\\
-1.33018858391933	-0.347536712974275\\
-1.32101614730795	-0.343553447804942\\
-1.31184060652837	-0.339577338341979\\
-1.30266204654417	-0.335608203432612\\
-1.29348055190216	-0.331645861738873\\
-1.2842962067432	-0.327690131743\\
-1.27510909481292	-0.323740831752757\\
-1.2659192994725	-0.319797779906755\\
-1.25672690370947	-0.315860794179705\\
-1.24753199014855	-0.311929692387642\\
-1.23833464106243	-0.308004292193123\\
-1.22913493838265	-0.304084411110377\\
-1.21993296371047	-0.300169866510426\\
-1.21072879832771	-0.296260475626173\\
-1.20152252320766	-0.292356055557462\\
-1.19231421902596	-0.288456423276094\\
-1.18310396617157	-0.284561395630833\\
-1.17389184475764	-0.280670789352369\\
-1.16467793463246	-0.276784421058251\\
-1.15546231539043	-0.272902107257811\\
-1.14624506638302	-0.269023664357044\\
-1.13702626672973	-0.265148908663475\\
-1.12780599532906	-0.261277656390994\\
-1.11858433086955	-0.257409723664683\\
-1.10936135184072	-0.253544926525603\\
-1.10013713654412	-0.24968308093558\\
-1.09091176310435	-0.245824002781959\\
-1.08168530948003	-0.241967507882342\\
-1.07245785347491	-0.238113411989322\\
-1.06322947274883	-0.234261530795177\\
-1.05400024482885	-0.230411679936581\\
-1.04477024712021	-0.22656367499927\\
-1.03553955691744	-0.222717331522721\\
-1.0263082514154	-0.218872465004807\\
-1.01707640772037	-0.215028890906441\\
-1.00784410286107	-0.211186424656222\\
-0.998611413799773	-0.207344881655058\\
-0.989378417443348	-0.203504077280801\\
-0.980145190654354	-0.199663826892853\\
-0.970911810262118	-0.195823945836794\\
-0.961678353073797	-0.191984249448981\\
};
\addplot [color=green, line width=2.0pt, forget plot]
  table[row sep=crcr]{%
-0.961678353073797	-0.191984249448981\\
-0.952444818367418	-0.188144556926445\\
-0.943211283661039	-0.184304864403908\\
-0.93397774895466	-0.180465171881372\\
-0.924744214248281	-0.176625479358835\\
-0.915510679541902	-0.172785786836299\\
-0.906277144835522	-0.168946094313762\\
-0.897043610129143	-0.165106401791226\\
-0.887810075422768	-0.161266709268691\\
-0.878576540716389	-0.157427016746154\\
-0.86934300601001	-0.153587324223618\\
-0.860109471303631	-0.149747631701081\\
-0.850875936597251	-0.145907939178545\\
-0.841642401890872	-0.142068246656009\\
-0.832408867184493	-0.138228554133472\\
-0.823175332478114	-0.134388861610936\\
-0.813941797771735	-0.130549169088399\\
-0.804708263065356	-0.126709476565863\\
-0.795474728358977	-0.122869784043326\\
-0.786241193652598	-0.11903009152079\\
-0.777007658946219	-0.115190398998253\\
-0.76777412423984	-0.111350706475717\\
-0.758540589533464	-0.107511013953182\\
-0.749307054827085	-0.103671321430645\\
-0.740073520120706	-0.099831628908109\\
-0.730839985414327	-0.0959919363855725\\
-0.721606450707948	-0.0921522438630361\\
-0.712372916001569	-0.0883125513404996\\
-0.70313938129519	-0.0844728588179632\\
-0.693905846588811	-0.0806331662954267\\
-0.684672311882432	-0.0767934737728902\\
-0.675438777176053	-0.0729537812503538\\
-0.666205242469674	-0.0691140887278173\\
-0.656971707763295	-0.0652743962052809\\
-0.647738173056916	-0.0614347036827444\\
-0.638504638350537	-0.057595011160208\\
-0.629271103644161	-0.0537553186376729\\
-0.620037568937782	-0.0499156261151364\\
-0.610804034231403	-0.0460759335926\\
-0.601570499525024	-0.0422362410700635\\
-0.592336964818645	-0.0383965485475271\\
-0.583103430112266	-0.0345568560249906\\
-0.573869895405887	-0.0307171635024542\\
-0.564636360699508	-0.0268774709799177\\
-0.555402825993129	-0.0230377784573813\\
-0.54616929128675	-0.0191980859348448\\
-0.536935756580371	-0.0153583934123083\\
-0.527702221873992	-0.0115187008897719\\
-0.518468687167612	-0.00767900836723544\\
-0.509235152461233	-0.00383931584469899\\
-0.500001617754858	3.76677836102584e-07\\
-0.490768083048479	0.00384006920037256\\
-0.4815345483421	0.00767976172290901\\
-0.47230101363572	0.0115194542454455\\
-0.463067478929341	0.0153591467679819\\
-0.453833944222962	0.0191988392905184\\
-0.444600409516583	0.0230385318130548\\
-0.435366874810204	0.0268782243355913\\
-0.426133340103825	0.0307179168581277\\
-0.416899805397446	0.0345576093806642\\
-0.407666270691067	0.0383973019032006\\
-0.398432735984688	0.0422369944257371\\
-0.389199201278309	0.0460766869482735\\
-0.37996566657193	0.04991637947081\\
-0.370732131865551	0.0537560719933465\\
-0.361498597159175	0.0575957645158815\\
-0.352265062452796	0.061435457038418\\
-0.343031527746417	0.0652751495609545\\
-0.333797993040038	0.0691148420834909\\
-0.324564458333659	0.0729545346060274\\
-0.31533092362728	0.0767942271285638\\
-0.306097388920901	0.0806339196511003\\
-0.296863854214522	0.0844736121736367\\
-0.287630319508143	0.0883133046961732\\
-0.278396784801764	0.0921529972187096\\
-0.269163250095385	0.0959926897412461\\
-0.259929715389006	0.0998323822637825\\
-0.250696180682627	0.103672074786319\\
-0.241462645976248	0.107511767308855\\
-0.232229111269872	0.111351459831391\\
-0.222995576563493	0.115191152353927\\
-0.213762041857114	0.119030844876463\\
-0.204528507150735	0.122870537399\\
-0.195294972444356	0.126710229921536\\
-0.186061437737977	0.130549922444073\\
-0.176827903031598	0.134389614966609\\
-0.167594368325219	0.138229307489146\\
-0.158360833618839	0.142069000011682\\
-0.14912729891246	0.145908692534219\\
-0.139893764206081	0.149748385056755\\
-0.130660229499702	0.153588077579292\\
-0.121426694793323	0.157427770101828\\
-0.112193160086944	0.161267462624364\\
-0.102959625380568	0.1651071551469\\
-0.0937260906741894	0.168946847669436\\
-0.0844925559678104	0.172786540191972\\
-0.0752590212614313	0.176626232714509\\
-0.0660254865550523	0.180465925237045\\
-0.0567919518486732	0.184305617759582\\
-0.0475584171422942	0.188145310282118\\
-0.0383248824359152	0.191985002804655\\
};
\addplot [color=red, line width=2.0pt, forget plot]
  table[row sep=crcr]{%
-0.0383248824359152	0.191985002804655\\
-0.029091425247594	0.195824699192467\\
-0.0198580448553583	0.199664580248527\\
-0.0106248180663638	0.203504830636475\\
-0.00139182170993877	0.207345635010732\\
0.00784086735136327	0.211187178011896\\
0.0170731722106604	0.215029644262115\\
0.0263050159056929	0.218873218360481\\
0.0355363214077274	0.222718084878396\\
0.0447670116104961	0.226564428354943\\
0.0539970093191399	0.230412433292255\\
0.0632262372391217	0.234262284150851\\
0.0724546179651937	0.238114165344995\\
0.0816820739703163	0.241968261238016\\
0.0909085275946352	0.245824756137633\\
0.100133901034414	0.249683834291255\\
0.109358116331007	0.253545679881276\\
0.118581095359837	0.257410477020356\\
0.127802759819348	0.261278409746667\\
0.137023031220014	0.265149662019148\\
0.146241830873308	0.269024417712718\\
0.155459079880719	0.272902860613485\\
0.164674699122748	0.276785174413926\\
0.173888609247928	0.280671542708043\\
0.183100730661864	0.284562148986508\\
0.19231098351625	0.288457176631768\\
0.201519287697944	0.292356808913136\\
0.210725562817997	0.296261228981847\\
0.219929728200761	0.3001706198661\\
0.229131702872942	0.304085164466052\\
0.238331405552714	0.308005045548797\\
0.247528754638837	0.311930445743317\\
0.25672366819976	0.315861547535379\\
0.265916063962788	0.31979853326243\\
0.275105859303209	0.323741585108431\\
0.284292971233492	0.327690885098674\\
0.29347731639245	0.331646615094549\\
0.302658811034454	0.335608956788285\\
0.311837371018662	0.339578091697654\\
0.321012911798236	0.343554201160615\\
0.330185348409623	0.34753746632995\\
0.339354595461806	0.351528068167824\\
0.348520567125624	0.355526187440334\\
0.357683177123066	0.359532004711993\\
0.366842338716615	0.363545700340173\\
0.375997964698614	0.367567454469514\\
0.385149967380627	0.37159744702627\\
0.394298258582858	0.375635857712623\\
0.40344274962356	0.379682866000927\\
0.412583351308498	0.383738651127934\\
0.421719973920408	0.38780339208893\\
0.4308525272085	0.391877267631854\\
0.43998092037799	0.395960456251347\\
0.449105062079628	0.400053136182741\\
0.4582248603993	0.404155485396015\\
0.467340222847607	0.408267681589665\\
0.476451056349516	0.412389902184551\\
0.485557267234017	0.416522324317658\\
0.494658761223793	0.420665124835803\\
0.503755443424975	0.424818480289303\\
0.512847218316875	0.428982566925557\\
0.521933989741754	0.433157560682568\\
0.531015660894672	0.437343637182426\\
0.540092134313314	0.441540971724697\\
0.549163311867886	0.445749739279771\\
0.558229094751004	0.449970114482116\\
0.567289383467689	0.454202271623502\\
0.576344077825333	0.458446384646134\\
0.585393076923709	0.4627026271357\\
0.594436279145073	0.466971172314399\\
0.603473582144238	0.471252193033847\\
0.612504882838732	0.475545861767947\\
0.621530077398951	0.479852350605651\\
0.630549061238421	0.484171831243697\\
0.639561729004044	0.488504474979229\\
0.648567974566385	0.492850452702352\\
0.657567691010064	0.497209934888628\\
0.666560770624117	0.501583091591481\\
0.67554710489246	0.505970092434519\\
0.684526584484341	0.510371106603783\\
0.693499099244916	0.514786302839928\\
0.70246453818581	0.519215849430305\\
0.711422789475726	0.523659914200962\\
0.720373740431165	0.528118664508584\\
0.729317277507132	0.532592267232327\\
0.738253286287928	0.537080888765578\\
0.747181651477959	0.541584695007618\\
0.756102256892667	0.54610385135523\\
0.765014985449448	0.550638522694187\\
0.773919719158636	0.555188873390657\\
0.782816339114592	0.559755067282549\\
0.791704725486795	0.564337267670736\\
0.800584757511025	0.568935637310207\\
0.809456313480566	0.573550338401109\\
0.818319270737539	0.578181532579735\\
0.827173505664235	0.582829380909386\\
0.836018893674511	0.587494043871137\\
0.844855309205313	0.592175681354548\\
0.853682625708191	0.59687445264824\\
0.862500715640927	0.601590516430401\\
0.871309450459186	0.606324030759171\\
};
\addplot [color=green, line width=2.0pt, forget plot]
  table[row sep=crcr]{%
0.871309450459186	0.606324030759171\\
1.11777268948622	0.748269899318639\\
1.35576610309427	0.904000091869693\\
1.5845196135408	1.07301071062244\\
1.80329304073323	1.25475488617501\\
2.01137849723725	1.44864454702446\\
2.20810267879554	1.65405232239585\\
2.39282904294527	1.87031357223229\\
2.56495986868511	2.09672853777769\\
2.72393819052725	2.33256460579328\\
2.86924960067637	2.57705867908193\\
3.00042391350425	2.82941964564953\\
3.11703668693422	3.0888309385143\\
3.21871059581284	3.35445317788093\\
3.30511665282469	3.62542688713049\\
3.37597527300013	3.90087527383776\\
3.43105717837117	4.17990706681751\\
3.47018413984862	4.46161940001991\\
3.49322955391976	4.74510073394349\\
3.50011885230063	5.02943380511288\\
3.49082974321743	5.31369859407775\\
3.46539228353621	5.59697530232918\\
3.42388878150756	5.87834732850123\\
3.36645353044096	6.15690423422736\\
3.29327237417056	6.43174469005532\\
3.2045821057183	6.70197939188814\\
3.10066970110038	6.96673393851449\\
2.98187139075593	7.22515166091778\\
2.84857157160275	7.4763963942089\\
2.70120156324022	7.71965518321359\\
2.54023821232404	7.95414091295985\\
2.36620234962856	8.17909485555398\\
2.17965710478934	8.39378912520428\\
1.98120608417868	8.59752903344856\\
1.77149141781036	8.78965533696483\\
1.55119168159285	8.96954637069162\\
1.32101970165431	9.13662005935616\\
1.08172024784368	9.29033580090118\\
0.834067623871253	9.43019621571671\\
0.578863161886137	9.55574875601632\\
0.316932629597567	9.66658717015073\\
0.0491235583298282	9.76235281712035\\
-0.223697499343607	9.84273582703381\\
-0.500647773513186	9.90747610375702\\
-0.78083113332093	9.95636416650901\\
-1.06334098658035	9.98924182768103\\
-1.34726321324609	10.0060027046858\\
-1.63167912324161	10.0065925641807\\
-1.91566842907411	9.99100949755137\\
-2.1983122236183	9.95930392708714\\
-2.47869595343363	9.91157844282931\\
-2.75591237799431	9.84798747061943\\
-3.02906450525679	9.768736772422\\
-3.29726849406624	9.6740827805384\\
-3.55965651401042	9.56433176786621\\
-3.81537955346754	9.43983885688884\\
-4.06361016676186	9.3010068706021\\
-4.30354515153828	9.14828502909558\\
-4.53440814769242	8.9821674960067\\
-4.75545214944701	8.80319177955021\\
-4.96596192244626	8.61193699329741\\
-5.16525631804694	8.40902198233232\\
-5.35269047731815	8.19510332084834\\
-5.52765791761796	7.97087318766431\\
-5.68959249499558	7.73705712653434\\
-5.83797023606917	7.4944116984985\\
-5.97231103345191	7.24372203387036\\
-6.09218019924042	6.98579929178279\\
-6.19718987153891	6.72147803551201\\
-6.28700026946789	6.45161353207252\\
-6.36132079259667	6.17707898482088\\
-6.41991096124221	5.89876270802255\\
-6.46258119459165	5.61756525252426\\
-6.48919342413094	5.33439649183243\\
-6.49966154039453	5.05017267802597\\
-6.4939516715907	4.76581347703004\\
-6.47208229320081	4.48223899284353\\
-6.43412416819803	4.20036679034908\\
-6.38020011807883	3.92110892633888\\
-6.31048462544824	3.64536899836311\\
-6.22520326944466	3.37403922095011\\
-6.1246319958311	3.1079975386585\\
-6.00909622411465	2.84810478530304\\
-5.87896979458323	2.59520189854568\\
-5.73467375866669	2.35010719886498\\
-5.5766750165364	2.11361374170803\\
-5.4054848063515	1.88648675139281\\
-5.22165705004035	1.66946114506395\\
-5.02578656096963	1.46323915471376\\
-4.81850711930057	1.26848805496291\\
-4.60048942126	1.08583800395308\\
-4.37243890896161	0.915880004337761\\
-4.13509348779962	0.759163990968867\\
-3.88922113880073	0.616197051466866\\
-3.63561743365996	0.487441785432138\\
-3.37510296050112	0.373314807606681\\
-3.10852066869136	0.274185399829483\\
-2.83673314130121	0.190374316147488\\
-2.56061980403565	0.122152744948402\\
-2.28107407966726	0.0697414314736332\\
-1.99900049717896	0.0333099635506408\\
};
\addplot [color=red, line width=2.0pt, forget plot]
  table[row sep=crcr]{%
-1.99900049717896	0.0333099635506408\\
-1.98904946646784	0.0323215533193271\\
-1.97909649817889	0.0313528480502308\\
-1.96914165031397	0.0304036524440793\\
-1.95918497971157	0.0294737710138607\\
-1.94922654205817	0.0285630080904961\\
-1.9392663918997	0.0276711678283905\\
-1.92930458265288	0.0267980542108766\\
-1.91934116661671	0.0259434710555407\\
-1.90937619498393	0.0251072220194393\\
-1.8994097178525	0.0242891106042117\\
-1.8894417842371	0.0234889401610726\\
-1.87947244208069	0.0227065138957079\\
-1.86950173826601	0.0219416348730575\\
-1.85952971862725	0.0211941060219967\\
-1.84955642796151	0.0204637301399081\\
-1.83958191004051	0.0197503098971562\\
-1.82960620762217	0.0190536478414563\\
-1.81962936246227	0.0183735464021451\\
-1.80965141532609	0.0177098078943481\\
-1.79967240600011	0.0170622345230493\\
-1.7896923733037	0.0164306283870698\\
-1.7797113551008	0.0158147914829388\\
-1.76972938831167	0.0152145257086787\\
-1.7597465089246	0.0146296328674909\\
-1.74976275200768	0.0140599146713536\\
-1.73977815172056	0.0135051727445218\\
-1.72979274132617	0.0129652086269413\\
-1.71980655320261	0.0124398237775728\\
-1.70981961885484	0.011928819577625\\
-1.69983196892654	0.0114319973337036\\
-1.6898436332119	0.0109491582808728\\
-1.67985464066751	0.010480103585634\\
-1.6698650194241	0.0100246343488172\\
-1.65987479679846	0.00958255160839297\\
-1.64988399930528	0.00915365634220425\\
-1.63989265266899	0.00873774947061368\\
-1.62990078183564	0.00833463185907619\\
-1.61990841098479	0.00794410432063179\\
-1.6099155635414	0.00756596761832346\\
-1.59992226218771	0.00720002246753704\\
-1.58992852887513	0.0068460695382695\\
-1.57993438483617	0.00650390945732298\\
-1.56993985059634	0.00617334281042809\\
-1.55994494598608	0.00585417014429445\\
-1.54994969015265	0.00554619196859391\\
-1.53995410157209	0.00524920875787537\\
-1.52995819806117	0.00496302095341067\\
-1.51996199678927	0.00468742896497674\\
-1.50996551429037	0.00442223317257177\\
-1.49996876647499	0.00416723392806916\\
-1.48997176864211	0.00392223155680708\\
-1.47997453549118	0.00368702635911879\\
-1.46997708113402	0.00346141861180297\\
-1.45997941910682	0.00324520856953369\\
-1.4499815623821	0.00303819646621456\\
-1.43998352338063	0.00284018251627568\\
-1.4299853139835	0.00265096691591614\\
-1.41998694554397	0.00247034984429129\\
-1.40998842889954	0.00229813146464837\\
-1.3999897743839	0.00213411192541084\\
-1.38999099183888	0.00197809136121084\\
-1.37999209062646	0.00182986989387391\\
-1.36999307964075	0.00168924763335497\\
-1.35999396731998	0.00155602467862829\\
-1.34999476165846	0.00143000111853073\\
-1.33999547021859	0.0013109770325615\\
-1.32999610014285	0.00119875249163843\\
-1.31999665816579	0.00109312755881263\\
-1.30999715062599	0.000993902289941395\\
-1.2999975834781	0.000900876734322534\\
-1.28999796230481	0.000813850935290047\\
-1.27999829232882	0.000732624930772135\\
-1.26999857842489	0.000656998753813726\\
-1.25999882513178	0.000586772433064021\\
-1.24999903666429	0.000521745993230698\\
-1.23999921692523	0.000461719455501083\\
-1.22999936951739	0.000406492837932722\\
-1.21999949775564	0.000355866155813918\\
-1.20999960467881	0.000309639421995075\\
-1.19999969306173	0.00026761264719283\\
-1.18999976542727	0.000229585840267476\\
-1.17999982405829	0.000195359008475629\\
-1.16999987100965	0.000164732157698412\\
-1.1599999081202	0.000137505292646998\\
-1.14999993702482	0.000113478417046796\\
-1.13999995916638	9.24515338008909e-05\\
-1.12999997580775	7.42246451345496e-05\\
-1.11999998804379	5.85977527213973e-05\\
-1.10999999681337	4.53708577933397e-05\\
-1.10000000291138	3.43439612342752e-05\\
-1.09000000700067	2.53170636598184e-05\\
-1.08000000962411	1.80901654835962e-05\\
-1.07000001121659	1.24632669718376e-05\\
-1.06000001211697	8.2363682866842e-06\\
-1.05000001258012	5.20946952047082e-06\\
-1.04000001278891	3.18257072102824e-06\\
-1.03000001286622	1.95567191036559e-06\\
-1.02000001288692	1.32877309686248e-06\\
-1.01000001288987	1.10187428293394e-06\\
-1.00000001288995	1.07497546891072e-06\\
};
\addplot [color=red, line width=2.0pt, only marks, mark size=2.5pt, mark=*, mark options={solid, fill=red, red}, forget plot]
  table[row sep=crcr]{%
0.99900046285614	-0.0333095313839588\\
};
\addplot [color=red, line width=2.0pt, only marks, mark size=2.5pt, mark=*, mark options={solid, fill=red, red}, forget plot]
  table[row sep=crcr]{%
-1.87131268596889	-0.606323277403495\\
};
\addplot [color=red, line width=2.0pt, only marks, mark size=2.5pt, mark=*, mark options={solid, fill=red, red}, forget plot]
  table[row sep=crcr]{%
-0.961678353073797	-0.191984249448981\\
};
\addplot [color=red, line width=2.0pt, only marks, mark size=2.5pt, mark=*, mark options={solid, fill=red, red}, forget plot]
  table[row sep=crcr]{%
-0.0383248824359152	0.191985002804655\\
};
\addplot [color=red, line width=2.0pt, only marks, mark size=2.5pt, mark=*, mark options={solid, fill=red, red}, forget plot]
  table[row sep=crcr]{%
0.871309450459182	0.606324030759169\\
};
\addplot [color=red, line width=2.0pt, only marks, mark size=2.5pt, mark=*, mark options={solid, fill=red, red}, forget plot]
  table[row sep=crcr]{%
-1.99900049717896	0.0333099635506408\\
};
\addplot [color=red, line width=2.0pt, only marks, mark size=2.5pt, mark=*, mark options={solid, fill=red, red}, forget plot]
  table[row sep=crcr]{%
-1.00000001288995	1.07497546893116e-06\\
};
\addplot [color=blue, line width=2.0pt, only marks, mark size=2.5pt, mark=*, mark options={solid, fill=blue, blue}, forget plot]
  table[row sep=crcr]{%
0	0\\
};
\addplot [color=blue, line width=2.0pt, only marks, mark size=2.5pt, mark=*, mark options={solid, fill=blue, blue}, forget plot]
  table[row sep=crcr]{%
-1	0\\
};
\end{axis}
\end{tikzpicture}%%
  \caption{Duboids solution example 3}
  \label{fig:DuboidsRes2}
\end{figure}
%
The fourth example (Figure \ref{fig:DuboidsRes3}) shows a connection from $x=0$, $y=0$, $\theta=0$ and $\kappa=0$ to $x=0$, $y=0$, $\theta=\pi$ and $\kappa=0$ obtaining a water drop like shape.\\
%
\begin{figure}[htb!]
  \centering
  % This file was created by matlab2tikz.
%
%The latest updates can be retrieved from
%  http://www.mathworks.com/matlabcentral/fileexchange/22022-matlab2tikz-matlab2tikz
%where you can also make suggestions and rate matlab2tikz.
%
\begin{tikzpicture}[scale = 0.7]

\begin{axis}[%
width=\linewidth,
height=0.776\linewidth,
at={(0\linewidth,0\linewidth)},
scale only axis,
xmin=0,
xmax=13.8717811499801,
xlabel style={font=\color{white!15!black}},
xlabel={x(m)},
ymin=-5.47040528366376,
ymax=5.47040275236895,
ylabel style={font=\color{white!15!black}},
ylabel={y(m)},
axis background/.style={fill=white},
title style={font=\bfseries},
title={$L_{tot}$ = 37.0618, $k_{max}$ = 0.2, $J_{max}$ = 0.5, Type = [LRL]},
axis x line*=bottom,
axis y line*=left,
xmajorgrids,
xminorgrids,
ymajorgrids,
yminorgrids
]
\addplot [color=red, line width=2.0pt, forget plot]
  table[row sep=crcr]{%
0	0\\
0.0039999999999936	5.33333333332724e-09\\
0.0079999999997952	4.26666666658865e-08\\
0.0119999999984448	1.4399999998667e-07\\
0.0159999999934464	3.41333333233469e-07\\
0.01999999998	6.66666666190477e-07\\
0.0239999999502336	1.15199999829372e-06\\
0.0279999998924352	1.82933332831364e-06\\
0.0319999997902848	2.73066665388403e-06\\
0.0359999996220864	3.88799997084666e-06\\
0.03999999936	5.33333327238096e-06\\
0.0439999989692736	7.09866654788772e-06\\
0.0479999984074752	9.2159997815966e-06\\
0.0519999976237249	1.17173329508662e-05\\
0.0559999965579265	1.46346660241463e-05\\
0.0599999951400002	1.79999989585715e-05\\
0.0639999932891139	2.18453316971554e-05\\
0.0679999909129158	2.62026641655549e-05\\
0.0719999879067657	3.11039962683733e-05\\
0.0759999841529679	3.65813278849726e-05\\
0.0799999795200024	4.26666588647626e-05\\
0.0839999738617574	4.93919890219376e-05\\
0.0879999670167609	5.67893181296299e-05\\
0.0919999588074134	6.48906459134479e-05\\
0.0959999490392189	7.37279720443694e-05\\
0.0999999375000181	8.33332961309598e-05\\
0.103999923959219	9.37386177108827e-05\\
0.107999908167031	0.000104975936241674\\
0.111999889853695	0.000117077251090747\\
0.115999868728715	0.000130074561524601\\
0.119999844480093	0.000143999866697198\\
0.123999816773559	0.000158885165637479\\
0.127999785251802	0.000174762457235992\\
0.131999749533705	0.0001916637402306\\
0.135999709213574	0.000209621013191226\\
0.139999663860374	0.000228666274503633\\
0.143999613016955	0.00024883152235217\\
0.147999556199291	0.000270148754701491\\
0.151999492895708	0.000292649969277189\\
0.155999422566116	0.000316367163545326\\
0.159999344641243	0.000341332334690826\\
0.163999258521866	0.000367577479594705\\
0.167999163578043	0.0003951345948101\\
0.171999059148347	0.000424035676537064\\
0.175998944539097	0.000454312720596112\\
0.179998819023587	0.000485997722400469\\
0.183998681841325	0.000519122676927002\\
0.187998532197261	0.000553719578685794\\
0.191998369261017	0.000589820421688346\\
0.195998192166126	0.000627457199414355\\
0.199998000009259	0.000666661904777056\\
0.203997791849459	0.000707466530087088\\
0.207997566707374	0.000749903067014852\\
0.211997323564488	0.000794003506551339\\
0.215997061362356	0.000839799838967381\\
0.219996779001833	0.000887324053771324\\
0.22399647534231	0.000936608139665051\\
0.227996149200946	0.000987684084498361\\
0.231995799351897	0.00104058387522166\\
0.235995424525555	0.0010953394978369\\
0.239995023407776	0.00115198293734686\\
0.243994594639112	0.0012105461777025\\
0.247994136814051	0.00127106120174863\\
0.25199364848024	0.0013335599911677\\
0.255993128137727	0.00139807452642166\\
0.259992574238189	0.00146463678669201\\
0.263991985184165	0.00153327874981787\\
0.267991359328293	0.00160403239223204\\
0.27199069497254	0.00167692968889515\\
0.275989990367434	0.00175200261322773\\
0.279989243711304	0.00182928313704023\\
0.283988453149507	0.00190880323046097\\
0.287987616773665	0.00199059486186195\\
0.291986732620897	0.00207468999778252\\
0.295985798673054	0.0021611206028509\\
0.299984812855953	0.00224991863970341\\
0.303983773038611	0.00234111606890158\\
0.307982677032479	0.00243474484884685\\
0.311981522590676	0.00253083693569309\\
0.315980307407224	0.00262942428325673\\
0.319979029116282	0.0027305388429245\\
0.323977685291382	0.00283421256355887\\
0.327976273444663	0.00294047739140096\\
0.331974791026105	0.00304936526997105\\
0.335973235422765	0.00316090813996663\\
0.339971603958015	0.00327513793915787\\
0.343969893890772	0.00339208660228061\\
0.347968102414737	0.00351178606092669\\
0.351966226657634	0.00363426824343178\\
0.355964263680437	0.0037595650747605\\
0.359962210476616	0.00388770847638889\\
0.363960063971367	0.00401873036618416\\
0.367957821020851	0.00415266265828177\\
0.371955478411431	0.00428953726295971\\
0.375953032858908	0.00442938608650997\\
0.379950481007759	0.00457224103110726\\
0.383947819430375	0.00471813399467477\\
0.387945044626296	0.00486709687074723\\
0.391942153021452	0.00501916154833079\\
0.395939140967401	0.00517435991176029\\
0.399936004740566	0.00533272384055325\\
};
\addplot [color=green, line width=2.0pt, forget plot]
  table[row sep=crcr]{%
0.399936004740566	0.00533272384055325\\
0.44627425662624	0.00740256711858648\\
0.492591313055718	0.0099021916352755\\
0.538883187950072	0.0128313822711515\\
0.585145897397518	0.0161898869379777\\
0.631375459996268	0.0199774166004455\\
0.677567897197179	0.0241936453010502\\
0.723719233646143	0.0288382101881415\\
0.769825497526213	0.0339107115471506\\
0.815882720899422	0.0394107128349907\\
0.861886940048259	0.0453377407176271\\
0.9078341958168	0.0516912851108105\\
0.953720533951426	0.0584707992239771\\
0.999542005441135	0.0656756996073041\\
1.04529466685739	0.0733053662019235\\
1.09097458069351	0.0813591423932847\\
1.13657781570351	0.0898363350676616\\
1.18210044724045	0.098736214671803\\
1.22753855759417	0.10805801527572\\
1.27288823632847	0.1178009346386\\
1.31814558061764	0.127964134277853\\
1.36330669558234	0.138546739541263\\
1.40836769462478	0.149547839682272\\
1.45332469976325	0.16096648793835\\
1.49817384196579	0.172801701612482\\
1.54291126148322	0.185052462157733\\
1.5875331081813	0.197717715264909\\
1.63203554187206	0.21079637095329\\
1.67641473264431	0.224287303664435\\
1.72066686119321	0.238189352359049\\
1.764788119149	0.252501320616901\\
1.80877470940476	0.26722197673979\\
1.85262284644312	0.282350053857546\\
1.89632875666213	0.297884250037057\\
1.93988867869997	0.313823228394317\\
1.98329886375867	0.330165617209474\\
2.02655557592671	0.346910010044884\\
2.0696550925006	0.364054965866153\\
2.11259370430517	0.38159900916615\\
2.15536771601286	0.399540630091988\\
2.19797344646169	0.417878284574967\\
2.24040722897211	0.43661039446346\\
2.2826654116625	0.455735347658721\\
2.32474435776353	0.475251498253633\\
2.36664044593104	0.495157166674348\\
2.40835007055781	0.515450639824839\\
2.44986964208378	0.536130171234327\\
2.49119558730499	0.557193981207581\\
2.53232434968111	0.578640256978085\\
2.5732523896415	0.600467152864042\\
2.61397618488984	0.622672790427219\\
2.65449223070725	0.645255258634602\\
2.69479704025392	0.668212614022863\\
2.73488714486917	0.691542880865615\\
2.77475909437	0.715244051343446\\
2.81440945734798	0.739314085716713\\
2.85383482146458	0.763750912501082\\
2.89303179374483	0.788552428645804\\
2.93199700086934	0.8137164997147\\
2.97072708946456	0.839240960069859\\
3.00921872639144	0.86512361305801\\
3.04746859903224	0.891362231199568\\
3.08547341557562	0.917954556380332\\
3.12322990529996	0.944898300045822\\
3.16073481885481	0.972191143398233\\
3.19798492854054	0.999830737595989\\
3.23497702858614	1.02781470395589\\
3.27170793542508	1.05614063415783\\
3.30817448796931	1.08480609045204\\
3.34437354788129	1.11380860586889\\
3.3803019998441	1.14314568443121\\
3.41595675182952	1.1728148013691\\
3.45133473536416	1.20281340333716\\
3.48643290579351	1.23313890863433\\
3.52124824254398	1.26378870742595\\
3.55577774938285	1.2947601619685\\
3.59001845467611	1.32605060683647\\
3.62396741164423	1.35765734915187\\
3.65762169861577	1.3895776688159\\
3.69097841927875	1.42180881874307\\
3.72403470293001	1.45434802509761\\
3.75678770472218	1.48719248753221\\
3.78923460590856	1.52033937942899\\
3.82137261408568	1.55378584814279\\
3.85319896343362	1.58752901524665\\
3.88471091495405	1.62156597677952\\
3.91590575670593	1.6558938034962\\
3.94678080403893	1.69050954111944\\
3.97733339982444	1.72541021059416\\
4.00756091468426	1.76059280834383\\
4.03746074721689	1.79605430652899\\
4.06703032422142	1.8317916533078\\
4.09626710091895	1.8678017730987\\
4.12516856117161	1.90408156684508\\
4.15373221769911	1.940627912282\\
4.18195561229278	1.9774376642049\\
4.20983631602716	2.01450765474027\\
4.23737192946898	2.05183469361825\\
4.26456008288369	2.08941556844726\\
4.29139843643942	2.12724704499039\\
4.31788468040832	2.16532586744377\\
};
\addplot [color=red, line width=2.0pt, forget plot]
  table[row sep=crcr]{%
4.31788468040832	2.16532586744377\\
4.32241705455914	2.17191810311729\\
4.32693908653588	2.17851743747424\\
4.3314509766231	2.18512370991175\\
4.3359529257954	2.19173676078417\\
4.34044513570058	2.19835643138484\\
4.34492780864309	2.2049825639277\\
4.34940114756749	2.21161500152886\\
4.35386535604215	2.21825358818794\\
4.35832063824307	2.22489816876946\\
4.36276719893779	2.23154858898401\\
4.36720524346948	2.2382046953694\\
4.37163497774116	2.24486633527164\\
4.37605660820004	2.25153335682591\\
4.38047034182193	2.25820560893738\\
4.38487638609588	2.26488294126199\\
4.38927494900885	2.27156520418708\\
4.39366623903056	2.27825224881205\\
4.39805046509835	2.28494392692877\\
4.40242783660231	2.29164009100212\\
4.40679856337035	2.29834059415029\\
4.4111628556535	2.30504529012508\\
4.41552092411123	2.31175403329215\\
4.41987297979693	2.31846667861113\\
4.42421923414341	2.32518308161574\\
4.42855989894857	2.33190309839384\\
4.43289518636112	2.33862658556735\\
4.43722530886637	2.3453534002722\\
4.44155047927213	2.35208340013819\\
4.44587091069467	2.35881644326879\\
4.45018681654478	2.3655523882209\\
4.45449841051388	2.37229109398453\\
4.45880590656019	2.37903241996254\\
4.463109518895	2.38577622595019\\
4.46740946196898	2.39252237211475\\
4.47170595045853	2.39927071897505\\
4.47599919925226	2.40602112738098\\
4.48028942343741	2.41277345849296\\
4.48457683828643	2.41952757376137\\
4.48886165924353	2.42628333490599\\
4.49314410191133	2.43304060389537\\
4.49742438203748	2.43979924292617\\
4.50170271550139	2.44655911440253\\
4.50597931830095	2.45332008091536\\
4.51025440653929	2.46008200522162\\
4.51452819641155	2.46684475022364\\
4.51880090419174	2.47360817894837\\
4.52307274621951	2.48037215452659\\
4.52734393888704	2.48713654017221\\
4.53161469862586	2.49390119916146\\
4.53588524189378	2.50066599481214\\
4.54015578516169	2.50743079046281\\
4.54442654490052	2.51419544945206\\
4.54869773756804	2.52095983509768\\
4.55296957959581	2.52772381067591\\
4.557242287376	2.53448723940063\\
4.56151607724826	2.54124998440266\\
4.5657911654866	2.54801190870892\\
4.57006776828616	2.55477287522174\\
4.57434610175007	2.5615327466981\\
4.57862638187622	2.56829138572891\\
4.58290882454402	2.57504865471828\\
4.58719364550112	2.58180441586291\\
4.59148106035014	2.58855853113132\\
4.59577128453529	2.59531086224329\\
4.60006453332902	2.60206127064922\\
4.60436102181857	2.60880961750952\\
4.60866096489255	2.61555576367408\\
4.61296457722736	2.62229956966173\\
4.61727207327367	2.62904089563974\\
4.62158366724277	2.63577960140338\\
4.62589957309288	2.64251554635548\\
4.63022000451542	2.64924858948608\\
4.63454517492118	2.65597858935207\\
4.63887529742643	2.66270540405693\\
4.64321058483898	2.66942889123043\\
4.64755124964414	2.67614890800853\\
4.65189750399062	2.68286531101315\\
4.65624955967632	2.68957795633212\\
4.66060762813405	2.69628669949919\\
4.6649719204172	2.70299139547398\\
4.66934264718524	2.70969189862215\\
4.6737200186892	2.7163880626955\\
4.67810424475699	2.72307974081223\\
4.6824955347787	2.72976678543719\\
4.68689409769167	2.73644904836229\\
4.69130014196563	2.74312638068689\\
4.69571387558751	2.74979863279837\\
4.70013550604639	2.75646565435264\\
4.70456524031807	2.76312729425487\\
4.70900328484977	2.76978340064026\\
4.71344984554448	2.77643382085481\\
4.7179051277454	2.78307840143634\\
4.72236933622006	2.78971698809542\\
4.72684267514446	2.79634942569657\\
4.73132534808697	2.80297555823943\\
4.73581755799215	2.8095952288401\\
4.74031950716445	2.81620827971252\\
4.74483139725167	2.82281455215003\\
4.74935342922841	2.82941388650698\\
4.75388580337923	2.8360061221805\\
};
\addplot [color=green, line width=2.0pt, forget plot]
  table[row sep=crcr]{%
4.75388580337923	2.8360061221805\\
4.90311336577669	3.04132673495169\\
5.06256825163365	3.23880979443824\\
5.2318395430342	3.4279463834828\\
5.41049102502064	3.60824909392241\\
5.59806230972679	3.77925328264745\\
5.79407002280521	3.94051826899474\\
5.998009049091	4.09162847038904\\
6.20935383429209	4.23219447330628\\
6.42755973935141	4.36185403679877\\
6.65206444399095	4.48027302599619\\
6.88228939582049	4.58714627317672\\
7.11764130127683	4.68219836418945\\
7.35751365455127	4.76518434820122\\
7.60128830056527	4.83589036893902\\
7.84833702796639	4.89413421580124\\
8.09802318803956	4.93976579341739\\
8.34970333536146	4.97266750844642\\
8.60272888597017	4.99275457261673\\
8.85644778877693	4.99997522122697\\
9.11020620591276	4.99431084654454\\
9.36335019767956	4.97577604575802\\
9.61522740776371	4.94441858335999\\
9.86518874436927	4.90031926805709\\
10.1125900529384	4.8435917445247\\
10.3567937761484	4.77438220054264\\
10.5971705969077	4.69286899026683\\
10.8331010601166	4.59926217460763\\
11.0639771690133	4.49380297989939\\
11.2892039519921	4.37676317625608\\
11.5082009958546	4.24844437721518\\
11.7204039415449	4.10917726247455\\
11.925265938512	3.95932072572525\\
12.1222590539528	3.79926094977653\\
12.310875633304	3.62941041135622\\
12.4906296084764	3.4502068181513\\
12.6610577504602	3.26211198082783\\
12.8217208630746	3.06561062293714\\
12.9722049147829	2.86120913177506\\
13.1121221056593	2.64943425341344\\
13.2411118667556	2.43083173526662\\
13.3588417892936	2.20596491969119\\
13.4650084812877	1.97541329224334\\
13.5593383493916	1.7397709883348\\
13.6415883039527	1.49964526213596\\
13.7115463854582	1.25565492167165\\
13.7690323107585	1.00842873414251\\
13.8138979376596	0.758603805581344\\
13.8460276466884	0.506823939020199\\
13.8653386390457	0.253737975399139\\
13.8717811499801	-1.87850773743139e-06\\
13.865338577033	-0.253741730840089\\
13.8460275228229	-0.50682768974164\\
13.8138977522605	-0.75860754845045\\
13.7690320643035	-1.00843246604669\\
13.7115460785824	-1.25565863952657\\
13.641587937447	-1.49964896289349\\
13.5593379242005	-1.73977466899088\\
13.4650079985068	-1.9754169498457\\
13.3588412501672	-2.20596855134697\\
13.241111272673	-2.43083533814983\\
13.1121214581514	-2.64943782477225\\
12.9722042155183	-2.86121266893885\\
12.8217201138554	-3.06561412332343\\
12.6610569532172	-3.26211544194891\\
12.4906287652639	-3.45021023762065\\
12.3108747462952	-3.62941378689466\\
12.1222581254334	-3.79926427921809\\
11.9252649708749	-3.95932400702274\\
11.7204029372837	-4.10918049370487\\
11.5081999575573	-4.24844755658424\\
11.2892028823343	-4.37676630210343\\
11.0639760707517	-4.49380605070252\\
10.8330999360812	-4.59926518898588\\
10.5971694499953	-4.69287194698494\\
10.3567926093147	-4.77438509851395\\
10.1125888691902	-4.84359458281393\\
9.86518754675724	-4.90032204588277\\
9.61522619937405	-4.94442130009645\\
9.36334898162631	-4.97577870093705\\
9.11020498532971	-4.99431343985654\\
8.85644656680955	-4.99997775252178\\
8.60272766576748	-4.992757041904\\
8.34970212006795	-4.9726699158956\\
8.09802198078706	-4.93976813935729\\
7.84833583186601	-4.89413650071919\\
7.60128711869938	-4.8358925934796\\
7.35751248996557	-4.76518651316458\\
7.11764015697245	-4.6822004705293\\
6.88228827474634	-4.58714832199783\\
6.65206334903606	-4.48027501855156\\
6.42755867333749	-4.36185597448641\\
6.20935279996627	-4.23219635766557\\
5.99800804911877	-4.0916303030968\\
5.79406905976351	-3.9405200518609\\
5.59806138609739	-3.77925501761039\\
5.41049014318375	-3.60825078304395\\
5.23183870526232	-3.42794802894288\\
5.06256746008575	-3.23881139852932\\
4.90311262249259	-3.04132830007286\\
4.75388511027439	-2.83600765083127\\
};
\addplot [color=red, line width=2.0pt, forget plot]
  table[row sep=crcr]{%
4.75388511027439	-2.83600765083127\\
4.74935273773467	-2.82941541405006\\
4.74483070737078	-2.82281607858794\\
4.74031881889809	-2.81620980504775\\
4.73581687134199	-2.80959675307508\\
4.73132466305462	-2.80297708137655\\
4.7268419917315	-2.79635094773814\\
4.72236865442804	-2.78971850904373\\
4.71790444757581	-2.78307992129362\\
4.71344916699879	-2.77643533962325\\
4.7090026079294	-2.76978491832198\\
4.70456456502443	-2.76312881085196\\
4.70013483238081	-2.75646716986712\\
4.69571320355132	-2.74980014723223\\
4.69129947156009	-2.74312789404206\\
4.68689342891805	-2.73645056064064\\
4.68249486763818	-2.72976829664056\\
4.67810357925076	-2.72308125094239\\
4.67371935481837	-2.71638957175418\\
4.66934198495092	-2.70969340661103\\
4.66497125982045	-2.70299290239468\\
4.66060696917589	-2.69628820535327\\
4.65624890235774	-2.68957946112112\\
4.65189684831258	-2.68286681473852\\
4.64755059560755	-2.67615041067171\\
4.64320993244472	-2.66943039283278\\
4.63887464667536	-2.66270690459975\\
4.63454452581411	-2.65598008883664\\
4.63021935705312	-2.64925008791359\\
4.6258989272761	-2.6425170437271\\
4.62158302307222	-2.63578109772022\\
4.61727143075002	-2.62904239090285\\
4.61296393635125	-2.62230106387211\\
4.60866032566459	-2.61555725683269\\
4.60436038423934	-2.60881110961724\\
4.60006389739905	-2.6020627617069\\
4.59577065025508	-2.59531235225173\\
4.59148042772017	-2.58856002009124\\
4.58719301452182	-2.58180590377501\\
4.58290819521578	-2.5750501415832\\
4.57862575419942	-2.56829287154722\\
4.57434547572505	-2.56153423147034\\
4.57006714391321	-2.55477435894837\\
4.565790542766	-2.54801339139037\\
4.56151545618024	-2.5412514660393\\
4.55724166796076	-2.53448871999279\\
4.55296896183352	-2.52772529022384\\
4.54869712145882	-2.5209613136016\\
4.54442593044448	-2.51419692691212\\
4.5401551723589	-2.50743226687912\\
4.53588463074427	-2.50066747018474\\
4.53161408912963	-2.49390267349037\\
4.52734333104405	-2.48713801345737\\
4.52307214002971	-2.4803736267679\\
4.51880029965502	-2.47360965014566\\
4.51452759352777	-2.46684622037671\\
4.51025380530829	-2.46008347433019\\
4.50597871872253	-2.45332154897912\\
4.50170211757532	-2.44656058142112\\
4.49742378576348	-2.43980070889915\\
4.49314350728911	-2.43304206882227\\
4.48886106627275	-2.42628479878629\\
4.48457624696672	-2.41952903659448\\
4.48028883376836	-2.41277492027825\\
4.47599861123345	-2.40602258811777\\
4.47170536408949	-2.39927217866259\\
4.46740887724919	-2.39252383075225\\
4.46310893582394	-2.38577768353681\\
4.45880532513728	-2.37903387649738\\
4.45449783073851	-2.37229254946664\\
4.45018623841631	-2.36555384264928\\
4.44587033421243	-2.35881789664239\\
4.44154990443541	-2.3520848524559\\
4.43722473567443	-2.34535485153286\\
4.43289461481317	-2.33862803576974\\
4.42855932904381	-2.33190454753672\\
4.42421866588098	-2.32518452969779\\
4.41987241317596	-2.31846812563097\\
4.4155203591308	-2.31175547924837\\
4.41116229231264	-2.30504673501622\\
4.40679800166809	-2.29834203797482\\
4.40242727653761	-2.29164153375846\\
4.39804990667016	-2.2849453686153\\
4.39366568223778	-2.2782536894271\\
4.38927439385035	-2.27156664372893\\
4.38487583257048	-2.26488437972885\\
4.38046978992844	-2.25820704632743\\
4.37605605793721	-2.25153479313726\\
4.37163442910772	-2.24486777050237\\
4.3672046964641	-2.23820612951753\\
4.36276665355913	-2.23155002204751\\
4.35832009448974	-2.22489960074624\\
4.35386481391272	-2.21825501907587\\
4.34940060706049	-2.21161643132577\\
4.34492726975703	-2.20498399263135\\
4.34044459843391	-2.19835785899295\\
4.33595239014654	-2.19173818729441\\
4.33145044259044	-2.18512513532174\\
4.32693855411775	-2.17851886178155\\
4.32241652375386	-2.17191952631943\\
4.31788415121414	-2.16532728953823\\
};
\addplot [color=green, line width=2.0pt, forget plot]
  table[row sep=crcr]{%
4.31788415121414	-2.16532728953823\\
4.29139791655541	-2.12724846061731\\
4.26455957224949	-2.08941697752051\\
4.23737142802345	-2.05183609605227\\
4.20983582370837	-2.01450905045006\\
4.181955129038	-1.97743905310606\\
4.15373174344484	-1.94062929429069\\
4.12516809585358	-1.90408294187806\\
4.09626664447211	-1.86780314107334\\
4.06702987657995	-1.83179301414206\\
4.03746030831423	-1.79605566014146\\
4.00756048445308	-1.76059415465371\\
3.97733297819666	-1.72541154952127\\
3.94678039094576	-1.69051087258425\\
3.91590535207782	-1.65589512741982\\
3.88471051872071	-1.62156729308369\\
3.85319857552406	-1.58753032385379\\
3.82137223442819	-1.55378714897598\\
3.7892342344307	-1.52034067241195\\
3.75678734135083	-1.48719377258935\\
3.72403434759133	-1.45434930215404\\
3.69097807189823	-1.42181008772456\\
3.65762135911818	-1.38957892964894\\
3.62396707995371	-1.35765860176363\\
3.59001813071608	-1.32605185115483\\
3.55577743307608	-1.29476139792205\\
3.52124793381261	-1.26378993494402\\
3.48643260455899	-1.23314012764694\\
3.45133444154729	-1.20281461377511\\
3.41595646535049	-1.17281600316387\\
3.38030172062245	-1.14314687751508\\
3.34437327583594	-1.11380979017483\\
3.30817422301855	-1.08480726591383\\
3.27170767748662	-1.05614180070998\\
3.23497677757704	-1.02781586153368\\
3.1979846843773	-0.999831886135469\\
3.16073458145333	-0.972192282836237\\
3.12322967457555	-0.944899430319965\\
3.08547319144301	-0.917955677429011\\
3.0474683814056	-0.891363342961982\\
3.00921851518439	-0.865124715474152\\
2.97072688459014	-0.839242053080521\\
2.93199680224007	-0.813717583261491\\
2.89303160127269	-0.788553502671144\\
2.85383463506099	-0.763751976948206\\
2.81440927692387	-0.739315140529686\\
2.77475891983577	-0.715245096467159\\
2.73488697613471	-0.691543916245787\\
2.69479687722862	-0.66821363960606\\
2.65449207330002	-0.645256274368229\\
2.61397603300909	-0.622673796259522\\
2.57325224319518	-0.600468148744128\\
2.53232420857668	-0.578641242855912\\
2.49119545144947	-0.557194957033968\\
2.44986951138371	-0.536131136960955\\
2.40834994491932	-0.515451595404263\\
2.36664032525981	-0.495158112059992\\
2.32474424196478	-0.475252433399796\\
2.28266530064109	-0.455736272520589\\
2.24040712263244	-0.436611308997099\\
2.19797334470777	-0.41787918873733\\
2.1553676187483	-0.399541523840925\\
2.11259361143321	-0.381599892460404\\
2.06965500392408	-0.364055838665364\\
2.02655549154812	-0.346910872309599\\
1.98329878348011	-0.330166468901145\\
1.93988860242322	-0.313824069475305\\
1.89632868428859	-0.297885080470633\\
1.85262277787388	-0.282350873607901\\
1.80877464454057	-0.267222785772033\\
1.76478805789029	-0.252502118897061\\
1.72066680344009	-0.238190139854083\\
1.6764146782966	-0.224288080342227\\
1.63203549082929	-0.210797136782651\\
1.5875330603427	-0.197718470215587\\
1.54291121674776	-0.185053206200413\\
1.49817380023214	-0.172802434718784\\
1.45332466092987	-0.160967210080837\\
1.40836765858984	-0.14954855083445\\
1.36330666224378	-0.138547439677583\\
1.31814554987316	-0.127964823373713\\
1.27288820807556	-0.11780161267035\\
1.22753853173009	-0.10805868222066\\
1.18210042366226	-0.0987368705081873\\
1.13657779430809	-0.0898369797747016\\
1.09097456137752	-0.081359775951149\\
1.04529464951731	-0.073305988591738\\
0.9995419899733	-0.0656763108111574\\
0.953720520251996	-0.0584713992249201\\
0.907834183781774	-0.0516918738928572\\
0.861886929573506	-0.0453383182657586\\
0.815882711880666	-0.0394112791351537\\
0.769825489859057	-0.0339112665862587\\
0.72371922722606	-0.0288387539540782\\
0.67756789191955	-0.0241941777826703\\
0.631375455756369	-0.019977937787572\\
0.585145894090524	-0.0161903968214064\\
0.538883185471093	-0.0128318808426537\\
0.492591311299785	-0.00990267888759281\\
0.446274255488311	-0.00740304304543541\\
0.39993600411556	-0.00533318843662593\\
};
\addplot [color=red, line width=2.0pt, forget plot]
  table[row sep=crcr]{%
0.39993600411556	-0.00533318843662593\\
0.395939140381123	-0.00517482353047837\\
0.39194215247312	-0.0050196241896639\\
0.38794504411515	-0.00486755853466557\\
0.383947818955657	-0.00471859468115007\\
0.379950480568712	-0.00457270074011162\\
0.375953032454796	-0.00442984481801698\\
0.371955478041518	-0.00428999501694311\\
0.36795782068441	-0.00415311943471652\\
0.363960063667672	-0.00401918616504569\\
0.359962210204962	-0.00388816329765356\\
0.355964263440121	-0.00376001891840599\\
0.351966226447952	-0.00363472110943548\\
0.347968102235008	-0.00351223794926726\\
0.343969893740315	-0.00339253751293776\\
0.339971603836151	-0.00327558787211185\\
0.335973235328836	-0.00316135709519803\\
0.331974790959454	-0.0030498132474611\\
0.327976273404634	-0.0029409243911315\\
0.32397768527734	-0.00283465858551325\\
0.319979029127594	-0.0027309838870861\\
0.315980307443256	-0.00262986834960889\\
0.311981522650818	-0.00253128002422104\\
0.307982677116121	-0.00243518695953558\\
0.303983773145151	-0.00234155720173657\\
0.299984812984789	-0.00225035879467128\\
0.295985798823606	-0.00216155977993842\\
0.291986732792586	-0.00207512819697662\\
0.287987616965913	-0.00199103208315052\\
0.283988453361758	-0.00190923947383243\\
0.279989243943003	-0.00182971840248333\\
0.275989990618025	-0.00175243690073175\\
0.271990695241491	-0.00167736299844988\\
0.267991359615072	-0.00160446472382805\\
0.263991985488241	-0.00153371010344491\\
0.259992574559052	-0.00146506716234164\\
0.255993128474869	-0.00139850392408526\\
0.25199364883316	-0.00133398841083698\\
0.247994137182249	-0.00127148864341565\\
0.24399459502211	-0.0012109726413599\\
0.239995023805097	-0.00115240842298755\\
0.235995424936723	-0.00109576400545401\\
0.231995799776456	-0.00104100740480884\\
0.227996149638443	-0.000988106636049526\\
0.223996475792292	-0.000937029713174306\\
0.219996779463868	-0.00088774464923326\\
0.215997061836015	-0.00084021945637679\\
0.211997324049341	-0.000794422145903208\\
0.207997567203013	-0.000750320728304604\\
0.203997792355478	-0.000707883213310317\\
0.199998000525258	-0.000667077609929607\\
0.195998192691707	-0.000627871926492247\\
0.191998369795804	-0.000590234170687964\\
0.187998532740877	-0.000554132349603673\\
0.183998682393397	-0.000519534469759833\\
0.179998819583761	-0.000486408537145264\\
0.175998945107021	-0.000454722557250015\\
0.171999059723671	-0.000424444535097356\\
0.167999164160436	-0.000395542475274345\\
0.163999259110999	-0.000367984381960582\\
0.159999345236789	-0.000341738258956127\\
0.15599942316777	-0.000316772109708099\\
0.151999493503164	-0.000293053937335573\\
0.147999556812244	-0.00027055174465373\\
0.143999613635123	-0.000249233534196711\\
0.139999664483475	-0.000229067308239018\\
0.135999709841335	-0.000210021068816108\\
0.131999750165852	-0.000192062817743722\\
0.127999785888085	-0.000175160556636257\\
0.123999817413726	-0.000159282286923856\\
0.119999845123896	-0.000144396009868737\\
0.115999869375925	-0.000130469726580475\\
0.111999890504086	-0.000117471438030191\\
0.107999908820376	-0.00010536914506398\\
0.103999924615315	-9.41308484154604e-05\\
0.09999993815866	-8.37245487172509e-05\\
0.0959999497002049	-7.41182465118738e-05\\
0.0919999594705627	-6.52799422617453e-05\\
0.0879999676818936	-5.7177636358339e-05\\
0.0839999745287012	-4.97793291307189e-05\\
0.0799999801885861	-4.30530208533207e-05\\
0.075999984823042	-3.69667117530689e-05\\
0.0719999885781815	-3.14884020157871e-05\\
0.0679999915855253	-2.65860917920986e-05\\
0.0639999939627913	-2.22277812026814e-05\\
0.0599999958146202	-1.83814703429509e-05\\
0.0559999972333646	-1.50151592872668e-05\\
0.0519999982998787	-1.20968480926506e-05\\
0.0479999990842431	-9.59453680197076e-06\\
0.0439999996465545	-7.4762254467949e-06\\
0.0400000000377148	-5.70991404978691e-06\\
0.036000000300157	-4.26360262671701e-06\\
0.0320000004686336	-3.10529118819289e-06\\
0.0280000005710068	-2.20297974104452e-06\\
0.0240000006289731	-1.52466828944284e-06\\
0.0200000006588606	-1.03835683574875e-06\\
0.0160000006723817	-7.12045381191291e-07\\
0.0120000006774307	-5.1373392635359e-07\\
0.00800000067880826	-4.11422471427183e-07\\
0.00400000067901098	-3.73111016492806e-07\\
6.79021061372976e-10	-3.66799561558748e-07\\
};
\addplot [color=red, line width=2.0pt, only marks, mark size=2.5pt, mark=*, mark options={solid, fill=red, red}, forget plot]
  table[row sep=crcr]{%
0.399936004740566	0.00533272384055325\\
};
\addplot [color=red, line width=2.0pt, only marks, mark size=2.5pt, mark=*, mark options={solid, fill=red, red}, forget plot]
  table[row sep=crcr]{%
4.31788468040832	2.16532586744377\\
};
\addplot [color=red, line width=2.0pt, only marks, mark size=2.5pt, mark=*, mark options={solid, fill=red, red}, forget plot]
  table[row sep=crcr]{%
4.75388580337923	2.8360061221805\\
};
\addplot [color=red, line width=2.0pt, only marks, mark size=2.5pt, mark=*, mark options={solid, fill=red, red}, forget plot]
  table[row sep=crcr]{%
4.75388511027439	-2.83600765083127\\
};
\addplot [color=red, line width=2.0pt, only marks, mark size=2.5pt, mark=*, mark options={solid, fill=red, red}, forget plot]
  table[row sep=crcr]{%
4.31788415121414	-2.16532728953823\\
};
\addplot [color=red, line width=2.0pt, only marks, mark size=2.5pt, mark=*, mark options={solid, fill=red, red}, forget plot]
  table[row sep=crcr]{%
0.39993600411556	-0.00533318843662593\\
};
\addplot [color=red, line width=2.0pt, only marks, mark size=2.5pt, mark=*, mark options={solid, fill=red, red}, forget plot]
  table[row sep=crcr]{%
6.7901873027764e-10	-3.66799561558196e-07\\
};
\addplot [color=blue, line width=2.0pt, only marks, mark size=2.5pt, mark=*, mark options={solid, fill=blue, blue}, forget plot]
  table[row sep=crcr]{%
0	0\\
};
\addplot [color=blue, line width=2.0pt, only marks, mark size=2.5pt, mark=*, mark options={solid, fill=blue, blue}, forget plot]
  table[row sep=crcr]{%
0	0\\
};
\end{axis}
\end{tikzpicture}%%
  \caption{Duboids solution example 4}
  \label{fig:DuboidsRes3}
\end{figure}
%
The fifth example (Figure \ref{fig:DuboidsRes4}) shows a connection from $x=0$, $y=0$, $\theta=0$ and $\kappa=0$ to $x=0$, $y=0$, $\theta=\pi/2$ and $\kappa=0$ obtaining a convoluted form.\\
%
\begin{figure}[htb!]
  \centering
  % This file was created by matlab2tikz.
%
%The latest updates can be retrieved from
%  http://www.mathworks.com/matlabcentral/fileexchange/22022-matlab2tikz-matlab2tikz
%where you can also make suggestions and rate matlab2tikz.
%
\begin{tikzpicture}[scale = 0.7]

\begin{axis}[%
width=\linewidth,
height=0.776\linewidth,
at={(0\linewidth,0\linewidth)},
scale only axis,
xmin=-4.60136435657085,
xmax=5.03889624988018,
xlabel style={font=\color{white!15!black}},
xlabel={x(m)},
ymin=-3.58291739226538,
ymax=4.02044944088712,
ylabel style={font=\color{white!15!black}},
ylabel={y(m)},
axis background/.style={fill=white},
title style={font=\bfseries},
title={$L_{tot}$ = 37.7126, $k_{max}$ = 0.5, $J_{max}$ = 0.5, Type = [LRL]},
axis x line*=bottom,
axis y line*=left,
xmajorgrids,
xminorgrids,
ymajorgrids,
yminorgrids
]
\addplot [color=red, line width=2.0pt, forget plot]
  table[row sep=crcr]{%
0	0\\
0.009999999999375	8.33333333296131e-08\\
0.01999999998	6.66666666190476e-07\\
0.029999999848125	2.24999999186384e-06\\
0.03999999936	5.33333327238095e-06\\
0.049999998046875	1.04166663760231e-05\\
0.0599999951400002	1.79999989585715e-05\\
0.0699999894956257	2.85833302695574e-05\\
0.0799999795200024	4.26666588647625e-05\\
0.089999963094382	6.07499822062188e-05\\
0.0999999375000181	8.33332961309598e-05\\
0.109999899343168	0.000110916594169772\\
0.119999844480093	0.000143999866697198\\
0.129999767942067	0.000183083099894043\\
0.139999663860374	0.000228666274503633\\
0.14999952539132	0.000281249364363084\\
0.159999344641243	0.000341332334690825\\
0.16999911259152	0.000409415140111639\\
0.179998819023587	0.000485997722400469\\
0.189998452443961	0.000571580007926256\\
0.199998000009259	0.000666661904777056\\
0.209997447451239	0.000771743299547711\\
0.219996779001833	0.000887324053771324\\
0.229995977318198	0.00101390399997582\\
0.239995023407776	0.00115198293734686\\
0.249993896553361	0.00130206062697838\\
0.259992574238189	0.00146463678669201\\
0.269991032071029	0.00164021108540679\\
0.279989243711304	0.00182928313704023\\
0.289987180794227	0.00203235249392229\\
0.299984812855953	0.00224991863970341\\
0.309982107258766	0.00248248098173797\\
0.319979029116282	0.0027305388429245\\
0.329975541218689	0.00299459145298403\\
0.339971603958015	0.00327513793915787\\
0.349967175253435	0.0035726773163062\\
0.359962210476616	0.00388770847638889\\
0.369956662377103	0.00422073017731001\\
0.379950481007759	0.00457224103110726\\
0.389943613650245	0.00494273949146801\\
0.399936004740566	0.00533272384055325\\
0.409927595794669	0.00574269217511096\\
0.419918325334115	0.00617314239186055\\
0.429908128811808	0.00662457217212975\\
0.439896938537814	0.00709747896572564\\
0.44988468360525	0.00759235997402141\\
0.459871289816262	0.00810971213224044\\
0.469856679608105	0.00865003209091956\\
0.479840771979308	0.00921381619653296\\
0.489823482415957	0.00980156047125886\\
0.499804722818087	0.0104137605918704\\
0.509784401426195	0.0110509118677329\\
0.519762422747881	0.0117135092178894\\
0.529738687484628	0.0124020471472162\\
0.539713092458725	0.0131170197216309\\
0.549685530540343	0.0138589205423346\\
0.559655890574774	0.0146282427190708\\
0.569624057309844	0.015425478842383\\
0.579589911323503	0.0162511209548533\\
0.589553328951615	0.0171056605213053\\
0.599514182215937	0.0179895883979516\\
0.609472338752324	0.0189033948004718\\
0.619427661739149	0.0198475692710004\\
0.629380009825963	0.0208226006440093\\
0.639329237062402	0.0218289770110674\\
0.64927519282735	0.0228671856844605\\
0.659217721758378	0.0239377131596537\\
0.669156663681469	0.0250410450765807\\
0.679091853541037	0.0261776661797427\\
0.689023121330264	0.0273480602771007\\
0.698950292021764	0.0285527101977446\\
0.708873185498583	0.0297920977483229\\
0.718791616485562	0.0310667036682178\\
0.72870539448107	0.0323770075834479\\
0.738614323689129	0.033723487959285\\
0.748518202951936	0.0351066220515678\\
0.758416825682818	0.0365268858566984\\
0.768309979799618	0.0379847540603047\\
0.77819744765854	0.0394806999845558\\
0.788079005988475	0.0410151955341141\\
0.79795442582581	0.0425887111407103\\
0.807823472449766	0.0442017157063264\\
0.817685905318254	0.045854676544973\\
0.827541478004292	0.0475480593230471\\
0.837389938132998	0.0492823279982562\\
0.847231027319174	0.0510579447570965\\
0.857064481105505	0.0528753699508699\\
0.866890028901404	0.0547350620302302\\
0.876707393922514	0.0566374774782424\\
0.886516293130892	0.0585830707419458\\
0.896316437175907	0.0605722941624074\\
0.906107530335867	0.0626055979032542\\
0.915889270460404	0.0646834298776731\\
0.925661348913641	0.0668062356738681\\
0.935423450518165	0.0689744584789635\\
0.945175253499836	0.0711885390013423\\
0.95491642943346	0.0734489153914118\\
0.964646643189338	0.0757560231607851\\
0.974365552880748	0.0781102950998698\\
0.984072809812352	0.0805121611938562\\
0.993768058429589	0.082962048537095\\
};
\addplot [color=green, line width=2.0pt, forget plot]
  table[row sep=crcr]{%
0.993768058429589	0.082962048537095\\
1.09032874389115	0.110215465856275\\
1.18540114459326	0.142277177556056\\
1.27874599385169	0.179066494610061\\
1.37012837267417	0.220490830128294\\
1.45931830097588	0.266445932368596\\
1.54609131636552	0.316816147104778\\
1.630229039045	0.37147470869115\\
1.71151972140121	0.43028405909092\\
1.78975878090672	0.493096194065571\\
1.8647493149883	0.559753035653971\\
1.93630259656738	0.630086830003813\\
2.0042385490255	0.703920569554155\\
2.06838619939931	0.781068438506577\\
2.12858410866454	0.861336280463842\\
2.18468077802617	0.944522087059157\\
2.23653503019222	1.03041650634633\\
2.28401636467164	1.11880336967134\\
2.32700528620211	1.20946023569939\\
2.3653936054812	1.30215895022827\\
2.39908471144409	1.39666622037919\\
2.4279938144026	1.49274420171994\\
2.45204815943357	1.59015109684299\\
2.47118720947963	1.68864176389178\\
2.4853627977015	1.78796833350406\\
2.49453924869849	1.88788083261944\\
2.49869346829199	1.98812781358121\\
2.49781500164611	2.08845698694935\\
2.49190605957911	2.18861585643193\\
2.48098151299955	2.28835235433722\\
2.46506885548093	2.38741547594701\\
2.4442081340692	2.48555591121486\\
2.41845184849727	2.58252667219939\\
2.38786481905998	2.67808371465353\\
2.35252402348231	2.77198655220553\\
2.31251840319112	2.86399886158584\\
2.2679486394782	2.95388907737686\\
2.21892690011781	3.04143097478869\\
2.1655765570764	3.12640423899423\\
2.10803187602504	3.20859501959082\\
2.04643767843585	3.287796468793\\
1.98094897711291	3.36380926200187\\
1.91173058607482	3.43644209944114\\
1.83895670577079	3.50551218759713\\
1.76281048467406	3.57084569925132\\
1.68348355835607	3.63227821094766\\
1.60117556720128	3.68965511679351\\
1.51609365397647	3.74283201755302\\
1.42845194251897	3.79167508405362\\
1.33847099885577	3.83606139399102\\
1.24637727610965	3.8758792412852\\
1.15240254458951	3.91102841720865\\
1.0567833084989	3.9414204625796\\
0.959760210730999	3.96697889038533\\
0.861577427247719	3.98763937827541\\
0.762482052567302	4.00334993044039\\
0.662723477906797	4.01407100846852\\
0.56255276354451	4.01977563085128\\
0.462222006981947	4.02044944088712\\
0.361983708495408	4.01609074281269\\
0.262090135673922	4.00671050607052\\
0.162792688542787	3.99233233770245\\
0.0643412668704865	3.97299242293833\\
-0.0330163587487385	3.94873943412937\\
-0.129035170453855	3.91963440825551\\
-0.223473519748267	3.88575059331494\\
-0.316093735651443	3.84717326398241\\
-0.406662722840391	3.80399950700032\\
-0.494952548275706	3.75633797684252\\
-0.580741014835778	3.70430862226598\\
-0.663812220515563	3.64804238443824\\
-0.743957101782546	3.58768086740062\\
-0.820973959722484	3.52337598169634\\
-0.894668967650758	3.45528956206047\\
-0.964856658911868	3.38359296013391\\
-1.0313603936394	3.30846661322638\\
-1.09401280330182	3.23009959021367\\
-1.15265621191528	3.14868911571208\\
-1.20714303286343	3.06444007372734\\
-1.25733614032547	2.97756449202751\\
-1.30310921437781	2.88828100853711\\
-1.34434705890073	2.79681432109582\\
-1.38094589148999	2.70339462196611\\
-1.41281360464377	2.60825701851338\\
-1.4398699975677	2.51164094151619\\
-1.46204697801448	2.41378954259598\\
-1.47928873365015	2.3149490822825\\
-1.49155187251585	2.21536831025525\\
-1.4988055322314	2.11529783932046\\
-1.50103145766603	2.01498951469912\\
-1.49822404688065	1.91469578021355\\
-1.49039036522613	1.81466904296736\\
-1.4775501275621	1.71516103811794\\
-1.45973564864098	1.61642219533989\\
-1.43699176178211	1.51870100857401\\
-1.40937570604074	1.42224341064793\\
-1.37695698215564	1.32729215434216\\
-1.33981717763817	1.2340862014594\\
-1.29804976144263	1.14286012143438\\
-1.25175984873503	1.05384350099807\\
-1.2010639363519	0.967260366381544\\
};
\addplot [color=red, line width=2.0pt, forget plot]
  table[row sep=crcr]{%
-1.2010639363519	0.967260366381544\\
-1.19044441022212	0.950312729443986\\
-1.17965930809285	0.933469973264685\\
-1.1687130132101	0.91673152590558\\
-1.15760986110355	0.900096709096101\\
-1.14635413899746	0.883564741290592\\
-1.13495008533281	0.867134740692135\\
-1.12340188939738	0.850805728241255\\
-1.11171369106024	0.834576630568183\\
-1.09988958060768	0.818446282907478\\
-1.08793359867698	0.802413431973994\\
-1.07584973628515	0.786476738799303\\
-1.06364193494926	0.770634781527816\\
-1.05131408689522	0.754886058172003\\
-1.03887003535216	0.73922898932623\\
-1.02631357492907	0.723661920838859\\
-1.01364845207091	0.708183126442392\\
-1.00087836559127	0.692790810341517\\
-0.988006967278539	0.677483109759086\\
-0.975037862572854	0.662258097440124\\
-0.961974611311084	0.647113784114066\\
-0.948820728536989	0.632048120915548\\
-0.935579685373988	0.617059001764162\\
-0.922254909957835	0.602144265703654\\
-0.90884978842668	0.587301699201183\\
-0.895367665965934	0.572529038407272\\
-0.881811847905518	0.557823971377232\\
-0.868185600867033	0.543184140254875\\
-0.854492153958497	0.528607143419402\\
-0.840734700014319	0.514090537596462\\
-0.826916396878226	0.499631839934369\\
-0.813040368726926	0.485228530046627\\
-0.799109707432308	0.470878052021859\\
-0.78512747396005	0.456577816402406\\
-0.771096699802522	0.442325202132794\\
-0.75702038844394	0.428117558479437\\
-0.742901516855746	0.413952206922887\\
-0.728743037020237	0.399826443024063\\
-0.714547877480485	0.385737538265869\\
-0.700318944914661	0.371682741871695\\
-0.686059125732857	0.35765928260231\\
-0.671771287694584	0.343664370532685\\
-0.657458281545093	0.329695198810315\\
-0.643122942668753	0.315748945396647\\
-0.628768092757696	0.301822774793228\\
-0.614396541493989	0.28791383975423\\
-0.600011088243579	0.27401928298699\\
-0.585614523760334	0.260136238842272\\
-0.571209631898434	0.246261834995928\\
-0.556799191331467	0.23239319412367\\
-0.542385977276509	0.218527435570659\\
-0.52797276322155	0.204661677017648\\
-0.513562322654583	0.19079303614539\\
-0.499157430792683	0.176918632299046\\
-0.484760866309437	0.163035588154328\\
-0.470375413059028	0.149141031387088\\
-0.45600386179532	0.135232096348089\\
-0.441649011884264	0.121305925744671\\
-0.427313673007924	0.107359672331003\\
-0.413000666858433	0.0933905006086325\\
-0.39871282882016	0.0793955885390077\\
-0.384453009638356	0.0653721292696225\\
-0.370224077072532	0.0513173328754485\\
-0.35602891753278	0.0372284281172545\\
-0.341870437697271	0.0231026642184313\\
-0.327751566109077	0.00893731266188101\\
-0.313675254750495	-0.00527033099147642\\
-0.299644480592967	-0.0195229452610881\\
-0.285662247120709	-0.0338231808805411\\
-0.271731585826092	-0.0481736589053074\\
-0.257855557674791	-0.0625769687930512\\
-0.244037254538698	-0.0770356664551436\\
-0.23027980059452	-0.0915522722780841\\
-0.216586353685984	-0.106129269113557\\
-0.202960106647499	-0.120769100235914\\
-0.189404288587083	-0.135474167265954\\
-0.175922166126337	-0.150246828059865\\
-0.162517044595182	-0.165089394562336\\
-0.149192269179029	-0.180004130622844\\
-0.135951226016028	-0.19499324977423\\
-0.122797343241934	-0.210058912972748\\
-0.109734091980162	-0.225203226298807\\
-0.0967649872744782	-0.240428238617768\\
-0.0838935889617432	-0.255735939200199\\
-0.0711235024821073	-0.271128255301074\\
-0.0584583796239513	-0.286607049697541\\
-0.0459019192008529	-0.302174118184912\\
-0.0334578676577944	-0.317831187030685\\
-0.0211300196037606	-0.333579910386498\\
-0.00892221826786305	-0.349421867657985\\
0.00316164412396199	-0.365358560832676\\
0.0151176260546627	-0.38139141176616\\
0.0269417365072282	-0.397521759426866\\
0.0386299348443596	-0.413750857099937\\
0.050178130779794	-0.430079869550816\\
0.0615821844444396	-0.446509870149274\\
0.0728379065505356	-0.463041837954784\\
0.0839410586570836	-0.479676654764262\\
0.0948873535398293	-0.496415102123367\\
0.105672455669105	-0.513257858302668\\
0.116291981798888	-0.530205495240226\\
};
\addplot [color=green, line width=2.0pt, forget plot]
  table[row sep=crcr]{%
0.116291981798888	-0.530205495240226\\
0.174712960895231	-0.630909378742977\\
0.227175324532961	-0.734841969168952\\
0.273501300281821	-0.841651083583531\\
0.31353390929599	-0.950974791747085\\
0.347137498248005	-1.06244264254118\\
0.374198198999658	-1.17567691926842\\
0.394624314452239	-1.29029391957216\\
0.40834662926864	-1.40590525563914\\
0.415318644414418	-1.52211917027898\\
0.415516734723067	-1.63854186442124\\
0.408940228951565	-1.75477883153131\\
0.395611412054924	-1.87043619442381\\
0.375575449672051	-1.98512203994322\\
0.348900235078778	-2.09844774698939\\
0.315676159126702	-2.21002930338757\\
0.276015803947382	-2.31948860714089\\
0.230053561459828	-2.42645474765567\\
0.177945177973979	-2.53056526259821\\
0.119867226433306	-2.63146736612411\\
0.0560165080849091	-2.72881914431796\\
-0.0133906143954297	-2.82229071379291\\
-0.0881189500634783	-2.91156533952382\\
-0.167915276667711	-2.99634050812631\\
-0.252509198710815	-3.07632895294478\\
-0.341614063703961	-3.15125962747573\\
-0.434927933509036	-3.22087862382811\\
-0.532134607477395	-3.28495003310807\\
-0.63290469391811	-3.343256744813\\
-0.736896726265025	-3.3956011825258\\
-0.843758320160366	-3.44180597341655\\
-0.953127367534065	-3.4817145492829\\
-1.06463326363258	-3.51519167709248\\
-1.17789816283929	-3.5421239172296\\
-1.29253825903114	-3.56242000789344\\
-1.40816508613279	-3.57601117434507\\
-1.52438683446142	-3.58285136195558\\
-1.64080967840156	-3.58291739226538\\
-1.75703911091101	-3.5762090415261\\
-1.87268128033591	-3.56274904145875\\
-1.9873443250049	-3.54258300222569\\
-2.1006397010801	-3.51577925787736\\
-2.21218349916541	-3.48242863479752\\
-2.32159774521067	-3.44264414393163\\
-2.42851168130343	-3.39656059784126\\
-2.53256302200839	-3.34433415388217\\
-2.63339918199721	-3.28614178505411\\
-2.73067847080883	-3.22218068031525\\
-2.82407125069181	-3.15266757639346\\
-2.9132610536051	-3.07783802335859\\
-2.99794565359245	-2.99794558644437\\
-3.0778380908964	-2.91326098682469\\
-3.15266764434174	-2.82407118425577\\
-3.22218074869334	-2.73067840469271\\
-3.28614185387989	-2.63339911617544\\
-3.34433422317201	-2.53256295645443\\
-3.39656066760996	-2.42851161598983\\
-3.44264421419236	-2.32159768010915\\
-3.48242870556179	-2.21218343424698\\
-3.51577932915497	-2.10063963631515\\
-3.5425830740247	-1.98734426036331\\
-3.56274911378546	-1.87268121578713\\
-3.57620911438501	-1.75703904642417\\
-3.5829174656592	-1.64080961394559\\
-3.58285143588519	-1.52438677000515\\
-3.57601124880955	-1.40816502164505\\
-3.56242008289004	-1.29253819448085\\
-3.5421239927538	-1.17789809819559\\
-3.51519175313794	-1.06463319886493\\
-3.48171462584153	-0.953127302612351\\
-3.44180605047851	-0.843758255054984\\
-3.39560126007955	-0.736896660947005\\
-3.34325682284533	-0.632904628359194\\
-3.28495011160416	-0.532134541650146\\
-3.22087870277156	-0.434927867386921\\
-3.15125970684863	-0.341613997261447\\
-3.07632903272774	-0.25250913192346\\
-2.99634058829859	-0.167915209512238\\
-2.91156542006333	-0.0881188825178569\\
-2.82229079467633	-0.0133905464389531\\
-2.7288192255208	0.056016576471553\\
-2.6314674476208	0.119867295267976\\
-2.53056534436219	0.177945247273014\\
-2.42645482965945	0.230053631237995\\
-2.3194886893562	0.276015874217821\\
-2.21002938578541	0.315676229900888\\
-2.09844782954012	0.348900306366479\\
-1.98512212261672	0.375575521481291\\
-1.87043627718951	0.395611484391965\\
-1.75477891435836	0.408940301820877\\
-1.63854194727855	0.415516808127318\\
-1.52211925313539	0.415318718354463\\
-1.40590533846345	0.408346703743517\\
-1.29029400233333	0.394624389459176\\
-1.17567700193558	0.374198274534077\\
-1.06244272508381	0.347137574303544\\
-0.95097487413506	0.31353398586452\\
-0.841651165787274	0.273501377353476\\
-0.734842051159495	0.227175402096166\\
-0.63090946049208	0.174713038936747\\
-0.530205576720467	0.116292060303857\\
};
\addplot [color=red, line width=2.0pt, forget plot]
  table[row sep=crcr]{%
-0.530205576720467	0.116292060303857\\
-0.513257939734036	0.105672534252069\\
-0.496415183505101	0.0948874322003065\\
-0.479676736095621	0.0839411373945944\\
-0.463041919235044	0.072837985364602\\
-0.446509951377731	0.0615822633345866\\
-0.430079950726791	0.050178209745555\\
-0.413750938222766	0.0386300138852692\\
-0.397521840495904	0.0269418156228262\\
-0.381391492780782	0.0151177052444948\\
-0.365358641792276	0.00316172338757965\\
-0.349421948561974	-0.00892213893090153\\
-0.333579991234305	-0.0211299401938921\\
-0.317831267821755	-0.0334577881754504\\
-0.302174198918713	-0.0459018396464527\\
-0.286607130373557	-0.0584582999979082\\
-0.271128335918803	-0.0711234227848285\\
-0.255736019759158	-0.0838935091936271\\
-0.240428319117491	-0.0967649074359139\\
-0.225203306738845	-0.10973401207153\\
-0.210058993352668	-0.122797263263604\\
-0.194993330093611	-0.135951145968366\\
-0.180004210881288	-0.149192189062386\\
-0.165089474759459	-0.162516964409897\\
-0.150246908195296	-0.175922085872745\\
-0.135474247339338	-0.189404208265505\\
-0.120769180246912	-0.202960026258247\\
-0.106129349061845	-0.216586273229357\\
-0.0915523521633546	-0.230279720070807\\
-0.0770357462771005	-0.244037173948178\\
-0.0625770485514119	-0.257855477017732\\
-0.0481737385998097	-0.271731505102746\\
-0.0338232605109327	-0.28566216633132\\
-0.0195230248271314	-0.299644399737766\\
-0.00527041049294841	-0.313675173829701\\
0.00893723322519029	-0.327751485122898\\
0.0231025848467164	-0.3418703566459\\
0.0372283488106988	-0.3560288364164\\
0.0513172536342236	-0.370223995891315\\
0.065372050093881	-0.384452928392458\\
0.0793955094288906	-0.398712747509722\\
0.0933904215642699	-0.413000585483589\\
0.10735959335251	-0.427313591568792\\
0.121305846832152	-0.441648930380949\\
0.135232017501633	-0.456003780227914\\
0.149140952606771	-0.470375331427611\\
0.163035509440217	-0.484760784614078\\
0.17691855365119	-0.499157349033432\\
0.190792957563827	-0.513562240831479\\
0.204661598502403	-0.52797268133462\\
0.218527357121746	-0.542385895325767\\
0.232393115741088	-0.556799109316912\\
0.246261756679665	-0.571209549820054\\
0.260136160592301	-0.585614441618101\\
0.274019204803274	-0.600011006037455\\
0.28791376163672	-0.614396459223922\\
0.301822696741858	-0.628768010423619\\
0.315748867411339	-0.643122860270584\\
0.329695120890981	-0.657458199082741\\
0.343664292679221	-0.671771205167944\\
0.357659204814601	-0.686059043141811\\
0.37168266414961	-0.700318862259075\\
0.385737460609268	-0.714547794760217\\
0.399826365432792	-0.728742954235133\\
0.413952129396775	-0.742901434005633\\
0.428117481018301	-0.757020305528635\\
0.44232512473644	-0.771096616821832\\
0.456577739070623	-0.785127390913767\\
0.470877974754424	-0.799109624320213\\
0.485228452843301	-0.813040285548787\\
0.499631762794903	-0.826916313633801\\
0.514090460520592	-0.840734616703355\\
0.528607066406846	-0.854492070580726\\
0.543184063305337	-0.868185517422176\\
0.557823894490404	-0.881811764393286\\
0.572528961582829	-0.895367582386027\\
0.587301622438787	-0.908849704778788\\
0.60214418900295	-0.922254826241635\\
0.617058925124779	-0.935579601589147\\
0.632048044337102	-0.948820644683166\\
0.647113707596159	-0.961974527387929\\
0.662258020982336	-0.975037778580003\\
0.677483033360983	-0.988006883215619\\
0.692790734002649	-1.00087828145791\\
0.708183050162294	-1.0136483678667\\
0.723661844617048	-1.02631349065362\\
0.739228913162204	-1.03886995100508\\
0.754885982065246	-1.05131400247608\\
0.770634705477797	-1.06364185045764\\
0.786476662805465	-1.07584965172063\\
0.802413356035767	-1.08793351403911\\
0.818446207024273	-1.09988949589603\\
0.834576554739395	-1.11171360627436\\
0.850805652466258	-1.1234018045368\\
0.867134664970283	-1.13495000039709\\
0.883564665621222	-1.14635405398612\\
0.900096633478536	-1.15760977601614\\
0.916731450339112	-1.16871292804613\\
0.933469897748592	-1.17965922285184\\
0.950312653977528	-1.1904443249036\\
0.967260290963958	-1.20106385095539\\
};
\addplot [color=green, line width=2.0pt, forget plot]
  table[row sep=crcr]{%
0.967260290963958	-1.20106385095539\\
1.05384342712276	-1.251759763663\\
1.14286004915029	-1.29804967655079\\
1.23408613080478	-1.33981709277827\\
1.32729208534435	-1.37695689717624\\
1.42224334332308	-1.4093756207879\\
1.51870094292695	-1.43699167610017\\
1.61642213136399	-1.45973556237326\\
1.71516097579506	-1.47755004055171\\
1.81466898226775	-1.4903902773167\\
1.91469572109594	-1.49822395791712\\
2.01498945711077	-1.5010313674954\\
2.11529778319723	-1.4988054407035\\
2.21536825552181	-1.49155177948409\\
2.31494902885251	-1.47928863897226\\
2.41378949037235	-1.46204688155327\\
2.51164089039141	-1.43986989919176\\
2.60825696836979	-1.41281350422815\\
2.70339457267636	-1.38094578891693\\
2.79681427252327	-1.34434695406028\\
2.88828096053633	-1.30310910716845\\
2.97756444444474	-1.25733603065476\\
3.0644400264011	-1.20714292064855\\
3.1486890684737	-1.15265609708358\\
3.23009954288798	-1.09401268579131\\
3.30846656563228	-1.03136027339924\\
3.38359291208513	-0.964856535902783\\
3.45528951336625	-0.894668841845432\\
3.52337593216215	-0.820973831105894\\
3.58768081682895	-0.74395697035228\\
3.64804233262935	-0.663812086282061\\
3.70430856901874	-0.580740877822547\\
3.75633792195521	-0.494952408519471\\
3.80399945027141	-0.406662580391225\\
3.84717320521146	-0.316093590572807\\
3.88575053230339	-0.223473372117042\\
3.91963434480757	-0.129035020360284\\
3.94873936805282	-0.0330162062963445\\
3.97299235404538	0.0643414215650395\\
3.9923322658106	0.162792845349893\\
4.00671043100335	0.262090294451278\\
4.0160906644007	0.361983869088301\\
4.02044935896853	0.462222169223607\\
4.01977554527279	0.562552927256499\\
4.0140709190861	0.662723642899457\\
4.00334983712002	0.762482218640231\\
3.98763928089378	0.861577594190325\\
3.96697878883052	0.959760378323061\\
3.94142035675176	1.05678347651121\\
3.91102830702058	1.15240271278451\\
3.87587912666296	1.24637744424217\\
3.83606127487448	1.33847116667373\\
3.79167496039692	1.4284521097642\\
3.74283188932508	1.51609382038547\\
3.6896549839784	1.6011757325061\\
3.63227807354499	1.68348372228515\\
3.57084555727654	1.76281064695313\\
3.50551204108181	1.83895686612378\\
3.43644194843319	1.9117307442248\\
3.36380910656567	1.98094913278303\\
3.28779630900951	2.04643783135033\\
3.20859485555768	2.10803202591011\\
3.12640407082566	2.1655767036613\\
3.04143080261553	2.21892704313576\\
2.95388890134639	2.26794877866741\\
2.86399868186159	2.31251853829573\\
2.77198636896709	2.35252415425344\\
2.67808352809622	2.38786494525663\\
2.58252648253393	2.4184519698873\\
2.48555571866696	2.4442082504303\\
2.38741528075691	2.46506896660147\\
2.28835215675916	2.48098161867957\\
2.1886156567336	2.49190615963112\\
2.08845678541124	2.49781509589596\\
1.98812761049596	2.49869355657972\\
1.88788062829111	2.49453933087906\\
1.78796812824736	2.48536287364554\\
1.68864155803123	2.47118727907417\\
1.5901508907121	2.45204822258268\\
1.49274399566029	2.427993871028\\
1.39666601473945	2.39908476148572\\
1.3021587453633	2.36539364889775\\
1.20946003196916	2.32700532297147\\
1.11880316743993	2.2840163947913\\
1.03041630598084	2.23653505367963\\
0.944521888928621	2.18468079491902\\
0.861336084938114	2.12858411902103\\
0.781068245955191	2.06838620329828\\
0.703920380345178	2.00423854656656\\
0.630086644502688	1.93630258787094\\
0.559752854222345	1.86474930019559\\
0.493096017060124	1.78975876017966\\
0.430283886862188	1.71151969492227\\
0.371474541582365	1.63022900701701\\
0.316815985450694	1.54609127901136\\
0.266445776494338	1.45931825853815\\
0.22049068034822	1.37012832541478\\
0.179066351226632	1.27874594205136\\
0.14227704085875	1.18540108855098\\
0.110215336120496	1.09032868392362\\
0.0829619260231352	0.993767994870484\\
};
\addplot [color=red, line width=2.0pt, forget plot]
  table[row sep=crcr]{%
0.0829619260231352	0.993767994870484\\
0.0805120394494035	0.984072746058801\\
0.0781101741258756	0.974365488936558\\
0.0757559029581759	0.964646579058293\\
0.0734487959610849	0.954916365119303\\
0.0711884203441675	0.945175189006276\\
0.0689743405957829	0.935423385848869\\
0.0668061185655005	0.925661284072257\\
0.0646833135449103	0.915889205450537\\
0.0626054823468618	0.906107465161078\\
0.0605721793831293	0.896316371839738\\
0.0585829567405	0.886516227636841\\
0.056637364255324	0.876707328274044\\
0.0547349495865097	0.866889963101936\\
0.0528752582869981	0.857064415158436\\
0.0510578338737009	0.847230961227858\\
0.0492822178959428	0.837389871900755\\
0.0475479500023997	0.827541411634396\\
0.0458545680065574	0.817685838813957\\
0.044201607950687	0.807823405814277\\
0.0425886041683709	0.797954359062293\\
0.0410150893455816	0.78807893910007\\
0.0394805945803175	0.778197380648346\\
0.0379846494408295	0.768309912670693\\
0.0365267820224367	0.758416758438178\\
0.0351065190029548	0.748518135594572\\
0.0337233856967376	0.738614256221988\\
0.0323769061073661	0.728705326907055\\
0.0310666029789874	0.718791548807551\\
0.0297919978463133	0.708873117719409\\
0.0285526110833097	0.698950224144223\\
0.0273479619505794	0.689023053357106\\
0.0261775686414609	0.679091785474986\\
0.0250409483268497	0.669156595525208\\
0.0239376171987705	0.659217653514541\\
0.0228670905127106	0.649275124498548\\
0.0218288826287227	0.6393291686512\\
0.0208225070513295	0.629379941334887\\
0.0198474764682326	0.61942759317068\\
0.0189033027878527	0.609472270108919\\
0.017989497175706	0.599514113500005\\
0.0171055700896473	0.589553260165529\\
0.016251031313986	0.579589842469588\\
0.0154253899925001	0.569623988390399\\
0.0146281546603561	0.559655821592055\\
0.0138588332749615	0.549685461496559\\
0.0131169332457649	0.539713023356059\\
0.0124019614630133	0.529738618325217\\
0.0117134243254973	0.519762353533822\\
0.0110508277672906	0.509784332159541\\
0.0104136772835099	0.499804653500865\\
0.00980147795510396	0.489823413050147\\
0.00921373447270001	0.479840702566844\\
0.00864995115951871	0.469856610150896\\
0.00810963199337461	0.45987122031617\\
0.00759228062778715	0.449884614064098\\
0.00709740041221289	0.439896868957379\\
0.00662449441142351	0.42990805919384\\
0.00617306542403963	0.419918255680319\\
0.00574261600024852	0.409927526106704\\
0.00533264845871796	0.399935935020063\\
0.00494266490272339	0.389943543898792\\
0.00457216723551227	0.379950411226902\\
0.00422065717491871	0.369956592568342\\
0.00388763626725179	0.359962140641424\\
0.0035726059004696	0.349967105393242\\
0.00327506731666439	0.339971534074208\\
0.00299452162387236	0.32997547131261\\
0.00273046980723061	0.319978959189248\\
0.00248241273949429	0.309982037312045\\
0.00224985119093942	0.299984742890769\\
0.00203228583866486	0.289987110811777\\
0.00182921727531368	0.279989173712739\\
0.00164014601723285	0.269990962057459\\
0.00146457251209006	0.259992504210679\\
0.00130199714596597	0.24999382651295\\
0.00115192024993965	0.239994953355454\\
0.00101384210618726	0.229995907254912\\
0.00088726295361375	0.219996708928503\\
0.000771682993031639	0.209997377368737\\
0.000666602391911615	0.199997929918419\\
0.000571521288719364	0.189998382345569\\
0.000485939796859026	0.179998748918405\\
0.000409358008241411	0.169999042480261\\
0.000341275996496673	0.159999274524576\\
0.00028119381984914	0.149999455269886\\
0.000228611523673179	0.139999593734768\\
0.00018302914274991	0.129999697812845\\
0.000143946703241396	0.119999774347765\\
0.000110864224404184	0.109999829208215\\
8.32817200568052e-05	0.0999998673628785\\
6.0699199824574e-05	0.0899998929554519\\
4.26166701764025e-05	0.0799999093796322\\
2.85341352748724e-05	0.0699999193541396\\
1.79515976578911e-05	0.059999924997676\\
1.0369058769739e-05	0.0499999279039441\\
5.28651936043693e-06	0.0399999292166675\\
2.20397977437651e-06	0.0299999297045498\\
6.21440143263748e-07	0.0199999298363011\\
3.89005049651121e-08	0.00999992985562482\\
-4.36391338898279e-08	-7.01437548045192e-08\\
};
\addplot [color=red, line width=2.0pt, only marks, mark size=2.5pt, mark=*, mark options={solid, fill=red, red}, forget plot]
  table[row sep=crcr]{%
0.993768058429589	0.082962048537095\\
};
\addplot [color=red, line width=2.0pt, only marks, mark size=2.5pt, mark=*, mark options={solid, fill=red, red}, forget plot]
  table[row sep=crcr]{%
-1.2010639363519	0.967260366381544\\
};
\addplot [color=red, line width=2.0pt, only marks, mark size=2.5pt, mark=*, mark options={solid, fill=red, red}, forget plot]
  table[row sep=crcr]{%
0.116291981798888	-0.530205495240226\\
};
\addplot [color=red, line width=2.0pt, only marks, mark size=2.5pt, mark=*, mark options={solid, fill=red, red}, forget plot]
  table[row sep=crcr]{%
-0.530205576720467	0.116292060303857\\
};
\addplot [color=red, line width=2.0pt, only marks, mark size=2.5pt, mark=*, mark options={solid, fill=red, red}, forget plot]
  table[row sep=crcr]{%
0.967260290963958	-1.20106385095539\\
};
\addplot [color=red, line width=2.0pt, only marks, mark size=2.5pt, mark=*, mark options={solid, fill=red, red}, forget plot]
  table[row sep=crcr]{%
0.0829619260231352	0.993767994870484\\
};
\addplot [color=red, line width=2.0pt, only marks, mark size=2.5pt, mark=*, mark options={solid, fill=red, red}, forget plot]
  table[row sep=crcr]{%
-4.36391338898279e-08	-7.01437548045192e-08\\
};
\addplot [color=blue, line width=2.0pt, only marks, mark size=2.5pt, mark=*, mark options={solid, fill=blue, blue}, forget plot]
  table[row sep=crcr]{%
0	0\\
};
\addplot [color=blue, line width=2.0pt, only marks, mark size=2.5pt, mark=*, mark options={solid, fill=blue, blue}, forget plot]
  table[row sep=crcr]{%
0	0\\
};
\end{axis}
\end{tikzpicture}%%
  \caption{Duboids solution example 5}
  \label{fig:DuboidsRes4}
\end{figure}
%

\section*{Compare PINS and Duboids}

In this section, we closely compare the results obtained with the two approaches in some specific cases.
%
\subsection*{Standard lane change/parallel parking}
%
In figure \ref{fig:Compare_traj1} we can see the trajectory obtained with the two approaches is pretty similar. However, the solution with PINS is slightly shorter than the Duboids approach. Thus PINS obtain the actual optimal solutions. On the other hand, The Duboids methods allow the manoeuvre to be stable at The boundary of the constraint as shown in figure \ref{fig:Compare_curv1}. Furthermore, the jerk on Duboids does not show any ringing or Fuller's phenomena in contrast with the PINS solution (Figure \ref{fig:Compare_jerk1}). This is a major advantage if we plan to use the trajectory for control purposes.
%
\begin{figure}[htb!]
  \centering
  
%
\begin{tikzpicture}[scale = 0.7]

\begin{axis}[%
width=\linewidth,
height=0.736\linewidth,
at={(0\linewidth,0\linewidth)},
scale only axis,
xmin=0,
xmax=40,
xlabel style={font=\color{white!15!black}},
xlabel={$x(m)$},
ymin=-5.23963145726288,
ymax=25.2396314572629,
ylabel style={font=\color{white!15!black}},
ylabel={$y(m)$},
axis background/.style={fill=white},
title style={font=\bfseries},
title={$L_{PINS}$ = 45.1439 $L_{DUB}$ = 45.1441, $k_{max}$ = 0.15, $J_{max}$ = 0.1},
axis x line*=bottom,
axis y line*=left,
xmajorgrids,
xminorgrids,
ymajorgrids,
yminorgrids,
legend style={at={(0.03,0.97)}, anchor=north west, legend cell align=left, align=left, draw=white!15!black}
]
\addplot [color=green, dashdotted, line width=2.0pt]
  table[row sep=crcr]{%
0	0\\
0.451436584219408	0.00153337829200633\\
0.902732553301484	0.0122656616699568\\
1.35318518300689	0.0413766551689649\\
1.80104133025803	0.0974795696296542\\
2.2440832982187	0.183707012407246\\
2.68027534112719	0.299715024913916\\
3.10761807219581	0.444971857680038\\
3.52417409303697	0.618764152649094\\
3.92957874534124	0.817263862843862\\
4.32659123134104	1.03212276277267\\
4.71916220671095	1.25502493685617\\
5.11097854315331	1.47925513182825\\
5.50279487959566	1.70348532680032\\
5.89461121603802	1.9277155217724\\
6.28642755248037	2.15194571674447\\
6.67824388892273	2.37617591171655\\
7.07006022536508	2.60040610668862\\
7.46187656180744	2.8246363016607\\
7.8536928982498	3.04886649663277\\
8.24550923469215	3.27309669160485\\
8.63732557113451	3.49732688657692\\
9.02914190757686	3.721557081549\\
9.42095824401922	3.94578727652107\\
9.81277458046157	4.17001747149315\\
10.2045909169039	4.39424766646522\\
10.5964072533463	4.6184778614373\\
10.9882235897886	4.84270805640937\\
11.380039926231	5.06693825138144\\
11.7718562626733	5.29116844635352\\
12.1636725991157	5.51539864132559\\
12.5554889355581	5.73962883629767\\
12.9473052720004	5.96385903126974\\
13.3391216084428	6.18808922624182\\
13.7309379448851	6.41231942121389\\
14.1227542813275	6.63654961618597\\
14.5145706177698	6.86077981115804\\
14.9063869542122	7.08501000613012\\
15.2982032906546	7.30924020110219\\
15.6900196270969	7.53347039607427\\
16.0818359635393	7.75770059104634\\
16.4736522999816	7.98193078601842\\
16.865468636424	8.20616098099049\\
17.2572849728663	8.43039117596257\\
17.6491013093087	8.65462137093464\\
18.040917645751	8.87885156590671\\
18.4327339821934	9.10308176087879\\
18.8245503186358	9.32731195585087\\
19.2163666550781	9.55154215082294\\
19.6081829915205	9.77577234579501\\
19.9999993279628	10.0000025407671\\
20.3918156644052	10.2242327357392\\
20.7836320008475	10.4484629307112\\
21.1754483372899	10.6726931256833\\
21.5672646737322	10.8969233206554\\
21.9590810101746	11.1211535156275\\
22.350897346617	11.3453837105995\\
22.7427136830593	11.5696139055716\\
23.1345300195017	11.7938441005437\\
23.526346355944	12.0180742955158\\
23.9181626923864	12.2423044904878\\
24.3099790288287	12.4665346854599\\
24.7017953652711	12.690764880432\\
25.0936117017134	12.9149950754041\\
25.4854280381558	13.1392252703761\\
25.8772443745981	13.3634554653482\\
26.2690607110405	13.5876856603203\\
26.6608770474829	13.8119158552924\\
27.0526933839252	14.0361460502644\\
27.4445097203676	14.2603762452365\\
27.8363260568099	14.4846064402086\\
28.2281423932523	14.7088366351807\\
28.6199587296946	14.9330668301527\\
29.011775066137	15.1572970251248\\
29.4035914025793	15.3815272200969\\
29.7954077390217	15.605757415069\\
30.1872240754641	15.829987610041\\
30.5790404119064	16.0542178050131\\
30.9708567483488	16.2784479999852\\
31.3626730847911	16.5026781949573\\
31.7544894212335	16.7269083899293\\
32.1463057576758	16.9511385849014\\
32.5381220941182	17.1753687798735\\
32.9299384305605	17.3995989748455\\
33.3217547670029	17.6238291698176\\
33.7135711034453	17.8480593647897\\
34.1053874398876	18.0722895597618\\
34.49720377633	18.2965197547338\\
34.8890201127723	18.5207499497059\\
35.2808364492147	18.744980144678\\
35.6734074674929	18.9678822428924\\
36.0704200777667	19.1827409122551\\
36.475824921426	19.3812402299503\\
36.8923811620748	19.5550319974477\\
37.3197240799956	19.7002882804971\\
37.7559162721318	19.8162957319036\\
38.1989583510117	19.9025226047697\\
38.646814570431	19.9586249431261\\
39.0972672375832	19.9877353571808\\
39.5485632204705	19.9984670600296\\
39.999999806662	19.9999998576117\\
};
\addlegendentry{Duboids}

\addplot [color=blue, line width=2.0pt, only marks, mark size=2.5pt, mark=*, mark options={solid, fill=blue, blue}, forget plot]
  table[row sep=crcr]{%
0	0\\
};
\addplot [color=blue, line width=2.0pt, only marks, mark size=2.5pt, mark=*, mark options={solid, fill=blue, blue}, forget plot]
  table[row sep=crcr]{%
40	20\\
};
\addplot [color=dodgerblue, dotted, line width=2.0pt]
  table[row sep=crcr]{%
0	0\\
0.0451439494672037	2.30005600159139e-06\\
0.0902878975281704	1.38003358302327e-05\\
0.135431837151716	4.37010613138671e-05\\
0.180575750056764	0.00010120244466115\\
0.225719601087421	0.000195504669431919\\
0.270863332588084	0.000335807854310382\\
0.316006858778659	0.000531311993949096\\
0.361150060129942	0.000791216871151912\\
0.406292777739323	0.00112472193466436\\
0.451434807706959	0.00154102613684039\\
0.496575895512668	0.00204932772545521\\
0.54171573039383	0.00265882398393508\\
0.586853939724675	0.00337871091427667\\
0.631990083397404	0.00421818285693089\\
0.677123648205686	0.00518643204192912\\
0.722254042231186	0.00629264806553381\\
0.767380589233844	0.00754601728670037\\
0.812502523046801	0.00895572213764356\\
0.857618981976928	0.0105309403428089\\
0.902729003212085	0.0122808440405585\\
0.947831517236369	0.0142145988018911\\
0.992925342254743	0.0163413625405273\\
1.03800917862861	0.0186702843087036\\
1.08308160332404	0.0212105029730306\\
1.12814106437456	0.0239711457647819\\
1.17318587536061	0.026961326698984\\
1.21821420990778	0.0301901448566669\\
1.26322409620659	0.0336666825245888\\
1.30821341155615	0.0374000031866074\\
1.353179876935	0.0413991493604699\\
1.39812105160225	0.0456731402724722\\
1.44303432773334	0.0502309693565426\\
1.48791692510246	0.0550816014998435\\
1.53276603266101	0.060232691674661\\
1.57757923908835	0.0656872960130874\\
1.62235449047873	0.0714451638229565\\
1.66708973469912	0.0775060303723118\\
1.71178292145205	0.0838696171483108\\
1.75643200240623	0.090535631604836\\
1.80103493125097	0.0975037674573684\\
1.84558966383604	0.104773704393863\\
1.89009415821562	0.112345108411987\\
1.93454637479967	0.120217631486098\\
1.97894427638486	0.128390911954736\\
2.02328582832032	0.13686457413483\\
2.06756899852237	0.145638228769269\\
2.11179175765824	0.154711472577221\\
2.15595207914053	0.164083888774143\\
2.20004793933369	0.173755046544168\\
2.24407731752276	0.183724501648237\\
2.28803819614883	0.19399179580061\\
2.33192856074505	0.204556457385287\\
2.37574640020934	0.21541800071342\\
2.41948970669819	0.226575926874252\\
2.46315647594775	0.238029722842976\\
2.50674470711268	0.24977886250086\\
2.55025240315081	0.261822805552918\\
2.59367757058951	0.274160998763796\\
2.63701821999496	0.286792874630098\\
2.68027236564157	0.299717852895809\\
2.72343802609574	0.312935338902739\\
2.76651322375323	0.326444725474876\\
2.80949598558079	0.340245390837951\\
2.85238434247006	0.354336701001528\\
2.89517633020199	0.368718007087491\\
2.93786998853918	0.383388648401857\\
2.98046336251427	0.398347948927137\\
3.02295450113696	0.413595221382719\\
3.06534145917239	0.429129762486048\\
3.10762229525553	0.444950858494177\\
3.14979507445171	0.461057778549152\\
3.19185786540341	0.47744978257982\\
3.23380874424381	0.494126111411599\\
3.27564579001384	0.511085998794013\\
3.3173670912554	0.52832865524534\\
3.35897073782266	0.545853288510971\\
3.40045483452144	0.563659071275962\\
3.44181748252866	0.581745185524694\\
3.48305691875304	0.600110508246684\\
3.52417312387076	0.618750085911047\\
3.56516793256497	0.637655160333337\\
3.60604353511476	0.65681661369626\\
3.64680228362084	0.676225390035985\\
3.68744668584134	0.695872495342523\\
3.72797939924859	0.715748997113039\\
3.76840322518083	0.735846023653986\\
3.80872110306345	0.756154763197094\\
3.84893610469155	0.77666646285461\\
3.88905142857006	0.797372427427877\\
3.92907039430941	0.818264018079153\\
3.96899643707493	0.839332650874698\\
4.00883310208882	0.860569795206253\\
4.04858403918306	0.881966972097544\\
4.08825299740198	0.903515752402092\\
4.12784381965265	0.925207754898477\\
4.16736043740165	0.947034644289023\\
4.20680686541609	0.968988129107823\\
4.24618719654708	0.991059959543949\\
4.28550559655339	1.0132419251857\\
4.32476629896297	1.03552585269173\\
4.36397359996995	1.05790360339496\\
4.40313185336426	1.08036707084534\\
4.44224546549116	1.1029081782976\\
4.48131889023732	1.12551887615047\\
4.52035662404005	1.1481911393444\\
4.55936320091562	1.17091696472519\\
4.59834318750202	1.19368836838223\\
4.63730117811058	1.21649738297144\\
4.67624178977931	1.23933605503557\\
4.71516965731851	1.26219644233847\\
4.75408942833525	1.28507061123686\\
4.79300575821659	1.30795063412483\\
4.83192330503866	1.33082858700834\\
4.87084672434425	1.35369654730979\\
4.9097806636797	1.37654659209337\\
4.94872975665856	1.39937079711592\\
4.9876986159825	1.42216123769479\\
5.02669182366745	1.44490999443501\\
5.0657139100489	1.46760917841211\\
5.10476884301339	1.49025180171258\\
5.14385781909431	1.51283560404374\\
5.18297805621419	1.53536521181822\\
5.22212503194231	1.55784832704121\\
5.26129423212693	1.5802927012959\\
5.30048117290074	1.60270608684064\\
5.33968141415188	1.62509620217398\\
5.37889058123564	1.64747068307043\\
5.41810441401168	1.66983698580021\\
5.45731890091532	1.69220214163854\\
5.49653071805034	1.71457197793113\\
5.53573809873965	1.73694958902506\\
5.57494120030349	1.75933469582764\\
5.61414127564248	1.7817251016347\\
5.65333968296528	1.80411842747397\\
5.69253745386882	1.82651286729686\\
5.73173519381684	1.8489073613022\\
5.77093315480288	1.87130146841378\\
5.81013137141132	1.89369512808578\\
5.84932978177252	1.91608844860647\\
5.88852830431806	1.93848157275284\\
5.92772687382091	1.96087461470157\\
5.96692545074817	1.98326764365402\\
6.00612401735404	2.00566069067388\\
6.0453225697065	2.02805376264399\\
6.08452111046941	2.05044685490134\\
6.12371964411437	2.07283995961849\\
6.16291817450314	2.0952330700355\\
6.20211670411679	2.11762618180932\\
6.24131523416031	2.14001929283068\\
6.28051376498072	2.16241240249212\\
6.31971229648504	2.1848055109564\\
6.35891082843464	2.20719861864121\\
6.39810936060412	2.22959172594114\\
6.43730789283985	2.25198483312509\\
6.47650642506183	2.27437794033311\\
6.51570495724265	2.29677104761319\\
6.55490348938368	2.31916415496292\\
6.59410202149722	2.34155726236076\\
6.63330055359623	2.36395036978404\\
6.6724990856901	2.38634347721633\\
6.71169761778403	2.40873658464849\\
6.75089614988007	2.43112969207697\\
6.79009468197839	2.45352279950147\\
6.82929321407838	2.47591590692303\\
6.86849174617931	2.49830901434294\\
6.90769027828063	2.52070212176219\\
6.94688881038198	2.54309522918137\\
6.98608734248323	2.56548833660072\\
7.02528587458436	2.5878814440203\\
7.06448440668538	2.61027455144006\\
7.10368293878634	2.63266765885992\\
7.14288147088728	2.65506076627983\\
7.18208000298821	2.67745387369975\\
7.22127853508915	2.69984698111966\\
7.26047706719009	2.72224008853955\\
7.29967559929104	2.74463319595944\\
7.33887413139199	2.76702630337932\\
7.37807266349294	2.7894194107992\\
7.4172711955939	2.81181251821907\\
7.45646972769485	2.83420562563895\\
7.49566825979581	2.85659873305883\\
7.53486679189676	2.87899184047871\\
7.57406532399771	2.90138494789858\\
7.61326385609867	2.92377805531846\\
7.65246238819962	2.94617116273834\\
7.69166092030058	2.96856427015822\\
7.73085945240153	2.9909573775781\\
7.77005798450248	3.01335048499798\\
7.80925651660344	3.03574359241785\\
7.84845504870439	3.05813669983773\\
7.88765358080534	3.08052980725761\\
7.9268521129063	3.10292291467749\\
7.96605064500725	3.12531602209737\\
8.0052491771082	3.14770912951725\\
8.04444770920916	3.17010223693712\\
8.08364624131011	3.192495344357\\
8.12284477341107	3.21488845177688\\
8.16204330551202	3.23728155919676\\
8.20124183761297	3.25967466661664\\
8.24044036971393	3.28206777403652\\
8.27963890181488	3.30446088145639\\
8.31883743391583	3.32685398887627\\
8.35803596601679	3.34924709629615\\
8.39723449811774	3.37164020371603\\
8.43643303021869	3.39403331113591\\
8.47563156231965	3.41642641855578\\
8.5148300944206	3.43881952597566\\
8.55402862652156	3.46121263339554\\
8.59322715862251	3.48360574081542\\
8.63242569072346	3.5059988482353\\
8.67162422282441	3.52839195565518\\
8.71082275492537	3.55078506307505\\
8.75002128702632	3.57317817049493\\
8.78921981912728	3.59557127791481\\
8.82841835122823	3.61796438533469\\
8.86761688332918	3.64035749275457\\
8.90681541543014	3.66275060017445\\
8.94601394753109	3.68514370759432\\
8.98521247963204	3.7075368150142\\
9.024411011733	3.72992992243408\\
9.06360954383395	3.75232302985396\\
9.1028080759349	3.77471613727384\\
9.14200660803586	3.79710924469372\\
9.18120514013681	3.81950235211359\\
9.22040367223777	3.84189545953347\\
9.25960220433872	3.86428856695335\\
9.29880073643967	3.88668167437323\\
9.33799926854063	3.90907478179311\\
9.37719780064158	3.93146788921299\\
9.41639633274253	3.95386099663286\\
9.45559486484349	3.97625410405274\\
9.49479339694444	3.99864721147262\\
9.5339919290454	4.0210403188925\\
9.57319046114635	4.04343342631238\\
9.6123889932473	4.06582653373225\\
9.65158752534825	4.08821964115213\\
9.69078605744921	4.11061274857201\\
9.72998458955016	4.13300585599189\\
9.76918312165112	4.15539896341177\\
9.80838165375207	4.17779207083165\\
9.84758018585302	4.20018517825152\\
9.88677871795398	4.2225782856714\\
9.92597725005493	4.24497139309128\\
9.96517578215588	4.26736450051116\\
10.0043743142568	4.28975760793104\\
10.0435728463578	4.31215071535092\\
10.0827713784587	4.33454382277079\\
10.1219699105597	4.35693693019067\\
10.1611684426607	4.37933003761055\\
10.2003669747616	4.40172314503043\\
10.2395655068626	4.42411625245031\\
10.2787640389635	4.44650935987019\\
10.3179625710645	4.46890246729006\\
10.3571611031654	4.49129557470994\\
10.3963596352664	4.51368868212982\\
10.4355581673673	4.5360817895497\\
10.4747566994683	4.55847489696958\\
10.5139552315692	4.58086800438946\\
10.5531537636702	4.60326111180933\\
10.5923522957711	4.62565421922921\\
10.6315508278721	4.64804732664909\\
10.670749359973	4.67044043406897\\
10.709947892074	4.69283354148885\\
10.749146424175	4.71522664890873\\
10.7883449562759	4.7376197563286\\
10.8275434883769	4.76001286374848\\
10.8667420204778	4.78240597116836\\
10.9059405525788	4.80479907858824\\
10.9451390846797	4.82719218600812\\
10.9843376167807	4.849585293428\\
11.0235361488816	4.87197840084787\\
11.0627346809826	4.89437150826775\\
11.1019332130835	4.91676461568763\\
11.1411317451845	4.93915772310751\\
11.1803302772854	4.96155083052739\\
11.2195288093864	4.98394393794726\\
11.2587273414874	5.00633704536714\\
11.2979258735883	5.02873015278702\\
11.3371244056893	5.0511232602069\\
11.3763229377902	5.07351636762678\\
11.4155214698912	5.09590947504666\\
11.4547200019921	5.11830258246653\\
11.4939185340931	5.14069568988641\\
11.533117066194	5.16308879730629\\
11.572315598295	5.18548190472617\\
11.6115141303959	5.20787501214605\\
11.6507126624969	5.23026811956592\\
11.6899111945978	5.2526612269858\\
11.7291097266988	5.27505433440568\\
11.7683082587997	5.29744744182556\\
11.8075067909007	5.31984054924544\\
11.8467053230017	5.34223365666532\\
11.8859038551026	5.3646267640852\\
11.9251023872036	5.38701987150507\\
11.9643009193045	5.40941297892495\\
12.0034994514055	5.43180608634483\\
12.0426979835064	5.45419919376471\\
12.0818965156074	5.47659230118459\\
12.1210950477083	5.49898540860447\\
12.1602935798093	5.52137851602434\\
12.1994921119102	5.54377162344422\\
12.2386906440112	5.5661647308641\\
12.2778891761121	5.58855783828398\\
12.3170877082131	5.61095094570386\\
12.3562862403141	5.63334405312373\\
12.395484772415	5.65573716054361\\
12.434683304516	5.67813026796349\\
12.4738818366169	5.70052337538337\\
12.5130803687179	5.72291648280325\\
12.5522789008188	5.74530959022313\\
12.5914774329198	5.767702697643\\
12.6306759650207	5.79009580506288\\
12.6698744971217	5.81248891248276\\
12.7090730292226	5.83488201990264\\
12.7482715613236	5.85727512732252\\
12.7874700934245	5.8796682347424\\
12.8266686255255	5.90206134216227\\
12.8658671576264	5.92445444958215\\
12.9050656897274	5.94684755700203\\
12.9442642218284	5.96924066442191\\
12.9834627539293	5.99163377184179\\
13.0226612860303	6.01402687926167\\
13.0618598181312	6.03641998668154\\
13.1010583502322	6.05881309410142\\
13.1402568823331	6.0812062015213\\
13.1794554144341	6.10359930894118\\
13.218653946535	6.12599241636106\\
13.257852478636	6.14838552378094\\
13.2970510107369	6.17077863120081\\
13.3362495428379	6.19317173862069\\
13.3754480749388	6.21556484604057\\
13.4146466070398	6.23795795346045\\
13.4538451391408	6.26035106088033\\
13.4930436712417	6.28274416830021\\
13.5322422033427	6.30513727572008\\
13.5714407354436	6.32753038313996\\
13.6106392675446	6.34992349055984\\
13.6498377996455	6.37231659797972\\
13.6890363317465	6.3947097053996\\
13.7282348638474	6.41710281281948\\
13.7674333959484	6.43949592023935\\
13.8066319280493	6.46188902765923\\
13.8458304601503	6.48428213507911\\
13.8850289922512	6.50667524249899\\
13.9242275243522	6.52906834991887\\
13.9634260564531	6.55146145733874\\
14.0026245885541	6.57385456475862\\
14.0418231206551	6.5962476721785\\
14.081021652756	6.61864077959838\\
14.120220184857	6.64103388701826\\
14.1594187169579	6.66342699443814\\
14.1986172490589	6.68582010185801\\
14.2378157811598	6.70821320927789\\
14.2770143132608	6.73060631669777\\
14.3162128453617	6.75299942411765\\
14.3554113774627	6.77539253153753\\
14.3946099095636	6.79778563895741\\
14.4338084416646	6.82017874637728\\
14.4730069737655	6.84257185379716\\
14.5122055058665	6.86496496121704\\
14.5514040379675	6.88735806863692\\
14.5906025700684	6.9097511760568\\
14.6298011021694	6.93214428347668\\
14.6689996342703	6.95453739089655\\
14.7081981663713	6.97693049831643\\
14.7473966984722	6.99932360573631\\
14.7865952305732	7.02171671315619\\
14.8257937626741	7.04410982057607\\
14.8649922947751	7.06650292799594\\
14.904190826876	7.08889603541582\\
14.943389358977	7.1112891428357\\
14.9825878910779	7.13368225025558\\
15.0217864231789	7.15607535767546\\
15.0609849552798	7.17846846509534\\
15.1001834873808	7.20086157251521\\
15.1393820194818	7.22325467993509\\
15.1785805515827	7.24564778735497\\
15.2177790836837	7.26804089477485\\
15.2569776157846	7.29043400219473\\
15.2961761478856	7.31282710961461\\
15.3353746799865	7.33522021703448\\
15.3745732120875	7.35761332445436\\
15.4137717441884	7.38000643187424\\
15.4529702762894	7.40239953929412\\
15.4921688083903	7.424792646714\\
15.5313673404913	7.44718575413388\\
15.5705658725922	7.46957886155375\\
15.6097644046932	7.49197196897363\\
15.6489629367942	7.51436507639351\\
15.6881614688951	7.53675818381339\\
15.7273600009961	7.55915129123327\\
15.766558533097	7.58154439865315\\
15.805757065198	7.60393750607302\\
15.8449555972989	7.6263306134929\\
15.8841541293999	7.64872372091278\\
15.9233526615008	7.67111682833266\\
15.9625511936018	7.69350993575254\\
16.0017497257027	7.71590304317242\\
16.0409482578037	7.73829615059229\\
16.0801467899046	7.76068925801217\\
16.1193453220056	7.78308236543205\\
16.1585438541066	7.80547547285193\\
16.1977423862075	7.82786858027181\\
16.2369409183085	7.85026168769169\\
16.2761394504094	7.87265479511156\\
16.3153379825104	7.89504790253144\\
16.3545365146113	7.91744100995132\\
16.3937350467123	7.9398341173712\\
16.4329335788132	7.96222722479108\\
16.4721321109142	7.98462033221095\\
16.5113306430151	8.00701343963083\\
16.5505291751161	8.02940654705071\\
16.589727707217	8.05179965447059\\
16.628926239318	8.07419276189047\\
16.6681247714189	8.09658586931035\\
16.7073233035199	8.11897897673022\\
16.7465218356209	8.1413720841501\\
16.7857203677218	8.16376519156998\\
16.8249188998228	8.18615829898986\\
16.8641174319237	8.20855140640974\\
16.9033159640247	8.23094451382961\\
16.9425144961256	8.25333762124949\\
16.9817130282266	8.27573072866937\\
17.0209115603275	8.29812383608925\\
17.0601100924285	8.32051694350913\\
17.0993086245294	8.34291005092901\\
17.1385071566304	8.36530315834889\\
17.1777056887313	8.38769626576876\\
17.2169042208323	8.41008937318864\\
17.2561027529332	8.43248248060852\\
17.2953012850342	8.4548755880284\\
17.3344998171352	8.47726869544828\\
17.3736983492361	8.49966180286816\\
17.4128968813371	8.52205491028803\\
17.452095413438	8.54444801770791\\
17.491293945539	8.56684112512779\\
17.5304924776399	8.58923423254767\\
17.5696910097409	8.61162733996755\\
17.6088895418418	8.63402044738742\\
17.6480880739428	8.6564135548073\\
17.6872866060437	8.67880666222718\\
17.7264851381447	8.70119976964706\\
17.7656836702456	8.72359287706694\\
17.8048822023466	8.74598598448682\\
17.8440807344476	8.7683790919067\\
17.8832792665485	8.79077219932657\\
17.9224777986495	8.81316530674645\\
17.9616763307504	8.83555841416633\\
18.0008748628514	8.85795152158621\\
18.0400733949523	8.88034462900609\\
18.0792719270533	8.90273773642596\\
18.1184704591542	8.92513084384584\\
18.1576689912552	8.94752395126572\\
18.1968675233561	8.9699170586856\\
18.2360660554571	8.99231016610548\\
18.275264587558	9.01470327352536\\
18.314463119659	9.03709638094523\\
18.3536616517599	9.05948948836511\\
18.3928601838609	9.08188259578499\\
18.4320587159619	9.10427570320487\\
18.4712572480628	9.12666881062475\\
18.5104557801638	9.14906191804462\\
18.5496543122647	9.1714550254645\\
18.5888528443657	9.19384813288438\\
18.6280513764666	9.21624124030426\\
18.6672499085676	9.23863434772414\\
18.7064484406685	9.26102745514402\\
18.7456469727695	9.2834205625639\\
18.7848455048704	9.30581366998377\\
18.8240440369714	9.32820677740365\\
18.8632425690723	9.35059988482353\\
18.9024411011733	9.37299299224341\\
18.9416396332743	9.39538609966329\\
18.9808381653752	9.41777920708316\\
19.0200366974762	9.44017231450304\\
19.0592352295771	9.46256542192292\\
19.0984337616781	9.4849585293428\\
19.137632293779	9.50735163676268\\
19.17683082588	9.52974474418256\\
19.2160293579809	9.55213785160243\\
19.2552278900819	9.57453095902231\\
19.2944264221828	9.59692406644219\\
19.3336249542838	9.61931717386207\\
19.3728234863847	9.64171028128195\\
19.4120220184857	9.66410338870183\\
19.4512205505867	9.6864964961217\\
19.4904190826876	9.70888960354158\\
19.5296176147886	9.73128271096146\\
19.5688161468895	9.75367581838134\\
19.6080146789905	9.77606892580122\\
19.6472132110914	9.7984620332211\\
19.6864117431924	9.82085514064097\\
19.7256102752933	9.84324824806085\\
19.7648088073943	9.86564135548073\\
19.8040073394952	9.88803446290061\\
19.8432058715962	9.91042757032049\\
19.8824044036971	9.93282067774036\\
19.9216029357981	9.95521378516024\\
19.960801467899	9.97760689258012\\
20	10\\
20.039198532101	10.0223931074199\\
20.0783970642019	10.0447862148398\\
20.1175955963029	10.0671793222596\\
20.1567941284038	10.0895724296795\\
20.1959926605048	10.1119655370994\\
20.2351911926057	10.1343586445193\\
20.2743897247067	10.1567517519391\\
20.3135882568076	10.179144859359\\
20.3527867889086	10.2015379667789\\
20.3919853210095	10.2239310741988\\
20.4311838531105	10.2463241816187\\
20.4703823852114	10.2687172890385\\
20.5095809173124	10.2911103964584\\
20.5487794494133	10.3135035038783\\
20.5879779815143	10.3358966112982\\
20.6271765136153	10.3582897187181\\
20.6663750457162	10.3806828261379\\
20.7055735778172	10.4030759335578\\
20.7447721099181	10.4254690409777\\
20.7839706420191	10.4478621483976\\
20.82316917412	10.4702552558174\\
20.862367706221	10.4926483632373\\
20.9015662383219	10.5150414706572\\
20.9407647704229	10.5374345780771\\
20.9799633025238	10.559827685497\\
21.0191618346248	10.5822207929168\\
21.0583603667257	10.6046139003367\\
21.0975588988267	10.6270070077566\\
21.1367574309277	10.6494001151765\\
21.1759559630286	10.6717932225963\\
21.2151544951296	10.6941863300162\\
21.2543530272305	10.7165794374361\\
21.2935515593315	10.738972544856\\
21.3327500914324	10.7613656522759\\
21.3719486235334	10.7837587596957\\
21.4111471556343	10.8061518671156\\
21.4503456877353	10.8285449745355\\
21.4895442198362	10.8509380819554\\
21.5287427519372	10.8733311893753\\
21.5679412840381	10.8957242967951\\
21.6071398161391	10.918117404215\\
21.6463383482401	10.9405105116349\\
21.685536880341	10.9629036190548\\
21.724735412442	10.9852967264746\\
21.7639339445429	11.0076898338945\\
21.8031324766439	11.0300829413144\\
21.8423310087448	11.0524760487343\\
21.8815295408458	11.0748691561542\\
21.9207280729467	11.097262263574\\
21.9599266050477	11.1196553709939\\
21.9991251371486	11.1420484784138\\
22.0383236692496	11.1644415858337\\
22.0775222013505	11.1868346932535\\
22.1167207334515	11.2092278006734\\
22.1559192655524	11.2316209080933\\
22.1951177976534	11.2540140155132\\
22.2343163297544	11.2764071229331\\
22.2735148618553	11.2988002303529\\
22.3127133939563	11.3211933377728\\
22.3519119260572	11.3435864451927\\
22.3911104581582	11.3659795526126\\
22.4303089902591	11.3883726600325\\
22.4695075223601	11.4107657674523\\
22.508706054461	11.4331588748722\\
22.547904586562	11.4555519822921\\
22.5871031186629	11.477945089712\\
22.6263016507639	11.5003381971318\\
22.6655001828648	11.5227313045517\\
22.7046987149658	11.5451244119716\\
22.7438972470667	11.5675175193915\\
22.7830957791677	11.5899106268114\\
22.8222943112687	11.6123037342312\\
22.8614928433696	11.6346968416511\\
22.9006913754706	11.657089949071\\
22.9398899075715	11.6794830564909\\
22.9790884396725	11.7018761639107\\
23.0182869717734	11.7242692713306\\
23.0574855038744	11.7466623787505\\
23.0966840359753	11.7690554861704\\
23.1358825680763	11.7914485935903\\
23.1750811001772	11.8138417010101\\
23.2142796322782	11.83623480843\\
23.2534781643791	11.8586279158499\\
23.2926766964801	11.8810210232698\\
23.3318752285811	11.9034141306897\\
23.371073760682	11.9258072381095\\
23.410272292783	11.9482003455294\\
23.4494708248839	11.9705934529493\\
23.4886693569849	11.9929865603692\\
23.5278678890858	12.015379667789\\
23.5670664211868	12.0377727752089\\
23.6062649532877	12.0601658826288\\
23.6454634853887	12.0825589900487\\
23.6846620174896	12.1049520974686\\
23.7238605495906	12.1273452048884\\
23.7630590816915	12.1497383123083\\
23.8022576137925	12.1721314197282\\
23.8414561458934	12.1945245271481\\
23.8806546779944	12.2169176345679\\
23.9198532100954	12.2393107419878\\
23.9590517421963	12.2617038494077\\
23.9982502742973	12.2840969568276\\
24.0374488063982	12.3064900642475\\
24.0766473384992	12.3288831716673\\
24.1158458706001	12.3512762790872\\
24.1550444027011	12.3736693865071\\
24.194242934802	12.396062493927\\
24.233441466903	12.4184556013469\\
24.2726399990039	12.4408487087667\\
24.3118385311049	12.4632418161866\\
24.3510370632058	12.4856349236065\\
24.3902355953068	12.5080280310264\\
24.4294341274078	12.5304211384462\\
24.4686326595087	12.5528142458661\\
24.5078311916097	12.575207353286\\
24.5470297237106	12.5976004607059\\
24.5862282558116	12.6199935681258\\
24.6254267879125	12.6423866755456\\
24.6646253200135	12.6647797829655\\
24.7038238521144	12.6871728903854\\
24.7430223842154	12.7095659978053\\
24.7822209163163	12.7319591052252\\
24.8214194484173	12.754352212645\\
24.8606179805182	12.7767453200649\\
24.8998165126192	12.7991384274848\\
24.9390150447202	12.8215315349047\\
24.9782135768211	12.8439246423245\\
25.0174121089221	12.8663177497444\\
25.056610641023	12.8887108571643\\
25.095809173124	12.9111039645842\\
25.1350077052249	12.9334970720041\\
25.1742062373259	12.9558901794239\\
25.2134047694268	12.9782832868438\\
25.2526033015278	13.0006763942637\\
25.2918018336287	13.0230695016836\\
25.3310003657297	13.0454626091034\\
25.3701988978306	13.0678557165233\\
25.4093974299316	13.0902488239432\\
25.4485959620325	13.1126419313631\\
25.4877944941335	13.135035038783\\
25.5269930262345	13.1574281462028\\
25.5661915583354	13.1798212536227\\
25.6053900904364	13.2022143610426\\
25.6445886225373	13.2246074684625\\
25.6837871546383	13.2470005758824\\
25.7229856867392	13.2693936833022\\
25.7621842188402	13.2917867907221\\
25.8013827509411	13.314179898142\\
25.8405812830421	13.3365730055619\\
25.879779815143	13.3589661129817\\
25.918978347244	13.3813592204016\\
25.9581768793449	13.4037523278215\\
25.9973754114459	13.4261454352414\\
26.0365739435469	13.4485385426613\\
26.0757724756478	13.4709316500811\\
26.1149710077488	13.493324757501\\
26.1541695398497	13.5157178649209\\
26.1933680719507	13.5381109723408\\
26.2325666040516	13.5605040797606\\
26.2717651361526	13.5828971871805\\
26.3109636682535	13.6052902946004\\
26.3501622003545	13.6276834020203\\
26.3893607324554	13.6500765094402\\
26.4285592645564	13.67246961686\\
26.4677577966573	13.6948627242799\\
26.5069563287583	13.7172558316998\\
26.5461548608593	13.7396489391197\\
26.5853533929602	13.7620420465396\\
26.6245519250612	13.7844351539594\\
26.6637504571621	13.8068282613793\\
26.7029489892631	13.8292213687992\\
26.742147521364	13.8516144762191\\
26.781346053465	13.8740075836389\\
26.8205445855659	13.8964006910588\\
26.8597431176669	13.9187937984787\\
26.8989416497678	13.9411869058986\\
26.9381401818688	13.9635800133185\\
26.9773387139697	13.9859731207383\\
27.0165372460707	14.0083662281582\\
27.0557357781716	14.0307593355781\\
27.0949343102726	14.053152442998\\
27.1341328423736	14.0755455504178\\
27.1733313744745	14.0979386578377\\
27.2125299065755	14.1203317652576\\
27.2517284386764	14.1427248726775\\
27.2909269707774	14.1651179800974\\
27.3301255028783	14.1875110875172\\
27.3693240349793	14.2099041949371\\
27.4085225670802	14.232297302357\\
27.4477210991812	14.2546904097769\\
27.4869196312821	14.2770835171968\\
27.5261181633831	14.2994766246166\\
27.565316695484	14.3218697320365\\
27.604515227585	14.3442628394564\\
27.6437137596859	14.3666559468763\\
27.6829122917869	14.3890490542961\\
27.7221108238879	14.411442161716\\
27.7613093559888	14.4338352691359\\
27.8005078880898	14.4562283765558\\
27.8397064201907	14.4786214839757\\
27.8789049522917	14.5010145913955\\
27.9181034843926	14.5234076988154\\
27.9573020164936	14.5458008062353\\
27.9965005485945	14.5681939136552\\
28.0356990806955	14.590587021075\\
28.0748976127964	14.6129801284949\\
28.1140961448974	14.6353732359148\\
28.1532946769983	14.6577663433347\\
28.1924932090993	14.6801594507546\\
28.2316917412003	14.7025525581744\\
28.2708902733012	14.7249456655943\\
28.3100888054022	14.7473387730142\\
28.3492873375031	14.7697318804341\\
28.3884858696041	14.792124987854\\
28.427684401705	14.8145180952738\\
28.466882933806	14.8369112026937\\
28.5060814659069	14.8593043101136\\
28.5452799980079	14.8816974175335\\
28.5844785301088	14.9040905249533\\
28.6236770622098	14.9264836323732\\
28.6628755943107	14.9488767397931\\
28.7020741264117	14.971269847213\\
28.7412726585126	14.9936629546329\\
28.7804711906136	15.0160560620527\\
28.8196697227146	15.0384491694726\\
28.8588682548155	15.0608422768925\\
28.8980667869165	15.0832353843124\\
28.9372653190174	15.1056284917322\\
28.9764638511184	15.1280215991521\\
29.0156623832193	15.150414706572\\
29.0548609153203	15.1728078139919\\
29.0940594474212	15.1952009214118\\
29.1332579795222	15.2175940288316\\
29.1724565116231	15.2399871362515\\
29.2116550437241	15.2623802436714\\
29.250853575825	15.2847733510913\\
29.290052107926	15.3071664585112\\
29.329250640027	15.329559565931\\
29.3684491721279	15.3519526733509\\
29.4076477042289	15.3743457807708\\
29.4468462363298	15.3967388881907\\
29.4860447684308	15.4191319956105\\
29.5252433005317	15.4415251030304\\
29.5644418326327	15.4639182104503\\
29.6036403647336	15.4863113178702\\
29.6428388968346	15.5087044252901\\
29.6820374289355	15.5310975327099\\
29.7212359610365	15.5534906401298\\
29.7604344931374	15.5758837475497\\
29.7996330252384	15.5982768549696\\
29.8388315573393	15.6206699623894\\
29.8780300894403	15.6430630698093\\
29.9172286215413	15.6654561772292\\
29.9564271536422	15.6878492846491\\
29.9956256857432	15.710242392069\\
30.0348242178441	15.7326354994888\\
30.0740227499451	15.7550286069087\\
30.113221282046	15.7774217143286\\
30.152419814147	15.7998148217485\\
30.1916183462479	15.8222079291684\\
30.2308168783489	15.8446010365882\\
30.2700154104498	15.8669941440081\\
30.3092139425508	15.889387251428\\
30.3484124746517	15.9117803588479\\
30.3876110067527	15.9341734662677\\
30.4268095388537	15.9565665736876\\
30.4660080709546	15.9789596811075\\
30.5052066030556	16.0013527885274\\
30.5444051351565	16.0237458959473\\
30.5836036672575	16.0461390033671\\
30.6228021993584	16.068532110787\\
30.6620007314594	16.0909252182069\\
30.7011992635603	16.1133183256268\\
30.7403977956613	16.1357114330467\\
30.7795963277622	16.1581045404665\\
30.8187948598632	16.1804976478864\\
30.8579933919641	16.2028907553063\\
30.8971919240651	16.2252838627262\\
30.936390456166	16.247676970146\\
30.975588988267	16.2700700775659\\
31.014787520368	16.2924631849858\\
31.0539860524689	16.3148562924057\\
31.0931845845699	16.3372493998256\\
31.1323831166708	16.3596425072454\\
31.1715816487718	16.3820356146653\\
31.2107801808727	16.4044287220852\\
31.2499787129737	16.4268218295051\\
31.2891772450746	16.4492149369249\\
31.3283757771756	16.4716080443448\\
31.3675743092765	16.4940011517647\\
31.4067728413775	16.5163942591846\\
31.4459713734784	16.5387873666045\\
31.4851699055794	16.5611804740243\\
31.5243684376804	16.5835735814442\\
31.5635669697813	16.6059666888641\\
31.6027655018823	16.628359796284\\
31.6419640339832	16.6507529037039\\
31.6811625660842	16.6731460111237\\
31.7203610981851	16.6955391185436\\
31.7595596302861	16.7179322259635\\
31.798758162387	16.7403253333834\\
31.837956694488	16.7627184408032\\
31.8771552265889	16.7851115482231\\
31.9163537586899	16.807504655643\\
31.9555522907908	16.8298977630629\\
31.9947508228918	16.8522908704828\\
32.0339493549927	16.8746839779026\\
32.0731478870937	16.8970770853225\\
32.1123464191947	16.9194701927424\\
32.1515449512956	16.9418633001623\\
32.1907434833966	16.9642564075821\\
32.2299420154975	16.986649515002\\
32.2691405475985	17.0090426224219\\
32.3083390796994	17.0314357298418\\
32.3475376118004	17.0538288372617\\
32.3867361439013	17.0762219446815\\
32.4259346760023	17.0986150521014\\
32.4651332081032	17.1210081595213\\
32.5043317402042	17.1434012669412\\
32.5435302723051	17.165794374361\\
32.5827288044061	17.1881874817809\\
32.6219273365071	17.2105805892008\\
32.661125868608	17.2329736966207\\
32.700324400709	17.2553668040406\\
32.7395229328099	17.2777599114604\\
32.7787214649109	17.3001530188803\\
32.8179199970118	17.3225461263003\\
32.8571185291127	17.3449392337202\\
32.8963170612137	17.3673323411401\\
32.9355155933146	17.3897254485599\\
32.9747141254156	17.4121185559797\\
33.0139126575168	17.4345116633993\\
33.053111189618	17.4569047708186\\
33.0923097217194	17.4792978782378\\
33.1315082538207	17.5016909856571\\
33.1707067859216	17.524084093077\\
33.2099053180216	17.5464772004985\\
33.2491038501199	17.568870307923\\
33.288302382216	17.5912634153515\\
33.3275009143099	17.6136565227837\\
33.3666994464038	17.636049630216\\
33.4058979785028	17.6584427376392\\
33.4450965106163	17.6808358450371\\
33.4842950427574	17.7032289523868\\
33.5234935749382	17.7256220596669\\
33.5626921071601	17.7480151668749\\
33.6018906393959	17.7704082740589\\
33.6410891715654	17.7928013813588\\
33.680287703515	17.8151944890436\\
33.7194862350193	17.8375875975079\\
33.7586847658397	17.8599807071693\\
33.7978832958832	17.8823738181907\\
33.8370818254969	17.9047669299645\\
33.8762803558856	17.9271600403815\\
33.9154788895306	17.9495531450987\\
33.9546774302935	17.971946237356\\
33.993875982646	17.9943393093261\\
34.0330745492518	18.016732356346\\
34.0722731261791	18.0391253852984\\
34.1114716956819	18.0615184272472\\
34.1506702182275	18.0839115513935\\
34.1898686285887	18.1063048719142\\
34.2290668451971	18.1286985315862\\
34.2682648061832	18.1510926386978\\
34.3074625461312	18.1734871327031\\
34.3466603170347	18.195881572526\\
34.3858587243575	18.2182748983653\\
34.4250587996965	18.2406653041724\\
34.4642619012604	18.2630504109749\\
34.5034692819497	18.2854280220689\\
34.5426810990847	18.3077978583615\\
34.5818955859883	18.3301630141998\\
34.6211094187644	18.3525293169296\\
34.6603185858481	18.374903797826\\
34.6995188270993	18.3972939131594\\
34.7387057678731	18.4197072987041\\
34.7778749680577	18.4421516729588\\
34.8170219437858	18.4646347881818\\
34.8561421809057	18.4871643959563\\
34.8952311569866	18.5097481982874\\
34.9342860899511	18.5323908215879\\
34.9733081763325	18.555090005565\\
35.0123013840175	18.5778387623052\\
35.0512702433414	18.6006292028841\\
35.0902193363203	18.6234534079066\\
35.1291532756558	18.6463034526902\\
35.1680766949613	18.6691714129917\\
35.2069942417834	18.6920493658752\\
35.2459105716648	18.7149293887631\\
35.2848303426815	18.7378035576615\\
35.3237582102207	18.7606639449644\\
35.3626988218894	18.7835026170286\\
35.401656812498	18.8063116316178\\
35.4406367990844	18.8290830352748\\
35.4796433759599	18.8518088606556\\
35.5186811097627	18.8744811238495\\
35.5577545345088	18.8970918217024\\
35.5968681466357	18.9196329291547\\
35.6360264000301	18.942096396605\\
35.675233701037	18.9644741473083\\
35.7144944034466	18.9867580748143\\
35.7538128034529	19.0089400404561\\
35.7931931345839	19.0310118708922\\
35.8326395625984	19.052965355711\\
35.8721561803474	19.0747922451015\\
35.911747002598	19.0964842475979\\
35.9514159608169	19.1180330279025\\
35.9911668979112	19.1394302047937\\
36.0310035629251	19.1606673491253\\
36.0709296056906	19.1817359819208\\
36.1109485714299	19.2026275725721\\
36.1510638953085	19.2233335371454\\
36.1912788969365	19.2438452368029\\
36.2315967748192	19.264153976346\\
36.2720206007514	19.284251002887\\
36.3125533141587	19.3041275046575\\
36.3531977163792	19.323774609964\\
36.3939564648852	19.3431833863037\\
36.434832067435	19.3623448396667\\
36.4758268761292	19.381249914089\\
36.516943081247	19.3998894917533\\
36.5581825174713	19.4182548144753\\
36.5995451654786	19.436340928724\\
36.6410292621773	19.454146711489\\
36.6826329087446	19.4716713447547\\
36.7243542099862	19.488914001206\\
36.7661912557562	19.5058738885884\\
36.8081421345966	19.5225502174202\\
36.8502049255483	19.5389422214508\\
36.8923777047445	19.5550491415058\\
36.9346585408276	19.570870237514\\
36.977045498863	19.5864047786173\\
37.0195366374857	19.6016520510729\\
37.0621300114608	19.6166113515981\\
37.104823669798	19.6312819929125\\
37.1476156575299	19.6456632989985\\
37.1905040144192	19.6597546091621\\
37.2334867762468	19.6735552745251\\
37.2765619739043	19.6870646610973\\
37.3197276343584	19.7002821471042\\
37.362981780005	19.7132071253699\\
37.4063224294105	19.7258390012362\\
37.4497475968492	19.7381771944471\\
37.4932552928873	19.7502211374991\\
37.5368435240522	19.761970277157\\
37.5805102933018	19.7734240731257\\
37.6242535997907	19.7845819992866\\
37.668071439255	19.7954435426147\\
37.7119618038512	19.8060082041994\\
37.7559226824772	19.8162754983518\\
37.7999520606663	19.8262449534558\\
37.8440479208595	19.8359161112259\\
37.8882082423418	19.8452885274228\\
37.9324310014776	19.8543617712307\\
37.9767141716797	19.8631354258652\\
38.0210557236151	19.8716090880453\\
38.0654536252003	19.8797823685139\\
38.1099058417844	19.887654891588\\
38.154410336164	19.8952262956061\\
38.198965068749	19.9024962325426\\
38.2435679975938	19.9094643683952\\
38.2882170785479	19.9161303828517\\
38.3329102653009	19.9224939696277\\
38.3776455095213	19.928554836177\\
38.4224207609117	19.9343127039869\\
38.467233967339	19.9397673083253\\
38.5120830748975	19.9449183985002\\
38.5569656722667	19.9497690306435\\
38.6018789483978	19.9543268597275\\
38.646820123065	19.9586008506395\\
38.6917865884439	19.9625999968134\\
38.7367759037934	19.9663333174754\\
38.7817857900922	19.9698098551433\\
38.8268141246394	19.973038673301\\
38.8718589356254	19.9760288542352\\
38.916918396676	19.978789497027\\
38.9619908213714	19.9813297156913\\
39.0070746577453	19.9836586374595\\
39.0521684827636	19.9857854011981\\
39.0972709967879	19.9877191559594\\
39.1423810180231	19.9894690596572\\
39.1874974769532	19.9910442778624\\
39.2326194107662	19.9924539827133\\
39.2777459577688	19.9937073519345\\
39.3228763517943	19.9948135679581\\
39.3680099166026	19.9957818171431\\
39.4131460602753	19.9966212890857\\
39.4582842696062	19.9973411760161\\
39.5034241044873	19.9979506722745\\
39.548565192293	19.9984589738632\\
39.5937072222607	19.9988752780653\\
39.6388499398701	19.9992087831288\\
39.6839931412213	19.9994686880061\\
39.7291366674119	19.9996641921457\\
39.7742803989126	19.9998044953306\\
39.8194242499432	19.9998987975553\\
39.8645681628483	19.9999562989387\\
39.9097121024718	19.9999861996642\\
39.9548560505328	19.999997699944\\
40	20\\
};
\addlegendentry{PINS}

\addplot [color=blue, line width=2.0pt, only marks, mark size=2.5pt, mark=*, mark options={solid, fill=blue, blue}, forget plot]
  table[row sep=crcr]{%
0	0\\
};
\addplot [color=blue, line width=2.0pt, only marks, mark size=2.5pt, mark=*, mark options={solid, fill=blue, blue}, forget plot]
  table[row sep=crcr]{%
40	20\\
};
\end{axis}
\end{tikzpicture}%%
  \caption{Trajectory analysis PINS vs Duboids}
  \label{fig:Compare_traj1}
\end{figure}
%
\begin{figure}[htb!]
  \centering
  
%
\begin{tikzpicture}[scale = 0.7]

\begin{axis}[%
width=0.985\linewidth,
height=\linewidth,
at={(0\linewidth,0\linewidth)},
scale only axis,
xmin=0,
xmax=50,
xlabel style={font=\color{white!15!black}},
xlabel={Time(s)},
ymin=-0.2,
ymax=0.2,
ylabel style={font=\color{white!15!black}},
ylabel={$\kappa(m^{-1})$},
axis background/.style={fill=white},
title style={font=\bfseries},
title={Curvature - PINS},
axis x line*=bottom,
axis y line*=left,
xmajorgrids,
xminorgrids,
ymajorgrids,
yminorgrids,
legend style={legend cell align=left, align=left, draw=white!15!black}
]
\addplot [color=green, dashdotted, line width=2.0pt]
  table[row sep=crcr]{%
0	0\\
0.451441271751181	0.0451441271751181\\
0.902882543502362	0.0902882543502362\\
1.35432381525354	0.135432381525354\\
1.80576508700472	0.15\\
2.25720635875591	0.15\\
2.70864763050709	0.15\\
3.16008890225827	0.15\\
3.61153017400945	0.135374108926985\\
4.06297144576063	0.0902299817518666\\
4.51441271751181	0.0450858545767485\\
4.96585398926299	-1.38777878078145e-17\\
5.41729526101417	-1.38777878078145e-17\\
5.86873653276535	-1.38777878078145e-17\\
6.32017780451654	-1.38777878078145e-17\\
6.77161907626772	-1.38777878078145e-17\\
7.2230603480189	-1.38777878078145e-17\\
7.67450161977008	-1.38777878078145e-17\\
8.12594289152126	-1.38777878078145e-17\\
8.57738416327244	-1.38777878078145e-17\\
9.02882543502362	-1.38777878078145e-17\\
9.4802667067748	-1.38777878078145e-17\\
9.93170797852598	-1.38777878078145e-17\\
10.3831492502772	-1.38777878078145e-17\\
10.8345905220283	-1.38777878078145e-17\\
11.2860317937795	-1.38777878078145e-17\\
11.7374730655307	-1.38777878078145e-17\\
12.1889143372819	-1.38777878078145e-17\\
12.6403556090331	-1.38777878078145e-17\\
13.0917968807842	-1.38777878078145e-17\\
13.5432381525354	-1.38777878078145e-17\\
13.9946794242866	-1.38777878078145e-17\\
14.4461206960378	-1.38777878078145e-17\\
14.897561967789	-1.38777878078145e-17\\
15.3490032395402	-1.38777878078145e-17\\
15.8004445112913	-1.38777878078145e-17\\
16.2518857830425	-1.38777878078145e-17\\
16.7033270547937	-1.38777878078145e-17\\
17.1547683265449	-1.38777878078145e-17\\
17.6062095982961	-1.38777878078145e-17\\
18.0576508700472	-1.38777878078145e-17\\
18.5090921417984	-1.38777878078145e-17\\
18.9605334135496	-1.38777878078145e-17\\
19.4119746853008	-1.38777878078145e-17\\
19.863415957052	-1.38777878078145e-17\\
20.3148572288031	-1.38777878078145e-17\\
20.7662985005543	-1.38777878078145e-17\\
21.2177397723055	-1.38777878078145e-17\\
21.6691810440567	-1.38777878078145e-17\\
22.1206223158079	-1.38777878078145e-17\\
22.5720635875591	-1.38777878078145e-17\\
23.0235048593102	-1.38777878078145e-17\\
23.4749461310614	-1.38777878078145e-17\\
23.9263874028126	-1.38777878078145e-17\\
24.3778286745638	-1.38777878078145e-17\\
24.829269946315	-1.38777878078145e-17\\
25.2807112180661	-1.38777878078145e-17\\
25.7321524898173	-1.38777878078145e-17\\
26.1835937615685	-1.38777878078145e-17\\
26.6350350333197	-1.38777878078145e-17\\
27.0864763050709	-1.38777878078145e-17\\
27.537917576822	-1.38777878078145e-17\\
27.9893588485732	-1.38777878078145e-17\\
28.4408001203244	-1.38777878078145e-17\\
28.8922413920756	-1.38777878078145e-17\\
29.3436826638268	-1.38777878078145e-17\\
29.7951239355779	-1.38777878078145e-17\\
30.2465652073291	-1.38777878078145e-17\\
30.6980064790803	-1.38777878078145e-17\\
31.1494477508315	-1.38777878078145e-17\\
31.6008890225827	-1.38777878078145e-17\\
32.0523302943339	-1.38777878078145e-17\\
32.503771566085	-1.38777878078145e-17\\
32.9552128378362	-1.38777878078145e-17\\
33.4066541095874	-1.38777878078145e-17\\
33.8580953813386	-1.38777878078145e-17\\
34.3095366530898	-1.38777878078145e-17\\
34.7609779248409	-1.38777878078145e-17\\
35.2124191965921	-1.38777878078145e-17\\
35.6638604683433	-1.38777878078145e-17\\
36.1153017400945	-1.38777878078145e-17\\
36.5667430118457	-1.38777878078145e-17\\
37.0181842835969	-1.38777878078145e-17\\
37.469625555348	-1.38777878078145e-17\\
37.9210668270992	-1.38777878078145e-17\\
38.3725080988504	-1.38777878078145e-17\\
38.8239493706016	-1.38777878078145e-17\\
39.2753906423528	-1.38777878078145e-17\\
39.7268319141039	-1.38777878078145e-17\\
40.1782731858551	-1.38777878078145e-17\\
40.6297144576063	-0.0450867121500984\\
41.0811557293575	-0.0902308393252163\\
41.5325970011087	-0.135374966500335\\
41.9840382728598	-0.15\\
42.435479544611	-0.15\\
42.8869208163622	-0.15\\
43.3383620881134	-0.15\\
43.7898033598646	-0.135432381525354\\
44.2412446316157	-0.0902882543502365\\
44.6926859033669	-0.0451441271751179\\
45.1441271751181	-1.38777878078145e-17\\
};
\addlegendentry{Duboids}

\addplot [color=dodgerblue, dotted, line width=2.0pt]
  table[row sep=crcr]{%
0	2.81173139635518e-29\\
0.0451439495257969	0.0045143923202374\\
0.0902878990515937	0.0090287845228991\\
0.135431848577391	0.013543176599467\\
0.180575798103187	0.0180575685405349\\
0.225719747628984	0.0225719603356831\\
0.270863697154781	0.0270863519733303\\
0.316007646680578	0.0316007434405574\\
0.361151596206375	0.0361151347228971\\
0.406295545732172	0.04062952580408\\
0.451439495257969	0.0451439166657267\\
0.496583444783766	0.049658307286972\\
0.541727394309562	0.0541726976440011\\
0.586871343835359	0.0586870877094722\\
0.632015293361156	0.0632014774517921\\
0.677159242886953	0.0677158668341946\\
0.72230319241275	0.0722302558135558\\
0.767447141938547	0.07674464433885\\
0.812591091464344	0.0812590323491071\\
0.857735040990141	0.0857734197706636\\
0.902878990515937	0.0902878065133953\\
0.948022940041734	0.0948021924654392\\
0.993166889567531	0.0993165774856178\\
1.03831083909333	0.103830961392247\\
1.08345478861912	0.10834534394604\\
1.12859873814492	0.112859724822931\\
1.17374268767072	0.117374103568729\\
1.21888663719652	0.121888479518762\\
1.26403058672231	0.126402851644199\\
1.30917453624811	0.13091721822676\\
1.35431848577391	0.135431576065813\\
1.3994624352997	0.139945918085612\\
1.4446063848255	0.1444602228483\\
1.4897503343513	0.148974334384146\\
1.53489428387709	0.150962645586409\\
1.58003823340289	0.148971276384297\\
1.62518218292869	0.150964352430091\\
1.67032613245448	0.14896793783944\\
1.71547008198028	0.150966266605675\\
1.76061403150608	0.148964280864826\\
1.80575798103187	0.150968419863841\\
1.85090193055767	0.148960260699899\\
1.89604588008347	0.150970850151575\\
1.94118982960927	0.148955824110167\\
1.98633377913506	0.150973603122211\\
2.03147772866086	0.148950907326594\\
2.07662167818666	0.150976734096714\\
2.12176562771245	0.148945433358697\\
2.16690957723825	0.150980310639437\\
2.21205352676405	0.148939308447185\\
2.25719747628984	0.150984415982574\\
2.30234142581564	0.148932417315298\\
2.34748537534144	0.150989153641403\\
2.39262932486723	0.148924616712156\\
2.43777327439303	0.150994653730277\\
2.48291722391883	0.148915726476986\\
2.52806117344462	0.151001081757312\\
2.57320512297042	0.148905516917896\\
2.61834907249622	0.151008651116694\\
2.66349302202202	0.148893690556952\\
2.70863697154781	0.151017641249164\\
2.75378092107361	0.148879854973553\\
2.79892487059941	0.151028424776752\\
2.8440688201252	0.148863481009218\\
2.889212769651	0.151041509412531\\
2.9343567191768	0.148843835685128\\
2.97950066870259	0.151057605399823\\
3.02464461822839	0.148819868638193\\
3.06978856775419	0.151077739848821\\
3.11493251727998	0.148790005931428\\
3.16007646680578	0.151103464398969\\
3.20522041633158	0.148751738004486\\
3.25036436585737	0.151137269907639\\
3.29550831538317	0.14870067299854\\
3.34065226490897	0.15118353771748\\
3.38579621443477	0.148627819637985\\
3.43094016396056	0.15125126480293\\
3.47608411348636	0.148507625876684\\
3.52122806301216	0.151366040851981\\
3.56637201253795	0.147759667788028\\
3.61151596206375	0.143247675469749\\
3.65665991158955	0.138734568355951\\
3.70180386111534	0.134221095600871\\
3.74694781064114	0.129707444923469\\
3.79209176016694	0.125193691898859\\
3.83723570969273	0.12067987479915\\
3.88237965921853	0.116166015981292\\
3.92752360874433	0.111652129918105\\
3.97266755827012	0.107138226780349\\
4.01781150779592	0.102624314242281\\
4.06295545732172	0.0981103984830695\\
4.10809940684751	0.0935964847900702\\
4.15324335637331	0.089082577952375\\
4.19838730589911	0.0845686825401873\\
4.24353125542491	0.0800548031226436\\
4.2886752049507	0.0755409444558334\\
4.3338191544765	0.0710271116624853\\
4.3789631040023	0.0665133104201206\\
4.42410705352809	0.0619995471732063\\
4.46925100305389	0.0574858293861005\\
4.51439495257969	0.0529721658572244\\
4.55953890210548	0.0484585671213825\\
4.60468285163128	0.0439450459775374\\
4.64982680115708	0.0394316181955717\\
4.69497075068287	0.0349183034809695\\
4.74011470020867	0.0304051268166733\\
4.78525864973447	0.0258921203667181\\
4.83040259926026	0.0213793262348368\\
4.87554654878606	0.0168668005570645\\
4.92069049831186	0.0123546197363159\\
4.96583444783766	0.00784289023240145\\
5.01097839736345	0.00333176448754518\\
5.05612234688925	-0.00117853205982067\\
5.10126629641505	-0.00568765276828508\\
5.14641024594084	-0.0101950381596387\\
5.19155419546664	-0.0146997269608737\\
5.23669814499244	-0.0191999211079566\\
5.28184209451823	-0.0236918114626388\\
5.32698604404403	-0.0281654821406633\\
5.37212999356983	-0.0325812075246882\\
5.41727394309562	-0.0350474792784714\\
5.46241789262142	-0.0307151009415113\\
5.50756184214722	-0.0263194206815274\\
5.55270579167301	-0.0219134576775913\\
5.59784974119881	-0.0175152649613942\\
5.64299369072461	-0.0131403335109906\\
5.68813764025041	-0.00881133003131632\\
5.7332815897762	-0.00457195769601708\\
5.778425539302	-0.000524795951922065\\
5.8235694888278	0.00303011743162835\\
5.86871343835359	0.00504044543509764\\
5.91385738787939	0.00445840820938415\\
5.95900133740519	0.00298334669134363\\
6.00414528693098	0.00155198145261172\\
6.04928923645678	0.000513092868876806\\
6.09443318598258	-6.00591855383552e-05\\
6.13957713550837	-0.000270496557444311\\
6.18472108503417	-0.000273508313330157\\
6.22986503455997	-0.000193903110070928\\
6.27500898408577	-0.000105316794209789\\
6.32015293361156	-3.93540067202554e-05\\
6.36529688313736	-1.77694811565289e-06\\
6.41044083266315	1.35307739868485e-05\\
6.45558478218895	1.55554202009504e-05\\
6.50072873171475	1.17566088413243e-05\\
6.54587268124055	6.78918391510053e-06\\
6.59101663076634	2.82948725434591e-06\\
6.63616058029214	4.35954736867342e-07\\
6.68130452981794	-6.34399947467269e-07\\
6.72644847934373	-8.68046419896515e-07\\
6.77159242886953	-7.03526428155559e-07\\
6.81673637839533	-4.30976966501718e-07\\
6.86188032792112	-1.96445411715375e-07\\
6.90702427744692	-4.61529216354279e-08\\
6.95216822697272	2.66033031848606e-08\\
6.99731217649851	4.73652006769724e-08\\
7.04245612602431	4.15402399026736e-08\\
7.08760007555011	2.69773058692302e-08\\
7.13274402507591	1.32791015993126e-08\\
7.1778879746017	3.96768700037266e-09\\
7.2230319241275	-8.6734264993107e-10\\
7.2681758736533	-2.51312295279528e-09\\
7.31331982317909	-2.41813192120557e-09\\
7.35846377270489	-1.66641766951061e-09\\
7.40360772223069	-8.78427470272426e-10\\
7.44875167175648	-3.08949360831695e-10\\
7.49389562128228	6.33936109079627e-12\\
7.53903957080808	1.28494844293211e-10\\
7.58418352033387	1.38589385792143e-10\\
7.62932746985967	1.01619299455379e-10\\
7.67447141938547	5.70564040837885e-11\\
7.71961536891127	2.26711258737573e-11\\
7.76475931843706	2.44719159349732e-12\\
7.80990326796286	-6.23160363093278e-12\\
7.85504721748865	-7.80478327848463e-12\\
7.90019116701445	-6.11787198904151e-12\\
7.94533511654025	-3.64678022547199e-12\\
7.99047906606605	-1.59722740934417e-12\\
8.03562301559184	-3.19033352280742e-13\\
8.08076696511764	2.78069769485129e-13\\
8.12591091464344	4.30790714940371e-13\\
8.17105486416923	3.63561711836459e-13\\
8.21619881369503	2.29669876378609e-13\\
8.26134276322083	1.09091916367425e-13\\
8.30648671274662	2.94247119738355e-14\\
8.35163066227242	-1.05685917070705e-14\\
8.39677461179822	-2.3227830897898e-14\\
8.44191856132401	-2.14619180971538e-14\\
8.48706251084981	-1.4483291018546e-14\\
8.53220646037561	-7.49248115640452e-15\\
8.57735040990141	-2.57234282860758e-15\\
8.6224943594272	1.55312864943911e-16\\
8.667638308953	1.27696326388953e-15\\
8.7127822584788	1.41788099647797e-15\\
8.75792620800459	1.08128271045236e-15\\
8.80307015753039	6.2944710890578e-16\\
8.84821410705619	2.65757377927211e-16\\
8.89335805658198	4.41810072285314e-17\\
8.93850200610778	-5.60343467880561e-17\\
8.98364595563358	-7.89038169101759e-17\\
9.02878990515937	-6.45910282535787e-17\\
9.07393385468517	-3.98796583135046e-17\\
9.11907780421097	-1.83777440521549e-17\\
9.16422175373676	-4.49030234577273e-18\\
9.20936570326256	2.29923231426946e-18\\
9.25450965278836	4.29087772484706e-18\\
9.29965360231416	3.80677195711145e-18\\
9.34479755183995	2.49176596387319e-18\\
9.38994150136575	1.23836973058571e-18\\
9.43508545089155	3.79478599757245e-19\\
9.48022940041734	-7.04948168391462e-20\\
9.52537334994314	-2.26680333632561e-19\\
9.57051729946894	-2.21156375267571e-19\\
9.61566124899473	-1.53650978848565e-19\\
9.66080519852053	-8.17052641808002e-20\\
9.70594914804633	-2.92664882954423e-20\\
9.75109309757212	7.57030421567667e-24\\
9.79623704709792	1.15211650935536e-20\\
9.84138099662372	1.26468867213463e-20\\
9.88652494614951	9.35367319761932e-21\\
9.93166889567531	5.29499525009941e-21\\
9.97681284520111	2.13418373181514e-21\\
10.0219567947269	2.60319035989825e-22\\
10.0671007442527	-5.53723172162463e-22\\
10.1122446937785	-7.10374416369915e-22\\
10.1573886433043	-5.62144563082603e-22\\
10.2025325928301	-3.37769223253022e-22\\
10.2476765423559	-1.49723142048625e-22\\
10.2928204918817	-3.15195647657677e-23\\
10.3379644414075	2.42904310236974e-23\\
10.3831083909333	3.90694407726045e-23\\
10.4282523404591	3.3341717499342e-23\\
10.4733962899849	2.12393500782956e-23\\
10.5185402395107	1.02039877163021e-23\\
10.5636841890365	2.84815244293326e-24\\
10.6088281385623	-8.87042922541216e-25\\
10.6539720880881	-2.09426227100842e-24\\
10.6991160376139	-1.94986331948614e-24\\
10.7442599871397	-1.31673132080988e-24\\
10.7894039366655	-6.78454762227716e-25\\
10.8345478861912	-2.27070435170678e-25\\
10.879691835717	1.68949926574643e-26\\
10.9248357852428	1.06095163559682e-25\\
10.9699797347686	1.09907167433286e-25\\
11.0151236842944	8.04848770807499e-26\\
11.0602676338202	4.53145631142603e-26\\
11.105411583346	1.82968026270436e-26\\
11.1505555328718	2.55519577805365e-27\\
11.1956994823976	-4.55487593718167e-27\\
11.2408434319234	-6.17135747044503e-27\\
11.2859873814492	-4.94820178446563e-27\\
11.331131330975	-3.16574504564398e-27\\
11.3762752805008	-1.27677237433138e-27\\
11.4214192300266	-2.56529499531783e-28\\
11.4665631795524	1.64018356437031e-30\\
11.5117071290782	5.11314759960679e-28\\
11.556851078604	2.39193155871842e-28\\
11.6019950281298	1.27266560530631e-28\\
11.6471389776556	4.41121668544045e-28\\
11.6922829271814	-2.47697666639845e-29\\
11.7374268767072	-1.73691804220184e-28\\
11.782570826233	-2.04660934104817e-28\\
11.8277147757588	-3.07854608051819e-28\\
11.8728587252846	3.66866836696702e-28\\
11.9180026748104	1.76237265135648e-28\\
11.9631466243362	-1.49728100823269e-28\\
12.008290573862	1.60396616867125e-28\\
12.0534345233878	2.32383141030667e-28\\
12.0985784729136	2.73266300955523e-28\\
12.1437224224394	5.83282689654397e-29\\
12.1888663719652	-4.64006824389659e-28\\
12.234010321491	-1.49039384384141e-27\\
12.2791542710167	-1.00170232236408e-27\\
12.3242982205425	1.85839072934094e-28\\
12.3694421700683	5.89830390069701e-28\\
12.4145861195941	6.6621755552982e-28\\
12.4597300691199	2.68476586211327e-28\\
12.5048740186457	9.41962379077956e-30\\
12.5500179681715	1.19901024451293e-28\\
12.5951619176973	9.85542126231576e-29\\
12.6403058672231	-7.54739787737211e-29\\
12.6854498167489	1.93885862310131e-28\\
12.7305937662747	1.31583147283992e-28\\
12.7757377158005	5.91853557603435e-29\\
12.8208816653263	-2.66033347430304e-28\\
12.8660256148521	-5.12864658019487e-28\\
12.9111695643779	-9.48241660128673e-28\\
12.9563135139037	-4.07630758054962e-28\\
13.0014574634295	3.66010481963388e-28\\
13.0466014129553	-7.40641282775901e-29\\
13.0917453624811	7.49185613419997e-28\\
13.1368893120069	3.09552065432828e-28\\
13.1820332615327	-3.79094971698411e-28\\
13.2271772110585	1.66056623535946e-28\\
13.2723211605843	-1.50317241211987e-28\\
13.3174651101101	1.63927501553229e-28\\
13.3626090596359	3.01374545245911e-28\\
13.4077530091617	-1.1523841136172e-28\\
13.4528969586875	-3.29990362969223e-29\\
13.4980409082133	1.30682994703936e-28\\
13.5431848577391	3.57690786546558e-29\\
13.5883288072649	-5.07954192787441e-28\\
13.6334727567907	-5.42186143001719e-28\\
13.6786167063165	-2.4512172147473e-28\\
13.7237606558422	-5.2615190618821e-28\\
13.768904605368	-7.07921603846378e-28\\
13.8140485548938	7.02956986663048e-29\\
13.8591925044196	3.73057128136774e-28\\
13.9043364539454	4.62232413022689e-29\\
13.9494804034712	-2.56813766742075e-28\\
13.994624352997	-1.89256836960315e-28\\
14.0397683025228	9.55944899340071e-29\\
14.0849122520486	-5.89496685652666e-29\\
14.1300562015744	-6.33954722826823e-29\\
14.1752001511002	4.86932193209212e-28\\
14.220344100626	-1.44606228248767e-29\\
14.2654880501518	-4.10498187384823e-28\\
14.3106319996776	-6.50537593969151e-30\\
14.3557759492034	2.30121956934978e-28\\
14.4009198987292	-1.97301561887422e-29\\
14.446063848255	-5.24923640571626e-28\\
14.4912077977808	-2.78145867482855e-28\\
14.5363517473066	5.15328342361442e-28\\
14.5814956968324	2.05782883048145e-28\\
14.6266396463582	-3.83244010657147e-28\\
14.671783595884	3.31101343891197e-29\\
14.7169275454098	1.7285722574246e-28\\
14.7620714949356	1.92883608263871e-28\\
14.8072154444614	-2.48753373514901e-29\\
14.8523593939872	-5.27195083653478e-29\\
14.897503343513	-4.8808568111499e-28\\
14.9426472930388	-1.74956322925195e-28\\
14.9877912425646	4.50946338305488e-28\\
15.0329351920904	2.40814667790226e-28\\
15.0780791416162	7.08855269811242e-28\\
15.123223091142	5.22372704162839e-28\\
15.1683670406677	-9.37935938575237e-29\\
15.2135109901935	-2.50907797667322e-28\\
15.2586549397193	9.57868852968567e-28\\
15.3037988892451	8.4391824583004e-28\\
15.3489428387709	-1.87515895663535e-28\\
15.3940867882967	3.73269639090769e-29\\
15.4392307378225	3.72909109186358e-28\\
15.4843746873483	1.32289978612333e-28\\
15.5295186368741	-3.97493206021948e-28\\
15.5746625863999	-1.144539250787e-28\\
15.6198065359257	3.76143228709713e-28\\
15.6649504854515	4.06205440293822e-28\\
15.7100944349773	-1.29526546436938e-28\\
15.7552383845031	-3.04466393555494e-28\\
15.8003823340289	-2.21038445840391e-28\\
15.8455262835547	2.17396369681426e-29\\
15.8906702330805	-1.56593069996003e-28\\
15.9358141826063	-2.28864953073123e-28\\
15.9809581321321	2.0102051284975e-28\\
16.0261020816579	3.24588326163336e-28\\
16.0712460311837	4.38955246655451e-28\\
16.1163899807095	6.40079087242005e-28\\
16.1615339302353	2.16957195573046e-28\\
16.2066778797611	-4.38778815524737e-28\\
16.2518218292869	-1.55021840106998e-27\\
16.2969657788127	-1.54189162907728e-27\\
16.3421097283385	-1.22742716239589e-28\\
16.3872536778643	3.35629422677982e-28\\
16.4323976273901	1.41022826551967e-28\\
16.4775415769159	-3.34421332962802e-28\\
16.5226855264417	7.25603530207424e-28\\
16.5678294759675	1.06606187739787e-27\\
16.6129734254932	-3.29837559955527e-28\\
16.658117375019	-4.59450597637913e-28\\
16.7032613245448	-4.30311208461644e-28\\
16.7484052740706	-5.07397178815634e-28\\
16.7935492235964	-3.83050868518929e-28\\
16.8386931731222	4.55940092901445e-28\\
16.883837122648	6.53623065444466e-28\\
16.9289810721738	2.21954078215765e-28\\
16.9741250216996	-1.03722466510522e-28\\
17.0192689712254	-8.3806903942176e-28\\
17.0644129207512	-7.95022798652444e-29\\
17.109556870277	3.25447051681877e-28\\
17.1547008198028	-2.24445574530946e-28\\
17.1998447693286	3.26653995969556e-28\\
17.2449887188544	1.82640220598221e-28\\
17.2901326683802	5.94488940389508e-29\\
17.335276617906	2.64640990873308e-28\\
17.3804205674318	7.12515017551277e-29\\
17.4255645169576	-1.92093169556501e-28\\
17.4707084664834	-4.40857775998532e-28\\
17.5158524160092	-2.500413133767e-28\\
17.560996365535	6.72961145194177e-28\\
17.6061403150608	9.45208089763467e-28\\
17.6512842645866	8.1952661885333e-29\\
17.6964282141124	-8.3183921064694e-28\\
17.7415721636382	-6.8551761207786e-28\\
17.786716113164	-4.85746819707466e-29\\
17.8318600626898	-3.43661653120013e-28\\
17.8770040122156	-4.50347340726842e-28\\
17.9221479617414	1.71358978548278e-28\\
17.9672919112672	4.96110663486462e-28\\
18.012435860793	2.08981934026951e-28\\
18.0575798103187	-1.39937516737482e-28\\
18.1027237598445	1.44874581739689e-28\\
18.1478677093703	-1.39783407677383e-28\\
18.1930116588961	-6.6534869842827e-28\\
18.2381556084219	-4.61203021231843e-29\\
18.2832995579477	7.41301856109094e-28\\
18.3284435074735	8.39490936875548e-28\\
18.3735874569993	-2.96534379929364e-28\\
18.4187314065251	-9.72748472453428e-28\\
18.4638753560509	-3.34064814872089e-28\\
18.5090193055767	2.11646196106835e-28\\
18.5541632551025	2.38004505134917e-29\\
18.5993072046283	-1.41873359100798e-28\\
18.6444511541541	3.30179989954705e-28\\
18.6895951036799	2.17927238628353e-28\\
18.7347390532057	7.89714566051107e-29\\
18.7798830027315	2.58499732955693e-28\\
18.8250269522573	3.25815989099788e-29\\
18.8701709017831	-3.39308650164207e-28\\
18.9153148513089	-5.93350863855922e-28\\
18.9604588008347	-3.95586554016499e-28\\
19.0056027503605	2.12930693586555e-29\\
19.0507466998863	-2.32769618231028e-28\\
19.0958906494121	7.9937902584567e-29\\
19.1410345989379	1.24549687831088e-27\\
19.1861785484637	1.19507906716102e-27\\
19.2313224979895	-1.19939752683435e-28\\
19.2764664475153	-7.8697859852025e-28\\
19.3216103970411	-3.22237346752918e-28\\
19.3667543465669	5.23619586324846e-28\\
19.4118982960927	5.18259476979882e-28\\
19.4570422456184	-1.87161491314186e-28\\
19.5021861951442	-2.58699442737797e-28\\
19.54733014467	-1.58791134468907e-28\\
19.5924740941958	-1.75486806823726e-29\\
19.6376180437216	-2.03216563491348e-28\\
19.6827619932474	-2.36151136065903e-28\\
19.7279059427732	2.70595817842381e-28\\
19.773049892299	2.73790979831564e-28\\
19.8181938418248	-1.9087517074275e-28\\
19.8633377913506	-5.84816087722778e-29\\
19.9084817408764	1.96035547143022e-28\\
19.9536256904022	1.59052872925476e-28\\
19.998769639928	-1.71337895278285e-28\\
20.0439135894538	-2.74223730491758e-28\\
20.0890575389796	-5.16399970109831e-29\\
20.1342014885054	2.05631242071138e-28\\
20.1793454380312	8.72072963000076e-29\\
20.224489387557	-3.41322390974222e-28\\
20.2696333370828	1.10903719941043e-29\\
20.3147772866086	4.92550187504384e-28\\
20.3599212361344	3.03930857874493e-28\\
20.4050651856602	-8.29195096660695e-29\\
20.450209135186	-6.49581840251205e-28\\
20.4953530847118	-5.96497689727651e-28\\
20.5404970342376	-1.34820826080527e-28\\
20.5856409837634	1.17185307580571e-29\\
20.6307849332892	8.05632605459397e-28\\
20.675928882815	7.94663776677697e-28\\
20.7210728323408	-2.63526980520647e-28\\
20.7662167818666	-3.88851291018297e-28\\
20.8113607313924	3.79670430695259e-29\\
20.8565046809182	-1.0855117092135e-28\\
20.901648630444	-4.31346892723376e-28\\
20.9467925799697	1.4208086395744e-28\\
20.9919365294955	6.68732487425415e-28\\
21.0370804790213	2.29588452897529e-29\\
21.0822244285471	-5.7699769157244e-28\\
21.1273683780729	-1.92705010814504e-28\\
21.1725123275987	3.10719868528619e-28\\
21.2176562771245	2.53782849488832e-28\\
21.2628002266503	-7.06967529942083e-29\\
21.3079441761761	6.3453036117606e-28\\
21.3530881257019	2.78220829452618e-28\\
21.3982320752277	-4.31417625600272e-28\\
21.4433760247535	-1.02107760325057e-28\\
21.4885199742793	7.85727183504492e-29\\
21.5336639238051	-1.33246757058502e-28\\
21.5788078733309	-4.20644241550092e-28\\
21.6239518228567	-1.02281890574513e-28\\
21.6690957723825	5.25046990565241e-28\\
21.7142397219083	4.57609812842464e-28\\
21.7593836714341	-1.20612420365304e-28\\
21.8045276209599	-3.56185585767615e-28\\
21.8496715704857	-1.66178394904365e-28\\
21.8948155200115	-8.58871275440173e-29\\
21.9399594695373	-3.96401655137903e-28\\
21.9851034190631	-1.66494087258937e-28\\
22.0302473685889	1.01808728518678e-27\\
22.0753913181147	7.15845915546576e-28\\
22.1205352676405	-4.84769304131182e-28\\
22.1656792171663	-4.18424612690513e-28\\
22.2108231666921	6.46998688758449e-29\\
22.2559671162179	3.22132407731123e-29\\
22.3011110657437	-4.19410149402398e-28\\
22.3462550152694	-2.06908051245301e-28\\
22.3913989647952	5.19834013798712e-28\\
22.436542914321	9.42984976665049e-30\\
22.4816868638468	-4.10400031833243e-28\\
22.5268308133726	2.69693498310672e-28\\
22.5719747628984	3.92731971966571e-28\\
22.6171187124242	-4.68111761567408e-29\\
22.66226266195	-5.38764512848591e-28\\
22.7074066114758	-3.14349487829847e-28\\
22.7525505610016	3.21648069060995e-28\\
22.7976945105274	6.2421076481343e-28\\
22.8428384600532	6.28677646779937e-30\\
22.887982409579	-5.65683394060972e-28\\
22.9331263591048	9.07120223374481e-29\\
22.9782703086306	1.1161358589433e-28\\
23.0234142581564	-4.39579931615855e-28\\
23.0685582076822	-9.25457032700909e-29\\
23.113702157208	-4.94254457601168e-30\\
23.1588461067338	2.93013685348484e-28\\
23.2039900562596	3.06528662586738e-28\\
23.2491340057854	-1.13638051722838e-28\\
23.2942779553112	1.24158656651033e-28\\
23.339421904837	-2.3733612596115e-28\\
23.3845658543628	-5.81216112637473e-28\\
23.4297098038886	-4.19775238616805e-29\\
23.4748537534144	1.22835206139812e-28\\
23.5199977029402	1.74255271582772e-28\\
23.565141652466	8.27741499386721e-29\\
23.6102856019918	-1.64138740108379e-28\\
23.6554295515176	1.1694490542797e-28\\
23.7005735010434	6.21138011793036e-30\\
23.7457174505692	-3.47295857535404e-28\\
23.790861400095	-1.06117832388275e-28\\
23.8360053496207	4.26954819628718e-28\\
23.8811492991465	3.14732510069742e-28\\
23.9262932486723	-4.06151736137377e-29\\
23.9714371981981	-4.83878748238334e-29\\
24.0165811477239	-1.44335678711192e-28\\
24.0617250972497	-4.40095934865859e-29\\
24.1068690467755	-2.22277704324556e-28\\
24.1520129963013	-2.69629575246337e-28\\
24.1971569458271	6.1301771392008e-29\\
24.2423008953529	8.90928549323993e-29\\
24.2874448448787	2.10125536738812e-28\\
24.3325887944045	4.36080885177529e-28\\
24.3777327439303	5.44076747120603e-28\\
24.4228766934561	8.26037466544829e-30\\
24.4680206429819	-6.6272387054662e-28\\
24.5131645925077	-4.91081321420945e-28\\
24.5583085420335	-2.37542772531694e-28\\
24.6034524915593	-2.15613414925277e-28\\
24.6485964410851	-1.56252222104373e-28\\
24.6937403906109	4.97691793288072e-28\\
24.7388843401367	2.78513251430983e-28\\
24.7840282896625	7.56418737422386e-29\\
24.8291722391883	1.72977050557345e-28\\
24.8743161887141	-1.14052890533569e-28\\
24.9194601382399	-1.80707673188549e-28\\
24.9646040877657	-3.09479706356906e-28\\
25.0097480372915	-1.40385878869347e-28\\
25.0548919868173	3.26476971428217e-28\\
25.1000359363431	5.29001713335074e-28\\
25.1451798858689	1.1348300845068e-28\\
25.1903238353947	-6.04739362653154e-28\\
25.2354677849205	-4.0015655237759e-28\\
25.2806117344462	-6.3323038900588e-29\\
25.325755683972	-4.60140216333249e-28\\
25.3708996334978	-3.52450925087274e-28\\
25.4160435830236	5.09844800919169e-28\\
25.4611875325494	1.27100768174248e-27\\
25.5063314820752	4.82443178480677e-28\\
25.551475431601	-5.98909786550309e-28\\
25.5966193811268	-4.37128994554977e-28\\
25.6417633306526	3.34860479021579e-28\\
25.6869072801784	1.85039524542784e-28\\
25.7320512297042	-2.84049549936452e-28\\
25.77719517923	-6.72324708387968e-29\\
25.8223391287558	3.18875196188882e-28\\
25.8674830782816	2.27936834651984e-28\\
25.9126270278074	-8.58919777282076e-29\\
25.9577709773332	2.33637248891291e-28\\
26.002914926859	1.99193142186269e-28\\
26.0480588763848	4.3662090473478e-29\\
26.0932028259106	3.73151533270884e-28\\
26.1383467754364	4.77915182766812e-28\\
26.1834907249622	6.3591405093373e-28\\
26.228634674488	5.22808656586108e-28\\
26.2737786240138	9.14855865058914e-29\\
26.3189225735396	-1.17130275927565e-28\\
26.3640665230654	-1.35610814719545e-28\\
26.4092104725912	-2.01587783272228e-28\\
26.454354422117	-2.50749369872111e-28\\
26.4994983716428	6.03106602085266e-28\\
26.5446423211686	2.19912008685328e-28\\
26.5897862706944	-4.97859589580967e-28\\
26.6349302202202	-5.38345879220029e-29\\
26.680074169746	4.15083750390171e-28\\
26.7252181192717	1.23800404152063e-28\\
26.7703620687975	-5.06292595507565e-28\\
26.8155060183233	-2.90607577832511e-28\\
26.8606499678491	4.4125820500809e-30\\
26.9057939173749	3.6427831746978e-28\\
26.9509378669007	1.97139031679053e-28\\
26.9960818164265	-2.51305686984096e-28\\
27.0412257659523	6.16135078327354e-29\\
27.0863697154781	1.58336924999682e-28\\
27.1315136650039	-8.33711181498925e-29\\
27.1766576145297	1.69210415800808e-28\\
27.2218015640555	5.50563904382355e-28\\
27.2669455135813	4.22633781112625e-28\\
27.3120894631071	-1.24636002507673e-29\\
27.3572334126329	-1.12629015814634e-27\\
27.4023773621587	-1.0841510453711e-27\\
27.4475213116845	-5.84454707705536e-28\\
27.4926652612103	-5.84232907423964e-28\\
27.5378092107361	1.33639368492104e-28\\
27.5829531602619	7.19955014113274e-28\\
27.6280971097877	8.1601535593274e-28\\
27.6732410593135	1.9031827328801e-28\\
27.7183850088393	-4.6798918675659e-28\\
27.7635289583651	-1.47239376981194e-28\\
27.8086729078909	-6.02292081828047e-29\\
27.8538168574167	-5.38690382007058e-28\\
27.8989608069425	-3.30859891697594e-28\\
27.9441047564683	2.56243618492e-28\\
27.9892487059941	2.26650243306008e-28\\
28.0343926555199	8.36626986970486e-29\\
28.0795366050457	3.3777036229184e-28\\
28.1246805545715	1.73355098302135e-28\\
28.1698245040972	-2.47199333555187e-29\\
28.214968453623	-2.65291478819732e-28\\
28.2601124031488	-2.34860482569759e-28\\
28.3052563526746	-4.3687788538049e-28\\
28.3504003022004	2.6866003256456e-30\\
28.3955442517262	6.24240031627548e-28\\
28.440688201252	4.93314624096234e-28\\
28.4858321507778	-1.90784990409862e-28\\
28.5309761003036	-2.61457515298471e-28\\
28.5761200498294	1.54748891459742e-28\\
28.6212639993552	-3.09369554215203e-28\\
28.666407948881	-7.35650815242395e-28\\
28.7115518984068	-8.0395521457974e-28\\
28.7566958479326	-5.66803339039898e-28\\
28.8018397974584	-1.9489981408891e-29\\
28.8469837469842	1.35264215288282e-28\\
28.89212769651	-3.40703478987677e-29\\
28.9372716460358	-8.31102599995071e-30\\
28.9824155955616	4.29727076976174e-28\\
29.0275595450874	9.6903785564997e-29\\
29.0727034946132	-4.86318779026213e-28\\
29.117847444139	-1.29257158108105e-28\\
29.1629913936648	2.10570057597722e-29\\
29.2081353431906	-1.74752576549525e-28\\
29.2532792927164	-3.35848963325314e-29\\
29.2984232422422	7.05072792854432e-29\\
29.343567191768	-2.0489250195994e-28\\
29.3887111412938	-1.06966392865541e-28\\
29.4338550908196	-1.85256722559409e-28\\
29.4789990403454	-2.54142200928266e-28\\
29.5241429898712	-9.44852291135334e-29\\
29.5692869393969	2.02445117118828e-28\\
29.6144308889227	5.34637052820902e-28\\
29.6595748384485	4.22722814249951e-28\\
29.7047187879743	-1.15305113270984e-28\\
29.7498627375001	-1.65461947390097e-28\\
29.7950066870259	-8.67046061045706e-29\\
29.8401506365517	-1.90772624638203e-28\\
29.8852945860775	6.01201439531914e-28\\
29.9304385356033	-1.84655918865461e-30\\
29.9755824851291	-6.55069780630643e-28\\
30.0207264346549	-1.5463013442113e-28\\
30.0658703841807	2.52183281717953e-28\\
30.1110143337065	-1.45426756245197e-29\\
30.1561582832323	-5.57593738840995e-28\\
30.2013022327581	-8.53489956325815e-29\\
30.2464461822839	1.32243420877421e-28\\
30.2915901318097	-7.12040753897771e-29\\
30.3367340813355	1.73398832802766e-28\\
30.3818780308613	6.24908914894182e-28\\
30.4270219803871	5.61355723668485e-28\\
30.4721659299129	-1.51041417249951e-28\\
30.5173098794387	-6.13085600039893e-28\\
30.5624538289645	-3.32480882055195e-28\\
30.6075977784903	-3.63073810747569e-28\\
30.6527417280161	-4.81353016977277e-29\\
30.6978856775419	4.16393788270275e-28\\
30.7430296270677	4.71898936179476e-28\\
30.7881735765935	3.79774713076543e-28\\
30.8333175261193	1.12804732794841e-29\\
30.8784614756451	-4.00305976526948e-28\\
30.9236054251709	-4.37837614075217e-28\\
30.9687493746967	7.39903056352933e-30\\
31.0138933242225	3.02053072993952e-28\\
31.0590372737482	-1.69168493862554e-29\\
31.104181223274	-3.19082038655972e-28\\
31.1493251727998	-7.88488403285459e-28\\
31.1944691223256	2.34847787744888e-28\\
31.2396130718514	1.06077997243865e-27\\
31.2847570213772	3.93972050777689e-28\\
31.329900970903	3.60069603015527e-28\\
31.3750449204288	1.00149000192829e-27\\
31.4201888699546	5.65892905745309e-28\\
31.4653328194804	-1.95823485353289e-27\\
31.5104767690062	-1.98169415801503e-27\\
31.555620718532	-4.93211692578297e-28\\
31.6007646680578	-3.56598137277607e-29\\
31.6459086175836	3.35109379216719e-28\\
31.6910525671094	3.77818794299512e-28\\
31.7361965166352	1.32992074205966e-28\\
31.781340466161	4.63697791973484e-29\\
31.8264844156868	2.66666575838571e-28\\
31.8716283652126	-8.91567508323686e-29\\
31.9167723147384	-7.04336883537918e-29\\
31.9619162642642	9.42407116764329e-29\\
32.00706021379	-3.72491743794885e-29\\
32.0522041633158	1.53048592517202e-28\\
32.0973481128416	2.76427117252684e-28\\
32.1424920623674	8.07209735363893e-29\\
32.1876360118932	-4.14349420458454e-28\\
32.232779961419	-2.75914218311161e-28\\
32.2779239109448	-1.39116595898462e-28\\
32.3230678604706	-4.12026315682814e-28\\
32.3682118099964	-3.10572190551461e-28\\
32.4133557595222	8.71727553328585e-28\\
32.4584997090479	7.22354668391796e-28\\
32.5036436585737	-8.25914095527239e-29\\
32.5487876080995	-5.1228246962774e-29\\
32.5939315576253	-1.24825484499374e-28\\
32.6390755071511	-2.67258340677934e-28\\
32.6842194566769	-7.41959327175401e-28\\
32.7293634062027	-4.56535137065303e-28\\
32.7745073557285	3.56382606107931e-28\\
32.8196513052543	3.04455728500845e-28\\
32.8647952547801	-1.83680175614153e-28\\
32.9099392043059	1.45361420279838e-28\\
32.9550831538317	2.98178734191325e-28\\
33.0002271033575	2.13080518707046e-28\\
33.0453710528833	-7.58255871164524e-29\\
33.0905150024091	-2.18860809335815e-28\\
33.1356589519349	-2.00808666997433e-28\\
33.1808029014607	-4.24750650235225e-28\\
33.2259468509865	-4.07750070978424e-28\\
33.2710908005123	1.69358218865396e-28\\
33.3162347500381	4.35756824730555e-28\\
33.3613786995639	2.68776025391013e-28\\
33.4065226490897	2.88722200176383e-28\\
33.4516665986155	4.57208669307563e-28\\
33.4968105481413	-1.38922548895213e-28\\
33.5419544976671	-7.92360653413578e-28\\
33.5870984471929	-5.84874232038609e-28\\
33.6322423967187	1.955994806665e-28\\
33.6773863462445	-4.46869800088983e-28\\
33.7225302957703	-2.45478769309135e-28\\
33.7676742452961	1.65670726169392e-27\\
33.8128181948219	3.20393141219285e-27\\
33.8579621443477	4.68423810908814e-27\\
33.9031060938734	5.74779722196114e-27\\
33.9482500433992	5.19366803176473e-27\\
33.993393992925	-5.98736482700335e-28\\
34.0385379424508	-1.63372317997608e-26\\
34.0836818919766	-4.45847028969566e-26\\
34.1288258415024	-8.14488422853452e-26\\
34.1739697910282	-1.1287247135823e-25\\
34.219113740554	-1.08493142335613e-25\\
34.2642576900798	-1.73780415549001e-26\\
34.3094016396056	2.27673707571859e-25\\
34.3545455891314	6.79795386050731e-25\\
34.3996895386572	1.31813290776907e-24\\
34.444833488183	1.95065215976871e-24\\
34.4899774377088	2.09388224262527e-24\\
34.5351213872346	8.86235418400064e-25\\
34.5802653367604	-2.84844929340171e-24\\
34.6254092862862	-1.02036839476833e-23\\
34.670553235812	-2.12389733065982e-23\\
34.7156971853378	-3.33419236411081e-23\\
34.7608411348636	-3.90691978445808e-23\\
34.8059850843894	-2.42900472844298e-23\\
34.8511290339152	3.15193760599168e-23\\
34.896272983441	1.49722780678386e-22\\
34.9414169329668	3.37769129454909e-22\\
34.9865608824926	5.62144582148122e-22\\
35.0317048320184	7.10374582723677e-22\\
35.0768487815442	5.53723669925238e-22\\
35.12199273107	-2.60318473460541e-22\\
35.1671366805958	-2.13418397069599e-21\\
35.2122806301216	-5.29499638681634e-21\\
35.2574245796474	-9.35367360737755e-21\\
35.3025685291732	-1.26468864006937e-20\\
35.347712478699	-1.15211649222191e-20\\
35.3928564282247	-7.56940197987303e-24\\
35.4380003777505	2.92664904807274e-20\\
35.4831443272763	8.17052657709811e-20\\
35.5282882768021	1.53650978718542e-19\\
35.5734322263279	2.21156373450594e-19\\
35.6185761758537	2.26680332984677e-19\\
35.6637201253795	7.04948177820821e-20\\
35.7088640749053	-3.79478601393508e-19\\
35.7540080244311	-1.2383697378179e-18\\
35.7991519739569	-2.49176597850005e-18\\
35.8442959234827	-3.80677198463122e-18\\
35.8894398730085	-4.29087776622404e-18\\
35.9345838225343	-2.29923235666601e-18\\
35.9797277720601	4.49030233485562e-18\\
36.0248717215859	1.83777440816391e-17\\
36.0700156711117	3.98796583549909e-17\\
36.1151596206375	6.45910282882587e-17\\
36.1603035701633	7.89038169302979e-17\\
36.2054475196891	5.60343467962104e-17\\
36.2505914692149	-4.41810072199033e-17\\
36.2957354187407	-2.65757377905e-16\\
36.3408793682665	-6.29447108885057e-16\\
36.3860233177923	-1.08128271045364e-15\\
36.4311672673181	-1.41788099649241e-15\\
36.4763112168439	-1.27696326388574e-15\\
36.5214551663697	-1.55312864937437e-16\\
36.5665991158955	2.57234282858155e-15\\
36.6117430654213	7.49248115636131e-15\\
36.6568870149471	1.44832910185246e-14\\
36.7020309644729	2.14619180971432e-14\\
36.7471749139986	2.32278308978856e-14\\
36.7923188635245	1.05685917070993e-14\\
36.8374628130502	-2.94247119737856e-14\\
36.882606762576	-1.09091916367384e-13\\
36.9277507121018	-2.29669876378588e-13\\
36.9728946616276	-3.63561711836464e-13\\
37.0180386111534	-4.30790714940378e-13\\
37.0631825606792	-2.78069769485155e-13\\
37.108326510205	3.19033352280703e-13\\
37.1534704597308	1.59722740934416e-12\\
37.1986144092566	3.64678022547201e-12\\
37.2437583587824	6.11787198904157e-12\\
37.2889023083082	7.80478327848476e-12\\
37.334046257834	6.23160363093299e-12\\
37.3791902073598	-2.44719159349712e-12\\
37.4243341568856	-2.2671125873758e-11\\
37.4694781064114	-5.70564040837902e-11\\
37.5146220559372	-1.01619299455382e-10\\
37.559766005463	-1.38589385792146e-10\\
37.6049099549888	-1.28494844293213e-10\\
37.6500539045146	-6.33936109079366e-12\\
37.6951978540404	3.08949360831706e-10\\
37.7403418035662	8.78427470272452e-10\\
37.785485753092	1.66641766951065e-09\\
37.8306297026178	2.41813192120562e-09\\
37.8757736521436	2.51312295279532e-09\\
37.9209176016694	8.67342649931046e-10\\
37.9660615511952	-3.96768700037283e-09\\
38.011205500721	-1.3279101599313e-08\\
38.0563494502468	-2.69773058692309e-08\\
38.1014933997726	-4.15402399026746e-08\\
38.1466373492984	-4.73652006769732e-08\\
38.1917812988242	-2.66033031848605e-08\\
38.2369252483499	4.61529216354302e-08\\
38.2820691978757	1.96445411715381e-07\\
38.3272131474015	4.3097696650173e-07\\
38.3723570969273	7.03526428155576e-07\\
38.4175010464531	8.68046419896532e-07\\
38.4626449959789	6.34399947467274e-07\\
38.5077889455047	-4.35954736867371e-07\\
38.5529328950305	-2.829487254346e-06\\
38.5980768445563	-6.78918391510072e-06\\
38.6432207940821	-1.17566088413246e-05\\
38.6883647436079	-1.55554202009507e-05\\
38.7335086931337	-1.35307739868487e-05\\
38.7786526426595	1.77694811565324e-06\\
38.8237965921853	3.93540067202582e-05\\
38.8689405417111	0.000105316794209794\\
38.9140844912369	0.000193903110070936\\
38.9592284407627	0.000273508313330164\\
39.0043723902885	0.000270496557444312\\
39.0495163398143	6.00591855383408e-05\\
39.0946602893401	-0.000513092868876848\\
39.1398042388659	-0.0015519814526118\\
39.1849481883917	-0.00298334669134374\\
39.2300921379175	-0.00445840820938428\\
39.2752360874433	-0.0050404454350977\\
39.3203800369691	-0.00303011743162828\\
39.3655239864949	0.000524795951922178\\
39.4106679360207	0.00457195769601721\\
39.4558118855465	0.00881133003131647\\
39.5009558350723	0.0131403335109908\\
39.5460997845981	0.0175152649613944\\
39.5912437341239	0.0219134576775915\\
39.6363876836497	0.0263194206815276\\
39.6815316331754	0.0307151009415115\\
39.7266755827012	0.0350474792784716\\
39.771819532227	0.0325812075246882\\
39.8169634817528	0.0281654821406633\\
39.8621074312786	0.0236918114626387\\
39.9072513808044	0.0191999211079565\\
39.9523953303302	0.0146997269608736\\
39.997539279856	0.0101950381596386\\
40.0426832293818	0.00568765276828499\\
40.0878271789076	0.00117853205982058\\
40.1329711284334	-0.00333176448754527\\
40.1781150779592	-0.00784289023240154\\
40.223259027485	-0.012354619736316\\
40.2684029770108	-0.0168668005570646\\
40.3135469265366	-0.0213793262348369\\
40.3586908760624	-0.0258921203667182\\
40.4038348255882	-0.0304051268166734\\
40.448978775114	-0.0349183034809696\\
40.4941227246398	-0.0394316181955718\\
40.5392666741656	-0.0439450459775375\\
40.5844106236914	-0.0484585671213826\\
40.6295545732172	-0.0529721658572246\\
40.674698522743	-0.0574858293861007\\
40.7198424722688	-0.0619995471732065\\
40.7649864217946	-0.0665133104201207\\
40.8101303713204	-0.0710271116624854\\
40.8552743208462	-0.0755409444558335\\
40.900418270372	-0.0800548031226438\\
40.9455622198978	-0.0845686825401874\\
40.9907061694236	-0.0890825779523751\\
41.0358501189494	-0.0935964847900704\\
41.0809940684752	-0.0981103984830697\\
41.126138018001	-0.102624314242281\\
41.1712819675267	-0.107138226780349\\
41.2164259170525	-0.111652129918105\\
41.2615698665783	-0.116166015981292\\
41.3067138161041	-0.12067987479915\\
41.3518577656299	-0.125193691898859\\
41.3970017151557	-0.129707444923469\\
41.4421456646815	-0.134221095600871\\
41.4872896142073	-0.138734568355951\\
41.5324335637331	-0.143247675469749\\
41.5775775132589	-0.147759667788028\\
41.6227214627847	-0.151366040851981\\
41.6678654123105	-0.148507625876685\\
41.7130093618363	-0.15125126480293\\
41.7581533113621	-0.148627819637985\\
41.8032972608879	-0.15118353771748\\
41.8484412104137	-0.14870067299854\\
41.8935851599395	-0.151137269907639\\
41.9387291094653	-0.148751738004486\\
41.9838730589911	-0.151103464398968\\
42.0290170085169	-0.148790005931429\\
42.0741609580427	-0.15107773984882\\
42.1193049075685	-0.148819868638194\\
42.1644488570943	-0.151057605399823\\
42.2095928066201	-0.148843835685128\\
42.2547367561459	-0.151041509412531\\
42.2998807056717	-0.148863481009218\\
42.3450246551975	-0.151028424776752\\
42.3901686047233	-0.148879854973553\\
42.4353125542491	-0.151017641249164\\
42.4804565037749	-0.148893690556952\\
42.5256004533006	-0.151008651116694\\
42.5707444028264	-0.148905516917897\\
42.6158883523522	-0.151001081757312\\
42.661032301878	-0.148915726476986\\
42.7061762514038	-0.150994653730277\\
42.7513202009296	-0.148924616712157\\
42.7964641504554	-0.150989153641403\\
42.8416080999812	-0.148932417315298\\
42.886752049507	-0.150984415982574\\
42.9318959990328	-0.148939308447185\\
42.9770399485586	-0.150980310639437\\
43.0221838980844	-0.148945433358697\\
43.0673278476102	-0.150976734096714\\
43.112471797136	-0.148950907326594\\
43.1576157466618	-0.150973603122211\\
43.2027596961876	-0.148955824110167\\
43.2479036457134	-0.150970850151575\\
43.2930475952392	-0.148960260699899\\
43.338191544765	-0.150968419863841\\
43.3833354942908	-0.148964280864826\\
43.4284794438166	-0.150966266605675\\
43.4736233933424	-0.14896793783944\\
43.5187673428682	-0.150964352430091\\
43.563911292394	-0.148971276384297\\
43.6090552419198	-0.150962645586408\\
43.6541991914456	-0.148974334384146\\
43.6993431409714	-0.1444602228483\\
43.7444870904972	-0.139945918085613\\
43.789631040023	-0.135431576065813\\
43.8347749895488	-0.130917218226761\\
43.8799189390746	-0.126402851644199\\
43.9250628886004	-0.121888479518762\\
43.9702068381262	-0.117374103568729\\
44.0153507876519	-0.112859724822931\\
44.0604947371777	-0.10834534394604\\
44.1056386867035	-0.103830961392247\\
44.1507826362293	-0.0993165774856179\\
44.1959265857551	-0.0948021924654393\\
44.2410705352809	-0.0902878065133954\\
44.2862144848067	-0.0857734197706637\\
44.3313584343325	-0.0812590323491072\\
44.3765023838583	-0.0767446443388501\\
44.4216463333841	-0.0722302558135558\\
44.4667902829099	-0.0677158668341946\\
44.5119342324357	-0.0632014774517922\\
44.5570781819615	-0.0586870877094722\\
44.6022221314873	-0.0541726976440011\\
44.6473660810131	-0.0496583072869721\\
44.6925100305389	-0.0451439166657267\\
44.7376539800647	-0.04062952580408\\
44.7827979295905	-0.0361151347228972\\
44.8279418791163	-0.0316007434405575\\
44.8730858286421	-0.0270863519733303\\
44.9182297781679	-0.0225719603356831\\
44.9633737276937	-0.0180575685405349\\
45.0085176772195	-0.013543176599467\\
45.0536616267453	-0.00902878452289911\\
45.0988055762711	-0.0045143923202374\\
45.1439495257969	-4.85812204855773e-29\\
};
\addlegendentry{PINS}

\end{axis}
\end{tikzpicture}%%
  \caption{Curvature analysis PINS vs Duboids}
  \label{fig:Compare_curv1}
\end{figure}
%
\begin{figure}[htb!]
  \centering
  %
\begin{tikzpicture}[scale = 0.7]

\begin{axis}[%
width=0.974\linewidth,
height=\linewidth,
at={(0\linewidth,0\linewidth)},
scale only axis,
xmin=0,
xmax=50,
xlabel style={font=\color{white!15!black}},
xlabel={Time(s)},
ymin=-0.1,
ymax=0.1,
ylabel style={font=\color{white!15!black}},
ylabel={$\kappa(m^{-1})$},
axis background/.style={fill=white},
title style={font=\bfseries},
title={Curvature - PINS},
axis x line*=bottom,
axis y line*=left,
xmajorgrids,
xminorgrids,
ymajorgrids,
yminorgrids,
legend style={legend cell align=left, align=left, draw=white!15!black}
]
\addplot [color=green, dashdotted, line width=2.0pt]
  table[row sep=crcr]{%
0	0.1\\
0.451441271751181	0.1\\
0.902882543502362	0.1\\
1.35432381525354	0.1\\
1.80576508700472	0\\
2.25720635875591	0\\
2.70864763050709	0\\
3.16008890225827	0\\
3.61153017400945	-0.1\\
4.06297144576063	-0.1\\
4.51441271751181	-0.1\\
4.96585398926299	0\\
5.41729526101417	0\\
5.86873653276535	0\\
6.32017780451654	0\\
6.77161907626772	0\\
7.2230603480189	0\\
7.67450161977008	0\\
8.12594289152126	0\\
8.57738416327244	0\\
9.02882543502362	0\\
9.4802667067748	0\\
9.93170797852598	0\\
10.3831492502772	0\\
10.8345905220283	0\\
11.2860317937795	0\\
11.7374730655307	0\\
12.1889143372819	0\\
12.6403556090331	0\\
13.0917968807842	0\\
13.5432381525354	0\\
13.9946794242866	0\\
14.4461206960378	0\\
14.897561967789	0\\
15.3490032395402	0\\
15.8004445112913	0\\
16.2518857830425	0\\
16.7033270547937	0\\
17.1547683265449	0\\
17.6062095982961	0\\
18.0576508700472	0\\
18.5090921417984	0\\
18.9605334135496	0\\
19.4119746853008	0\\
19.863415957052	0\\
20.3148572288031	0\\
20.7662985005543	0\\
21.2177397723055	0\\
21.6691810440567	0\\
22.1206223158079	0\\
22.5720635875591	0\\
23.0235048593102	0\\
23.4749461310614	0\\
23.9263874028126	0\\
24.3778286745638	0\\
24.829269946315	0\\
25.2807112180661	0\\
25.7321524898173	0\\
26.1835937615685	0\\
26.6350350333197	0\\
27.0864763050709	0\\
27.537917576822	0\\
27.9893588485732	0\\
28.4408001203244	0\\
28.8922413920756	0\\
29.3436826638268	0\\
29.7951239355779	0\\
30.2465652073291	0\\
30.6980064790803	0\\
31.1494477508315	0\\
31.6008890225827	0\\
32.0523302943339	0\\
32.503771566085	0\\
32.9552128378362	0\\
33.4066541095874	0\\
33.8580953813386	0\\
34.3095366530898	0\\
34.7609779248409	0\\
35.2124191965921	0\\
35.6638604683433	0\\
36.1153017400945	0\\
36.5667430118457	0\\
37.0181842835969	0\\
37.469625555348	0\\
37.9210668270992	0\\
38.3725080988504	0\\
38.8239493706016	0\\
39.2753906423528	0\\
39.7268319141039	0\\
40.1782731858551	0\\
40.6297144576063	-0.1\\
41.0811557293575	-0.1\\
41.5325970011087	-0.1\\
41.9840382728598	0\\
42.435479544611	0\\
42.8869208163622	0\\
43.3383620881134	0\\
43.7898033598646	0.1\\
44.2412446316157	0.1\\
44.6926859033669	0.1\\
45.1441271751181	0.1\\
};
\addlegendentry{Duboids}

\addplot [color=dodgerblue, dotted, line width=2.0pt]
  table[row sep=crcr]{%
0	0.0999999416900312\\
0.0451439495257969	0.0999999403878002\\
0.0902878990515937	0.0999999376889948\\
0.135431848577391	0.0999999347916653\\
0.180575798103187	0.09999993167475\\
0.225719747628984	0.0999999283141585\\
0.270863697154781	0.0999999246822094\\
0.316007646680578	0.0999999207469372\\
0.361151596206375	0.0999999164712336\\
0.406295545732172	0.0999999118117719\\
0.451439495257969	0.0999999067176508\\
0.496583444783766	0.0999999011286663\\
0.541727394309562	0.0999998949730883\\
0.586871343835359	0.0999998881647658\\
0.632015293361156	0.099999880599315\\
0.677159242886953	0.0999998721490273\\
0.72230319241275	0.0999998626559701\\
0.767447141938547	0.0999998519224819\\
0.812591091464344	0.099999839697834\\
0.857735040990141	0.0999998256591261\\
0.902878990515937	0.0999998093832732\\
0.948022940041734	0.0999997903048241\\
0.993166889567531	0.0999997676504584\\
1.03831083909333	0.0999997403335577\\
1.08345478861912	0.0999997067771539\\
1.12859873814492	0.0999996646010087\\
1.17374268767072	0.0999996100326979\\
1.21888663719652	0.0999995367076965\\
1.26403058672231	0.0999994329565569\\
1.30917453624811	0.0999992747251104\\
1.35431848577391	0.0999990026757921\\
1.3994624352997	0.0999984148188906\\
1.4446063848255	0.0999958620520616\\
1.4897503343513	0.0720187622750279\\
1.53489428387709	-3.38694319047983e-05\\
1.58003823340289	1.89044567483559e-05\\
1.62518218292869	-3.69766590222975e-05\\
1.67032613245448	2.12007988186991e-05\\
1.71547008198028	-4.05034855425701e-05\\
1.76061403150608	2.38488013229315e-05\\
1.80575798103187	-4.45260657233354e-05\\
1.85090193055767	2.6917092539408e-05\\
1.89604588008347	-4.91382541745272e-05\\
1.94118982960927	3.04910255430418e-05\\
1.98633377913506	-5.44567281345498e-05\\
2.03147772866086	3.46776759197726e-05\\
2.07662167818666	-6.06279241750968e-05\\
2.12176562771245	3.96126475492586e-05\\
2.16690957723825	-6.78375682236221e-05\\
2.21205352676405	4.54694724311311e-05\\
2.25719747628984	-7.63239809500854e-05\\
2.30234142581564	5.24727995500185e-05\\
2.34748537534144	-8.63969947600685e-05\\
2.39262932486723	6.09172317932422e-05\\
2.43777327439303	-9.84654119082329e-05\\
2.48291722391883	7.11947791738804e-05\\
2.52806117344462	-0.000113077823240773\\
2.57320512297042	8.38358125605565e-05\\
2.61834907249622	-0.000130985005397197\\
2.66349302202202	9.9571842573648e-05\\
2.70863697154781	-0.000153238513067247\\
2.75378092107361	0.000119434915431002\\
2.79892487059941	-0.000181352811480626\\
2.8440688201252	0.000144921256520048\\
2.889212769651	-0.000217585349711369\\
2.9343567191768	0.000178274026319525\\
2.97950066870259	-0.000265451374837319\\
3.02464461822839	0.00022300274132967\\
3.06978856775419	-0.000330749824489746\\
3.11493251727998	0.000284916920408418\\
3.16007646680578	-0.000423843364886539\\
3.20522041633158	0.000374419042036961\\
3.25036436585737	-0.000565579734184485\\
3.29550831538317	0.00051244751873678\\
3.34065226490897	-0.000806900607063271\\
3.38579621443477	0.000750123617471993\\
3.43094016396056	-0.00133122780088163\\
3.47608411348636	0.00127122294633428\\
3.52122806301216	-0.00828414545595681\\
3.56637201253795	-0.0899164280873723\\
3.61151596206375	-0.099959125496094\\
3.65665991158955	-0.0999755223423666\\
3.70180386111534	-0.0999815426796485\\
3.74694781064114	-0.0999846468556474\\
3.79209176016694	-0.0999864900960899\\
3.83723570969273	-0.0999876618283936\\
3.88237965921853	-0.0999884256459014\\
3.92752360874433	-0.0999889165189687\\
3.97266755827012	-0.0999892097463174\\
4.01781150779592	-0.0999893495375342\\
4.06295545732172	-0.0999893623291809\\
4.10809940684751	-0.0999892635173151\\
4.15324335637331	-0.0999890610448708\\
4.19838730589911	-0.0999887573480087\\
4.24353125542491	-0.0999883503679175\\
4.2886752049507	-0.0999878339731839\\
4.3338191544765	-0.0999871979583229\\
4.3789631040023	-0.0999864276841824\\
4.42410705352809	-0.0999855033603272\\
4.46925100305389	-0.099984398915112\\
4.51439495257969	-0.099983080341249\\
4.55953890210548	-0.0999815033300161\\
4.60468285163128	-0.0999796098993568\\
4.64982680115708	-0.09997732355484\\
4.69497075068287	-0.0999745422555498\\
4.74011470020867	-0.0999711280145467\\
4.78525864973447	-0.0999668912074119\\
4.83040259926026	-0.0999615663279107\\
4.87554654878606	-0.0999547734892341\\
4.92069049831186	-0.0999459553212824\\
4.96583444783766	-0.0999342696368951\\
5.01097839736345	-0.0999183986667687\\
5.05612234688925	-0.0998961915225898\\
5.10126629641505	-0.0998639484862272\\
5.14641024594084	-0.0998148620939642\\
5.19155419546664	-0.0997352141638841\\
5.23669814499244	-0.099593462648042\\
5.28184209451823	-0.0992996971563545\\
5.32698604404403	-0.0984561182110328\\
5.37212999356983	-0.0762228073761637\\
5.41727394309562	0.0206684018875061\\
5.46241789262142	0.0966691958570933\\
5.50756184214722	0.0974841961810453\\
5.55270579167301	0.0975120233454786\\
5.59784974119881	0.0971683277466388\\
5.64299369072461	0.0964020098097989\\
5.68813764025041	0.0949006002462994\\
5.7332815897762	0.0917790109908198\\
5.778425539302	0.0841981617414901\\
5.8235694888278	0.0616388402596401\\
5.86871343835359	0.0158192935350022\\
5.91385738787939	-0.0227837701991328\\
5.95900133740519	-0.0321906566361855\\
6.00414528693098	-0.0273597442006622\\
6.04928923645678	-0.0178544484375353\\
6.09443318598258	-0.00867878679814384\\
6.13957713550837	-0.00236409452466968\\
6.18472108503417	0.000848324616897042\\
6.22986503455997	0.00186283567218966\\
6.27500898408577	0.00171173662222839\\
6.32015293361156	0.00114677434276071\\
6.36529688313736	0.000585734979577768\\
6.41044083266315	0.00019196778858149\\
6.45558478218895	-1.96500878208545e-05\\
6.50072873171475	-9.70920397742392e-05\\
6.54587268124055	-9.88739541040498e-05\\
6.59101663076634	-7.03663419458096e-05\\
6.63616058029214	-3.83649108928073e-05\\
6.68130452981794	-1.44427012973101e-05\\
6.72644847934373	-7.65622873213485e-07\\
6.77159242886953	4.84084199528266e-06\\
6.81673637839533	5.61626775865523e-06\\
6.86188032792112	4.26218849822153e-06\\
6.90702427744692	2.47041649261079e-06\\
6.95216822697272	1.03577692353834e-06\\
6.99731217649851	1.65436751488453e-07\\
7.04245612602431	-2.25809826365456e-07\\
7.08760007555011	-3.13011362543851e-07\\
7.13274402507591	-2.54847206663973e-07\\
7.1778879746017	-1.56681508793996e-07\\
7.2230319241275	-7.17793859558585e-08\\
7.2681758736533	-1.71760478155364e-08\\
7.31331982317909	9.37783791824447e-09\\
7.35846377270489	1.70532758775713e-08\\
7.40360772223069	1.50348864348168e-08\\
7.44875167175648	9.79939549659511e-09\\
7.49389562128228	4.84499262603213e-09\\
7.53903957080808	1.46475913262501e-09\\
7.58418352033387	-2.97664970833735e-10\\
7.62932746985967	-9.03033325227381e-10\\
7.67447141938547	-8.74404814054959e-10\\
7.71961536891127	-6.0483423652472e-10\\
7.76475931843706	-3.20117422249182e-10\\
7.80990326796286	-1.13547606929292e-10\\
7.85504721748865	1.25965542543283e-12\\
7.90019116701445	4.60527168833179e-11\\
7.94533511654025	5.00692188785352e-11\\
7.99047906606605	3.68570639935885e-11\\
8.03562301559184	2.07701939964035e-11\\
8.08076696511764	8.30481243995539e-12\\
8.12591091464344	9.46881511801228e-13\\
8.17105486416923	-2.22755032152021e-12\\
8.21619881369503	-2.81842636878305e-12\\
8.26134276322083	-2.21785163358764e-12\\
8.30648671274662	-1.32532165806756e-12\\
8.35163066227242	-5.83162787314897e-13\\
8.39677461179822	-1.20651011979561e-13\\
8.44191856132401	9.68517372893471e-14\\
8.48706251084981	1.54721032247817e-13\\
8.53220646037561	1.31921866773443e-13\\
8.57735040990141	8.47045296399931e-14\\
8.6224943594272	4.26336877137597e-14\\
8.667638308953	1.39838023123407e-14\\
8.7127822584788	-2.16729545700643e-15\\
8.75792620800459	-8.73244250729156e-15\\
8.80307015753039	-9.03249871900472e-15\\
8.84821410705619	-6.48222084936107e-15\\
8.89335805658198	-3.56406260523776e-15\\
8.93850200610778	-1.36324829164949e-15\\
8.98364595563358	-9.47710773581404e-17\\
9.02878990515937	4.32219145717097e-16\\
9.07393385468517	5.11843610127818e-16\\
9.11907780421097	3.91961230014988e-16\\
9.16422175373676	2.29011601594683e-16\\
9.20936570326256	9.72575523725731e-17\\
9.25450965278836	1.66970287123474e-17\\
9.29965360231416	-1.99263885844303e-17\\
9.34479755183995	-2.84468046494034e-17\\
9.38994150136575	-2.33950217726176e-17\\
9.43508545089155	-1.44965666625704e-17\\
9.48022940041734	-6.71362319598803e-18\\
9.52537334994314	-1.66867941342096e-18\\
9.57051729946894	8.08849863061545e-19\\
9.61566124899473	1.54451607083119e-18\\
9.66080519852053	1.3776429827218e-18\\
9.70594914804633	9.05025317272275e-19\\
9.75109309757212	4.51751052105979e-19\\
9.79623704709792	1.3998904116606e-19\\
9.84138099662372	-2.40064495763239e-20\\
9.88652494614951	-8.14272072832903e-20\\
9.93166889567531	-7.99607648604017e-20\\
9.97681284520111	-5.5762469467061e-20\\
10.0219567947269	-2.97704003771495e-20\\
10.0671007442527	-1.07510913705617e-20\\
10.1122446937785	-9.3272642387306e-23\\
10.1573886433043	4.12685638973582e-21\\
10.2025325928301	4.56784824285596e-21\\
10.2476765423559	3.39192363211655e-21\\
10.2928204918817	1.92731888658617e-21\\
10.3379644414075	7.81821332425013e-22\\
10.3831083909333	1.00249164846247e-22\\
10.4282523404591	-1.97480403039616e-22\\
10.4733962899849	-2.56266122327401e-22\\
10.5185402395107	-2.03695044724132e-22\\
10.5636841890365	-1.22840721241123e-22\\
10.6088281385623	-5.47406104899784e-23\\
10.6539720880881	-1.17714600528869e-23\\
10.6991160376139	8.61168504712053e-24\\
10.7442599871397	1.40817160506957e-23\\
10.7894039366655	1.20687367530452e-23\\
10.8345478861912	7.70147231455493e-24\\
10.879691835717	3.69003600958725e-24\\
10.9248357852428	1.03017320984145e-24\\
10.9699797347686	-2.83651372420326e-25\\
11.0151236842944	-7.15407103249962e-25\\
11.0602676338202	-6.8877529665597e-25\\
11.105411583346	-4.73589127506135e-25\\
11.1505555328718	-2.53097910176939e-25\\
11.1956994823976	-9.66525230973822e-26\\
11.2408434319234	-4.35635175273049e-27\\
11.2859873814492	3.32892054901343e-26\\
11.331131330975	4.06635822596332e-26\\
11.3762752805008	3.22215443782741e-26\\
11.4214192300266	1.41592901299584e-26\\
11.4665631795524	8.50439834749002e-27\\
11.5117071290782	2.63106102592934e-27\\
11.556851078604	-4.25359548138871e-27\\
11.6019950281298	2.23649586260545e-27\\
11.6471389776556	-1.68390591421063e-27\\
11.6922829271814	-6.80947811636678e-27\\
11.7374268767072	-1.99241724894154e-27\\
11.782570826233	-1.48594446477095e-27\\
11.8277147757588	6.33005947425019e-27\\
11.8728587252846	5.36164733339334e-27\\
11.9180026748104	-5.72164091695802e-27\\
11.9631466243362	-1.75445972664029e-28\\
12.008290573862	4.23214235647228e-27\\
12.0534345233878	1.25010865546498e-27\\
12.0985784729136	-1.92777630106791e-27\\
12.1437224224394	-8.16580220703812e-27\\
12.1888663719652	-1.71531526270329e-26\\
12.234010321491	-5.9553440009374e-27\\
12.2791542710167	1.85654216609638e-26\\
12.3242982205425	1.76273091870669e-26\\
12.3694421700683	5.32051900245014e-27\\
12.4145861195941	-3.55921233336699e-27\\
12.4597300691199	-7.27448460577859e-27\\
12.5048740186457	-1.64557557901733e-27\\
12.5500179681715	9.8722630350777e-28\\
12.5951619176973	-2.16391128021918e-27\\
12.6403058672231	1.05586297486828e-27\\
12.6854498167489	2.29329875025146e-27\\
12.7305937662747	-1.49189988873992e-27\\
12.7757377158005	-4.40387359647292e-27\\
12.8208816653263	-6.33584367106783e-27\\
12.8660256148521	-7.55592188836624e-27\\
12.9111695643779	1.16553714362697e-27\\
12.9563135139037	1.45562379443716e-26\\
13.0014574634295	3.69447770167846e-27\\
13.0466014129553	4.24392565871538e-27\\
13.0917453624811	4.24881072369518e-27\\
13.1368893120069	-1.24964762384567e-26\\
13.1820332615327	-1.58930979017263e-27\\
13.2271772110585	2.53386924371525e-27\\
13.2723211605843	-2.35814766457867e-29\\
13.3174651101101	5.00279429693593e-27\\
13.3626090596359	-3.09195269628664e-27\\
13.4077530091617	-3.70341524229435e-27\\
13.4528969586875	2.72374713166462e-27\\
13.4980409082133	7.61653728507602e-28\\
13.5431848577391	-7.07334198934455e-27\\
13.5883288072649	-6.40124787183383e-27\\
13.6334727567907	2.91104870169196e-27\\
13.6786167063165	1.77590097694286e-28\\
13.7237606558422	-5.12582402772675e-27\\
13.768904605368	6.60606361561219e-27\\
13.8140485548938	1.19725759856886e-26\\
13.8591925044196	-2.66618867163519e-28\\
13.9043364539454	-6.97624932571264e-27\\
13.9494804034712	-2.60810231200578e-27\\
13.994624352997	3.90316155739159e-27\\
14.0397683025228	1.44324067525742e-27\\
14.0849122520486	-1.76092215996513e-27\\
14.1300562015744	6.04601355783847e-27\\
14.1752001511002	5.41986799692983e-28\\
14.220344100626	-9.93965293268336e-27\\
14.2654880501518	8.81097795924956e-29\\
14.3106319996776	7.09530458731447e-27\\
14.3557759492034	-1.46473452018789e-28\\
14.4009198987292	-8.36264444558048e-27\\
14.446063848255	-2.86213007511018e-27\\
14.4912077977808	1.15214994906311e-26\\
14.5363517473066	5.35984063882824e-27\\
14.5814956968324	-9.95230105537024e-27\\
14.6266396463582	-1.91246834263296e-27\\
14.671783595884	6.1592000948155e-27\\
14.7169275454098	1.76960008542532e-27\\
14.7620714949356	-2.19002286209953e-27\\
14.8072154444614	-2.72022185928763e-27\\
14.8523593939872	-5.13037016731083e-27\\
14.897503343513	-1.35385600776764e-27\\
14.9426472930388	1.04004194281228e-26\\
14.9877912425646	4.60494700932316e-27\\
15.0329351920904	2.85651714365902e-27\\
15.0780791416162	3.11844709346838e-27\\
15.123223091142	-8.88988305298793e-27\\
15.1683670406677	-8.56460843538307e-27\\
15.2135109901935	1.16478781528115e-26\\
15.2586549397193	1.21259443956293e-26\\
15.3037988892451	-1.26859165033566e-26\\
15.3489428387709	-8.9335480213207e-27\\
15.3940867882967	6.20708877641203e-27\\
15.4392307378225	1.05178009124491e-27\\
15.4843746873483	-8.53273055747901e-27\\
15.5295186368741	-2.73285685327383e-27\\
15.5746625863999	8.56855063478555e-27\\
15.6198065359257	5.76665722474767e-27\\
15.6649504854515	-5.60063729977452e-27\\
15.7100944349773	-7.8711747788444e-27\\
15.7552383845031	-1.01355663787435e-27\\
15.8003823340289	3.61295404976935e-27\\
15.8455262835547	7.13776447577073e-28\\
15.8906702330805	-2.77561658509906e-27\\
15.9358141826063	3.96081409139175e-27\\
15.9809581321321	6.12987216504423e-27\\
16.0261020816579	2.63528929463703e-27\\
16.0712460311837	3.49427513977609e-27\\
16.1163899807095	-2.45877967495558e-27\\
16.1615339302353	-1.19490863570794e-26\\
16.2066778797611	-1.95726738046373e-26\\
16.2518218292869	-1.22177260202063e-26\\
16.2969657788127	1.58102658254855e-26\\
16.3421097283385	2.07948249043016e-26\\
16.3872536778643	2.92138310407539e-27\\
16.4323976273901	-7.42126866035285e-27\\
16.4775415769159	6.4746295992693e-27\\
16.5226855264417	1.55113057793104e-26\\
16.5678294759675	-1.16897292023627e-26\\
16.6129734254932	-1.68960900747576e-26\\
16.658117375019	-1.11281411530931e-27\\
16.7032613245448	-5.31041055128764e-28\\
16.7484052740706	5.23440465880252e-28\\
16.7935492235964	1.06696166577848e-26\\
16.8386931731222	1.1481870160376e-26\\
16.883837122648	-2.59155453990496e-27\\
16.9289810721738	-8.38811778666417e-27\\
16.9741250216996	-1.17404782786247e-26\\
17.0192689712254	2.68255069612866e-28\\
17.0644129207512	1.2886733475092e-26\\
17.109556870277	-1.60534574608666e-27\\
17.1547008198028	1.33677303436075e-29\\
17.1998447693286	4.50875255051498e-27\\
17.2449887188544	-2.95947856509199e-27\\
17.2901326683802	9.08214402332219e-28\\
17.335276617906	1.30721922207106e-28\\
17.3804205674318	-5.05864202431852e-27\\
17.4255645169576	-5.67195917872543e-27\\
17.4707084664834	-6.41815175994926e-28\\
17.5158524160092	1.2336303456971e-26\\
17.560996365535	1.32382015275102e-26\\
17.6061403150608	-6.54582163852194e-27\\
17.6512842645866	-1.96820096499864e-26\\
17.6964282141124	-8.50025620293778e-27\\
17.7415721636382	8.67518833535506e-27\\
17.786716113164	3.78628767032126e-27\\
17.8318600626898	-4.44990594505243e-27\\
17.8770040122156	5.70420440699439e-27\\
17.9221479617414	1.04826672694244e-26\\
17.9672919112672	4.16699866465177e-28\\
18.012435860793	-7.04466696982965e-27\\
18.0575798103187	-7.10032606367302e-28\\
18.1027237598445	1.7068628427318e-30\\
18.1478677093703	-8.97377487658007e-27\\
18.1930116588961	1.03738271172492e-27\\
18.2381556084219	1.55796133182067e-26\\
18.2832995579477	9.80874788649674e-27\\
18.3284435074735	-1.14947434477914e-26\\
18.3735874569993	-2.00717862345349e-26\\
18.4187314065251	-4.15675138495542e-28\\
18.4638753560509	1.31179779461161e-26\\
18.5090193055767	3.96360164700271e-27\\
18.5541632551025	-3.91546994581909e-27\\
18.5993072046283	3.39336215217772e-27\\
18.6444511541541	3.98503677135992e-27\\
18.6895951036799	-2.78230566873463e-27\\
18.7347390532057	4.49368018898427e-28\\
18.7798830027315	-5.13799281883808e-28\\
18.8250269522573	-6.62113516207373e-27\\
18.8701709017831	-6.93262850659722e-27\\
18.9153148513089	-6.23316130326912e-28\\
18.9604588008347	6.80760035033526e-27\\
19.0056027503605	1.80330849976521e-27\\
19.0507466998863	6.49531485856882e-28\\
19.0958906494121	1.63728086717026e-26\\
19.1410345989379	1.23509482033587e-26\\
19.1861785484637	-1.51231410337064e-26\\
19.2313224979895	-2.19526391299506e-26\\
19.2764664475153	-2.24058369055563e-27\\
19.3216103970411	1.45157678782193e-26\\
19.3667543465669	9.30907499855212e-27\\
19.4118982960927	-7.87238472824386e-27\\
19.4570422456184	-8.6053494197917e-27\\
19.5021861951442	3.14221032309416e-28\\
19.54733014467	2.67090899875164e-27\\
19.5924740941958	-4.92041895857997e-28\\
19.6376180437216	-2.42117113898633e-27\\
19.6827619932474	5.24779495714131e-27\\
19.7279059427732	5.64795638456754e-27\\
19.773049892299	-5.11110562359396e-27\\
19.8181938418248	-3.68014531398164e-27\\
19.8633377913506	4.28529982367329e-27\\
19.9084817408764	2.40934260274976e-27\\
19.9536256904022	-4.06891118610761e-27\\
19.998769639928	-4.79883359750842e-27\\
20.0439135894538	1.32573577993196e-27\\
20.0890575389796	5.31472077214588e-27\\
20.1342014885054	1.53782837754592e-27\\
20.1793454380312	-6.05788415491678e-27\\
20.224489387557	-8.43046799243881e-28\\
20.2696333370828	9.23570696890271e-27\\
20.3147772866086	3.24340790910745e-27\\
20.3599212361344	-6.37371899462504e-27\\
20.4050651856602	-1.05608028112465e-26\\
20.450209135186	-5.68822827263045e-27\\
20.4953530847118	5.70132896631632e-27\\
20.5404970342376	6.73640905231781e-27\\
20.5856409837634	1.04161625358292e-26\\
20.6307849332892	8.67165206127917e-27\\
20.675928882815	-1.18416708906807e-26\\
20.7210728323408	-1.31082357672262e-26\\
20.7662167818666	3.33925173536055e-27\\
20.8113607313924	3.104514813627e-27\\
20.8565046809182	-5.19797160774173e-27\\
20.901648630444	2.77592055537336e-27\\
20.9467925799697	1.21841286784203e-26\\
20.9919365294955	-1.31935752098869e-27\\
21.0370804790213	-1.37973105154033e-26\\
21.0822244285471	-2.38862414974106e-27\\
21.1273683780729	9.83207682786099e-27\\
21.1725123275987	4.94515726906289e-27\\
21.2176562771245	-4.224448963035e-27\\
21.2628002266503	4.21703811570064e-27\\
21.3079441761761	3.86449996192059e-27\\
21.3530881257019	-1.18061002412666e-26\\
21.3982320752277	-4.21239827011591e-27\\
21.4433760247535	5.648490543118e-27\\
21.4885199742793	-3.44885605498631e-28\\
21.5336639238051	-5.5291679742724e-27\\
21.5788078733309	3.42956994340456e-28\\
21.6239518228567	1.04741747459976e-26\\
21.6690957723825	6.20118209968801e-27\\
21.7142397219083	-7.15111789855004e-27\\
21.7593836714341	-9.01333852219633e-27\\
21.8045276209599	-5.04674214570815e-28\\
21.8496715704857	2.99373959370887e-27\\
21.8948155200115	-2.54987947058068e-27\\
21.9399594695373	-8.92777000701377e-28\\
21.9851034190631	1.56664287815186e-26\\
22.0302473685889	9.77251671679269e-27\\
22.0753913181147	-1.66451607037465e-26\\
22.1205352676405	-1.25628189397668e-26\\
22.1656792171663	6.08574547396038e-27\\
22.2108231666921	4.99112127092477e-27\\
22.2559671162179	-5.36184830263506e-27\\
22.3011110657437	-2.64843123530548e-27\\
22.3462550152694	1.04027690650386e-26\\
22.3913989647952	2.39608965635928e-27\\
22.436542914321	-1.03029758738814e-26\\
22.4816868638468	2.88259723925494e-27\\
22.5268308133726	8.89523415913273e-27\\
22.5719747628984	-3.50550492139155e-27\\
22.6171187124242	-1.03169582480063e-26\\
22.66226266195	-2.96316909002643e-27\\
22.7074066114758	9.52965559003893e-27\\
22.7525505610016	1.03951942896244e-26\\
22.7976945105274	-3.49284118808592e-27\\
22.8428384600532	-1.31788885484433e-26\\
22.887982409579	9.35067121469492e-28\\
22.9331263591048	7.50152553187503e-27\\
22.9782703086306	-5.8733447064759e-27\\
23.0234142581564	-2.26120323219968e-27\\
23.0685582076822	4.81390520330803e-27\\
23.113702157208	4.27033293130982e-27\\
23.1588461067338	3.44975584142493e-27\\
23.2039900562596	-4.50394506177492e-27\\
23.2491340057854	-2.01987207424064e-27\\
23.2942779553112	-1.37004045434313e-27\\
23.339421904837	-7.81250617965089e-27\\
23.3845658543628	2.16372962657996e-27\\
23.4297098038886	7.79784806350443e-27\\
23.4748537534144	2.39492554058604e-27\\
23.5199977029402	-4.43703493179787e-28\\
23.565141652466	-3.7479442455257e-27\\
23.6102856019918	3.78464399414392e-28\\
23.6554295515176	1.88674365020483e-27\\
23.7005735010434	-5.14178276202878e-27\\
23.7457174505692	-1.24412256444053e-27\\
23.790861400095	8.57535378823772e-27\\
23.8360053496207	4.66120429070871e-27\\
23.8811492991465	-5.17865625574608e-27\\
23.9262932486723	-4.02180567615215e-27\\
23.9714371981981	-1.14877526431968e-27\\
24.0165811477239	4.8492448966913e-29\\
24.0617250972497	-8.63261039763624e-28\\
24.1068690467755	-2.49889502502163e-27\\
24.1520129963013	3.14083591151624e-27\\
24.1971569458271	3.97309533112064e-27\\
24.2423008953529	1.64832460285678e-27\\
24.2874448448787	3.84312885658318e-27\\
24.3325887944045	3.69873719390308e-27\\
24.3777327439303	-4.73840365106018e-27\\
24.4228766934561	-1.33661390988582e-26\\
24.4680206429819	-5.53054951251721e-27\\
24.5131645925077	4.70917035927556e-27\\
24.5583085420335	3.05099475554514e-27\\
24.6034524915593	9.00348233608217e-28\\
24.6485964410851	7.90034119418063e-27\\
24.6937403906109	4.81532384851567e-27\\
24.7388843401367	-4.67449042428753e-27\\
24.7840282896625	-1.16888533216896e-27\\
24.8291722391883	-2.10099876360751e-27\\
24.8743161887141	-3.91729929991761e-27\\
24.9194601382399	-2.16448514004219e-27\\
24.9646040877657	4.46591345495795e-28\\
25.0097480372915	7.04365351797433e-27\\
25.0548919868173	7.41392367344933e-27\\
25.1000359363431	-2.35905326422257e-27\\
25.1451798858689	-1.25569548953654e-26\\
25.1903238353947	-5.6889081064456e-27\\
25.2354677849205	5.99655468163031e-27\\
25.2806117344462	-6.64359948404461e-28\\
25.325755683972	-3.20228833790178e-27\\
25.3708996334978	1.07432449690577e-26\\
25.4160435830236	1.79809102203387e-26\\
25.4611875325494	-3.0349163871195e-28\\
25.5063314820752	-2.07106100367231e-26\\
25.551475431601	-1.01848883703665e-26\\
25.5966193811268	1.03421419191317e-26\\
25.6417633306526	6.89094026589659e-27\\
25.6869072801784	-6.85485026741245e-27\\
25.7320512297042	-2.79408423533538e-27\\
25.77719517923	6.67780236840742e-27\\
25.8223391287558	3.26920117304047e-27\\
25.8674830782816	-4.48307224078466e-27\\
25.9126270278074	6.31359716974054e-29\\
25.9577709773332	3.15751194688118e-27\\
26.002914926859	-2.104104319774e-27\\
26.0480588763848	1.92670770847536e-27\\
26.0932028259106	4.80964887714884e-27\\
26.1383467754364	2.9102739173588e-27\\
26.1834907249622	4.97225810887378e-28\\
26.228634674488	-6.02991619194147e-27\\
26.2737786240138	-7.08775970241377e-27\\
26.3189225735396	-2.51524737657102e-27\\
26.3640665230654	-9.35424439287201e-28\\
26.4092104725912	-1.27523794840616e-27\\
26.454354422117	8.91253859941932e-27\\
26.4994983716428	5.21289545444644e-27\\
26.5446423211686	-1.21939507202111e-26\\
26.5897862706944	-3.03193007571488e-27\\
26.6349302202202	1.01114695275971e-26\\
26.680074169746	1.9674285695002e-27\\
26.7252181192717	-1.02048708140938e-26\\
26.7703620687975	-4.58985075893441e-27\\
26.8155060183233	5.65640781236686e-27\\
26.8606499678491	7.2533074994706e-27\\
26.9057939173749	2.134576744543e-27\\
26.9509378669007	-6.81801227983064e-27\\
26.9960818164265	-1.50103751743578e-27\\
27.0412257659523	4.53707103926837e-27\\
27.0863697154781	-1.60580351858101e-27\\
27.1315136650039	1.20431319319164e-28\\
27.1766576145297	7.02126230858459e-27\\
27.2218015640555	2.80683644180352e-27\\
27.2669455135813	-6.23591323474082e-27\\
27.3120894631071	-1.71553879925159e-26\\
27.3572334126329	-1.18696686530402e-26\\
27.4023773621587	6.00119679526716e-27\\
27.4475213116845	5.53693399889073e-27\\
27.4926652612103	7.95338117002352e-27\\
27.5378092107361	1.4444769844434e-26\\
27.5829531602619	7.5577790003906e-27\\
27.6280971097877	-5.866087774646e-27\\
27.6732410593135	-1.42212251716689e-26\\
27.7183850088393	-3.73868097292335e-27\\
27.7635289583651	4.51621959151847e-27\\
27.8086729078909	-4.33558659729299e-27\\
27.8538168574167	-2.99741921517177e-27\\
27.8989608069425	8.80443568683597e-27\\
27.9441047564683	6.17480460682136e-27\\
27.9892487059941	-1.91145127539489e-27\\
28.0343926555199	1.23073102988213e-27\\
28.0795366050457	9.93404437883491e-28\\
28.1246805545715	-4.01482700843783e-27\\
28.1698245040972	-4.85830971513802e-27\\
28.214968453623	-2.32744976260982e-27\\
28.2601124031488	-1.90043636371094e-27\\
28.3052563526746	2.63099579667736e-27\\
28.3504003022004	1.17526039275948e-26\\
28.3955442517262	5.43403965452875e-27\\
28.440688201252	-9.02695743946644e-27\\
28.4858321507778	-8.3596157106664e-27\\
28.5309761003036	3.82702317253784e-27\\
28.5761200498294	-5.30658476051087e-28\\
28.6212639993552	-9.86178342896995e-27\\
28.666407948881	-5.4778731763645e-27\\
28.7115518984068	1.87010084382038e-27\\
28.7566958479326	8.68848695574263e-27\\
28.8018397974584	7.77587652058795e-27\\
28.8469837469842	-1.61487493261314e-28\\
28.89212769651	-1.59019362280396e-27\\
28.9372716460358	5.13687249062598e-27\\
28.9824155955616	1.16532572659035e-27\\
29.0275595450874	-1.01458320065549e-26\\
29.0727034946132	-2.50488654679676e-27\\
29.117847444139	5.6195325189287e-27\\
29.1629913936648	-5.03892757714429e-28\\
29.2081353431906	-6.05196296139603e-28\\
29.2532792927164	2.71642001211043e-27\\
29.2984232422422	-1.89734845341111e-27\\
29.343567191768	-1.96564184143862e-27\\
29.3887111412938	2.17479635768142e-28\\
29.4338550908196	-1.6300723530907e-27\\
29.4789990403454	1.00535613741853e-27\\
29.5241429898712	5.05701564488034e-27\\
29.5692869393969	6.96795792728598e-27\\
29.6144308889227	2.43972558277205e-27\\
29.6595748384485	-7.19855233004306e-27\\
29.7047187879743	-6.51454699709478e-27\\
29.7498627375001	3.16770103932931e-28\\
29.7950066870259	-2.80332996048641e-28\\
29.8401506365517	7.61902816282534e-27\\
29.8852945860775	2.09248490035077e-27\\
29.9304385356033	-1.39140597284705e-26\\
29.9755824851291	-1.69218219537101e-27\\
30.0207264346549	1.00484458258313e-26\\
30.0658703841807	1.55156405528916e-27\\
30.1110143337065	-8.96883224734654e-27\\
30.1561582832323	-7.84228238242294e-28\\
30.2013022327581	7.64041656705694e-27\\
30.2464461822839	1.56664629385869e-28\\
30.2915901318097	4.55824228469314e-28\\
30.3367340813355	7.70992566662982e-27\\
30.3818780308613	4.29688690223863e-27\\
30.4270219803871	-8.59417862520932e-27\\
30.4721659299129	-1.30077378701321e-26\\
30.5173098794387	-2.00956569718634e-27\\
30.5624538289645	2.76905091289794e-27\\
30.6075977784903	3.14932104240106e-27\\
30.6527417280161	8.63313475233309e-27\\
30.6978856775419	5.75973351179908e-27\\
30.7430296270677	-4.05581208313531e-28\\
30.7881735765935	-5.10166332076438e-27\\
30.8333175261193	-8.63992514830535e-27\\
30.8784614756451	-4.97428882576861e-27\\
30.9236054251709	4.51561074488542e-27\\
30.9687493746967	8.19479348662923e-27\\
31.0138933242225	-2.69314938158181e-28\\
31.0590372737482	-6.87949457429472e-27\\
31.104181223274	-8.54568067265326e-27\\
31.1493251727998	6.13515024958479e-27\\
31.1944691223256	2.04819072671992e-26\\
31.2396130718514	1.76240963301165e-27\\
31.2847570213772	-7.76084477304258e-27\\
31.329900970903	6.72867524366226e-27\\
31.3750449204288	2.27963331622139e-27\\
31.4201888699546	-3.27809693940338e-26\\
31.4653328194804	-2.82162625392962e-26\\
31.5104767690062	1.62261297066838e-26\\
31.555620718532	2.15536563008997e-26\\
31.6007646680578	9.17422024984402e-27\\
31.6459086175836	4.57955730912306e-27\\
31.6910525671094	-2.23858686638815e-27\\
31.7361965166352	-3.67102367629858e-27\\
31.781340466161	1.48053618520947e-27\\
31.8264844156868	-1.50104866160183e-27\\
31.8716283652126	-3.73361511047769e-27\\
31.9167723147384	2.03125185584693e-27\\
31.9619162642642	3.6754110266311e-28\\
32.00706021379	6.51337349287196e-28\\
32.0522041633158	3.4741786543661e-27\\
32.0973481128416	-8.01077660906335e-28\\
32.1424920623674	-7.65082081837764e-27\\
32.1876360118932	-3.94997774446679e-27\\
32.232779961419	3.04839106293102e-27\\
32.2779239109448	-1.50753421888276e-27\\
32.3230678604706	-1.89898753270121e-27\\
32.3682118099964	1.42184487898856e-26\\
32.4133557595222	1.14403687514449e-26\\
32.4584997090479	-1.05697327427645e-26\\
32.5036436585737	-8.56795787121721e-27\\
32.5487876080995	-4.67771156388305e-28\\
32.5939315576253	-2.39268048082281e-27\\
32.6390755071511	-6.83517779412317e-27\\
32.6842194566769	-2.09636948445183e-27\\
32.7293634062027	1.21648852705674e-26\\
32.7745073557285	8.42849234016509e-27\\
32.8196513052543	-5.98156328140408e-27\\
32.8647952547801	-1.76207786305946e-27\\
32.9099392043059	5.33691574249947e-27\\
32.9550831538317	7.5003515574814e-28\\
33.0002271033575	-4.14235268775048e-27\\
33.0453710528833	-4.78404451294436e-27\\
33.0905150024091	-1.38427276738472e-27\\
33.1356589519349	-2.28037027179135e-27\\
33.1808029014607	-2.29201704940118e-27\\
33.2259468509865	6.5801605236324e-27\\
33.2710908005123	9.34241359661316e-27\\
33.3162347500381	1.10111994597166e-27\\
33.3613786995639	-1.62850864953635e-27\\
33.4065226490897	2.08701992067866e-27\\
33.4516665986155	-4.73645697334656e-27\\
33.4968105481413	-1.38398316479523e-26\\
33.5419544976671	-4.93921874169021e-27\\
33.5870984471929	1.09423316353353e-26\\
33.6322423967187	1.52849311368761e-27\\
33.6773863462445	-4.88524214882673e-27\\
33.7225302957703	2.32985492394806e-26\\
33.7676742452961	3.82045680288882e-26\\
33.8128181948219	3.35319669545536e-26\\
33.8579621443477	2.81750471158288e-26\\
33.9031060938734	5.64228349565877e-27\\
33.9482500433992	-7.02921850140123e-26\\
33.993393992925	-2.38469385794678e-25\\
34.0385379424508	-4.87174548043489e-25\\
34.0836818919766	-7.21155450171425e-25\\
34.1288258415024	-7.56333563839499e-25\\
34.1739697910282	-2.9953404979347e-25\\
34.219113740554	1.05766587556501e-24\\
34.2642576900798	3.72327691128769e-24\\
34.3094016396056	7.72167073250027e-24\\
34.3545455891314	1.20775786307098e-23\\
34.3996895386572	1.40756046720254e-23\\
34.444833488183	8.59195244329374e-24\\
34.4899774377088	-1.1789140655055e-23\\
34.5351213872346	-5.47396892378984e-23\\
34.5802653367604	-1.22828413138134e-22\\
34.6254092862862	-2.03687583899671e-22\\
34.670553235812	-2.56271769932345e-22\\
34.7156971853378	-1.97481885449497e-22\\
34.7608411348636	1.00255698180613e-22\\
34.8059850843894	7.81816551785757e-22\\
34.8511290339152	1.9273106339907e-21\\
34.896272983441	3.39192468328441e-21\\
34.9414169329668	4.56785245644118e-21\\
34.9865608824926	4.12685927109509e-21\\
35.0317048320184	-9.32673404890238e-23\\
35.0768487815442	-1.07510869826479e-20\\
35.12199273107	-2.97704085359796e-20\\
35.1671366805958	-5.57624882873705e-20\\
35.2122806301216	-7.99607667529851e-20\\
35.2574245796474	-8.142719114193e-20\\
35.3025685291732	-2.40064431403258e-20\\
35.347712478699	1.39989047607491e-19\\
35.3928564282247	4.51751074411853e-19\\
35.4380003777505	9.05025324891736e-19\\
35.4831443272763	1.37764295707819e-18\\
35.5282882768021	1.5445160330946e-18\\
35.5734322263279	8.08849857325879e-19\\
35.6185761758537	-1.66867938285305e-18\\
35.6637201253795	-6.713623206935e-18\\
35.7088640749053	-1.44965667531154e-17\\
35.7540080244311	-2.33950219164974e-17\\
35.7991519739569	-2.84468048741021e-17\\
35.8442959234827	-1.9926388880706e-17\\
35.8894398730085	1.6697028547577e-17\\
35.9345838225343	9.72575527099347e-17\\
35.9797277720601	2.29011602390809e-16\\
36.0248717215859	3.91961230595392e-16\\
36.0700156711117	5.11843610185365e-16\\
36.1151596206375	4.32219145480472e-16\\
36.1603035701633	-9.477107765193e-17\\
36.2054475196891	-1.3632482917768e-15\\
36.2505914692149	-3.56406260508207e-15\\
36.2957354187407	-6.4822208492271e-15\\
36.3408793682665	-9.03249871926493e-15\\
36.3860233177923	-8.732442507681e-15\\
36.4311672673181	-2.16729545695016e-15\\
36.4763112168439	1.39838023125723e-14\\
36.5214551663697	4.26336877134294e-14\\
36.5665991158955	8.47045296394428e-14\\
36.6117430654213	1.31921866773494e-13\\
36.6568870149471	1.54721032248187e-13\\
36.7020309644729	9.68517372894549e-14\\
36.7471749139986	-1.20651011979125e-13\\
36.7923188635245	-5.83162787314207e-13\\
36.8374628130502	-1.32532165806743e-12\\
36.882606762576	-2.21785163358796e-12\\
36.9277507121018	-2.81842636878356e-12\\
36.9728946616276	-2.22755032152051e-12\\
37.0180386111534	9.46881511801001e-13\\
37.0631825606792	8.30481243995503e-12\\
37.108326510205	2.07701939964036e-11\\
37.1534704597308	3.68570639935891e-11\\
37.1986144092566	5.0069218878536e-11\\
37.2437583587824	4.60527168833191e-11\\
37.2889023083082	1.25965542543445e-12\\
37.334046257834	-1.13547606929291e-10\\
37.3791902073598	-3.20117422249187e-10\\
37.4243341568856	-6.04834236524734e-10\\
37.4694781064114	-8.74404814054978e-10\\
37.5146220559372	-9.03033325227395e-10\\
37.559766005463	-2.97664970833726e-10\\
37.6049099549888	1.46475913262507e-09\\
37.6500539045146	4.84499262603228e-09\\
37.6951978540404	9.79939549659537e-09\\
37.7403418035662	1.50348864348171e-08\\
37.785485753092	1.70532758775716e-08\\
37.8306297026178	9.37783791824444e-09\\
37.8757736521436	-1.71760478155372e-08\\
37.9209176016694	-7.17793859558608e-08\\
37.9660615511952	-1.56681508794e-07\\
38.011205500721	-2.5484720666398e-07\\
38.0563494502468	-3.13011362543857e-07\\
38.1014933997726	-2.25809826365458e-07\\
38.1466373492984	1.65436751488464e-07\\
38.1917812988242	1.03577692353838e-06\\
38.2369252483499	2.47041649261086e-06\\
38.2820691978757	4.26218849822164e-06\\
38.3272131474015	5.61626775865534e-06\\
38.3723570969273	4.84084199528272e-06\\
38.4175010464531	-7.65622873213616e-07\\
38.4626449959789	-1.44427012973106e-05\\
38.5077889455047	-3.83649108928084e-05\\
38.5529328950305	-7.03663419458113e-05\\
38.5980768445563	-9.88739541040519e-05\\
38.6432207940821	-9.70920397742406e-05\\
38.6883647436079	-1.96500878208532e-05\\
38.7335086931337	0.000191967788581497\\
38.7786526426595	0.00058573497957779\\
38.8237965921853	0.00114677434276075\\
38.8689405417111	0.00171173662222844\\
38.9140844912369	0.00186283567218968\\
38.9592284407627	0.000848324616896976\\
39.0043723902885	-0.00236409452466992\\
39.0495163398143	-0.00867878679814433\\
39.0946602893401	-0.017854448437536\\
39.1398042388659	-0.027359744200663\\
39.1849481883917	-0.032190656636186\\
39.2300921379175	-0.0227837701991322\\
39.2752360874433	0.0158192935350042\\
39.3203800369691	0.061638840259642\\
39.3655239864949	0.0841981617414908\\
39.4106679360207	0.0917790109908201\\
39.4558118855465	0.0949006002462996\\
39.5009558350723	0.096402009809799\\
39.5460997845981	0.0971683277466389\\
39.5912437341239	0.0975120233454787\\
39.6363876836497	0.0974841961810453\\
39.6815316331754	0.0966691958570934\\
39.7266755827012	0.020668401887504\\
39.771819532227	-0.0762228073761658\\
39.8169634817528	-0.0984561182110329\\
39.8621074312786	-0.0992996971563545\\
39.9072513808044	-0.099593462648042\\
39.9523953303302	-0.0997352141638841\\
39.997539279856	-0.0998148620939642\\
40.0426832293818	-0.0998639484862272\\
40.0878271789076	-0.0998961915225898\\
40.1329711284334	-0.0999183986667687\\
40.1781150779592	-0.0999342696368951\\
40.223259027485	-0.0999459553212824\\
40.2684029770108	-0.0999547734892341\\
40.3135469265366	-0.0999615663279107\\
40.3586908760624	-0.0999668912074119\\
40.4038348255882	-0.0999711280145467\\
40.448978775114	-0.0999745422555498\\
40.4941227246398	-0.09997732355484\\
40.5392666741656	-0.0999796098993568\\
40.5844106236914	-0.0999815033300161\\
40.6295545732172	-0.099983080341249\\
40.674698522743	-0.099984398915112\\
40.7198424722688	-0.0999855033603272\\
40.7649864217946	-0.0999864276841824\\
40.8101303713204	-0.0999871979583229\\
40.8552743208462	-0.0999878339731839\\
40.900418270372	-0.0999883503679175\\
40.9455622198978	-0.0999887573480087\\
40.9907061694236	-0.0999890610448708\\
41.0358501189494	-0.0999892635173151\\
41.0809940684752	-0.0999893623291809\\
41.126138018001	-0.0999893495375342\\
41.1712819675267	-0.0999892097463174\\
41.2164259170525	-0.0999889165189687\\
41.2615698665783	-0.0999884256459014\\
41.3067138161041	-0.0999876618283936\\
41.3518577656299	-0.0999864900960899\\
41.3970017151557	-0.0999846468556474\\
41.4421456646815	-0.0999815426796485\\
41.4872896142073	-0.0999755223423666\\
41.5324335637331	-0.099959125496094\\
41.5775775132589	-0.0899164280873678\\
41.6227214627847	-0.00828414545595664\\
41.6678654123105	0.00127122294633482\\
41.7130093618363	-0.00133122780088222\\
41.7581533113621	0.00075012361747116\\
41.8032972608879	-0.000806900607064193\\
41.8484412104137	0.000512447518736342\\
41.8935851599395	-0.000565579734184019\\
41.9387291094653	0.00037441904203846\\
41.9838730589911	-0.000423843364886012\\
42.0290170085169	0.000284916920408523\\
42.0741609580427	-0.000330749824489129\\
42.1193049075685	0.000223002741329745\\
42.1644488570943	-0.0002654513748374\\
42.2095928066201	0.000178274026320352\\
42.2547367561459	-0.000217585349710672\\
42.2998807056717	0.000144921256516969\\
42.3450246551975	-0.00018135281148223\\
42.3901686047233	0.000119434915431818\\
42.4353125542491	-0.000153238513068096\\
42.4804565037749	9.9571842572066e-05\\
42.5256004533006	-0.000130985005398038\\
42.5707444028264	8.38358125605843e-05\\
42.6158883523522	-0.000113077823239979\\
42.661032301878	7.11947791755513e-05\\
42.7061762514038	-9.84654119066064e-05\\
42.7513202009296	6.09172317932533e-05\\
42.7964641504554	-8.6396994758417e-05\\
42.8416080999812	5.24727995508623e-05\\
42.886752049507	-7.63239809475735e-05\\
42.9318959990328	4.54694724328242e-05\\
42.9770399485586	-6.78375682219318e-05\\
43.0221838980844	3.96126475509628e-05\\
43.0673278476102	-6.06279241742558e-05\\
43.112471797136	3.46776759189232e-05\\
43.1576157466618	-5.44567281336922e-05\\
43.2027596961876	3.04910255439023e-05\\
43.2479036457134	-4.91382541745244e-05\\
43.2930475952392	2.6917092539408e-05\\
43.338191544765	-4.45260657233409e-05\\
43.3833354942908	2.3848801323792e-05\\
43.4284794438166	-4.05034855417125e-05\\
43.4736233933424	2.1200798819565e-05\\
43.5187673428682	-3.69766590231718e-05\\
43.563911292394	1.89044567483532e-05\\
43.6090552419198	-3.38694319048066e-05\\
43.6541991914456	0.0720187622750253\\
43.6993431409714	0.0999958620520616\\
43.7444870904972	0.0999984148188906\\
43.789631040023	0.0999990026757921\\
43.8347749895488	0.0999992747251104\\
43.8799189390746	0.0999994329565569\\
43.9250628886004	0.0999995367076965\\
43.9702068381262	0.0999996100326979\\
44.0153507876519	0.0999996646010087\\
44.0604947371777	0.0999997067771539\\
44.1056386867035	0.0999997403335577\\
44.1507826362293	0.0999997676504584\\
44.1959265857551	0.0999997903048241\\
44.2410705352809	0.0999998093832732\\
44.2862144848067	0.0999998256591261\\
44.3313584343325	0.099999839697834\\
44.3765023838583	0.0999998519224819\\
44.4216463333841	0.0999998626559701\\
44.4667902829099	0.0999998721490273\\
44.5119342324357	0.099999880599315\\
44.5570781819615	0.0999998881647658\\
44.6022221314873	0.0999998949730883\\
44.6473660810131	0.0999999011286663\\
44.6925100305389	0.0999999067176508\\
44.7376539800647	0.0999999118117719\\
44.7827979295905	0.0999999164712336\\
44.8279418791163	0.0999999207469372\\
44.8730858286421	0.0999999246822094\\
44.9182297781679	0.0999999283141585\\
44.9633737276937	0.09999993167475\\
45.0085176772195	0.0999999347916653\\
45.0536616267453	0.0999999376889948\\
45.0988055762711	0.0999999403878002\\
45.1439495257969	0.0999999416900312\\
};
\addlegendentry{PINS}

\end{axis}
\end{tikzpicture}%%
  \caption{Jerk analysis PINS vs Duboids}
  \label{fig:Compare_jerk1}
\end{figure}
%
\subsection*{Around same orientation}
%
In figure \ref{fig:Compare_traj2} we can see the trajectory obtained with the two approaches is almost identical. However, in this case, Duboids achieved a better solution than PINS. Furthermore, by looking at figure \ref{fig:Compare_curv2} and \ref{fig:Compare_jerk2} we can see that the Duboids solution is smoother than the PINS one. As for the previous example Duboids is not affected by the ringing or Fuller's phenomena. Hence, also in this case Duboids is a better choice for control purposes.
%
\begin{figure}[htb!]
  \centering
  
\begin{tikzpicture}[scale = 0.7]

\begin{axis}[%
width=\linewidth,
height=0.736\linewidth,
at={(0\linewidth,0\linewidth)},
scale only axis,
xmin=-3.8997965457983,
xmax=2.8997965457983,
xlabel style={font=\color{white!15!black}},
xlabel={$x(m)$},
ymin=-0.0799252268713069,
ymax=5.10123911176031,
ylabel style={font=\color{white!15!black}},
ylabel={$y(m)$},
axis background/.style={fill=white},
title style={font=\bfseries},
title={$L_{PINS}$ = 18.3089 $L_{DUB}$ = 18.308, $k_{max}$ = 0.4, $J_{max}$ = 0.5},
axis x line*=bottom,
axis y line*=left,
xmajorgrids,
xminorgrids,
ymajorgrids,
yminorgrids,
legend style={at={(0.03,0.97)}, anchor=north west, legend cell align=left, align=left, draw=white!15!black}
]
\addplot [color=green, dashdotted, line width=2.0pt]
  table[row sep=crcr]{%
0	0\\
0.183078342331855	0.000511371639625034\\
0.366118121080859	0.00409066535454819\\
0.548926583513928	0.0138014953501555\\
0.731003230063397	0.0326859516788782\\
0.911413921530549	0.0635960687943143\\
1.08908495251861	0.107598091561118\\
1.26306030987848	0.164481729730947\\
1.43240739736366	0.233942058122162\\
1.59621842860407	0.315606734108076\\
1.75361529330185	0.409037993562485\\
1.90375426434845	0.513734997497648\\
2.04583052063025	0.629136516815542\\
2.1790824612784	0.754623940780781\\
2.30279578823614	0.889524593088236\\
2.41630733525905	1.03311533774957\\
2.51900862282283	1.18462645546927\\
2.61034911988247	1.34324576973089\\
2.68983919499818	1.50812300047555\\
2.75705274100853	1.67837432203472\\
2.81162945918113	1.85308710088458\\
2.85327679059679	2.03132478782504\\
2.88177148441388	2.21213193835897\\
2.89696079460616	2.39453933435983\\
2.89876329875907	2.57756917957286\\
2.88716933453518	2.76024034109954\\
2.86224105146923	2.94157360876798\\
2.82411207781508	3.12059694419667\\
2.77298680423042	3.29635069141368\\
2.70913928813908	3.46789272109998\\
2.63291178464405	3.63430348088085\\
2.54471291186643	3.79469092459353\\
2.44501546054478	3.94819529410755\\
2.33435385963672	4.09399372806483\\
2.21332131150842	4.23130467283436\\
2.08256661206866	4.35939207203642\\
1.94279067289344	4.47756931217858\\
1.794742763984	4.58520290325251\\
1.63921649729921	4.68171587456218\\
1.47704557259254	4.76659086757964\\
1.3090993083581	4.83937290924945\\
1.13627798184213	4.89967185087525\\
0.959508003099883	4.94716445951485\\
0.779737029403942	4.9815966671735\\
0.597998319582188	5.00346070028529\\
0.415321083206872	5.01534135261698\\
0.232314523577601	5.02026896764499\\
0.0492391235307252	5.02130390355428\\
-0.133840502511438	5.02131385843297\\
-0.316920130362916	5.02131386504698\\
-0.499999758214393	5.02131387166099\\
-0.683079386065869	5.021313878275\\
-0.866159013917348	5.02131388488901\\
-1.04923863996024	5.02130394327558\\
-1.23231404008748	5.02026902138596\\
-1.41532060012436	5.01534142140182\\
-1.59799783754713	5.00346078511443\\
-1.77973654941759	4.98159676898653\\
-1.95950752644698	4.94716457873091\\
-2.13627750978819	4.8996719872089\\
-2.30909884214322	4.83937306231831\\
-2.47704511342554	4.76659103691164\\
-2.63921604635111	4.68171605959806\\
-2.79474232238177	4.58520310334884\\
-2.94279024171395	4.47756952661118\\
-3.0825661923329	4.35939230000428\\
-3.21332090417605	4.23130491346389\\
-3.33435346560091	4.0939939804146\\
-3.44501508062741	3.94819555717328\\
-3.54471254681371	3.7946911973135\\
-3.63291143512251	3.63430376214159\\
-3.70913895473198	3.46789300974224\\
-3.77298648743466	3.29635098623864\\
-3.8241117780385	3.12059724397235\\
-3.86224076902844	2.9415739122359\\
-3.88716906965386	2.7602406469814\\
-3.89876305156679	2.57756948657744\\
-3.89696056513764	2.39453964118985\\
-3.88177127260884	2.21213224371813\\
-3.85327659630028	2.03132509042491\\
-3.81162928214433	1.85308739945152\\
-3.75705258089011	1.6783746153167\\
-3.68983905136611	1.50812328724888\\
-3.61034899221634	1.34324604880678\\
-3.51900851051666	1.18462672570016\\
-3.41630723762452	1.03311559803534\\
-3.30279570450627	0.889524842382081\\
-3.17908239061169	0.754624178094802\\
-3.04583046211516	0.629136741226062\\
-2.90375421700831	0.513735208150164\\
-2.75361525610008	0.409038189676239\\
-2.59621840044975	0.315606914980245\\
-2.43240737711737	0.233942223131632\\
-2.26306029635841	0.164481878341631\\
-2.08908494450688	0.107598223324834\\
-1.91141391777978	0.0635961833531912\\
-1.73100322930582	0.0326860487676197\\
-1.54892658458504	0.0138015748074383\\
-1.36611812309232	0.00409072710950661\\
-1.18307834468992	0.000511415669859639\\
-1.00000000240758	2.63017764508866e-08\\
};
\addlegendentry{Duboids}

\addplot [color=blue, line width=2.0pt, only marks, mark size=2.5pt, mark=*, mark options={solid, fill=blue, blue}, forget plot]
  table[row sep=crcr]{%
0	0\\
};
\addplot [color=blue, line width=2.0pt, only marks, mark size=2.5pt, mark=*, mark options={solid, fill=blue, blue}, forget plot]
  table[row sep=crcr]{%
-1	0\\
};
\addplot [color=dodgerblue, dotted, line width=2.0pt]
  table[row sep=crcr]{%
0	0\\
0.0183088928047699	7.67178182485602e-07\\
0.0366177852237847	4.6030690672929e-06\\
0.0549266753282691	1.45763849451829e-05\\
0.0732355581034078	3.37558359919779e-05\\
0.091544423905329	6.52101259796034e-05\\
0.109853256918095	0.000112007944107793\\
0.128162033610709	0.000177217951663365\\
0.146470721194162	0.000263908762213996\\
0.164779276078525	0.000375148914043488\\
0.183087642330141	0.000514006833535589\\
0.201395750128943	0.000683550788213572\\
0.219703514225962	0.000886848828142996\\
0.238010832401089	0.00112696871440531\\
0.256317583921181	0.00140697783335043\\
0.274623627998608	0.00172994309533678\\
0.292928802250349	0.00209893081666821\\
0.311232921157807	0.00251700658343776\\
0.329535774527462	0.0029872350959895\\
0.347837125952579	0.00351267999271088\\
0.366136711276155	0.00409640365186966\\
0.384434237055354	0.00474146697021127\\
0.402729379027674	0.0054509291170348\\
0.421021780579149	0.00622784726246839\\
0.439311051214897	0.00707527627866779\\
0.457596765032362	0.00799626841266557\\
0.475878459197644	0.00899387292960233\\
0.494155632425323	0.010071135725076\\
0.512427743462243	0.0112310989053508\\
0.530694209575735	0.0124768003341724\\
0.548954405046825	0.0138112731449448\\
0.567207659668988	0.0152375452170286\\
0.585453257253062	0.0167586386149316\\
0.603690434138986	0.0183775689891707\\
0.621918377715055	0.020097344937594\\
0.640136224945443	0.0219209673259634\\
0.658343060906792	0.0238514285666109\\
0.676537917334704	0.0258917118539944\\
0.694719771181035	0.0280447903559923\\
0.712887543182934	0.0303136263597931\\
0.731040096444624	0.032701170371248\\
0.749176235032983	0.0352103601665689\\
0.767294702588035	0.037844119795247\\
0.785394180949532	0.0406053585329895\\
0.803473288801067	0.0434969697823957\\
0.821530785254224	0.0465206063626313\\
0.839565655564497	0.0496763961554924\\
0.857576932586538	0.0529641699251492\\
0.875563650440273	0.0563837513579016\\
0.893524844562725	0.0599349570715595\\
0.911459551759721	0.0636175966253486\\
0.929366810257565	0.067431472530058\\
0.947245659754601	0.0713763802586936\\
0.965095141472728	0.0754521082573898\\
0.982914298208806	0.0796584379568085\\
1.000702174386	0.083995143783813\\
1.01845781610501	0.0884619931736094\\
1.03618027119526	0.0930587465821808\\
1.05386858926592	0.0977851574991682\\
1.07152182175691	0.102640972461062\\
1.08913902198975	0.107625931064822\\
1.10671924521834	0.11273976598182\\
1.1242615486796	0.117982202972193\\
1.14176499164408	0.123352960899546\\
1.15922863546637	0.128851751746028\\
1.17665154363544	0.134478280627782\\
1.19403278182489	0.140232245810755\\
1.21137141794302	0.146113338726891\\
1.22866652218285	0.152121243990666\\
1.24591716707197	0.158255639416017\\
1.26312242752228	0.164516196033605\\
1.28028138087961	0.17090257810848\\
1.29739310697317	0.17741444315806\\
1.31445668816495	0.18405144197053\\
1.33147120939888	0.190813218623535\\
1.34843575824994	0.197699410503311\\
1.36534942497308	0.204709648324083\\
1.38221130255199	0.211843556147927\\
1.39902048674777	0.219100751404869\\
1.41577607614741	0.226480844913473\\
1.43247717221214	0.233983440901637\\
1.44912287932561	0.2416081370279\\
1.46571230484191	0.249354524402937\\
1.48224455913347	0.257222187611572\\
1.49871875563874	0.265210704734973\\
1.51513401090974	0.273319647373369\\
1.53148944465949	0.28154858066893\\
1.54778417980911	0.289897063329191\\
1.56401734253499	0.298364647650614\\
1.58018806231552	0.306950879542707\\
1.59629547197788	0.315655298552267\\
1.61233870774448	0.324477437888189\\
1.62831690927933	0.333416824446382\\
1.64422921973412	0.342472978835266\\
1.66007478579426	0.351645415401356\\
1.67585275772453	0.360933642255437\\
1.69156228941474	0.370337161298813\\
1.70720253842507	0.379855468250158\\
1.72277266603127	0.389488052672419\\
1.73827183726956	0.399234398000338\\
1.75369922098156	0.409093981568008\\
1.7690539898587	0.419066274637053\\
1.7843353204867	0.429150742424836\\
1.79954239338966	0.439346844133294\\
1.81467439307407	0.44965403297778\\
1.82973050807248	0.460071756216558\\
1.84470993098709	0.470599455180272\\
1.85961185853293	0.481236565302087\\
1.8744354915811	0.491982516147788\\
1.88918003520148	0.502836731446554\\
1.90384469870546	0.513798629121682\\
1.91842869568827	0.524867621321989\\
1.93293124407125	0.536043114453153\\
1.94735156614363	0.547324509209739\\
1.96168888860444	0.558711200607145\\
1.9759424426038	0.57020257801425\\
1.99011146378429	0.581798025185957\\
2.00419519232182	0.593496920296452\\
2.0181928729665	0.605298635972342\\
2.03210375508301	0.617202539326519\\
2.04592709269102	0.62920799199188\\
2.05966214450498	0.641314350155792\\
2.07330817397412	0.653520964594388\\
2.0868644493217	0.665827180707621\\
2.10033024358451	0.678232338554138\\
2.11370483465157	0.690735772886904\\
2.12698750530317	0.703336813188651\\
2.14017754324903	0.716034783708072\\
2.15327424116674	0.728829003495823\\
2.1662768967395	0.741718786441289\\
2.17918481269399	0.754703441309134\\
2.1919972968375	0.767782271776622\\
2.20471366209536	0.780954576470708\\
2.21733322654747	0.794219649005914\\
2.22985531346515	0.807576778021951\\
2.24227925134719	0.821025247222134\\
2.25460437395611	0.834564335411532\\
2.26683002035359	0.848193316535919\\
2.27895553493625	0.861911459720436\\
2.29098026747046	0.875718029309068\\
2.30290357312757	0.889612284903822\\
2.31472481251811	0.903593481404712\\
2.32644335172648	0.91766086904944\\
2.33805856234454	0.931813693453893\\
2.34956982150569	0.946051195652312\\
2.36097651191791	0.960372612138288\\
2.37227802189724	0.974777174905417\\
2.38347374540019	0.989264111488788\\
2.39456308205663	1.00383264500611\\
2.40554543720159	1.01848199419968\\
2.41642022190757	1.03321137347797\\
2.42718685301567	1.04801999295809\\
2.43784475316735	1.06290705850783\\
2.44839335083488	1.07787177178853\\
2.45883208035248	1.09291333029765\\
2.46916038194623	1.10803092741207\\
2.47937770176446	1.12322375243104\\
2.48948349190708	1.13849099062\\
2.49947721045535	1.15383182325396\\
2.50935832150053	1.16924542766169\\
2.51912629517307	1.18473097726955\\
2.52878060767054	1.20028764164615\\
2.53832074128622	1.21591458654656\\
2.54774618443633	1.23161097395733\\
2.55705643168803	1.24737596214119\\
2.56625098378596	1.26320870568246\\
2.57532934767955	1.27910835553209\\
2.58429103654893	1.29507405905349\\
2.59313556983159	1.31110496006801\\
2.60186247324757	1.32720019890107\\
2.61047127882547	1.34335891242805\\
2.61896152492698	1.35958023412084\\
2.62733275627221	1.375863294094\\
2.63558452396351	1.39220721915177\\
2.64371638551014	1.40861113283453\\
2.6517279048514	1.42507415546617\\
2.65961865238063	1.44159540420091\\
2.66738820496759	1.458173993071\\
2.67503614598185	1.47480903303391\\
2.6825620653144	1.49149963202031\\
2.68996555940039	1.50824489498164\\
2.69724623124001	1.52504392393836\\
2.70440369042055	1.54189581802787\\
2.71143755313655	1.55879967355309\\
2.71834744221119	1.57575458403063\\
2.72513298711567	1.59275964023972\\
2.73179382398995	1.60981393027066\\
2.73832959566139	1.62691653957404\\
2.74473995166475	1.64406655100947\\
2.75102454826014	1.66126304489509\\
2.75718304845234	1.67850509905658\\
2.76321512200792	1.69579178887693\\
2.76912044547392	1.71312218734572\\
2.77489870219423	1.73049536510914\\
2.78054958232753	1.74791039051952\\
2.78607278286293	1.76536632968563\\
2.7914680076372	1.78286224652243\\
2.79673496734967	1.8003972028016\\
2.80187337957873	1.81797025820154\\
2.80688296879595	1.83558047035812\\
2.81176346638192	1.85322689491491\\
2.81651461063956	1.87090858557413\\
2.82113614680924	1.8886245941471\\
2.82562782708136	1.90637397060539\\
2.82998941061073	1.92415576313149\\
2.83422066352838	1.94196901817014\\
2.83832135895518	1.95981278047919\\
2.84229127701295	1.97768609318108\\
2.8461302048373	1.99558799781396\\
2.84983793658796	2.01351753438327\\
2.85341427346095	2.03147374141304\\
2.85685902369806	2.04945565599767\\
2.86017200259833	2.06746231385335\\
2.86335303252674	2.08549274936999\\
2.86640194292496	2.1035459956628\\
2.86931857031925	2.12162108462435\\
2.87210275833048	2.1397170469763\\
2.87475435768123	2.15783291232158\\
2.87727322620512	2.17596770919622\\
2.87965922885305	2.19412046512167\\
2.88191223770187	2.21229020665677\\
2.88403213195977	2.23047595945011\\
2.88601879797424	2.24867674829214\\
2.88787212923664	2.26689159716763\\
2.88959202638948	2.28511952930786\\
2.89117839723014	2.30335956724315\\
2.89263115671741	2.32161073285517\\
2.89395022697448	2.33987204742948\\
2.89513553729466	2.35814253170796\\
2.89618702414359	2.37642120594138\\
2.89710463116428	2.39470708994191\\
2.89788830917844	2.41299920313572\\
2.89853801619083	2.43129656461554\\
2.89905371738978	2.4495981931933\\
2.89943538515079	2.46790310745274\\
2.89968299903624	2.48621032580204\\
2.8997965457983	2.50451886652646\\
2.89977601937776	2.52282774784104\\
2.89962142090627	2.54113598794318\\
2.89933275870433	2.55944260506537\\
2.8989100482828	2.57774661752779\\
2.89835331234008	2.596047043791\\
2.8976625807629	2.61434290250858\\
2.89683789062267	2.6326332125797\\
2.89587928617557	2.65091699320185\\
2.89478681885804	2.66919326392334\\
2.89356054728619	2.68746104469598\\
2.89220053725043	2.70571935592752\\
2.89070686171422	2.72396721853435\\
2.88907960080782	2.74220365399382\\
2.88731884182633	2.76042768439695\\
2.88542467922266	2.77863833250063\\
2.88339721460484	2.79683462178028\\
2.88123655672812	2.81501557648197\\
2.87894282149162	2.83318022167503\\
2.87651613192959	2.85132758330406\\
2.8739566182074	2.86945668824145\\
2.87126441761184	2.88756656433927\\
2.86843967454657	2.90565624048177\\
2.8654825405215	2.92372474663703\\
2.86239317414755	2.9417711139095\\
2.85917174112522	2.9597943745914\\
2.85581841423864	2.9777935622152\\
2.85233337334335	2.9957677116049\\
2.84871680535961	3.01371585892835\\
2.84496890425935	3.03163704174835\\
2.84108987105888	3.04953029907491\\
2.83707991380488	3.06739467141615\\
2.83293924756653	3.08522920083039\\
2.82866809442063	3.10303293097689\\
2.82426668344308	3.12080490716785\\
2.81973525069313	3.1385441764188\\
2.81507403920422	3.15624978750062\\
2.81028329896738	3.17392079098962\\
2.80536328692145	3.19155623931942\\
2.80031426693559	3.20915518683078\\
2.79513650979887	3.22671668982336\\
2.78983029320197	3.24423980660525\\
2.78439590172606	3.26172359754462\\
2.77883362682373	3.27916712511894\\
2.77314376680723	3.29656945396648\\
2.76732662682846	3.31392965093519\\
2.76138251886672	3.33124678513413\\
2.75531176170776	3.34851992798198\\
2.74911468093098	3.36574815325829\\
2.74279160888756	3.38293053715165\\
2.73634288468711	3.40006615831085\\
2.72976885417493	3.41715409789261\\
2.72306986991805	3.43419343961262\\
2.71624629118167	3.45118326979286\\
2.70929848391461	3.46812267741252\\
2.70222682072485	3.48501075415488\\
2.69503168086446	3.50184659445807\\
2.68771345020422	3.51862929556155\\
2.68027252121808	3.53535795755671\\
2.67270929295684	3.55203168343287\\
2.66502417103206	3.56864957912776\\
2.65721756758892	3.58521075357303\\
2.64928990128958	3.60171431874458\\
2.64124159728513	3.61815938970755\\
2.6330730871985	3.63454508466656\\
2.62478480909544	3.65087052501013\\
2.616377207467	3.66713483536079\\
2.60785073319963	3.68333714361896\\
2.5992058435571	3.69947658101292\\
2.59044300214975	3.71555228214211\\
2.58156267891599	3.73156338502695\\
2.57256535009056	3.74750903115159\\
2.56345149818564	3.76338836551359\\
2.55422161195817	3.779200536666\\
2.54487618639057	3.79494469676699\\
2.53541572265707	3.8106200016212\\
2.52584072810412	3.82622561072933\\
2.51615171621569	3.84176068732879\\
2.50634920659336	3.85722439844318\\
2.49643372492068	3.8726159149222\\
2.48640580294285	3.88793441149111\\
2.47626597843014	3.90317906678982\\
2.46601479515729	3.91834906342237\\
2.45565280286589	3.93344358799515\\
2.44518055724351	3.94846183116636\\
2.43459861988511	3.96340298768344\\
2.42390755827181	3.97826625643246\\
2.41310794573138	3.9930508404747\\
2.40220036141678	4.00775594709604\\
2.39118539026555	4.02238078784261\\
2.38006362297814	4.03692457857026\\
2.36883565597634	4.0513865394792\\
2.35750209138137	4.06576589516354\\
2.34606353697123	4.0800618746449\\
2.3345206061587	4.09427371142212\\
2.32287391794757	4.10840064350371\\
2.31112409691052	4.12244191345775\\
2.29927177314427	4.13639676844321\\
2.28731758224739	4.15026446026002\\
2.27526216527429	4.16404424537916\\
2.26310616871301	4.17773538499306\\
2.25085024443808	4.19133714504433\\
2.23849504968827	4.20484879627649\\
2.22604124701831	4.21826961426135\\
2.21348950427668	4.23159887945014\\
2.20084049455607	4.24483587719936\\
2.18809489617134	4.25797989782246\\
2.17525339260871	4.27103023661408\\
2.16231667250383	4.28398619390225\\
2.14928542958972	4.29684707507108\\
2.13616036267507	4.30961219061348\\
2.12294217559077	4.32228085615212\\
2.10963157716858	4.33485239249292\\
2.09622928118622	4.34732612564408\\
2.08273600634634	4.35970138687044\\
2.06915247622028	4.37197751271044\\
2.05547941922738	4.38415384503154\\
2.04171756857725	4.39622973104494\\
2.02786766224959	4.40820452336215\\
2.01393044293488	4.42007758000729\\
1.9999066580148	4.43184826447509\\
1.98579705950128	4.44351594574055\\
1.97160240401756	4.45507999831842\\
1.95732345273556	4.46653980227012\\
1.94296097135772	4.47789474326496\\
1.92851573005254	4.48914421258401\\
1.91398850343731	4.50028760718327\\
1.89938007051174	4.51132432969424\\
1.88469121464172	4.52225378848939\\
1.86992272349093	4.53307539767907\\
1.85507538900573	4.54378857717957\\
1.84015000734465	4.55439275270603\\
1.82514737886452	4.56488735584347\\
1.81006830804774	4.57527182403532\\
1.79491360348987	4.58554560065781\\
1.77968407782442	4.59570813500366\\
1.76438054771211	4.60575888236029\\
1.74900383376305	4.61569730398819\\
1.73355476052782	4.62552286720348\\
1.7180341564169	4.63523504535037\\
1.70244285369375	4.64483331788871\\
1.68678168839125	4.65431717035978\\
1.67105150030714	4.66368609447964\\
1.65525313291712	4.67293958809746\\
1.63938743337293	4.68207715529549\\
1.62345525241187	4.69109830633896\\
1.60745744435786	4.70000255778376\\
1.59139486702702	4.70878943241676\\
1.57526838173208	4.71745845937237\\
1.55907885318363	4.72600917406204\\
1.54282714949833	4.73444111830109\\
1.52651414209543	4.74275384022581\\
1.51014070570916	4.75094689443227\\
1.4937077182801	4.75901984187924\\
1.47721606097236	4.76697225004087\\
1.46066661805906	4.77480369279331\\
1.44406027694507	4.78251375058367\\
1.4273979280459	4.79010201029768\\
1.41068046481671	4.7975680654479\\
1.3939087836239	4.80491151601911\\
1.37708378378122	4.81213196867937\\
1.36020636741304	4.8192290365993\\
1.34327743949876	4.82620233969024\\
1.32629790772654	4.8330515043927\\
1.30926868254731	4.83977616394691\\
1.29219067701756	4.84637595814454\\
1.27506480686449	4.85285053363837\\
1.25789199031622	4.85919954364973\\
1.2406731481799	4.86542264832583\\
1.22340920365731	4.87151951439343\\
1.20610108243833	4.87748981557461\\
1.1887497124992	4.88333323217455\\
1.17135602421432	4.88904945156972\\
1.15392095013404	4.89463816771391\\
1.13644542511839	4.90009908171731\\
1.11893038609018	4.90543190124971\\
1.10137677219554	4.91063634123526\\
1.08378552452689	4.91571212312433\\
1.06615758631662	4.92065897573786\\
1.04849390262258	4.92547663436856\\
1.03079542056355	4.93016484182252\\
1.01306308895728	4.93472334729365\\
0.995297858609743	4.93915190767193\\
0.977500681891464	4.94345028610844\\
0.959672513098229	4.9476182536953\\
0.941814307944349	4.95165558759368\\
0.923927024022149	4.95556207325367\\
0.906011620178297	4.95933750189646\\
0.88806905711774	4.96298167356408\\
0.870100296604461	4.96649439358445\\
0.852106302295373	4.96987547700987\\
0.834088038650386	4.97312474331857\\
0.816046472188447	4.97624202353128\\
0.797982569829728	4.97922715131745\\
0.779897301193722	4.98207997705497\\
0.761791635375281	4.98480034864651\\
0.743666613552837	4.98738861838922\\
0.725523416232016	4.9898462488136\\
0.70736331817158	4.99217571114652\\
0.689187593884732	4.99437993504023\\
0.670997455694889	4.99646186418397\\
0.652794055104613	4.99842445514365\\
0.63457848419369	5.00027067634228\\
0.616351777034957	5.00200350711154\\
0.598114911124636	5.00362593680003\\
0.579868808825409	5.00514096393222\\
0.561614338820948	5.0065515954154\\
0.543352317580799	5.00786084579232\\
0.525083510834687	5.00907173653815\\
0.506808635055334	5.01018729540004\\
0.488528358949	5.0112105557781\\
0.470243304952988	5.01214455614642\\
0.451954050739414	5.01299233951285\\
0.433661130724595	5.01375695291617\\
0.415365037583442	5.01444144695952\\
0.397066223768316	5.01504887537874\\
0.378765103031808	5.01558229464431\\
0.36046205195298	5.01604476359562\\
0.342157411466622	5.01643934310635\\
0.323851488395124	5.01676909577962\\
0.305544556982598	5.01703708567156\\
0.287236860430909	5.0172463780421\\
0.268928612437322	5.01740003913164\\
0.250619998733495	5.01750113596221\\
0.232311178625565	5.0175527361619\\
0.214002286535131	5.01755790781123\\
0.195693433540924	5.01751971930994\\
0.177384708921023	5.01744123926318\\
0.15907618169546	5.01732553638532\\
0.140767902169105	5.01717567942031\\
0.122459903474722	5.01699473707701\\
0.104152203116128	5.01678577797804\\
0.0858448045113761	5.01655187062066\\
0.0675376985359299	5.01629608334798\\
0.0492308650657771	5.01602148432903\\
0.0309242745204798	5.0157311415455\\
0.0126178894061291	5.01542812278338\\
-0.00568833414178451	5.01511549562696\\
-0.0239944448155839	5.01479632745229\\
-0.0423004945926899	5.01447368541653\\
-0.0606065371929575	5.01415063643835\\
-0.0789126265360417	5.01383024716242\\
-0.0972188151987585	5.01351558389774\\
-0.115525152872396	5.01320971251323\\
-0.1338316848199	5.0129156982616\\
-0.152138450332759	5.01263660547723\\
-0.170445481187211	5.01237549703371\\
-0.18875280009853	5.01213543328305\\
-0.207060419168824	5.01191946962384\\
-0.225368338300139	5.01173064852767\\
-0.243676539250817	5.01157148108531\\
-0.261984977426946	5.01144245443696\\
-0.280293592238421	5.01134155840057\\
-0.298602328232194	5.01126577822214\\
-0.316911142328875	5.0112120874569\\
-0.335220002399749	5.01117745755222\\
-0.353528885777202	5.01115886196281\\
-0.371837777747004	5.01115327955906\\
-0.390146670034522	5.01115769874159\\
-0.408455559290816	5.01116912353395\\
-0.42676444558428	5.01118458374493\\
-0.4450733309072	5.01120115351493\\
-0.463382217717913	5.01121598852852\\
-0.481691107571264	5.01122641268158\\
-0.499999999999999	5.01123020189079\\
-0.518308892428735	5.01122641268158\\
-0.536617782282086	5.01121598852852\\
-0.554926669092799	5.01120115351493\\
-0.573235554415719	5.01118458374493\\
-0.591544440709183	5.01116912353395\\
-0.609853329965477	5.01115769874159\\
-0.628162222252995	5.01115327955906\\
-0.646471114222797	5.01115886196281\\
-0.66477999760025	5.01117745755222\\
-0.683088857671123	5.0112120874569\\
-0.701397671767805	5.01126577822214\\
-0.719706407761578	5.01134155840057\\
-0.738015022573053	5.01144245443696\\
-0.756323460749182	5.01157148108531\\
-0.77463166169986	5.01173064852767\\
-0.792939580831174	5.01191946962384\\
-0.811247199901469	5.01213543328305\\
-0.829554518812788	5.01237549703371\\
-0.847861549667239	5.01263660547723\\
-0.866168315180099	5.0129156982616\\
-0.884474847127602	5.01320971251323\\
-0.90278118480124	5.01351558389774\\
-0.921087373463957	5.01383024716242\\
-0.939393462807041	5.01415063643835\\
-0.957699505407309	5.01447368541653\\
-0.976005555184415	5.01479632745229\\
-0.994311665858214	5.01511549562696\\
-1.01261788940613	5.01542812278338\\
-1.03092427452048	5.0157311415455\\
-1.04923086506578	5.01602148432903\\
-1.06753769853593	5.01629608334798\\
-1.08584480451138	5.01655187062066\\
-1.10415220311613	5.01678577797804\\
-1.12245990347472	5.01699473707701\\
-1.14076790216911	5.01717567942031\\
-1.15907618169546	5.01732553638532\\
-1.17738470892102	5.01744123926318\\
-1.19569343354092	5.01751971930994\\
-1.21400228653513	5.01755790781123\\
-1.23231117862557	5.01755273616191\\
-1.25061999873349	5.01750113596221\\
-1.26892861243732	5.01740003913164\\
-1.28723686043091	5.0172463780421\\
-1.3055445569826	5.01703708567156\\
-1.32385148839512	5.01676909577962\\
-1.34215741146662	5.01643934310635\\
-1.36046205195298	5.01604476359562\\
-1.37876510303181	5.01558229464431\\
-1.39706622376832	5.01504887537874\\
-1.41536503758344	5.01444144695952\\
-1.43366113072459	5.01375695291617\\
-1.45195405073941	5.01299233951285\\
-1.47024330495299	5.01214455614642\\
-1.488528358949	5.0112105557781\\
-1.50680863505533	5.01018729540004\\
-1.52508351083469	5.00907173653815\\
-1.5433523175808	5.00786084579232\\
-1.56161433882095	5.0065515954154\\
-1.57986880882541	5.00514096393222\\
-1.59811491112464	5.00362593680003\\
-1.61635177703496	5.00200350711154\\
-1.63457848419369	5.00027067634228\\
-1.65279405510461	4.99842445514365\\
-1.67099745569489	4.99646186418397\\
-1.68918759388473	4.99437993504023\\
-1.70736331817158	4.99217571114652\\
-1.72552341623201	4.9898462488136\\
-1.74366661355284	4.98738861838922\\
-1.76179163537528	4.98480034864651\\
-1.77989730119372	4.98207997705497\\
-1.79798256982973	4.97922715131745\\
-1.81604647218845	4.97624202353128\\
-1.83408803865038	4.97312474331857\\
-1.85210630229537	4.96987547700987\\
-1.87010029660446	4.96649439358445\\
-1.88806905711774	4.96298167356408\\
-1.9060116201783	4.95933750189646\\
-1.92392702402215	4.95556207325367\\
-1.94181430794435	4.95165558759368\\
-1.95967251309823	4.9476182536953\\
-1.97750068189146	4.94345028610844\\
-1.99529785860974	4.93915190767193\\
-2.01306308895728	4.93472334729365\\
-2.03079542056355	4.93016484182252\\
-2.04849390262258	4.92547663436857\\
-2.06615758631662	4.92065897573787\\
-2.08378552452689	4.91571212312433\\
-2.10137677219554	4.91063634123526\\
-2.11893038609017	4.90543190124971\\
-2.13644542511839	4.90009908171731\\
-2.15392095013404	4.89463816771391\\
-2.17135602421432	4.88904945156972\\
-2.18874971249919	4.88333323217455\\
-2.20610108243833	4.87748981557461\\
-2.22340920365731	4.87151951439343\\
-2.2406731481799	4.86542264832583\\
-2.25789199031621	4.85919954364973\\
-2.27506480686449	4.85285053363837\\
-2.29219067701756	4.84637595814454\\
-2.30926868254731	4.83977616394691\\
-2.32629790772654	4.8330515043927\\
-2.34327743949876	4.82620233969024\\
-2.36020636741304	4.8192290365993\\
-2.37708378378122	4.81213196867937\\
-2.3939087836239	4.80491151601911\\
-2.41068046481671	4.7975680654479\\
-2.42739792804589	4.79010201029768\\
-2.44406027694507	4.78251375058367\\
-2.46066661805906	4.77480369279331\\
-2.47721606097236	4.76697225004087\\
-2.4937077182801	4.75901984187924\\
-2.51014070570916	4.75094689443227\\
-2.52651414209543	4.74275384022581\\
-2.54282714949833	4.73444111830109\\
-2.55907885318363	4.72600917406204\\
-2.57526838173208	4.71745845937237\\
-2.59139486702702	4.70878943241676\\
-2.60745744435786	4.70000255778376\\
-2.62345525241187	4.69109830633896\\
-2.63938743337293	4.68207715529549\\
-2.65525313291712	4.67293958809746\\
-2.67105150030714	4.66368609447964\\
-2.68678168839125	4.65431717035978\\
-2.70244285369375	4.64483331788871\\
-2.71803415641689	4.63523504535037\\
-2.73355476052782	4.62552286720348\\
-2.74900383376305	4.61569730398819\\
-2.76438054771211	4.60575888236029\\
-2.77968407782442	4.59570813500367\\
-2.79491360348987	4.58554560065781\\
-2.81006830804774	4.57527182403533\\
-2.82514737886452	4.56488735584347\\
-2.84015000734465	4.55439275270603\\
-2.85507538900573	4.54378857717957\\
-2.86992272349093	4.53307539767907\\
-2.88469121464171	4.52225378848939\\
-2.89938007051173	4.51132432969424\\
-2.91398850343731	4.50028760718327\\
-2.92851573005254	4.48914421258401\\
-2.94296097135772	4.47789474326496\\
-2.95732345273556	4.46653980227012\\
-2.97160240401756	4.45507999831842\\
-2.98579705950128	4.44351594574055\\
-2.9999066580148	4.4318482644751\\
-3.01393044293488	4.42007758000729\\
-3.02786766224959	4.40820452336215\\
-3.04171756857725	4.39622973104494\\
-3.05547941922738	4.38415384503154\\
-3.06915247622028	4.37197751271044\\
-3.08273600634634	4.35970138687044\\
-3.09622928118621	4.34732612564408\\
-3.10963157716858	4.33485239249292\\
-3.12294217559077	4.32228085615212\\
-3.13616036267507	4.30961219061348\\
-3.14928542958972	4.29684707507108\\
-3.16231667250383	4.28398619390226\\
-3.17525339260871	4.27103023661408\\
-3.18809489617134	4.25797989782246\\
-3.20084049455607	4.24483587719936\\
-3.21348950427668	4.23159887945014\\
-3.22604124701831	4.21826961426135\\
-3.23849504968827	4.20484879627649\\
-3.25085024443808	4.19133714504433\\
-3.26310616871301	4.17773538499306\\
-3.27526216527429	4.16404424537916\\
-3.28731758224739	4.15026446026002\\
-3.29927177314427	4.13639676844321\\
-3.31112409691052	4.12244191345775\\
-3.32287391794757	4.10840064350371\\
-3.3345206061587	4.09427371142212\\
-3.34606353697123	4.0800618746449\\
-3.35750209138137	4.06576589516354\\
-3.36883565597634	4.0513865394792\\
-3.38006362297814	4.03692457857026\\
-3.39118539026555	4.02238078784261\\
-3.40220036141678	4.00775594709604\\
-3.41310794573138	3.99305084047469\\
-3.42390755827181	3.97826625643246\\
-3.43459861988511	3.96340298768344\\
-3.44518055724351	3.94846183116636\\
-3.45565280286589	3.93344358799514\\
-3.46601479515728	3.91834906342237\\
-3.47626597843014	3.90317906678982\\
-3.48640580294285	3.8879344114911\\
-3.49643372492068	3.8726159149222\\
-3.50634920659336	3.85722439844318\\
-3.51615171621569	3.84176068732879\\
-3.52584072810411	3.82622561072934\\
-3.53541572265707	3.8106200016212\\
-3.54487618639057	3.79494469676699\\
-3.55422161195817	3.779200536666\\
-3.56345149818564	3.7633883655136\\
-3.57256535009056	3.74750903115159\\
-3.58156267891599	3.73156338502695\\
-3.59044300214975	3.71555228214211\\
-3.5992058435571	3.69947658101293\\
-3.60785073319962	3.68333714361896\\
-3.616377207467	3.66713483536079\\
-3.62478480909543	3.65087052501013\\
-3.63307308719849	3.63454508466656\\
-3.64124159728513	3.61815938970755\\
-3.64928990128958	3.60171431874458\\
-3.65721756758892	3.58521075357304\\
-3.66502417103206	3.56864957912777\\
-3.67270929295684	3.55203168343287\\
-3.68027252121808	3.53535795755671\\
-3.68771345020422	3.51862929556155\\
-3.69503168086446	3.50184659445807\\
-3.70222682072485	3.48501075415488\\
-3.70929848391461	3.46812267741252\\
-3.71624629118167	3.45118326979287\\
-3.72306986991805	3.43419343961262\\
-3.72976885417492	3.41715409789261\\
-3.73634288468711	3.40006615831085\\
-3.74279160888756	3.38293053715165\\
-3.74911468093098	3.36574815325829\\
-3.75531176170776	3.34851992798198\\
-3.76138251886672	3.33124678513413\\
-3.76732662682846	3.31392965093519\\
-3.77314376680723	3.29656945396648\\
-3.77883362682373	3.27916712511894\\
-3.78439590172606	3.26172359754462\\
-3.78983029320196	3.24423980660525\\
-3.79513650979887	3.22671668982336\\
-3.80031426693559	3.20915518683078\\
-3.80536328692145	3.19155623931942\\
-3.81028329896738	3.17392079098962\\
-3.81507403920421	3.15624978750062\\
-3.81973525069313	3.1385441764188\\
-3.82426668344308	3.12080490716785\\
-3.82866809442063	3.1030329309769\\
-3.83293924756653	3.08522920083039\\
-3.83707991380488	3.06739467141615\\
-3.84108987105888	3.04953029907491\\
-3.84496890425935	3.03163704174835\\
-3.84871680535961	3.01371585892835\\
-3.85233337334335	2.9957677116049\\
-3.85581841423864	2.9777935622152\\
-3.85917174112521	2.9597943745914\\
-3.86239317414755	2.9417711139095\\
-3.8654825405215	2.92372474663704\\
-3.86843967454657	2.90565624048177\\
-3.87126441761184	2.88756656433927\\
-3.8739566182074	2.86945668824145\\
-3.87651613192959	2.85132758330406\\
-3.87894282149162	2.83318022167503\\
-3.88123655672812	2.81501557648197\\
-3.88339721460484	2.79683462178028\\
-3.88542467922266	2.77863833250063\\
-3.88731884182633	2.76042768439696\\
-3.88907960080781	2.74220365399382\\
-3.89070686171422	2.72396721853435\\
-3.89220053725043	2.70571935592752\\
-3.89356054728619	2.68746104469598\\
-3.89478681885804	2.66919326392334\\
-3.89587928617557	2.65091699320185\\
-3.89683789062267	2.6326332125797\\
-3.8976625807629	2.61434290250858\\
-3.89835331234008	2.596047043791\\
-3.8989100482828	2.57774661752779\\
-3.89933275870433	2.55944260506537\\
-3.89962142090627	2.54113598794318\\
-3.89977601937776	2.52282774784104\\
-3.8997965457983	2.50451886652646\\
-3.89968299903624	2.48621032580204\\
-3.89943538515079	2.46790310745274\\
-3.89905371738978	2.4495981931933\\
-3.89853801619083	2.43129656461554\\
-3.89788830917844	2.41299920313572\\
-3.89710463116428	2.39470708994191\\
-3.89618702414359	2.37642120594138\\
-3.89513553729466	2.35814253170796\\
-3.89395022697448	2.33987204742948\\
-3.89263115671741	2.32161073285517\\
-3.89117839723013	2.30335956724315\\
-3.88959202638948	2.28511952930786\\
-3.88787212923664	2.26689159716763\\
-3.88601879797424	2.24867674829214\\
-3.88403213195977	2.23047595945011\\
-3.88191223770187	2.21229020665677\\
-3.87965922885305	2.19412046512167\\
-3.87727322620512	2.17596770919622\\
-3.87475435768123	2.15783291232158\\
-3.87210275833048	2.1397170469763\\
-3.86931857031925	2.12162108462435\\
-3.86640194292496	2.1035459956628\\
-3.86335303252674	2.08549274936999\\
-3.86017200259833	2.06746231385335\\
-3.85685902369806	2.04945565599767\\
-3.85341427346095	2.03147374141304\\
-3.84983793658796	2.01351753438327\\
-3.8461302048373	1.99558799781396\\
-3.84229127701295	1.97768609318108\\
-3.83832135895518	1.95981278047919\\
-3.83422066352838	1.94196901817014\\
-3.82998941061073	1.92415576313149\\
-3.82562782708136	1.90637397060539\\
-3.82113614680924	1.88862459414709\\
-3.81651461063956	1.87090858557413\\
-3.81176346638192	1.85322689491491\\
-3.80688296879595	1.83558047035812\\
-3.80187337957873	1.81797025820154\\
-3.79673496734967	1.8003972028016\\
-3.7914680076372	1.78286224652243\\
-3.78607278286293	1.76536632968563\\
-3.78054958232753	1.74791039051952\\
-3.77489870219423	1.73049536510914\\
-3.76912044547392	1.71312218734572\\
-3.76321512200792	1.69579178887693\\
-3.75718304845234	1.67850509905658\\
-3.75102454826014	1.66126304489509\\
-3.74473995166475	1.64406655100946\\
-3.73832959566139	1.62691653957404\\
-3.73179382398995	1.60981393027066\\
-3.72513298711567	1.59275964023972\\
-3.71834744221119	1.57575458403063\\
-3.71143755313655	1.55879967355309\\
-3.70440369042055	1.54189581802787\\
-3.69724623124001	1.52504392393836\\
-3.68996555940039	1.50824489498164\\
-3.6825620653144	1.49149963202032\\
-3.67503614598185	1.47480903303391\\
-3.66738820496759	1.458173993071\\
-3.65961865238063	1.44159540420091\\
-3.6517279048514	1.42507415546617\\
-3.64371638551014	1.40861113283453\\
-3.63558452396351	1.39220721915177\\
-3.62733275627221	1.375863294094\\
-3.61896152492698	1.35958023412084\\
-3.61047127882547	1.34335891242805\\
-3.60186247324757	1.32720019890107\\
-3.59313556983159	1.31110496006801\\
-3.58429103654893	1.2950740590535\\
-3.57532934767955	1.27910835553209\\
-3.56625098378596	1.26320870568246\\
-3.55705643168803	1.24737596214119\\
-3.54774618443633	1.23161097395733\\
-3.53832074128622	1.21591458654656\\
-3.52878060767054	1.20028764164616\\
-3.51912629517307	1.18473097726955\\
-3.50935832150053	1.16924542766169\\
-3.49947721045535	1.15383182325396\\
-3.48948349190708	1.13849099062\\
-3.47937770176446	1.12322375243104\\
-3.46916038194623	1.10803092741207\\
-3.45883208035248	1.09291333029765\\
-3.44839335083488	1.07787177178853\\
-3.43784475316735	1.06290705850783\\
-3.42718685301567	1.04801999295809\\
-3.41642022190757	1.03321137347797\\
-3.40554543720159	1.01848199419968\\
-3.39456308205663	1.00383264500611\\
-3.38347374540019	0.989264111488789\\
-3.37227802189724	0.974777174905418\\
-3.36097651191791	0.960372612138289\\
-3.34956982150569	0.946051195652313\\
-3.33805856234454	0.931813693453894\\
-3.32644335172648	0.917660869049441\\
-3.31472481251811	0.903593481404713\\
-3.30290357312757	0.889612284903823\\
-3.29098026747046	0.875718029309069\\
-3.27895553493625	0.861911459720436\\
-3.26683002035359	0.848193316535919\\
-3.25460437395611	0.834564335411533\\
-3.24227925134719	0.821025247222134\\
-3.22985531346515	0.807576778021951\\
-3.21733322654747	0.794219649005914\\
-3.20471366209536	0.780954576470708\\
-3.1919972968375	0.767782271776622\\
-3.17918481269399	0.754703441309135\\
-3.1662768967395	0.74171878644129\\
-3.15327424116674	0.728829003495823\\
-3.14017754324903	0.716034783708072\\
-3.12698750530317	0.703336813188652\\
-3.11370483465157	0.690735772886905\\
-3.10033024358451	0.678232338554138\\
-3.0868644493217	0.665827180707622\\
-3.07330817397412	0.653520964594389\\
-3.05966214450498	0.641314350155792\\
-3.04592709269102	0.629207991991881\\
-3.03210375508301	0.617202539326519\\
-3.0181928729665	0.605298635972343\\
-3.00419519232182	0.593496920296452\\
-2.99011146378429	0.581798025185957\\
-2.9759424426038	0.570202578014251\\
-2.96168888860444	0.558711200607146\\
-2.94735156614363	0.547324509209739\\
-2.93293124407124	0.536043114453153\\
-2.91842869568827	0.524867621321989\\
-2.90384469870546	0.513798629121682\\
-2.88918003520148	0.502836731446555\\
-2.8744354915811	0.491982516147789\\
-2.85961185853293	0.481236565302088\\
-2.84470993098709	0.470599455180272\\
-2.82973050807248	0.460071756216558\\
-2.81467439307407	0.449654032977781\\
-2.79954239338966	0.439346844133294\\
-2.7843353204867	0.429150742424837\\
-2.7690539898587	0.419066274637053\\
-2.75369922098156	0.409093981568008\\
-2.73827183726956	0.399234398000338\\
-2.72277266603127	0.389488052672419\\
-2.70720253842507	0.379855468250158\\
-2.69156228941474	0.370337161298813\\
-2.67585275772453	0.360933642255437\\
-2.66007478579426	0.351645415401356\\
-2.64422921973412	0.342472978835266\\
-2.62831690927933	0.333416824446382\\
-2.61233870774448	0.324477437888189\\
-2.59629547197788	0.315655298552267\\
-2.58018806231552	0.306950879542707\\
-2.56401734253499	0.298364647650614\\
-2.54778417980911	0.289897063329191\\
-2.53148944465949	0.28154858066893\\
-2.51513401090974	0.273319647373369\\
-2.49871875563874	0.265210704734973\\
-2.48224455913347	0.257222187611572\\
-2.46571230484191	0.249354524402937\\
-2.44912287932561	0.2416081370279\\
-2.43247717221214	0.233983440901637\\
-2.41577607614741	0.226480844913473\\
-2.39902048674777	0.219100751404869\\
-2.38221130255199	0.211843556147927\\
-2.36534942497308	0.204709648324083\\
-2.34843575824994	0.197699410503311\\
-2.33147120939888	0.190813218623535\\
-2.31445668816495	0.18405144197053\\
-2.29739310697317	0.17741444315806\\
-2.28028138087961	0.17090257810848\\
-2.26312242752228	0.164516196033605\\
-2.24591716707197	0.158255639416017\\
-2.22866652218285	0.152121243990666\\
-2.21137141794302	0.146113338726891\\
-2.19403278182489	0.140232245810755\\
-2.17665154363544	0.134478280627782\\
-2.15922863546637	0.128851751746028\\
-2.14176499164408	0.123352960899546\\
-2.1242615486796	0.117982202972194\\
-2.10671924521834	0.11273976598182\\
-2.08913902198975	0.107625931064823\\
-2.07152182175692	0.102640972461063\\
-2.05386858926592	0.0977851574991687\\
-2.03618027119526	0.0930587465821813\\
-2.01845781610501	0.0884619931736099\\
-2.000702174386	0.0839951437838135\\
-1.98291429820881	0.079658437956809\\
-1.96509514147273	0.0754521082573903\\
-1.9472456597546	0.071376380258694\\
-1.92936681025757	0.0674314725300584\\
-1.91145955175972	0.063617596625349\\
-1.89352484456273	0.0599349570715599\\
-1.87556365044027	0.056383751357902\\
-1.85757693258654	0.0529641699251496\\
-1.8395656555645	0.0496763961554928\\
-1.82153078525422	0.0465206063626317\\
-1.80347328880107	0.043496969782396\\
-1.78539418094953	0.0406053585329898\\
-1.76729470258804	0.0378441197952474\\
-1.74917623503298	0.0352103601665692\\
-1.73104009644462	0.0327011703712483\\
-1.71288754318293	0.0303136263597934\\
-1.69471977118104	0.0280447903559926\\
-1.6765379173347	0.0258917118539947\\
-1.65834306090679	0.0238514285666112\\
-1.64013622494544	0.0219209673259636\\
-1.62191837771506	0.0200973449375942\\
-1.60369043413899	0.0183775689891709\\
-1.58545325725306	0.0167586386149318\\
-1.56720765966899	0.0152375452170288\\
-1.54895440504683	0.013811273144945\\
-1.53069420957574	0.0124768003341726\\
-1.51242774346224	0.0112310989053509\\
-1.49415563242532	0.0100711357250762\\
-1.47587845919764	0.00899387292960246\\
-1.45759676503236	0.00799626841266569\\
-1.4393110512149	0.0070752762786679\\
-1.42102178057915	0.00622784726246849\\
-1.40272937902767	0.0054509291170349\\
-1.38443423705535	0.00474146697021135\\
-1.36613671127616	0.00409640365186975\\
-1.34783712595258	0.00351267999271097\\
-1.32953577452746	0.00298723509598959\\
-1.31123292115781	0.00251700658343784\\
-1.29292880225035	0.00209893081666829\\
-1.27462362799861	0.00172994309533685\\
-1.25631758392118	0.0014069778333505\\
-1.23801083240109	0.00112696871440538\\
-1.21970351422596	0.000886848828143052\\
-1.20139575012894	0.000683550788213631\\
-1.18308764233014	0.000514006833535643\\
-1.16477927607853	0.00037514891404353\\
-1.14647072119416	0.000263908762214025\\
-1.12816203361071	0.000177217951663389\\
-1.10985325691809	0.000112007944107812\\
-1.09154442390533	6.52101259796178e-05\\
-1.07323555810341	3.37558359919959e-05\\
-1.05492667532827	1.45763849452045e-05\\
-1.03661778522378	4.60306906731815e-06\\
-1.01830889280477	7.67178182498219e-07\\
-1	0\\
};
\addlegendentry{PINS}

\addplot [color=blue, line width=2.0pt, only marks, mark size=2.5pt, mark=*, mark options={solid, fill=blue, blue}, forget plot]
  table[row sep=crcr]{%
0	0\\
};
\addplot [color=blue, line width=2.0pt, only marks, mark size=2.5pt, mark=*, mark options={solid, fill=blue, blue}, forget plot]
  table[row sep=crcr]{%
-1	0\\
};
\end{axis}
\end{tikzpicture}%%
  \caption{Trajectory analysis PINS vs Duboids}
  \label{fig:Compare_traj2}
\end{figure}
%
\begin{figure}[htb!]
  \centering
  
%
\begin{tikzpicture}[scale = 0.7]

\begin{axis}[%
width=0.985\linewidth,
height=\linewidth,
at={(0\linewidth,0\linewidth)},
scale only axis,
xmin=0,
xmax=20,
xlabel style={font=\color{white!15!black}},
xlabel={Time(s)},
ymin=-0.1,
ymax=0.5,
ylabel style={font=\color{white!15!black}},
ylabel={$\kappa(m^{-1})$},
axis background/.style={fill=white},
title style={font=\bfseries},
title={Curvature - PINS},
axis x line*=bottom,
axis y line*=left,
xmajorgrids,
xminorgrids,
ymajorgrids,
yminorgrids,
legend style={legend cell align=left, align=left, draw=white!15!black}
]
\addplot [color=green, dashdotted, line width=2.0pt]
  table[row sep=crcr]{%
0	0\\
0.183079627851477	0.0915398139257387\\
0.366159255702955	0.183079627851477\\
0.549238883554432	0.274619441777216\\
0.73231851140591	0.366159255702955\\
0.915398139257387	0.4\\
1.09847776710886	0.4\\
1.28155739496034	0.4\\
1.46463702281182	0.4\\
1.6477166506633	0.4\\
1.83079627851477	0.4\\
2.01387590636625	0.4\\
2.19695553421773	0.4\\
2.38003516206921	0.4\\
2.56311478992068	0.4\\
2.74619441777216	0.4\\
2.92927404562364	0.4\\
3.11235367347512	0.4\\
3.29543330132659	0.4\\
3.47851292917807	0.4\\
3.66159255702955	0.4\\
3.84467218488103	0.4\\
4.0277518127325	0.4\\
4.21083144058398	0.4\\
4.39391106843546	0.4\\
4.57699069628694	0.4\\
4.76007032413841	0.4\\
4.94314995198989	0.4\\
5.12622957984137	0.4\\
5.30930920769284	0.4\\
5.49238883554432	0.4\\
5.6754684633958	0.4\\
5.85854809124728	0.4\\
6.04162771909875	0.4\\
6.22470734695023	0.4\\
6.40778697480171	0.4\\
6.59086660265319	0.4\\
6.77394623050466	0.4\\
6.95702585835614	0.4\\
7.14010548620762	0.4\\
7.3231851140591	0.4\\
7.50626474191057	0.4\\
7.68934436976205	0.4\\
7.87242399761353	0.390778773022471\\
8.05550362546501	0.299238959096732\\
8.23858325331648	0.207699145170993\\
8.42166288116796	0.116159331245254\\
8.60474250901944	0.0246195173195157\\
8.78782213687091	1.11022302462516e-16\\
8.97090176472239	1.11022302462516e-16\\
9.15398139257387	1.11022302462516e-16\\
9.33706102042535	1.11022302462516e-16\\
9.52014064827683	1.11022302462516e-16\\
9.7032202761283	0.0246194865913162\\
9.88629990397978	0.116159300517055\\
10.0693795318313	0.207699114442794\\
10.2524591596827	0.299238928368533\\
10.4355387875342	0.390778742294272\\
10.6186184153857	0.4\\
10.8016980432372	0.4\\
10.9847776710886	0.4\\
11.1678572989401	0.4\\
11.3509369267916	0.4\\
11.5340165546431	0.4\\
11.7170961824946	0.4\\
11.900175810346	0.4\\
12.0832554381975	0.4\\
12.266335066049	0.4\\
12.4494146939005	0.4\\
12.6324943217519	0.4\\
12.8155739496034	0.4\\
12.9986535774549	0.4\\
13.1817332053064	0.4\\
13.3648128331579	0.4\\
13.5478924610093	0.4\\
13.7309720888608	0.4\\
13.9140517167123	0.4\\
14.0971313445638	0.4\\
14.2802109724152	0.4\\
14.4632906002667	0.4\\
14.6463702281182	0.4\\
14.8294498559697	0.4\\
15.0125294838211	0.4\\
15.1956091116726	0.4\\
15.3786887395241	0.4\\
15.5617683673756	0.4\\
15.7448479952271	0.4\\
15.9279276230785	0.4\\
16.11100725093	0.4\\
16.2940868787815	0.4\\
16.477166506633	0.4\\
16.6602461344844	0.4\\
16.8433257623359	0.4\\
17.0264053901874	0.4\\
17.2094850180389	0.4\\
17.3925646458904	0.4\\
17.5756442737418	0.366159255702957\\
17.7587239015933	0.274619441777219\\
17.9418035294448	0.183079627851479\\
18.1248831572963	0.0915398139257403\\
18.3079627851477	1.88737914186277e-15\\
};
\addlegendentry{Duboids}

\addplot [color=dodgerblue, dotted, line width=2.0pt]
  table[row sep=crcr]{%
0	0\\
0.018308892820843	0.00915444606593419\\
0.036617785641686	0.0183088921237435\\
0.054926678462529	0.0274633381730355\\
0.073235571283372	0.0366177842133885\\
0.091544464104215	0.0457722302443487\\
0.109853356925058	0.054926676265427\\
0.128162249745901	0.0640811222760952\\
0.146471142566744	0.0732355682757812\\
0.164780035387587	0.082390014263865\\
0.18308892820843	0.0915444602396722\\
0.201397821029273	0.100698906202468\\
0.219706713850116	0.10985335215145\\
0.238015606670959	0.119007798085739\\
0.256324499491802	0.12816224400437\\
0.274633392312645	0.137316689906278\\
0.292942285133488	0.146471135790287\\
0.311251177954331	0.155625581655092\\
0.329560070775174	0.164780027499242\\
0.347868963596017	0.173934473321114\\
0.36617785641686	0.183088919118884\\
0.384486749237703	0.192243364890498\\
0.402795642058546	0.201397810633627\\
0.421104534879389	0.210552256345616\\
0.439413427700232	0.219706702023423\\
0.457722320521075	0.228861147663536\\
0.476031213341918	0.238015593261874\\
0.494340106162761	0.247170038813657\\
0.512648998983604	0.256324484313232\\
0.530957891804447	0.265478929753856\\
0.54926678462529	0.274633375127388\\
0.567575677446133	0.283787820423875\\
0.585884570266976	0.292942265630979\\
0.604193463087819	0.302096710733137\\
0.622502355908662	0.31125115571034\\
0.640811248729505	0.320405600536254\\
0.659120141550348	0.329560045175249\\
0.677429034371191	0.338714489577427\\
0.695737927192034	0.347868933669847\\
0.714046820012877	0.35702337733974\\
0.73235571283372	0.366177820399075\\
0.750664605654563	0.375332262498408\\
0.768973498475406	0.384486702868094\\
0.787282391296249	0.393641139193881\\
0.805591284117092	0.402795555131127\\
0.823900176937935	0.397146347682761\\
0.842209069758778	0.402795130456504\\
0.860517962579621	0.397146747057585\\
0.878826855400464	0.402794705187685\\
0.897135748221307	0.397147144912013\\
0.91544464104215	0.402794279264965\\
0.933753533862993	0.397147541299805\\
0.952062426683836	0.402793852628503\\
0.970371319504679	0.397147936274391\\
0.988680212325522	0.402793425218302\\
1.00698910514636	0.397148329888899\\
1.02529799796721	0.402792996974165\\
1.04360689078805	0.397148722196181\\
1.06191578360889	0.402792567835674\\
1.08022467642974	0.397149113248844\\
1.09853356925058	0.402792137742151\\
1.11684246207142	0.397149503099274\\
1.13515135489227	0.402791706632626\\
1.15346024771311	0.397149891799665\\
1.17176914053395	0.40279127444581\\
1.19007803335479	0.397150279402047\\
1.20838692617564	0.402790841120056\\
1.22669581899648	0.397150665958311\\
1.24500471181732	0.402790406593329\\
1.26331360463817	0.397151051520241\\
1.28162249745901	0.402789970803171\\
1.29993139027985	0.397151436139537\\
1.3182402831007	0.402789533686667\\
1.33654917592154	0.397151819867846\\
1.35485806874238	0.402789095180412\\
1.37316696156322	0.397152202756787\\
1.39147585438407	0.402788655220473\\
1.40978474720491	0.397152584857982\\
1.42809364002575	0.402788213742357\\
1.4464025328466	0.397152966223083\\
1.46471142566744	0.402787770680971\\
1.48302031848828	0.3971533469038\\
1.50132921130913	0.402787325970586\\
1.51963810412997	0.397153726951931\\
1.53794699695081	0.402786879544804\\
1.55625588977165	0.39715410641939\\
1.5745647825925	0.402786431336512\\
1.59287367541334	0.397154485358238\\
1.61118256823418	0.402785981277848\\
1.62949146105503	0.397154863820711\\
1.64780035387587	0.402785529300161\\
1.66610924669671	0.397155241859252\\
1.68441813951756	0.402785075333968\\
1.7027270323384	0.397155619526541\\
1.72103592515924	0.402784619308913\\
1.73934481798008	0.397155996875527\\
1.75765371080093	0.402784161153724\\
1.77596260362177	0.397156373959457\\
1.79427149644261	0.402783700796172\\
1.81258038926346	0.397156750831915\\
1.8308892820843	0.402783238163021\\
1.84919817490514	0.397157127546846\\
1.86750706772599	0.402782773179987\\
1.88581596054683	0.397157504158599\\
1.90412485336767	0.402782305771688\\
1.92243374618851	0.397157880721955\\
1.94074263900936	0.402781835861594\\
1.9590515318302	0.397158257292165\\
1.97736042465104	0.402781363371982\\
1.99566931747189	0.397158633924985\\
2.01397821029273	0.40278088822388\\
2.03228710311357	0.397159010676713\\
2.05059599593442	0.402780410337015\\
2.06890488875526	0.39715938760423\\
2.0872137815761	0.402779929629762\\
2.10552267439694	0.397159764765034\\
2.12383156721779	0.402779446019082\\
2.14214046003863	0.397160142217284\\
2.16044935285947	0.40277895942047\\
2.17875824568032	0.397160520019838\\
2.19706713850116	0.402778469747891\\
2.215376031322	0.397160898232297\\
2.23368492414285	0.402777976913723\\
2.25199381696369	0.39716127691505\\
2.27030270978453	0.402777480828686\\
2.28861160260537	0.397161656129316\\
2.30692049542622	0.402776981401784\\
2.32522938824706	0.39716203593719\\
2.3435382810679	0.402776478540232\\
2.36184717388875	0.397162416401696\\
2.38015606670959	0.40277597214939\\
2.39846495953043	0.39716279758683\\
2.41677385235128	0.402775462132685\\
2.43508274517212	0.397163179557612\\
2.45339163799296	0.402774948391538\\
2.4717005308138	0.397163562380144\\
2.49000942363465	0.402774430825288\\
2.50831831645549	0.397163946121657\\
2.52662720927633	0.402773909331108\\
2.54493610209718	0.397164330850574\\
2.56324499491802	0.402773383803922\\
2.58155388773886	0.397164716636564\\
2.59986278055971	0.40277285413632\\
2.61817167338055	0.397165103550606\\
2.63648056620139	0.402772320218463\\
2.65478945902223	0.397165491665048\\
2.67309835184308	0.402771781937994\\
2.69140724466392	0.397165881053678\\
2.70971613748476	0.402771239179936\\
2.72802503030561	0.397166271791784\\
2.74633392312645	0.402770691826596\\
2.76464281594729	0.397166663956234\\
2.78295170876814	0.402770139757454\\
2.80126060158898	0.397167057625541\\
2.81956949440982	0.402769582849058\\
2.83787838723066	0.397167452879946\\
2.85618728005151	0.402769020974907\\
2.87449617287235	0.39716784980149\\
2.89280506569319	0.402768454005334\\
2.91111395851404	0.397168248474104\\
2.92942285133488	0.402767881807381\\
2.94773174415572	0.397168648983694\\
2.96604063697657	0.40276730424467\\
2.98434952979741	0.397169051418224\\
3.00265842261825	0.402766721177271\\
3.02096731543909	0.397169455867822\\
3.03927620825994	0.40276613246156\\
3.05758510108078	0.397169862424864\\
3.07589399390162	0.402765537950072\\
3.09420288672247	0.39717027118409\\
3.11251177954331	0.402764937491353\\
3.13082067236415	0.397170682242699\\
3.149129565185	0.402764330929794\\
3.16743845800584	0.39717109570047\\
3.18574735082668	0.402763718105471\\
3.20405624364752	0.397171511659872\\
3.22236513646837	0.402763098853967\\
3.24067402928921	0.397171930226191\\
3.25898292211005	0.402762473006193\\
3.2772918149309	0.397172351507655\\
3.29560070775174	0.402761840388199\\
3.31390960057258	0.39717277561557\\
3.33221849339342	0.402761200820969\\
3.35052738621427	0.39717320266446\\
3.36883627903511	0.402760554120223\\
3.38714517185595	0.397173632772212\\
3.4054540646768	0.402759900096188\\
3.42376295749764	0.397174066060234\\
3.44207185031848	0.402759238553382\\
3.46038074313933	0.397174502653616\\
3.47868963596017	0.402758569290364\\
3.49699852878101	0.397174942681297\\
3.51530742160186	0.402757892099492\\
3.5336163144227	0.397175386276249\\
3.55192520724354	0.402757206766652\\
3.57023410006438	0.397175833575661\\
3.58854299288523	0.402756513070992\\
3.60685188570607	0.397176284721137\\
3.62516077852691	0.402755810784625\\
3.64346967134776	0.397176739858905\\
3.6617785641686	0.402755099672327\\
3.68008745698944	0.397177199140034\\
3.69839634981029	0.40275437949122\\
3.71670524263113	0.397177662720664\\
3.73501413545197	0.402753649990434\\
3.75332302827281	0.397178130762251\\
3.77163192109366	0.402752910910754\\
3.7899408139145	0.397178603431819\\
3.80824970673534	0.402752161984251\\
3.82655859955619	0.39717908090223\\
3.84486749237703	0.40275140293389\\
3.86317638519787	0.39717956335247\\
3.88148527801871	0.402750633473117\\
3.89979417083956	0.397180050967948\\
3.9181030636604	0.402749853305429\\
3.93641195648124	0.397180543940809\\
3.95472084930209	0.402749062123914\\
3.97302974212293	0.397181042470273\\
3.99133863494377	0.40274825961077\\
4.00964752776462	0.397181546762987\\
4.02795642058546	0.4027474454368\\
4.0462653134063	0.397182057033398\\
4.06457420622715	0.402746619260871\\
4.08288309904799	0.397182573504148\\
4.10119199186883	0.402745780729348\\
4.11950088468967	0.397183096406496\\
4.13780977751052	0.402744929475499\\
4.15611867033136	0.397183625980755\\
4.1744275631522	0.402744065118861\\
4.19273645597305	0.397184162476771\\
4.21104534879389	0.402743187264567\\
4.22935424161473	0.397184706154411\\
4.24766313443557	0.402742295502643\\
4.26597202725642	0.397185257284101\\
4.28428092007726	0.402741389407255\\
4.3025898128981	0.397185816147381\\
4.32089870571895	0.402740468535915\\
4.33920759853979	0.397186383037504\\
4.35751649136063	0.402739532428638\\
4.37582538418148	0.397186958260069\\
4.39413427700232	0.402738580607053\\
4.41244316982316	0.397187542133694\\
4.430752062644	0.402737612573448\\
4.44906095546485	0.397188134990735\\
4.46736984828569	0.402736627809771\\
4.48567874110653	0.397188737178047\\
4.50398763392738	0.402735625776555\\
4.52229652674822	0.397189349057796\\
4.54060541956906	0.402734605911784\\
4.55891431238991	0.397189971008331\\
4.57722320521075	0.40273356762968\\
4.59553209803159	0.397190603425103\\
4.61384099085243	0.402732510319421\\
4.63214988367328	0.397191246721656\\
4.65045877649412	0.402731433343766\\
4.66876766931496	0.397191901330679\\
4.68707656213581	0.402730336037591\\
4.70538545495665	0.397192567705134\\
4.72369434777749	0.402729217706336\\
4.74200324059834	0.397193246319459\\
4.76031213341918	0.402728077624334\\
4.77862102624002	0.397193937670863\\
4.79692991906087	0.402726915033039\\
4.81523881188171	0.397194642280701\\
4.83354770470255	0.402725729139123\\
4.85185659752339	0.397195360695963\\
4.87016549034424	0.402724519112445\\
4.88847438316508	0.397196093490853\\
4.90678327598592	0.402723284083874\\
4.92509216880677	0.397196841268506\\
4.94340106162761	0.402722023142964\\
4.96170995444845	0.39719760466281\\
4.9800188472693	0.402720735335447\\
4.99832774009014	0.397198384340383\\
5.01663663291098	0.402719419660561\\
5.03494552573182	0.39719918100269\\
5.05325441855267	0.402718075068168\\
5.07156331137351	0.397199995388328\\
5.08987220419435	0.40271670045566\\
5.1081810970152	0.397200828275484\\
5.12648998983604	0.402715294664635\\
5.14479888265688	0.397201680484595\\
5.16310777547772	0.402713856477311\\
5.18141666829857	0.397202552881212\\
5.19972556111941	0.402712384612671\\
5.21803445394025	0.397203446379096\\
5.2363433467611	0.4027108777223\\
5.25465223958194	0.397204361943577\\
5.27296113240278	0.402709334385887\\
5.29127002522362	0.397205300595176\\
5.30957891804447	0.402707753106379\\
5.32788781086531	0.397206263413541\\
5.34619670368615	0.402706132304722\\
5.364505596507	0.397207251541718\\
5.38281448932784	0.40270447031418\\
5.40112338214868	0.39720826619078\\
5.41943227496953	0.402702765374174\\
5.43774116779037	0.397209308644871\\
5.45605006061121	0.402701015623595\\
5.47435895343206	0.397210380266686\\
5.4926678462529	0.402699219093549\\
5.51097673907374	0.397211482503451\\
5.52928563189458	0.402697373699457\\
5.54759452471543	0.397212616893432\\
5.56590341753627	0.402695477232467\\
5.58421231035711	0.397213785073062\\
5.60252120317796	0.402693527350089\\
5.6208300959988	0.397214988784715\\
5.63913898881964	0.402691521565973\\
5.65744788164049	0.397216229885227\\
5.67575677446133	0.402689457238753\\
5.69406566728217	0.397217510355231\\
5.71237456010301	0.402687331559831\\
5.73068345292386	0.397218832309405\\
5.7489923457447	0.402685141540003\\
5.76730123856554	0.397220198007729\\
5.78561013138639	0.402682883994794\\
5.80391902420723	0.39722160986788\\
5.82222791702807	0.402680555528343\\
5.84053680984892	0.39722307047888\\
5.85884570266976	0.402678152515683\\
5.8771545954906	0.39722458261617\\
5.89546348831144	0.402675671083224\\
5.91377238113229	0.39722614925826\\
5.93208127395313	0.402673107087223\\
5.95039016677397	0.397227773605167\\
5.96869905959482	0.402670456089986\\
5.98700795241566	0.397229459098851\\
6.0053168452365	0.402667713333547\\
6.02362573805735	0.397231209445914\\
6.04193463087819	0.402664873710474\\
6.06024352369903	0.397233028642849\\
6.07855241651987	0.402661931731461\\
6.09686130934072	0.397234921004174\\
6.11517020216156	0.402658881489274\\
6.1334790949824	0.397236891193839\\
6.15178798780325	0.402655716618566\\
6.17009688062409	0.397238944260361\\
6.18840577344493	0.40265243025101\\
6.20671466626578	0.397241085676184\\
6.22502355908662	0.402649014965092\\
6.24333245190746	0.397243321381886\\
6.2616413447283	0.402645462729835\\
6.27995023754915	0.397245657835915\\
6.29825913036999	0.402641764841559\\
6.31656802319083	0.397248102070673\\
6.33487691601168	0.402637911852676\\
6.35318580883252	0.397250661755898\\
6.37149470165336	0.402633893491333\\
6.3898035944742	0.397253345270466\\
6.40811248729505	0.402629698570495\\
6.42642138011589	0.397256161783919\\
6.44473027293673	0.402625314884838\\
6.46303916575758	0.397259121349274\\
6.48134805857842	0.40262072909351\\
6.49965695139926	0.397262235008953\\
6.51796584422011	0.402615926586462\\
6.53627473704095	0.397265514916015\\
6.55458362986179	0.402610891331608\\
6.57289252268264	0.397268974473297\\
6.59120141550348	0.402605605699569\\
6.60951030832432	0.397272628493587\\
6.62781920114516	0.402600050262065\\
6.64612809396601	0.397276493384581\\
6.66443698678685	0.402594203559265\\
6.68274587960769	0.397280587363155\\
6.70105477242854	0.402588041830384\\
6.71936366524938	0.397284930704444\\
6.73767255807022	0.402581538700624\\
6.75598145089107	0.397289546032436\\
6.77429034371191	0.402574664816005\\
6.79259923653275	0.397294458660255\\
6.81090812935359	0.402567387415723\\
6.82921702217444	0.397299696990241\\
6.84752591499528	0.402559669829263\\
6.86583480781612	0.397305292986291\\
6.88414370063697	0.402551470882399\\
6.90245259345781	0.397311282733996\\
6.92076148627865	0.402542744192282\\
6.93907037909949	0.397317707108008\\
6.95737927192034	0.402533437326727\\
6.97568816474118	0.397324612571133\\
6.99399705756202	0.402523490796219\\
7.01230595038287	0.397332052136242\\
7.03061484320371	0.402512836838465\\
7.04892373602455	0.397340086530718\\
7.0672326288454	0.402501397943894\\
7.08554152166624	0.397348785614645\\
7.10385041448708	0.402489085055173\\
7.12215930730793	0.39735823011928\\
7.14046820012877	0.402475795353129\\
7.15877709294961	0.397368513793096\\
7.17708598577045	0.402461409513242\\
7.1953948785913	0.397379746071073\\
7.21370377141214	0.402445788277753\\
7.23201266423298	0.397392055422227\\
7.25032155705383	0.40242876813363\\
7.26863044987467	0.397405593585579\\
7.28693934269551	0.402410155808439\\
7.30524823551636	0.397420540983501\\
7.3235571283372	0.402389721182948\\
7.34186602115804	0.397437113715536\\
7.36017491397888	0.402367188051859\\
7.37848380679973	0.397455572704492\\
7.39679269962057	0.402342221911413\\
7.41510159244141	0.397476235821316\\
7.43341048526226	0.402314413561726\\
7.4517193780831	0.397499494209056\\
7.47002827090394	0.40228325669\\
7.48833716372478	0.397525834652291\\
7.50664605654563	0.402248116579846\\
7.52495494936647	0.397555870865681\\
7.54326384218731	0.402208185352056\\
7.56157273500816	0.397590388324516\\
7.579881627829	0.402162416044818\\
7.59819052064984	0.397630410370954\\
7.61649941347069	0.402109422034122\\
7.63480830629153	0.397677299156447\\
7.65311719911237	0.40204731670276\\
7.67142609193322	0.397732916601268\\
7.68973498475406	0.401973443265887\\
7.7080438775749	0.397799895592608\\
7.72635277039574	0.401883885195761\\
7.74466166321659	0.39788213116521\\
7.76297055603743	0.401772486799793\\
7.78127944885827	0.397985762379368\\
7.79958834167912	0.401628592868526\\
7.81789723449996	0.398121436375314\\
7.8362061273208	0.401430503835516\\
7.85451502014164	0.398310858805647\\
7.87282391296249	0.401116153382826\\
7.89113280578333	0.398616125289521\\
7.90944169860417	0.394807579878867\\
7.92775059142502	0.385656267944226\\
7.94605948424586	0.376503474026801\\
7.9643683770667	0.367350190192656\\
7.98267726988755	0.358196664812719\\
8.00098616270839	0.34904299750743\\
8.01929505552923	0.33988923833419\\
8.03760394835007	0.330735416132871\\
8.05591284117092	0.321581549159197\\
8.07422173399176	0.312427649814119\\
8.0925306268126	0.303273727011973\\
8.11083951963345	0.294119787475859\\
8.12914841245429	0.284965836497201\\
8.14745730527513	0.275811878407434\\
8.16576619809598	0.26665791688593\\
8.18407509091682	0.257503955170488\\
8.20238398373766	0.248349996207874\\
8.22069287655851	0.23919604276667\\
8.23900176937935	0.230042097526249\\
8.25731066220019	0.220888163150916\\
8.27561955502103	0.211734242355403\\
8.29392844784188	0.202580337966284\\
8.31223734066272	0.193426452982912\\
8.33054623348356	0.184272590641031\\
8.34885512630441	0.175118754482073\\
8.36716401912525	0.1659649484313\\
8.38547291194609	0.15681117688836\\
8.40378180476694	0.147657444834523\\
8.42209069758778	0.13850375796188\\
8.44039959040862	0.129350122831278\\
8.45870848322947	0.120196547067805\\
8.47701737605031	0.111043039605539\\
8.49532626887115	0.101889610997356\\
8.51363516169199	0.0927362738113144\\
8.53194405451284	0.0835830431434889\\
8.55025294733368	0.0744299372891218\\
8.56856184015452	0.0652769786318448\\
8.58687073297537	0.0561241948374978\\
8.60517962579621	0.0469716204801463\\
8.62348851861705	0.0378192992921482\\
8.64179741143789	0.0286672873329923\\
8.66010630425874	0.0195156575408037\\
8.67841519707958	0.0103645064169277\\
8.69672408990042	0.00121396409614429\\
8.71503298272127	-0.00793579002961699\\
8.73334187554211	-0.0170845022189134\\
8.75165076836295	-0.0262318023809368\\
8.76995966118379	-0.0353771301335943\\
8.78826855400464	-0.0445195960993527\\
8.80657744682548	-0.0536576993260843\\
8.82488633964632	-0.0627886830217425\\
8.84319523246717	-0.0719068041341361\\
8.86150412528801	-0.0809972768724237\\
8.87981301810885	-0.090000365036115\\
8.8981219109297	-0.092863806708639\\
8.91643080375054	-0.0839418565437226\\
8.93473969657138	-0.0749310457032785\\
8.95304858939222	-0.0658975641370096\\
8.97135748221307	-0.0568600730246737\\
8.98966637503391	-0.047828976110916\\
9.00797526785475	-0.0388135810845879\\
9.0262841606756	-0.0298253041290661\\
9.04459305349644	-0.0208809450689012\\
9.06290194631728	-0.0120080971591629\\
9.08121083913813	-0.00325599229492927\\
9.09951973195897	0.00528012993119749\\
9.11782862477981	0.0133959298847554\\
9.13613751760066	0.0205611474642791\\
9.1544464104215	0.024654048172014\\
9.17275530324234	0.020561147464279\\
9.19106419606318	0.0133959298847552\\
9.20937308888403	0.0052801299311974\\
9.22768198170487	-0.00325599229492935\\
9.24599087452571	-0.012008097159163\\
9.26429976734656	-0.0208809450689013\\
9.2826086601674	-0.0298253041290661\\
9.30091755298824	-0.0388135810845879\\
9.31922644580908	-0.0478289761109161\\
9.33753533862993	-0.0568600730246737\\
9.35584423145077	-0.0658975641370096\\
9.37415312427161	-0.0749310457032785\\
9.39246201709246	-0.0839418565437226\\
9.4107709099133	-0.092863806708639\\
9.42907980273414	-0.0900003650361148\\
9.44738869555499	-0.0809972768724235\\
9.46569758837583	-0.071906804134136\\
9.48400648119667	-0.0627886830217423\\
9.50231537401752	-0.0536576993260841\\
9.52062426683836	-0.0445195960993526\\
9.5389331596592	-0.0353771301335942\\
9.55724205248004	-0.0262318023809366\\
9.57555094530089	-0.0170845022189133\\
9.59385983812173	-0.00793579002961684\\
9.61216873094257	0.00121396409614444\\
9.63047762376342	0.0103645064169279\\
9.64878651658426	0.0195156575408038\\
9.6670954094051	0.0286672873329924\\
9.68540430222595	0.0378192992921483\\
9.70371319504679	0.0469716204801465\\
9.72202208786763	0.056124194837498\\
9.74033098068847	0.0652769786318454\\
9.75863987350932	0.0744299372891225\\
9.77694876633016	0.0835830431434896\\
9.795257659151	0.092736273811315\\
9.81356655197185	0.101889610997356\\
9.83187544479269	0.11104303960554\\
9.85018433761353	0.120196547067805\\
9.86849323043437	0.129350122831279\\
9.88680212325522	0.138503757961881\\
9.90511101607606	0.147657444834523\\
9.9234199088969	0.156811176888361\\
9.94172880171775	0.165964948431301\\
9.96003769453859	0.175118754482074\\
9.97834658735943	0.184272590641032\\
9.99665548018028	0.193426452982912\\
10.0149643730011	0.202580337966285\\
10.033273265822	0.211734242355404\\
10.0515821586428	0.220888163150916\\
10.0698910514636	0.23004209752625\\
10.0881999442845	0.23919604276667\\
10.1065088371053	0.248349996207875\\
10.1248177299262	0.257503955170488\\
10.143126622747	0.26665791688593\\
10.1614355155679	0.275811878407434\\
10.1797444083887	0.284965836497201\\
10.1980533012095	0.29411978747586\\
10.2163621940304	0.303273727011974\\
10.2346710868512	0.31242764981412\\
10.2529799796721	0.321581549159198\\
10.2712888724929	0.330735416132872\\
10.2895977653138	0.339889238334191\\
10.3079066581346	0.34904299750743\\
10.3262155509554	0.358196664812718\\
10.3445244437763	0.367350190192656\\
10.3628333365971	0.376503474026801\\
10.381142229418	0.385656267944226\\
10.3994511222388	0.394807579878867\\
10.4177600150597	0.39861612528952\\
10.4360689078805	0.401116153382828\\
10.4543778007013	0.398310858805646\\
10.4726866935222	0.401430503835518\\
10.490995586343	0.398121436375313\\
10.5093044791639	0.401628592868527\\
10.5276133719847	0.397985762379366\\
10.5459222648056	0.401772486799795\\
10.5642311576264	0.397882131165208\\
10.5825400504472	0.401883885195763\\
10.6008489432681	0.397799895592606\\
10.6191578360889	0.401973443265888\\
10.6374667289098	0.397732916601267\\
10.6557756217306	0.402047316702761\\
10.6740845145515	0.397677299156446\\
10.6923934073723	0.402109422034123\\
10.7107023001932	0.397630410370953\\
10.729011193014	0.40216241604482\\
10.7473200858348	0.397590388324515\\
10.7656289786557	0.402208185352057\\
10.7839378714765	0.39755587086568\\
10.8022467642974	0.402248116579847\\
10.8205556571182	0.39752583465229\\
10.8388645499391	0.402283256690001\\
10.8571734427599	0.397499494209055\\
10.8754823355807	0.402314413561728\\
10.8937912284016	0.397476235821314\\
10.9121001212224	0.402342221911414\\
10.9304090140433	0.39745557270449\\
10.9487179068641	0.40236718805186\\
10.967026799685	0.397437113715534\\
10.9853356925058	0.40238972118295\\
11.0036445853266	0.3974205409835\\
11.0219534781475	0.402410155808441\\
11.0402623709683	0.397405593585577\\
11.0585712637892	0.402428768133632\\
11.07688015661	0.397392055422225\\
11.0951890494309	0.402445788277755\\
11.1134979422517	0.397379746071071\\
11.1318068350725	0.402461409513243\\
11.1501157278934	0.397368513793094\\
11.1684246207142	0.40247579535313\\
11.1867335135351	0.397358230119278\\
11.2050424063559	0.402489085055174\\
11.2233512991768	0.397348785614644\\
11.2416601919976	0.402501397943896\\
11.2599690848184	0.397340086530717\\
11.2782779776393	0.402512836838467\\
11.2965868704601	0.397332052136241\\
11.314895763281	0.402523490796221\\
11.3332046561018	0.397324612571131\\
11.3515135489227	0.402533437326729\\
11.3698224417435	0.397317707108006\\
11.3881313345643	0.402542744192283\\
11.4064402273852	0.397311282733995\\
11.424749120206	0.4025514708824\\
11.4430580130269	0.39730529298629\\
11.4613669058477	0.402559669829264\\
11.4796757986686	0.397299696990239\\
11.4979846914894	0.402567387415724\\
11.5162935843102	0.397294458660253\\
11.5346024771311	0.402574664816007\\
11.5529113699519	0.397289546032434\\
11.5712202627728	0.402581538700626\\
11.5895291555936	0.397284930704442\\
11.6078380484145	0.402588041830385\\
11.6261469412353	0.397280587363153\\
11.6444558340561	0.402594203559266\\
11.662764726877	0.39727649338458\\
11.6810736196978	0.402600050262067\\
11.6993825125187	0.397272628493586\\
11.7176914053395	0.402605605699571\\
11.7360002981604	0.397268974473296\\
11.7543091909812	0.40261089133161\\
11.772618083802	0.397265514916014\\
11.7909269766229	0.402615926586464\\
11.8092358694437	0.397262235008952\\
11.8275447622646	0.402620729093512\\
11.8458536550854	0.397259121349272\\
11.8641625479063	0.40262531488484\\
11.8824714407271	0.397256161783917\\
11.9007803335479	0.402629698570496\\
11.9190892263688	0.397253345270464\\
11.9373981191896	0.402633893491335\\
11.9557070120105	0.397250661755896\\
11.9740159048313	0.402637911852678\\
11.9923247976522	0.397248102070671\\
12.010633690473	0.40264176484156\\
12.0289425832938	0.397245657835914\\
12.0472514761147	0.402645462729837\\
12.0655603689355	0.397243321381884\\
12.0838692617564	0.402649014965094\\
12.1021781545772	0.397241085676182\\
12.1204870473981	0.402652430251012\\
12.1387959402189	0.397238944260359\\
12.1571048330397	0.402655716618568\\
12.1754137258606	0.397236891193837\\
12.1937226186814	0.402658881489276\\
12.2120315115023	0.397234921004172\\
12.2303404043231	0.402661931731462\\
12.248649297144	0.397233028642848\\
12.2669581899648	0.402664873710475\\
12.2852670827857	0.397231209445913\\
12.3035759756065	0.402667713333549\\
12.3218848684273	0.397229459098849\\
12.3401937612482	0.402670456089988\\
12.358502654069	0.397227773605165\\
12.3768115468899	0.402673107087224\\
12.3951204397107	0.397226149258259\\
12.4134293325316	0.402675671083226\\
12.4317382253524	0.397224582616169\\
12.4500471181732	0.402678152515684\\
12.4683560109941	0.397223070478879\\
12.4866649038149	0.402680555528344\\
12.5049737966358	0.397221609867879\\
12.5232826894566	0.402682883994796\\
12.5415915822775	0.397220198007728\\
12.5599004750983	0.402685141540005\\
12.5782093679191	0.397218832309403\\
12.59651826074	0.402687331559832\\
12.6148271535608	0.397217510355229\\
12.6331360463817	0.402689457238754\\
12.6514449392025	0.397216229885226\\
12.6697538320234	0.402691521565974\\
12.6880627248442	0.397214988784714\\
12.706371617665	0.40269352735009\\
12.7246805104859	0.397213785073061\\
12.7429894033067	0.402695477232469\\
12.7612982961276	0.397212616893431\\
12.7796071889484	0.402697373699458\\
12.7979160817693	0.397211482503449\\
12.8162249745901	0.40269921909355\\
12.8345338674109	0.397210380266685\\
12.8528427602318	0.402701015623597\\
12.8711516530526	0.397209308644869\\
12.8894605458735	0.402702765374175\\
12.9077694386943	0.397208266190778\\
12.9260783315152	0.402704470314181\\
12.944387224336	0.397207251541716\\
12.9626961171568	0.402706132304724\\
12.9810050099777	0.39720626341354\\
12.9993139027985	0.402707753106381\\
13.0176227956194	0.397205300595174\\
13.0359316884402	0.402709334385889\\
13.0542405812611	0.397204361943576\\
13.0725494740819	0.402710877722301\\
13.0908583669027	0.397203446379095\\
13.1091672597236	0.402712384612673\\
13.1274761525444	0.39720255288121\\
13.1457850453653	0.402713856477312\\
13.1640939381861	0.397201680484594\\
13.182402831007	0.402715294664636\\
13.2007117238278	0.397200828275482\\
13.2190206166486	0.402716700455662\\
13.2373295094695	0.397199995388326\\
13.2556384022903	0.40271807506817\\
13.2739472951112	0.397199181002688\\
13.292256187932	0.402719419660563\\
13.3105650807529	0.397198384340381\\
13.3288739735737	0.402720735335449\\
13.3471828663945	0.397197604662808\\
13.3654917592154	0.402722023142965\\
13.3838006520362	0.397196841268504\\
13.4021095448571	0.402723284083876\\
13.4204184376779	0.397196093490852\\
13.4387273304988	0.402724519112447\\
13.4570362233196	0.397195360695961\\
13.4753451161404	0.402725729139125\\
13.4936540089613	0.3971946422807\\
13.5119629017821	0.402726915033041\\
13.530271794603	0.397193937670861\\
13.5485806874238	0.402728077624336\\
13.5668895802447	0.397193246319457\\
13.5851984730655	0.402729217706337\\
13.6035073658863	0.397192567705132\\
13.6218162587072	0.402730336037593\\
13.640125151528	0.397191901330677\\
13.6584340443489	0.402731433343768\\
13.6767429371697	0.397191246721654\\
13.6950518299906	0.402732510319423\\
13.7133607228114	0.397190603425101\\
13.7316696156322	0.402733567629682\\
13.7499785084531	0.397189971008329\\
13.7682874012739	0.402734605911785\\
13.7865962940948	0.397189349057794\\
13.8049051869156	0.402735625776557\\
13.8232140797365	0.397188737178045\\
13.8415229725573	0.402736627809773\\
13.8598318653781	0.397188134990734\\
13.878140758199	0.40273761257345\\
13.8964496510198	0.397187542133693\\
13.9147585438407	0.402738580607054\\
13.9330674366615	0.397186958260067\\
13.9513763294824	0.40273953242864\\
13.9696852223032	0.397186383037502\\
13.987994115124	0.402740468535916\\
14.0063030079449	0.397185816147379\\
14.0246119007657	0.402741389407257\\
14.0429207935866	0.397185257284099\\
14.0612296864074	0.402742295502645\\
14.0795385792283	0.397184706154409\\
14.0978474720491	0.402743187264569\\
14.1161563648699	0.397184162476769\\
14.1344652576908	0.402744065118862\\
14.1527741505116	0.397183625980754\\
14.1710830433325	0.402744929475501\\
14.1893919361533	0.397183096406494\\
14.2077008289742	0.402745780729349\\
14.226009721795	0.397182573504146\\
14.2443186146159	0.402746619260872\\
14.2626275074367	0.397182057033396\\
14.2809364002575	0.402747445436802\\
14.2992452930784	0.397181546762985\\
14.3175541858992	0.402748259610772\\
14.3358630787201	0.397181042470271\\
14.3541719715409	0.402749062123915\\
14.3724808643618	0.397180543940807\\
14.3907897571826	0.402749853305431\\
14.4090986500034	0.397180050967946\\
14.4274075428243	0.402750633473119\\
14.4457164356451	0.397179563352469\\
14.464025328466	0.402751402933892\\
14.4823342212868	0.397179080902228\\
14.5006431141077	0.402752161984253\\
14.5189520069285	0.397178603431817\\
14.5372608997493	0.402752910910755\\
14.5555697925702	0.397178130762249\\
14.573878685391	0.402753649990435\\
14.5921875782119	0.397177662720662\\
14.6104964710327	0.402754379491222\\
14.6288053638536	0.397177199140032\\
14.6471142566744	0.402755099672329\\
14.6654231494952	0.397176739858903\\
14.6837320423161	0.402755810784627\\
14.7020409351369	0.397176284721136\\
14.7203498279578	0.402756513070994\\
14.7386587207786	0.39717583357566\\
14.7569676135995	0.402757206766654\\
14.7752765064203	0.397175386276248\\
14.7935853992411	0.402757892099493\\
14.811894292062	0.397174942681295\\
14.8302031848828	0.402758569290366\\
14.8485120777037	0.397174502653614\\
14.8668209705245	0.402759238553383\\
14.8851298633454	0.397174066060233\\
14.9034387561662	0.40275990009619\\
14.921747648987	0.39717363277221\\
14.9400565418079	0.402760554120224\\
14.9583654346287	0.397173202664458\\
14.9766743274496	0.402761200820971\\
14.9949832202704	0.397172775615569\\
15.0132921130913	0.4027618403882\\
15.0316010059121	0.397172351507654\\
15.0499098987329	0.402762473006195\\
15.0682187915538	0.39717193022619\\
15.0865276843746	0.402763098853968\\
15.1048365771955	0.397171511659871\\
15.1231454700163	0.402763718105472\\
15.1414543628372	0.397171095700469\\
15.159763255658	0.402764330929795\\
15.1780721484788	0.397170682242698\\
15.1963810412997	0.402764937491354\\
15.2146899341205	0.397170271184089\\
15.2329988269414	0.402765537950074\\
15.2513077197622	0.397169862424863\\
15.2696166125831	0.402766132461561\\
15.2879255054039	0.39716945586782\\
15.3062343982247	0.402766721177272\\
15.3245432910456	0.397169051418223\\
15.3428521838664	0.402767304244671\\
15.3611610766873	0.397168648983692\\
15.3794699695081	0.402767881807382\\
15.397778862329	0.397168248474103\\
15.4160877551498	0.402768454005336\\
15.4343966479706	0.397167849801488\\
15.4527055407915	0.402769020974909\\
15.4710144336123	0.397167452879944\\
15.4893233264332	0.402769582849059\\
15.507632219254	0.39716705762554\\
15.5259411120749	0.402770139757455\\
15.5442500048957	0.397166663956233\\
15.5625588977165	0.402770691826597\\
15.5808677905374	0.397166271791783\\
15.5991766833582	0.402771239179937\\
15.6174855761791	0.397165881053676\\
15.6357944689999	0.402771781937995\\
15.6541033618208	0.397165491665047\\
15.6724122546416	0.402772320218464\\
15.6907211474624	0.397165103550605\\
15.7090300402833	0.402772854136322\\
15.7273389331041	0.397164716636563\\
15.745647825925	0.402773383803924\\
15.7639567187458	0.397164330850573\\
15.7822656115667	0.402773909331109\\
15.8005745043875	0.397163946121656\\
15.8188833972083	0.402774430825289\\
15.8371922900292	0.397163562380142\\
15.85550118285	0.402774948391539\\
15.8738100756709	0.397163179557611\\
15.8921189684917	0.402775462132686\\
15.9104278613126	0.397162797586829\\
15.9287367541334	0.402775972149391\\
15.9470456469542	0.397162416401695\\
15.9653545397751	0.402776478540234\\
15.9836634325959	0.397162035937189\\
16.0019723254168	0.402776981401785\\
16.0202812182376	0.397161656129314\\
16.0385901110585	0.402777480828687\\
16.0568990038793	0.397161276915049\\
16.0752078967001	0.402777976913724\\
16.093516789521	0.397160898232296\\
16.1118256823418	0.402778469747893\\
16.1301345751627	0.397160520019836\\
16.1484434679835	0.402778959420471\\
16.1667523608044	0.397160142217283\\
16.1850612536252	0.402779446019083\\
16.2033701464461	0.397159764765033\\
16.2216790392669	0.402779929629763\\
16.2399879320877	0.397159387604229\\
16.2582968249086	0.402780410337016\\
16.2766057177294	0.397159010676712\\
16.2949146105503	0.402780888223881\\
16.3132235033711	0.397158633924984\\
16.331532396192	0.402781363371983\\
16.3498412890128	0.397158257292164\\
16.3681501818336	0.402781835861595\\
16.3864590746545	0.397157880721954\\
16.4047679674753	0.402782305771689\\
16.4230768602962	0.397157504158598\\
16.441385753117	0.402782773179989\\
16.4596946459379	0.397157127546845\\
16.4780035387587	0.402783238163022\\
16.4963124315795	0.397156750831913\\
16.5146213244004	0.402783700796173\\
16.5329302172212	0.397156373959456\\
16.5512391100421	0.402784161153725\\
16.5695480028629	0.397155996875525\\
16.5878568956838	0.402784619308914\\
16.6061657885046	0.39715561952654\\
16.6244746813254	0.402785075333969\\
16.6427835741463	0.397155241859251\\
16.6610924669671	0.402785529300163\\
16.679401359788	0.39715486382071\\
16.6977102526088	0.40278598127785\\
16.7160191454297	0.397154485358237\\
16.7343280382505	0.402786431336513\\
16.7526369310713	0.397154106419389\\
16.7709458238922	0.402786879544805\\
16.789254716713	0.39715372695193\\
16.8075636095339	0.402787325970587\\
16.8258725023547	0.397153346903799\\
16.8441813951756	0.402787770680971\\
16.8624902879964	0.397152966223082\\
16.8807991808172	0.402788213742358\\
16.8991080736381	0.397152584857981\\
16.9174169664589	0.402788655220474\\
16.9357258592798	0.397152202756786\\
16.9540347521006	0.402789095180413\\
16.9723436449215	0.397151819867845\\
16.9906525377423	0.402789533686668\\
17.0089614305631	0.397151436139536\\
17.027270323384	0.402789970803172\\
17.0455792162048	0.39715105152024\\
17.0638881090257	0.40279040659333\\
17.0821970018465	0.39715066595831\\
17.1005058946674	0.402790841120057\\
17.1188147874882	0.397150279402046\\
17.137123680309	0.402791274445811\\
17.1554325731299	0.397149891799664\\
17.1737414659507	0.402791706632627\\
17.1920503587716	0.397149503099273\\
17.2103592515924	0.402792137742151\\
17.2286681444133	0.397149113248843\\
17.2469770372341	0.402792567835675\\
17.2652859300549	0.39714872219618\\
17.2835948228758	0.402792996974166\\
17.3019037156966	0.397148329888898\\
17.3202126085175	0.402793425218302\\
17.3385215013383	0.397147936274391\\
17.3568303941592	0.402793852628504\\
17.37513928698	0.397147541299805\\
17.3934481798008	0.402794279264965\\
17.4117570726217	0.397147144912013\\
17.4300659654425	0.402794705187686\\
17.4483748582634	0.397146747057585\\
17.4666837510842	0.402795130456504\\
17.4849926439051	0.39714634768276\\
17.5033015367259	0.402795555131128\\
17.5216104295467	0.393641139193881\\
17.5399193223676	0.384486702868095\\
17.5582282151884	0.375332262498408\\
17.5765371080093	0.366177820399075\\
17.5948460008301	0.357023377339741\\
17.613154893651	0.347868933669847\\
17.6314637864718	0.338714489577427\\
17.6497726792926	0.329560045175249\\
17.6680815721135	0.320405600536255\\
17.6863904649343	0.31125115571034\\
17.7046993577552	0.302096710733137\\
17.723008250576	0.292942265630979\\
17.7413171433969	0.283787820423876\\
17.7596260362177	0.274633375127388\\
17.7779349290385	0.265478929753856\\
17.7962438218594	0.256324484313232\\
17.8145527146802	0.247170038813657\\
17.8328616075011	0.238015593261875\\
17.8511705003219	0.228861147663536\\
17.8694793931428	0.219706702023423\\
17.8877882859636	0.210552256345616\\
17.9060971787844	0.201397810633627\\
17.9244060716053	0.192243364890498\\
17.9427149644261	0.183088919118884\\
17.961023857247	0.173934473321114\\
17.9793327500678	0.164780027499243\\
17.9976416428887	0.155625581655093\\
18.0159505357095	0.146471135790287\\
18.0342594285304	0.137316689906278\\
18.0525683213512	0.12816224400437\\
18.070877214172	0.11900779808574\\
18.0891861069929	0.10985335215145\\
18.1074949998137	0.100698906202468\\
18.1258038926346	0.0915444602396723\\
18.1441127854554	0.0823900142638651\\
18.1624216782763	0.0732355682757813\\
18.1807305710971	0.0640811222760952\\
18.1990394639179	0.0549266762654271\\
18.2173483567388	0.0457722302443487\\
18.2356572495596	0.0366177842133885\\
18.2539661423805	0.0274633381730355\\
18.2722750352013	0.0183088921237435\\
18.2905839280222	0.0091544460659342\\
18.308892820843	0\\
};
\addlegendentry{PINS}

\end{axis}
\end{tikzpicture}%%
  \caption{Curvature analysis PINS vs Duboids}
  \label{fig:Compare_curv2}
\end{figure}
%
\begin{figure}[htb!]
  \centering
  
%
\begin{tikzpicture}[scale = 0.7]

\begin{axis}[%
width=0.985\linewidth,
height=\linewidth,
at={(0\linewidth,0\linewidth)},
scale only axis,
xmin=0,
xmax=20,
xlabel style={font=\color{white!15!black}},
xlabel={Time(s)},
ymin=-0.5,
ymax=0.5,
ylabel style={font=\color{white!15!black}},
ylabel={$\kappa(m^{-1})$},
axis background/.style={fill=white},
title style={font=\bfseries},
title={Curvature - PINS},
axis x line*=bottom,
axis y line*=left,
xmajorgrids,
xminorgrids,
ymajorgrids,
yminorgrids,
legend style={legend cell align=left, align=left, draw=white!15!black}
]
\addplot [color=green, dashdotted, line width=2.0pt]
  table[row sep=crcr]{%
0	0.5\\
0.183079627851477	0.5\\
0.366159255702955	0.5\\
0.549238883554432	0.5\\
0.73231851140591	0.5\\
0.915398139257387	0\\
1.09847776710886	0\\
1.28155739496034	0\\
1.46463702281182	0\\
1.6477166506633	0\\
1.83079627851477	0\\
2.01387590636625	0\\
2.19695553421773	0\\
2.38003516206921	0\\
2.56311478992068	0\\
2.74619441777216	0\\
2.92927404562364	0\\
3.11235367347512	0\\
3.29543330132659	0\\
3.47851292917807	0\\
3.66159255702955	0\\
3.84467218488103	0\\
4.0277518127325	0\\
4.21083144058398	0\\
4.39391106843546	0\\
4.57699069628694	0\\
4.76007032413841	0\\
4.94314995198989	0\\
5.12622957984137	0\\
5.30930920769284	0\\
5.49238883554432	0\\
5.6754684633958	0\\
5.85854809124728	0\\
6.04162771909875	0\\
6.22470734695023	0\\
6.40778697480171	0\\
6.59086660265319	0\\
6.77394623050466	0\\
6.95702585835614	0\\
7.14010548620762	0\\
7.3231851140591	0\\
7.50626474191057	0\\
7.68934436976205	0\\
7.87242399761353	-0.5\\
8.05550362546501	-0.5\\
8.23858325331648	-0.5\\
8.42166288116796	-0.5\\
8.60474250901944	-0.5\\
8.78782213687091	0\\
8.97090176472239	0\\
9.15398139257387	0\\
9.33706102042535	0\\
9.52014064827683	0\\
9.7032202761283	0.5\\
9.88629990397978	0.5\\
10.0693795318313	0.5\\
10.2524591596827	0.5\\
10.4355387875342	0.5\\
10.6186184153857	0\\
10.8016980432372	0\\
10.9847776710886	0\\
11.1678572989401	0\\
11.3509369267916	0\\
11.5340165546431	0\\
11.7170961824946	0\\
11.900175810346	0\\
12.0832554381975	0\\
12.266335066049	0\\
12.4494146939005	0\\
12.6324943217519	0\\
12.8155739496034	0\\
12.9986535774549	0\\
13.1817332053064	0\\
13.3648128331579	0\\
13.5478924610093	0\\
13.7309720888608	0\\
13.9140517167123	0\\
14.0971313445638	0\\
14.2802109724152	0\\
14.4632906002667	0\\
14.6463702281182	0\\
14.8294498559697	0\\
15.0125294838211	0\\
15.1956091116726	0\\
15.3786887395241	0\\
15.5617683673756	0\\
15.7448479952271	0\\
15.9279276230785	0\\
16.11100725093	0\\
16.2940868787815	0\\
16.477166506633	0\\
16.6602461344844	0\\
16.8433257623359	0\\
17.0264053901874	0\\
17.2094850180389	0\\
17.3925646458904	0\\
17.5756442737418	-0.5\\
17.7587239015933	-0.5\\
17.9418035294448	-0.5\\
18.1248831572963	-0.5\\
18.3079627851477	-0.5\\
};
\addlegendentry{Duboids}

\addplot [color=dodgerblue, dotted, line width=2.0pt]
  table[row sep=crcr]{%
0	0.4999999811847\\
0.018308892820843	0.499999980962817\\
0.036617785641686	0.499999980508333\\
0.054926678462529	0.499999980031614\\
0.073235571283372	0.499999979530991\\
0.091544464104215	0.499999979004618\\
0.109853356925058	0.499999978450458\\
0.128162249745901	0.499999977866253\\
0.146471142566744	0.499999977249494\\
0.164780035387587	0.499999976597383\\
0.18308892820843	0.499999975906796\\
0.201397821029273	0.499999975174225\\
0.219706713850116	0.499999974395728\\
0.238015606670959	0.49999997356685\\
0.256324499491802	0.499999972682543\\
0.274633392312645	0.499999971737061\\
0.292942285133488	0.499999970723836\\
0.311251177954331	0.499999969635325\\
0.329560070775174	0.499999968462821\\
0.347868963596017	0.499999967196224\\
0.36617785641686	0.499999965823745\\
0.384486749237703	0.49999996433154\\
0.402795642058546	0.499999962703241\\
0.421104534879389	0.499999960919354\\
0.439413427700232	0.499999958956473\\
0.457722320521075	0.499999956786251\\
0.476031213341918	0.499999954374012\\
0.494340106162761	0.499999951676893\\
0.512648998983604	0.49999994864126\\
0.530957891804447	0.499999945199104\\
0.54926678462529	0.499999941262868\\
0.567575677446133	0.499999936717868\\
0.585884570266976	0.499999931410893\\
0.604193463087819	0.499999925132494\\
0.622502355908662	0.499999917588522\\
0.640811248729505	0.499999908352343\\
0.659120141550348	0.499999896780475\\
0.677429034371191	0.499999881854003\\
0.695737927192034	0.499999861855943\\
0.714046820012877	0.499999833643282\\
0.73235571283372	0.499999790752621\\
0.750664605654563	0.499999717300666\\
0.768973498475406	0.499999559630121\\
0.787282391296249	0.499998892401344\\
0.805591284117092	0.0957242068971353\\
0.823900176937935	-1.15974960211429e-05\\
0.842209069758778	1.09065804300146e-05\\
0.860517962579621	-1.16137229809465e-05\\
0.878826855400464	1.08650597353543e-05\\
0.897135748221307	-1.16315804812306e-05\\
0.91544464104215	1.08250071662319e-05\\
0.933753533862993	-1.16510721190344e-05\\
0.952062426683836	1.0786413734365e-05\\
0.970371319504679	-1.16722022924232e-05\\
0.988680212325522	1.0749271172894e-05\\
1.00698910514636	-1.16949763328267e-05\\
1.02529799796721	1.07135719802087e-05\\
1.04360689078805	-1.17194003847465e-05\\
1.06191578360889	1.06793094109825e-05\\
1.08022467642974	-1.17454814967943e-05\\
1.09853356925058	1.06464774671799e-05\\
1.11684246207142	-1.17732276014026e-05\\
1.13515135489227	1.06150709067721e-05\\
1.15346024771311	-1.18026475000588e-05\\
1.17176914053395	1.05850852168132e-05\\
1.19007803335479	-1.18337509018851e-05\\
1.20838692617564	1.05565166692856e-05\\
1.22669581899648	-1.18665484384883e-05\\
1.24500471181732	1.05293622588432e-05\\
1.26331360463817	-1.19010516435591e-05\\
1.28162249745901	1.05036197469432e-05\\
1.29993139027985	-1.19372730032485e-05\\
1.3182402831007	1.04792876406412e-05\\
1.33654917592154	-1.19752259004069e-05\\
1.35485806874238	1.0456365242828e-05\\
1.37316696156322	-1.20149247417878e-05\\
1.39147585438407	1.0434852577873e-05\\
1.40978474720491	-1.20563848612365e-05\\
1.42809364002575	1.04147504889074e-05\\
1.4464025328466	-1.20996225936587e-05\\
1.46471142566744	1.03960605579434e-05\\
1.48302031848828	-1.21446552965865e-05\\
1.50132921130913	1.03787851793435e-05\\
1.51963810412997	-1.21915013298612e-05\\
1.53794699695081	1.03629275449435e-05\\
1.55625588977165	-1.22401801185001e-05\\
1.5745647825925	1.03484916291474e-05\\
1.59287367541334	-1.22907121684057e-05\\
1.61118256823418	1.0335482232976e-05\\
1.62949146105503	-1.23431190403311e-05\\
1.64780035387587	1.0323905016818e-05\\
1.66610924669671	-1.23974234252366e-05\\
1.68441813951756	1.03137664438369e-05\\
1.7027270323384	-1.24536491658001e-05\\
1.72103592515924	1.03050738542998e-05\\
1.73934481798008	-1.2511821248895e-05\\
1.75765371080093	1.02978354387662e-05\\
1.77596260362177	-1.25719658929102e-05\\
1.79427149644261	1.02920602827183e-05\\
1.81258038926346	-1.2634110510612e-05\\
1.8308892820843	1.02877583995353e-05\\
1.84919817490514	-1.26982837969358e-05\\
1.86750706772599	1.02849406741212e-05\\
1.88581596054683	-1.2764515697844e-05\\
1.90412485336767	1.02836189971867e-05\\
1.92243374618851	-1.28328375223474e-05\\
1.94074263900936	1.02838061553367e-05\\
1.9590515318302	-1.2903281941784e-05\\
1.97736042465104	1.02855159895832e-05\\
1.99566931747189	-1.29758830005877e-05\\
2.01397821029273	1.02887632856541e-05\\
2.03228710311357	-1.30506762291982e-05\\
2.05059599593442	1.02935639205437e-05\\
2.06890488875526	-1.31276986020112e-05\\
2.0872137815761	1.02999347890709e-05\\
2.10552267439694	-1.32069886563391e-05\\
2.12383156721779	1.03078939089896e-05\\
2.14214046003863	-1.32885864985999e-05\\
2.16044935285947	1.0317460413134e-05\\
2.17875824568032	-1.33725338642698e-05\\
2.19706713850116	1.0328654577757e-05\\
2.215376031322	-1.34588741602937e-05\\
2.23368492414285	1.0341497886257e-05\\
2.25199381696369	-1.35476525555955e-05\\
2.27030270978453	1.03560130339242e-05\\
2.28861160260537	-1.36389159822725e-05\\
2.30692049542622	1.03722239983561e-05\\
2.32522938824706	-1.37327132863907e-05\\
2.3435382810679	1.03901560204733e-05\\
2.36184717388875	-1.38290951583186e-05\\
2.38015606670959	1.04098357351923e-05\\
2.39846495953043	-1.39281143499426e-05\\
2.41677385235128	1.04312911169691e-05\\
2.43508274517212	-1.40298256176841e-05\\
2.45339163799296	1.04545516528554e-05\\
2.4717005308138	-1.41342858629434e-05\\
2.49000942363465	1.04796482527647e-05\\
2.50831831645549	-1.42415542313534e-05\\
2.52662720927633	1.05066133871956e-05\\
2.54493610209718	-1.43516921167208e-05\\
2.56324499491802	1.05354811473501e-05\\
2.58155388773886	-1.44647633091854e-05\\
2.59986278055971	1.05662872522394e-05\\
2.61817167338055	-1.45808340897835e-05\\
2.63648056620139	1.05990691632585e-05\\
2.65478945902223	-1.46999732481001e-05\\
2.67309835184308	1.0633866097759e-05\\
2.69140724466392	-1.48222523039809e-05\\
2.70971613748476	1.06707191508681e-05\\
2.72802503030561	-1.49477454963742e-05\\
2.74633392312645	1.07096713512778e-05\\
2.76464281594729	-1.50765299403999e-05\\
2.78295170876814	1.07507677122864e-05\\
2.80126060158898	-1.52086857735656e-05\\
2.81956949440982	1.07940553307484e-05\\
2.83787838723066	-1.53442962425587e-05\\
2.85618728005151	1.08395834685637e-05\\
2.87449617287235	-1.54834478209021e-05\\
2.89280506569319	1.08874036950923e-05\\
2.91111395851404	-1.56262303515342e-05\\
2.92942285133488	1.09375698797709e-05\\
2.94773174415572	-1.57727372324112e-05\\
2.96604063697657	1.0990138365502e-05\\
2.98434952979741	-1.59230654889764e-05\\
3.00265842261825	1.1045168082896e-05\\
3.02096731543909	-1.60773160039762e-05\\
3.03927620825994	1.11027205988157e-05\\
3.05758510108078	-1.62355936341174e-05\\
3.07589399390162	1.11628602922909e-05\\
3.09420288672247	-1.63980073881465e-05\\
3.11251177954331	1.12256544768097e-05\\
3.13082067236415	-1.65646706485889e-05\\
3.149129565185	1.12911734935217e-05\\
3.16743845800584	-1.67357013160507e-05\\
3.18574735082668	1.13594908717762e-05\\
3.20405624364752	-1.6911222052246e-05\\
3.22236513646837	1.14306835204137e-05\\
3.24067402928921	-1.70913604574108e-05\\
3.25898292211005	1.15048317820277e-05\\
3.2772918149309	-1.72762493404754e-05\\
3.29560070775174	1.15820197048311e-05\\
3.31390960057258	-1.74660269126603e-05\\
3.33221849339342	1.16623351655576e-05\\
3.35052738621427	-1.76608370973108e-05\\
3.36883627903511	1.17458700658324e-05\\
3.38714517185595	-1.78608297151106e-05\\
3.4054540646768	1.18327204875202e-05\\
3.42376295749764	-1.80661608774058e-05\\
3.44207185031848	1.19229869769422e-05\\
3.46038074313933	-1.82769931637017e-05\\
3.47868963596017	1.20167747275357e-05\\
3.49699852878101	-1.84934960197336e-05\\
3.51530742160186	1.21141938130842e-05\\
3.5336163144227	-1.87158460615844e-05\\
3.55192520724354	1.22153593980212e-05\\
3.57023410006438	-1.89442274493024e-05\\
3.58854299288523	1.23203920837922e-05\\
3.60685188570607	-1.91788322209396e-05\\
3.62516077852691	1.24294181052809e-05\\
3.64346967134776	-1.9419860739639e-05\\
3.6617785641686	1.25425696998194e-05\\
3.68008745698944	-1.96675220653642e-05\\
3.69839634981029	1.26599853690901e-05\\
3.71670524263113	-1.99220344334339e-05\\
3.73501413545197	1.27818102232669e-05\\
3.75332302827281	-2.01836257194832e-05\\
3.77163192109366	1.29081963640976e-05\\
3.7899408139145	-2.04525339149164e-05\\
3.80824970673534	1.30393032591603e-05\\
3.82655859955619	-2.072900771663e-05\\
3.84486749237703	1.31752980649935e-05\\
3.86317638519787	-2.10133070371599e-05\\
3.88148527801871	1.33163562093808e-05\\
3.89979417083956	-2.13057036235753e-05\\
3.9181030636604	1.34626617158962e-05\\
3.93641195648124	-2.16064816951911e-05\\
3.95472084930209	1.36144077339806e-05\\
3.97302974212293	-2.19159386486423e-05\\
3.99133863494377	1.37717970919715e-05\\
4.00964752776462	-2.22343857214646e-05\\
4.02795642058546	1.39350428187135e-05\\
4.0462653134063	-2.25621488405547e-05\\
4.06457420622715	1.41043687255094e-05\\
4.08288309904799	-2.28995693812295e-05\\
4.10119199186883	1.42800100617069e-05\\
4.11950088468967	-2.3247005070004e-05\\
4.13780977751052	1.44622142132789e-05\\
4.15611867033136	-2.36048309032455e-05\\
4.1744275631522	1.46512413472522e-05\\
4.19273645597305	-2.3973440187175e-05\\
4.21104534879389	1.48473653092118e-05\\
4.22935424161473	-2.43532455154738e-05\\
4.24766313443557	1.50508743207156e-05\\
4.26597202725642	-2.4744680011457e-05\\
4.28428092007726	1.52620719676699e-05\\
4.3025898128981	-2.5148198461139e-05\\
4.32089870571895	1.54812781013591e-05\\
4.33920759853979	-2.55642786761712e-05\\
4.35751649136063	1.57088298726182e-05\\
4.37582538418148	-2.59934228363234e-05\\
4.39413427700232	1.59450828457752e-05\\
4.41244316982316	-2.64361590222584e-05\\
4.430752062644	1.61904121319945e-05\\
4.44906095546485	-2.68930428165282e-05\\
4.46736984828569	1.6445213727706e-05\\
4.48567874110653	-2.73646589695198e-05\\
4.50398763392738	1.67099058134001e-05\\
4.52229652674822	-2.78516232977422e-05\\
4.54060541956906	1.69849302482372e-05\\
4.55891431238991	-2.83545846758715e-05\\
4.57722320521075	1.72707541206962e-05\\
4.59553209803159	-2.8874227105874e-05\\
4.61384099085243	1.75678715117755e-05\\
4.63214988367328	-2.94112720574669e-05\\
4.65045877649412	1.78768052557232e-05\\
4.66876766931496	-2.99664809231825e-05\\
4.68707656213581	1.819810900483e-05\\
4.70538545495665	-3.05406576456557e-05\\
4.72369434777749	1.85323692783457e-05\\
4.74200324059834	-3.11346516897737e-05\\
4.76031213341918	1.88802078394112e-05\\
4.77862102624002	-3.17493609844899e-05\\
4.79692991906087	1.92422842237028e-05\\
4.81523881188171	-3.23857353671519e-05\\
4.83354770470255	1.96192983549781e-05\\
4.85185659752339	-3.304478019997e-05\\
4.87016549034424	2.00119935810095e-05\\
4.88847438316508	-3.37275602206588e-05\\
4.90678327598592	2.04211598111015e-05\\
4.92509216880677	-3.44352037887352e-05\\
4.94340106162761	2.08476370345545e-05\\
4.96170995444845	-3.51689075172024e-05\\
4.9800188472693	2.12923190013592e-05\\
4.99832774009014	-3.59299412218961e-05\\
5.01663663291098	2.17561573767067e-05\\
5.03494552573182	-3.67196532921332e-05\\
5.05325441855267	2.22401661658378e-05\\
5.07156331137351	-3.75394766173875e-05\\
5.08987220419435	2.2745426622256e-05\\
5.1081810970152	-3.83909349192524e-05\\
5.12648998983604	2.32730924682456e-05\\
5.14479888265688	-3.92756497522295e-05\\
5.16310777547772	2.38243957507711e-05\\
5.18141666829857	-4.01953480754225e-05\\
5.19972556111941	2.44006531004703e-05\\
5.21803445394025	-4.11518704612757e-05\\
5.2363433467611	2.50032727275584e-05\\
5.25465223958194	-4.21471802564688e-05\\
5.27296113240278	2.56337619055635e-05\\
5.29127002522362	-4.31833733382236e-05\\
5.30957891804447	2.62937353633641e-05\\
5.32788781086531	-4.42626889813613e-05\\
5.34619670368615	2.69849243788312e-05\\
5.364505596507	-4.53875217506505e-05\\
5.38281448932784	2.77091867852131e-05\\
5.40112338214868	-4.65604343984349e-05\\
5.41943227496953	2.84685180367839e-05\\
5.43774116779037	-4.77841722969774e-05\\
5.45605006061121	2.92650633312019e-05\\
5.47435895343206	-4.90616790341791e-05\\
5.4926678462529	3.01011310569199e-05\\
5.51097673907374	-5.03961137838238e-05\\
5.52928563189458	3.09792075667559e-05\\
5.54759452471543	-5.17908703756842e-05\\
5.56590341753627	3.19019735754711e-05\\
5.58421231035711	-5.32495983760484e-05\\
5.60252120317796	3.28723223352589e-05\\
5.6208300959988	-5.47762263748697e-05\\
5.63913898881964	3.38933796743546e-05\\
5.65744788164049	-5.63749878352837e-05\\
5.67575677446133	3.49685263992261e-05\\
5.69406566728217	-5.80504496697221e-05\\
5.71237456010301	3.61014231296808e-05\\
5.73068345292386	-5.98075440463031e-05\\
5.7489923457447	3.72960379965703e-05\\
5.76730123856554	-6.16516037053039e-05\\
5.78561013138639	3.85566774704915e-05\\
5.80391902420723	-6.35884013866495e-05\\
5.82222791702807	3.98880209408481e-05\\
5.84053680984892	-6.56241937569135e-05\\
5.85884570266976	4.12951592628774e-05\\
5.8771545954906	-6.77657705032142e-05\\
5.89546348831144	4.27836381328706e-05\\
5.91377238113229	-7.00205093416562e-05\\
5.93208127395313	4.43595066815239e-05\\
5.95039016677397	-7.23964376767305e-05\\
5.96869905959482	4.60293721932514e-05\\
5.98700795241566	-7.4902302008184e-05\\
6.0053168452365	4.78004617965322e-05\\
6.02362573805735	-7.7547645869841e-05\\
6.04193463087819	4.96806921309045e-05\\
6.06024352369903	-8.03428978996545e-05\\
6.07855241651987	5.16787482162939e-05\\
6.09686130934072	-8.32994713709234e-05\\
6.11517020216156	5.38041727804195e-05\\
6.1334790949824	-8.64298769561178e-05\\
6.15178798780325	5.60674679220274e-05\\
6.17009688062409	-8.97478506341476e-05\\
6.18840577344493	5.84802107911897e-05\\
6.20671466626578	-9.32684993867061e-05\\
6.22502355908662	6.10551857094799e-05\\
6.24333245190746	-9.70084671901017e-05\\
6.2616413447283	6.38065352125128e-05\\
6.27995023754915	-0.000100986124963731\\
6.29825913036999	6.67499335335331e-05\\
6.31656802319083	-0.000105221788129883\\
6.33487691601168	6.99027857720969e-05\\
6.35318580883252	-0.000109737966730583\\
6.37149470165336	7.32844578550662e-05\\
6.3898035944742	-0.000114559653579122\\
6.40811248729505	7.69165421599383e-05\\
6.42642138011589	-0.000119714657230513\\
6.44473027293673	8.08231656703462e-05\\
6.46303916575758	-0.000125233987997408\\
6.48134805857842	8.50313481408904e-05\\
6.49965695139926	-0.000131152306581644\\
6.51796584422011	8.95714201216702e-05\\
6.53627473704095	-0.000137508447489998\\
6.55458362986179	9.44775119825214e-05\\
6.57289252268264	-0.000144346031479115\\
6.59120141550348	9.97881282622792e-05\\
6.60951030832432	-0.000151714184955354\\
6.62781920114516	0.000105546824480884\\
6.64612809396601	-0.000159668387863798\\
6.66443698678685	0.000111803007784878\\
6.68274587960769	-0.000168271477184762\\
6.70105477242854	0.000118612887525099\\
6.71936366524938	-0.000177594839372713\\
6.73767255807022	0.000126040608716377\\
6.75598145089107	-0.000187719833367184\\
6.77429034371191	0.000134159609413986\\
6.79259923653275	-0.000198739496525963\\
6.81090812935359	0.000143054253388786\\
6.82921702217444	-0.000210760599665516\\
6.84752591499528	0.000152821803724945\\
6.86583480781612	-0.000223906135233348\\
6.88414370063697	0.000163574820275791\\
6.90245259345781	-0.000238318346244659\\
6.92076148627865	0.000175444088128113\\
6.93907037909949	-0.000254162434760868\\
6.95737927192034	0.000188582215016081\\
6.97568816474118	-0.000271631130424416\\
6.99399705756202	0.000203168077463206\\
7.01230595038287	-0.000290950355608111\\
7.03061484320371	0.000219412352096549\\
7.04892373602455	-0.000312386299997996\\
7.0672326288454	0.000237564445115634\\
7.08554152166624	-0.00033625432301912\\
7.10385041448708	0.000257921238791231\\
7.12215930730793	-0.00036293024854045\\
7.14046820012877	0.000280838222070756\\
7.15877709294961	-0.000392864823342137\\
7.17708598577045	0.000306743779851182\\
7.1953948785913	-0.000426602406846721\\
7.21370377141214	0.000336157715109947\\
7.23201266423298	-0.00046480538965879\\
7.25032155705383	0.000369715511593233\\
7.26863044987467	-0.000508286475127878\\
7.28693934269551	0.000408200486749527\\
7.30524823551636	-0.000558051917462082\\
7.3235571283372	0.000452586953143513\\
7.34186602115804	-0.000615360287212791\\
7.36017491397888	0.000504098995402713\\
7.37848380679973	-0.000681803664764363\\
7.39679269962057	0.000564291817819973\\
7.41510159244141	-0.000759421936616461\\
7.43341048526226	0.00063516641797251\\
7.4517193780831	-0.000850867172348169\\
7.47002827090394	0.000719334683215916\\
7.48833716372478	-0.00095964596270795\\
7.50664605654563	0.000820262964126184\\
7.52495494936647	-0.00109048723427646\\
7.54326384218731	0.000942641894647989\\
7.56157273500816	-0.00124992012585086\\
7.579881627829	0.0010929674128683\\
7.59819052064984	-0.0014472205177844\\
7.61649941347069	0.00128049210708432\\
7.63480830629153	-0.0016960427910583\\
7.65311719911237	0.00151886423075558\\
7.67142609193322	-0.00201741955660481\\
7.68973498475406	0.00182913822247163\\
7.7080438775749	-0.00244575330146994\\
7.72635277039574	0.00224578223836493\\
7.74466166321659	-0.00304219367762401\\
7.76297055603743	0.00283007867192381\\
7.78127944885827	-0.00392961859234081\\
7.79958834167912	0.00370513928051314\\
7.81789723449996	-0.00540963986593267\\
7.8362061273208	0.00517296245563699\\
7.85451502014164	-0.00858463850780507\\
7.87282391296249	0.00833656318983776\\
7.89113280578333	-0.172281676606298\\
7.90944169860417	-0.353922475599984\\
7.92775059142502	-0.499869272030161\\
7.94605948424586	-0.499923122897157\\
7.9643683770667	-0.499943098504611\\
7.98267726988755	-0.4999535707693\\
8.00098616270839	-0.499959955461847\\
8.01929505552923	-0.499964185538236\\
8.03760394835007	-0.499967129474669\\
8.05591284117092	-0.499969236203897\\
8.07422173399176	-0.499970760831107\\
8.0925306268126	-0.499971858413473\\
8.11083951963345	-0.49997262788962\\
8.12914841245429	-0.499973134573815\\
8.14745730527513	-0.499973422489781\\
8.16576619809598	-0.499973521503827\\
8.18407509091682	-0.499973451622751\\
8.20238398373766	-0.49997322566048\\
8.22069287655851	-0.499972850919287\\
8.23900176937935	-0.499972330246866\\
8.25731066220019	-0.499971662677605\\
8.27561955502103	-0.499970843780081\\
8.29392844784188	-0.499969865781549\\
8.31223734066272	-0.499968717507899\\
8.33054623348356	-0.49996738415544\\
8.34885512630441	-0.499965846894062\\
8.36716401912525	-0.499964082286606\\
8.38547291194609	-0.499962061494405\\
8.40378180476694	-0.499959749221942\\
8.42209069758778	-0.499957102332343\\
8.44039959040862	-0.499954068037204\\
8.45870848322947	-0.49995058152556\\
8.47701737605031	-0.499946562842086\\
8.49532626887115	-0.499941912745927\\
8.51363516169199	-0.499936507166239\\
8.53194405451284	-0.499930189698649\\
8.55025294733368	-0.499922761326244\\
8.56856184015452	-0.499913966146141\\
8.58687073297537	-0.499903471248135\\
8.60517962579621	-0.49989083786954\\
8.62348851861705	-0.499875479262083\\
8.64179741143789	-0.499856597841555\\
8.66010630425874	-0.499833089175892\\
8.67841519707958	-0.499803391274009\\
8.69672408990042	-0.499765240465865\\
8.71503298272127	-0.499715261160592\\
8.73334187554211	-0.499648245537038\\
8.75165076836295	-0.499555819504728\\
8.76995966118379	-0.49942380179311\\
8.78826855400464	-0.499226506249443\\
8.80657744682548	-0.498912935401317\\
8.82488633964632	-0.498367241171375\\
8.84319523246717	-0.497260922024524\\
8.86150412528801	-0.494119471860717\\
8.87981301810885	-0.324064648592688\\
8.8981219109297	0.165452617798257\\
8.91643080375054	0.489728166002089\\
8.93473969657138	0.492773991941534\\
8.95304858939222	0.493502607050947\\
8.97135748221307	0.493437484256943\\
8.98966637503391	0.492834059292257\\
9.00797526785475	0.491664683332311\\
9.0262841606756	0.489724752642387\\
9.04459305349644	0.486572485410473\\
9.06290194631728	0.48132218988981\\
9.08121083913813	0.47212650321158\\
9.09951973195897	0.454749567399508\\
9.11782862477981	0.417311349260988\\
9.13613751760066	0.307449456322184\\
9.1544464104215	-3.53883589099269e-15\\
9.17275530324234	-0.307449456322188\\
9.19106419606318	-0.417311349260987\\
9.20937308888403	-0.454749567399507\\
9.22768198170487	-0.47212650321158\\
9.24599087452571	-0.481322189889809\\
9.26429976734656	-0.486572485410472\\
9.2826086601674	-0.489724752642387\\
9.30091755298824	-0.49166468333231\\
9.31922644580908	-0.492834059292256\\
9.33753533862993	-0.493437484256942\\
9.35584423145077	-0.493502607050947\\
9.37415312427161	-0.492773991941533\\
9.39246201709246	-0.489728166002089\\
9.4107709099133	-0.165452617798253\\
9.42907980273414	0.324064648592692\\
9.44738869555499	0.494119471860717\\
9.46569758837583	0.497260922024524\\
9.48400648119667	0.498367241171375\\
9.50231537401752	0.498912935401317\\
9.52062426683836	0.499226506249443\\
9.5389331596592	0.49942380179311\\
9.55724205248004	0.499555819504728\\
9.57555094530089	0.499648245537038\\
9.59385983812173	0.499715261160592\\
9.61216873094257	0.499765240465865\\
9.63047762376342	0.499803391274009\\
9.64878651658426	0.499833089175892\\
9.6670954094051	0.499856597841555\\
9.68540430222595	0.499875479262083\\
9.70371319504679	0.49989083786954\\
9.72202208786763	0.499903471248135\\
9.74033098068847	0.499913966146141\\
9.75863987350932	0.499922761326244\\
9.77694876633016	0.499930189698649\\
9.795257659151	0.499936507166239\\
9.81356655197185	0.499941912745927\\
9.83187544479269	0.499946562842086\\
9.85018433761353	0.49995058152556\\
9.86849323043437	0.499954068037204\\
9.88680212325522	0.499957102332343\\
9.90511101607606	0.499959749221942\\
9.9234199088969	0.499962061494405\\
9.94172880171775	0.499964082286606\\
9.96003769453859	0.499965846894062\\
9.97834658735943	0.49996738415544\\
9.99665548018028	0.499968717507899\\
10.0149643730011	0.499969865781549\\
10.033273265822	0.499970843780081\\
10.0515821586428	0.499971662677605\\
10.0698910514636	0.499972330246866\\
10.0881999442845	0.499972850919287\\
10.1065088371053	0.49997322566048\\
10.1248177299262	0.499973451622751\\
10.143126622747	0.499973521503827\\
10.1614355155679	0.499973422489781\\
10.1797444083887	0.499973134573815\\
10.1980533012095	0.49997262788962\\
10.2163621940304	0.499971858413473\\
10.2346710868512	0.499970760831107\\
10.2529799796721	0.499969236203897\\
10.2712888724929	0.499967129474669\\
10.2895977653138	0.499964185538236\\
10.3079066581346	0.499959955461847\\
10.3262155509554	0.4999535707693\\
10.3445244437763	0.499943098504611\\
10.3628333365971	0.499923122897157\\
10.381142229418	0.499869272030161\\
10.3994511222388	0.353922475599955\\
10.4177600150597	0.172281676606337\\
10.4360689078805	-0.00833656318985601\\
10.4543778007013	0.0085846385078\\
10.4726866935222	-0.00517296245563378\\
10.490995586343	0.00540963986592922\\
10.5093044791639	-0.0037051392805228\\
10.5276133719847	0.00392961859233197\\
10.5459222648056	-0.00283007867193306\\
10.5642311576264	0.00304219367762694\\
10.5825400504472	-0.00224578223835215\\
10.6008489432681	0.00244575330147322\\
10.6191578360889	-0.00182913822246496\\
10.6374667289098	0.00201741955660818\\
10.6557756217306	-0.00151886423075427\\
10.6740845145515	0.00169604279105683\\
10.6923934073723	-0.00128049210709272\\
10.7107023001932	0.00144722051777833\\
10.729011193014	-0.00109296741287646\\
10.7473200858348	0.00124992012584507\\
10.7656289786557	-0.000942641894651486\\
10.7839378714765	0.00109048723427976\\
10.8022467642974	-0.000820262964125268\\
10.8205556571182	0.000959645962711198\\
10.8388645499391	-0.000719334683210837\\
10.8571734427599	0.000850867172343034\\
10.8754823355807	-0.000635166417988553\\
10.8937912284016	0.000759421936615517\\
10.9121001212224	-0.000564291817819251\\
10.9304090140433	0.000681803664763503\\
10.9487179068641	-0.000504098995402075\\
10.967026799685	0.00061536028721601\\
10.9853356925058	-0.000452586953142931\\
11.0036445853266	0.000558051917461333\\
11.0219534781475	-0.000408200486745086\\
11.0402623709683	0.000508286475135039\\
11.0585712637892	-0.000369715511584989\\
11.07688015661	0.00046480538966584\\
11.0951890494309	-0.000336157715109503\\
11.1134979422517	0.000426602406853743\\
11.1318068350725	-0.000306743779835611\\
11.1501157278934	0.000392864823349187\\
11.1684246207142	-0.000280838222062874\\
11.1867335135351	0.000362930248540005\\
11.2050424063559	-0.000257921238798309\\
11.2233512991768	0.000336254323018648\\
11.2416601919976	-0.000237564445107974\\
11.2599690848184	0.000312386300004935\\
11.2782779776393	-0.000219412352088971\\
11.2965868704601	0.000290950355614994\\
11.314895763281	-0.00020316807744844\\
11.3332046561018	0.000271631130438543\\
11.3515135489227	-0.000188582215008615\\
11.3698224417435	0.000254162434760591\\
11.3881313345643	-0.000175444088135024\\
11.4064402273852	0.000238318346237248\\
11.424749120206	-0.000163574820282647\\
11.4430580130269	0.000223906135225965\\
11.4613669058477	-0.000152821803731856\\
11.4796757986686	0.000210760599665183\\
11.4979846914894	-0.000143054253388619\\
11.5162935843102	0.000198739496525685\\
11.5346024771311	-0.00013415960941382\\
11.5529113699519	0.000187719833366906\\
11.5712202627728	-0.00012604060871621\\
11.5895291555936	0.000177594839379402\\
11.6078380484145	-0.000118612887511138\\
11.6261469412353	0.000168271477184567\\
11.6444558340561	-0.000111803007784711\\
11.662764726877	0.00015966838787046\\
11.6810736196978	-0.000105546824480718\\
11.6993825125187	0.000151714184968815\\
11.7176914053395	-9.97881282485125e-05\\
11.7360002981604	0.000144346031478892\\
11.7543091909812	-9.44775119959829e-05\\
11.772618083802	0.000137508447476259\\
11.7909269766229	-8.95714201283038e-05\\
11.8092358694437	0.00013115230658145\\
11.8275447622646	-8.50313481407516e-05\\
11.8458536550854	0.000125233988003959\\
11.8641625479063	-8.08231656568015e-05\\
11.8824714407271	0.000119714657243752\\
11.9007803335479	-7.69165421531104e-05\\
11.9190892263688	0.000114559653585672\\
11.9373981191896	-7.32844578416048e-05\\
11.9557070120105	0.000109737966750401\\
11.9740159048313	-6.99027857653245e-05\\
11.9923247976522	0.000105221788136378\\
12.010633690473	-6.67499335201549e-05\\
12.0289425832938	0.000100986124970198\\
12.0472514761147	-6.38065352190353e-05\\
12.0655603689355	9.70084671899352e-05\\
12.0838692617564	-6.10551857027908e-05\\
12.1021781545772	9.32684993931177e-05\\
12.1204870473981	-5.84802107911064e-05\\
12.1387959402189	8.97478506405869e-05\\
12.1571048330397	-5.6067467908788e-05\\
12.1754137258606	8.64298769625294e-05\\
12.1937226186814	-5.38041727737859e-05\\
12.2120315115023	8.32994713642898e-05\\
12.2303404043231	-5.16787482227055e-05\\
12.248649297144	8.03428978864984e-05\\
12.2669581899648	-4.9680692130849e-05\\
12.2852670827857	7.75476458762248e-05\\
12.3035759756065	-4.78004617834593e-05\\
12.3218848684273	7.49023020211181e-05\\
12.3401937612482	-4.60293721801786e-05\\
12.358502654069	7.23964376831421e-05\\
12.3768115468899	-4.43595066749736e-05\\
12.3951204397107	7.00205093351058e-05\\
12.4134293325316	-4.2783638132815e-05\\
12.4317382253524	6.77657705031587e-05\\
12.4500471181732	-4.12951592628219e-05\\
12.4683560109941	6.56241937632973e-05\\
12.4866649038149	-3.98880209343255e-05\\
12.5049737966358	6.35884013930332e-05\\
12.5232826894566	-3.85566774704638e-05\\
12.5415915822775	6.16516037116321e-05\\
12.5599004750983	-3.72960379901033e-05\\
12.5782093679191	5.98075440654822e-05\\
12.59651826074	-3.61014231168022e-05\\
12.6148271535608	5.80504496632273e-05\\
12.6331360463817	-3.49685264056099e-05\\
12.6514449392025	5.63749878288167e-05\\
12.6697538320234	-3.38933796742991e-05\\
12.6880627248442	5.47762263747864e-05\\
12.706371617665	-3.28723223288474e-05\\
12.7246805104859	5.32495983760206e-05\\
12.7429894033067	-3.19019735817716e-05\\
12.7612982961276	5.17908703756009e-05\\
12.7796071889484	-3.09792075667559e-05\\
12.7979160817693	5.03961137837405e-05\\
12.8162249745901	-3.01011310505361e-05\\
12.8345338674109	4.90616790468357e-05\\
12.8528427602318	-2.92650633248182e-05\\
12.8711516530526	4.77841722968941e-05\\
12.8894605458735	-2.84685180430844e-05\\
12.9077694386943	4.65604343983794e-05\\
12.9260783315152	-2.77091867851853e-05\\
12.944387224336	4.5387521750595e-05\\
12.9626961171568	-2.69849243851039e-05\\
12.9810050099777	4.42626889813336e-05\\
12.9993139027985	-2.62937353696369e-05\\
13.0176227956194	4.31833733318399e-05\\
13.0359316884402	-2.56337619055358e-05\\
13.0542405812611	4.2147180262686e-05\\
13.0725494740819	-2.50032727212302e-05\\
13.0908583669027	4.11518704674929e-05\\
13.1091672597236	-2.44006531004703e-05\\
13.1274761525444	4.01953480753392e-05\\
13.1457850453653	-2.38243957444428e-05\\
13.1640939381861	3.92756497584468e-05\\
13.182402831007	-2.32730924807634e-05\\
13.2007117238278	3.83909349191969e-05\\
13.2190206166486	-2.27454266284732e-05\\
13.2373295094695	3.75394766110593e-05\\
13.2556384022903	-2.22401661783278e-05\\
13.2739472951112	3.6719653285805e-05\\
13.292256187932	-2.17561573704061e-05\\
13.3105650807529	3.59299412218406e-05\\
13.3288739735737	-2.12923190075487e-05\\
13.3471828663945	3.51689075108741e-05\\
13.3654917592154	-2.08476370407995e-05\\
13.3838006520362	3.44352037886797e-05\\
13.4021095448571	-2.0421159817291e-05\\
13.4204184376779	3.37275602143583e-05\\
13.4387273304988	-2.00119935685195e-05\\
13.4570362233196	3.3044780218594e-05\\
13.4753451161404	-1.96192983362709e-05\\
13.4936540089613	3.23857353795309e-05\\
13.5119629017821	-1.924228421743e-05\\
13.530271794603	3.17493609782449e-05\\
13.5485806874238	-1.88802078517902e-05\\
13.5668895802447	3.11346516896904e-05\\
13.5851984730655	-1.85323692721284e-05\\
13.6035073658863	3.05406576580347e-05\\
13.6218162587072	-1.81981090047745e-05\\
13.640125151528	2.99664809169098e-05\\
13.6584340443489	-1.78768052556955e-05\\
13.6767429371697	2.94112720574113e-05\\
13.6950518299906	-1.75678715055583e-05\\
13.7133607228114	2.88742271057907e-05\\
13.7316696156322	-1.72707541206962e-05\\
13.7499785084531	2.8354584675816e-05\\
13.7682874012739	-1.69849302420477e-05\\
13.7865962940948	2.78516233038484e-05\\
13.8049051869156	-1.67099058195896e-05\\
13.8232140797365	2.73646589570853e-05\\
13.8415229725573	-1.64452137338955e-05\\
13.8598318653781	2.68930428102832e-05\\
13.878140758199	-1.61904121319667e-05\\
13.8964496510198	2.64361590222029e-05\\
13.9147585438407	-1.59450828457475e-05\\
13.9330674366615	2.59934228301339e-05\\
13.9513763294824	-1.57088298725905e-05\\
13.9696852223032	2.55642786761157e-05\\
13.987994115124	-1.54812781013591e-05\\
14.0063030079449	2.51481984611113e-05\\
14.0246119007657	-1.52620719676422e-05\\
14.0429207935866	2.47446800114015e-05\\
14.0612296864074	-1.50508743268218e-05\\
14.0795385792283	2.43532455092843e-05\\
14.0978474720491	-1.48473653092396e-05\\
14.1161563648699	2.39734401809577e-05\\
14.1344652576908	-1.46512413533584e-05\\
14.1527741505116	2.36048309032455e-05\\
14.1710830433325	-1.44622142193573e-05\\
14.1893919361533	2.3247005070004e-05\\
14.2077008289742	-1.42800100616514e-05\\
14.226009721795	2.28995693873635e-05\\
14.2443186146159	-1.41043687254538e-05\\
14.2626275074367	2.2562148840527e-05\\
14.2809364002575	-1.3935042818658e-05\\
14.2992452930784	2.22343857153307e-05\\
14.3175541858992	-1.37717970980777e-05\\
14.3358630787201	2.19159386363743e-05\\
14.3541719715409	-1.36144077400591e-05\\
14.3724808643618	2.16064816951633e-05\\
14.3907897571826	-1.34626617158684e-05\\
14.4090986500034	2.13057036174413e-05\\
14.4274075428243	-1.33163562154592e-05\\
14.4457164356451	2.10133070309981e-05\\
14.464025328466	-1.3175298077206e-05\\
14.4823342212868	2.0729007710496e-05\\
14.5006431141077	-1.3039303253054e-05\\
14.5189520069285	2.04525339148609e-05\\
14.5372608997493	-1.29081963640421e-05\\
14.5555697925702	2.01836257195109e-05\\
14.573878685391	-1.27818102293176e-05\\
14.5921875782119	1.99220344273277e-05\\
14.6104964710327	-1.26599853629561e-05\\
14.6288053638536	1.96675220714426e-05\\
14.6471142566744	-1.25425696936854e-05\\
14.6654231494952	1.9419860739639e-05\\
14.6837320423161	-1.24294181052531e-05\\
14.7020409351369	1.91788322269903e-05\\
14.7203498279578	-1.23203920716075e-05\\
14.7386587207786	1.89442274553531e-05\\
14.7569676135995	-1.22153593919427e-05\\
14.7752765064203	1.87158460676629e-05\\
14.7935853992411	-1.2114193806978e-05\\
14.811894292062	1.84934960196781e-05\\
14.8302031848828	-1.20167747275635e-05\\
14.8485120777037	1.82769931575955e-05\\
14.8668209705245	-1.19229870072513e-05\\
14.8851298633454	1.80661608591703e-05\\
14.9034387561662	-1.18327204814694e-05\\
14.921747648987	1.7860829727212e-05\\
14.9400565418079	-1.17458700597539e-05\\
14.9583654346287	1.76608370851539e-05\\
14.9766743274496	-1.16623351776313e-05\\
14.9949832202704	1.74660269066096e-05\\
15.0132921130913	-1.15820197229832e-05\\
15.0316010059121	1.72762493222955e-05\\
15.0499098987329	-1.15048317941013e-05\\
15.0682187915538	1.70913604452816e-05\\
15.0865276843746	-1.14306835204137e-05\\
15.1048365771955	1.69112220461953e-05\\
15.1231454700163	-1.13594908777714e-05\\
15.1414543628372	1.67357013160507e-05\\
15.159763255658	-1.1291173487471e-05\\
15.1780721484788	1.65646706485612e-05\\
15.1963810412997	-1.12256544828049e-05\\
15.2146899341205	1.63980073821235e-05\\
15.2329988269414	-1.11628602982861e-05\\
15.2513077197622	1.62355936341174e-05\\
15.2696166125831	-1.1102720586742e-05\\
15.2879255054039	1.60773160099992e-05\\
15.3062343982247	-1.10451680768453e-05\\
15.3245432910456	1.59230654949716e-05\\
15.3428521838664	-1.0990138371525e-05\\
15.3611610766873	1.57727372323557e-05\\
15.3794699695081	-1.09375698737757e-05\\
15.397778862329	1.56262303575017e-05\\
15.4160877551498	-1.08874036890694e-05\\
15.4343966479706	1.54834478209021e-05\\
15.4527055407915	-1.0839583462513e-05\\
15.4710144336123	1.53442962485539e-05\\
15.4893233264332	-1.07940553307484e-05\\
15.507632219254	1.52086857735378e-05\\
15.5259411120749	-1.07507677062635e-05\\
15.5442500048957	1.5076529952418e-05\\
15.5625588977165	-1.07096713452548e-05\\
15.5808677905374	1.49477455023694e-05\\
15.5991766833582	-1.06707191448174e-05\\
15.6174855761791	1.48222523099761e-05\\
15.6357944689999	-1.06338660917915e-05\\
15.6541033618208	1.46999732540676e-05\\
15.6724122546416	-1.05990691632307e-05\\
15.6907211474624	1.45808340837605e-05\\
15.7090300402833	-1.05662872522116e-05\\
15.7273389331041	1.44647633091577e-05\\
15.745647825925	-1.05354811473501e-05\\
15.7639567187458	1.43516921167486e-05\\
15.7822656115667	-1.05066133871679e-05\\
15.8005745043875	1.42415542253305e-05\\
15.8188833972083	-1.04796482527647e-05\\
15.8371922900292	1.41342858629157e-05\\
15.85550118285	-1.04545516588506e-05\\
15.8738100756709	1.40298256116611e-05\\
15.8921189684917	-1.04312911229365e-05\\
15.9104278613126	1.39281143499426e-05\\
15.9287367541334	-1.04098357291971e-05\\
15.9470456469542	1.38290951642861e-05\\
15.9653545397751	-1.03901560144781e-05\\
15.9836634325959	1.37327132863907e-05\\
16.0019723254168	-1.03722239863657e-05\\
16.0202812182376	1.36389159942074e-05\\
16.0385901110585	-1.03560130398916e-05\\
16.0568990038793	1.35476525496281e-05\\
16.0752078967001	-1.03414978981642e-05\\
16.093516789521	1.34588741542985e-05\\
16.1118256823418	-1.0328654577757e-05\\
16.1301345751627	1.33725338642698e-05\\
16.1484434679835	-1.03174604071388e-05\\
16.1667523608044	1.32885864985999e-05\\
16.1850612536252	-1.03078939149293e-05\\
16.2033701464461	1.32069886563113e-05\\
16.2216790392669	-1.02999347890709e-05\\
16.2399879320877	1.31276986019835e-05\\
16.2582968249086	-1.0293563914604e-05\\
16.2766057177294	1.30506762351101e-05\\
16.2949146105503	-1.02887632796866e-05\\
16.3132235033711	1.29758830064997e-05\\
16.331532396192	-1.02855159836712e-05\\
16.3498412890128	1.29032819417563e-05\\
16.3681501818336	-1.02838061612764e-05\\
16.3864590746545	1.28328375282871e-05\\
16.4047679674753	-1.0283618991247e-05\\
16.4230768602962	1.27645156977885e-05\\
16.441385753117	-1.02849406800887e-05\\
16.4596946459379	1.26982837968803e-05\\
16.4780035387587	-1.02877583935956e-05\\
16.4963124315795	1.2634110516524e-05\\
16.5146213244004	-1.02920602827183e-05\\
16.5329302172212	1.25719658988221e-05\\
16.5512391100421	-1.02978354387662e-05\\
16.5695480028629	1.25118212488673e-05\\
16.5878568956838	-1.03050738542998e-05\\
16.6061657885046	1.24536491598326e-05\\
16.6244746813254	-1.03137664497766e-05\\
16.6427835741463	1.23974234192692e-05\\
16.6610924669671	-1.032390502273e-05\\
16.679401359788	1.23431190343914e-05\\
16.6977102526088	-1.03354822330037e-05\\
16.7160191454297	1.22907121742899e-05\\
16.7343280382505	-1.0348491617268e-05\\
16.7526369310713	1.2240180130324e-05\\
16.7709458238922	-1.03629275330919e-05\\
16.789254716713	1.21915013298612e-05\\
16.8075636095339	-1.03787851793158e-05\\
16.8258725023547	1.21446552847626e-05\\
16.8441813951756	-1.03960605578879e-05\\
16.8624902879964	1.20996225936865e-05\\
16.8807991808172	-1.04147504829677e-05\\
16.8991080736381	1.20563848612365e-05\\
16.9174169664589	-1.04348525778453e-05\\
16.9357258592798	1.20149247417878e-05\\
16.9540347521006	-1.0456365242828e-05\\
16.9723436449215	1.19752259003791e-05\\
16.9906525377423	-1.04792876406412e-05\\
17.0089614305631	1.19372730032485e-05\\
17.027270323384	-1.05036197410036e-05\\
17.0455792162048	1.19010516494711e-05\\
17.0638881090257	-1.05293622588709e-05\\
17.0821970018465	1.18665484325486e-05\\
17.1005058946674	-1.05565166692578e-05\\
17.1188147874882	1.18337509018851e-05\\
17.137123680309	-1.05850852227252e-05\\
17.1554325731299	1.18026475118271e-05\\
17.1737414659507	-1.06150708831798e-05\\
17.1920503587716	1.17732276013749e-05\\
17.2103592515924	-1.06464774671799e-05\\
17.2286681444133	1.17454814967943e-05\\
17.2469770372341	-1.06793094168667e-05\\
17.2652859300549	1.17194003788346e-05\\
17.2835948228758	-1.07135719860929e-05\\
17.3019037156966	1.16949763210583e-05\\
17.3202126085175	-1.07492711787505e-05\\
17.3385215013383	1.1672202286539e-05\\
17.3568303941592	-1.0786413734365e-05\\
17.37513928698	1.1651072107266e-05\\
17.3934481798008	-1.08250071720883e-05\\
17.4117570726217	1.16315804812306e-05\\
17.4300659654425	-1.08650597353821e-05\\
17.4483748582634	1.16137229750346e-05\\
17.4666837510842	-1.09065804300146e-05\\
17.4849926439051	1.15974960270271e-05\\
17.5033015367259	-0.0957242068971118\\
17.5216104295467	-0.499998892401344\\
17.5399193223676	-0.499999559630121\\
17.5582282151884	-0.499999717300666\\
17.5765371080093	-0.499999790752621\\
17.5948460008301	-0.499999833643282\\
17.613154893651	-0.499999861855943\\
17.6314637864718	-0.499999881854003\\
17.6497726792926	-0.499999896780475\\
17.6680815721135	-0.499999908352343\\
17.6863904649343	-0.499999917588522\\
17.7046993577552	-0.499999925132494\\
17.723008250576	-0.499999931410893\\
17.7413171433969	-0.499999936717868\\
17.7596260362177	-0.499999941262868\\
17.7779349290385	-0.499999945199104\\
17.7962438218594	-0.49999994864126\\
17.8145527146802	-0.499999951676893\\
17.8328616075011	-0.499999954374012\\
17.8511705003219	-0.499999956786251\\
17.8694793931428	-0.499999958956473\\
17.8877882859636	-0.499999960919354\\
17.9060971787844	-0.499999962703241\\
17.9244060716053	-0.49999996433154\\
17.9427149644261	-0.499999965823745\\
17.961023857247	-0.499999967196224\\
17.9793327500678	-0.499999968462821\\
17.9976416428887	-0.499999969635325\\
18.0159505357095	-0.499999970723836\\
18.0342594285304	-0.499999971737061\\
18.0525683213512	-0.499999972682543\\
18.070877214172	-0.49999997356685\\
18.0891861069929	-0.499999974395728\\
18.1074949998137	-0.499999975174225\\
18.1258038926346	-0.499999975906796\\
18.1441127854554	-0.499999976597383\\
18.1624216782763	-0.499999977249494\\
18.1807305710971	-0.499999977866253\\
18.1990394639179	-0.499999978450458\\
18.2173483567388	-0.499999979004618\\
18.2356572495596	-0.499999979530991\\
18.2539661423805	-0.499999980031614\\
18.2722750352013	-0.499999980508333\\
18.2905839280222	-0.499999980962817\\
18.308892820843	-0.4999999811847\\
};
\addlegendentry{PINS}

\end{axis}
\end{tikzpicture}%%
  \caption{Jerk analysis PINS vs Duboids}
  \label{fig:Compare_jerk2}
\end{figure}
%
\section*{Conclusions}
%
In this work, we have presented a new approach to the trajectory planning problem for autonomous vehicles suitable for constant velocity manoeuvres or parking. The approach is based on the Duboids algorithm, and it can find a feasible suboptimal solution to the trajectory planning problem. 
%
The new approach was compared with the solution of the numerical optimal control with PINS, and it was shown that the Duboids algorithm can find a better solution in some cases. Furthermore, the Duboids solution is smoother than the PINS one, and it is not affected by the ringing or Fuller's phenomena. Hence, the Duboids solution is a better choice for control purposes.
%
\section*{Future work}
%
In the future development of this work, we plan to extend the Duboids algorithm and find a closed-form solution or a least a semi-analytical form to compute the length of the arcs without relying on numerical integration and or/ constraint minimization. This will allow for speeding up the computation time of the Duboids algorithm making it more suitable for real-time applications.\\
We also plan to extend the Duboids algorithm to the case of non-constant velocity manoeuvres. Furthermore, we plan to extend the algorithm to the case of multiple Duboids (a spline of Duboids).
%
%%%% References
\bibliographystyle{ieeetr}
\bibliography{Bibliography}
%
\end{document}