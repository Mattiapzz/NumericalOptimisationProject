\documentclass[11pt,twocolumn]{scrartcl}

% \usepackage{mathpazo}
% \usepackage{multicol}
\usepackage[dvipsnames]{xcolor}
\usepackage[top = 2.5cm, bottom = 2.0cm, right = 1.5cm, left = 1.5cm]{geometry}
\usepackage{amsmath,amssymb} %packages for Mathematics
\usepackage{physics}
\usepackage{cases}
\usepackage{amsfonts}

% \definecolor{UniTnRed}{RGB}{177,10,37}
\definecolor{UniTnRed}{RGB}{125,02,12}
\definecolor{UniTnGray}{RGB}{218,218,218}
\definecolor{UniTnOrange}{RGB}{206,176,123}

\definecolor{customgreen}{RGB}{0,128,0} % Define the custom green color

\makeatletter
\renewcommand\maketitle
{
    \noindent
    {\color{UniTnOrange}\rule{\textwidth}{2pt}}
    \vspace{0.5cm}\\
    {\Large\bfseries\textcolor{UniTnRed}{\@title}}%
    \medskip\par\noindent
    {\Large\bfseries\textcolor{UniTnRed}{\@subtitle}}\\
    \;
    \medskip\par\noindent
    {\large\bfseries\textcolor{UniTnRed}{\@author}}%
    {\large\quad University of Trento}
    \hfill
    {\large\@date}\\
    \;\\%
    {\color{UniTnOrange}\rule{\textwidth}{2pt}}
    \bigskip\par\noindent
}

% avoid indent
\setlength{\parindent}{0pt}

% create a command \rm to remove the math font
\newcommand{\rm}{\mathrm}
% add package 

\usepackage{cancel}



\definecolor{darkolivegreen}{RGB}{ 85,107, 47}
\definecolor{darksalmon}{RGB}{233,150,122}
\definecolor{darkseagreen}{RGB}{143,188,143}
\definecolor{darkslateblue}{RGB}{ 72, 61,139}
\definecolor{deepskyblue}{RGB}{  0,191,255}
\definecolor{dodgerblue}{RGB}{ 30,144,255}
%\definecolor{green}{RGB}{  0,128,  0}
\definecolor{indigo}{RGB}{ 75,  0,130}
\definecolor{lawngreen}{RGB}{124,252,  0}
\definecolor{lightgreen}{RGB}{144,238,144}
\definecolor{lightseagreen}{RGB}{ 32,178,170}
\definecolor{lightskyblue}{RGB}{135,206,250}
\definecolor{lightsteelblue}{RGB}{176,196,222}
\definecolor{lime}{RGB}{  0,255,  0}
\definecolor{limegreen}{RGB}{ 50,205, 50}
\definecolor{midnightblue}{RGB}{ 25, 25,112}
\definecolor{orange}{RGB}{255,165,  0}
\definecolor{orchid}{RGB}{218,112,214}
\definecolor{skyblue}{RGB}{135,206,235}
\definecolor{slateblue}{RGB}{106, 90,205}
\definecolor{springgreen}{RGB}{  0,255,127}
\definecolor{steelblue}{RGB}{ 70,130,180}
\definecolor{turquoise}{RGB}{ 64,224,208}
\definecolor{violet}{RGB}{238,130,238} 

\usepackage{pgfplots}
\usepackage{grffile}
\pgfplotsset{compat=newest}
\usetikzlibrary{plotmarks}
\usetikzlibrary{arrows.meta}
\usepgfplotslibrary{patchplots}
\usepackage{siunitx}
\usepackage{graphicx}
\usepackage{subcaption}
\usepackage{tikz}
\usetikzlibrary{calc}

\makeatother


\title{Report of Course Numerical Optimization}
\subtitle{Duboids: Extended Dubins path with Clothoids Junctions}
\author{Mattia Piazza}
\date{\today}
 
\begin{document}
\twocolumn[\maketitle ]
%
% \begin{multicols}{2}
 % 
\section*{Introduction}
%
The Dubins path is a well known problem in robotics and trajectory planning. The problem consists in finding the shortest path to connect two points in a plane with a prescribed heading (initial and final) with a maxim curvature constraint. This problem is suitable for mobile robots that are not omnidirectional. The solution can be derived analytically and is composed of a sequence of circular arcs and straight lines.\cite{shkel2001classification,chen2019shortest,jha2020shortest} The Dubins path is a special case of the Reeds-Shepp path\cite{duits2018optimal} where also backward motion is allowed.\\
However, the Dubins path is not suitable for vehicles that have a limited steering rate. In this case, the curvature of the path cannot change instantaneously.\\
Duboids is the proposed suboptimal solution to account for limits in curvature rate and therefore in steering rate of a vehicle. Duboids is a combination of Dubins path and clothoids. 
%
\subsection*{Dubins}
The problem can be stated as follows:
%
\begin{equation}
  \begin{split}
    \min_{\kappa}  \quad & \int_0^T v \differential t = \min_{\kappa}  \quad v T \\ % \min_{|\kappa|\le\kappa_{max}}
    \text{s.t.} \quad 
      &\dot{x}(t)      = v \cos(\theta(t)) \\
      &\dot{y}(t)      = v \sin(\theta(t)) \\
      &\dot{\theta}(t) = v \kappa(t)       \\
      &x(0)      = x_0,      \; x(T)      = x_T\\
      &y(0)      = y_0,      \; y(T)      = y_T\\
      &\theta(0) = \theta_0, \; \theta(T) = \theta_T\\
      &-\kappa_{max} \le \kappa(t) \le \kappa_{max}   
  \end{split}
\end{equation}
%
Where $v$ is a fixed velocity, $x$, $y$ and $\theta$ are the position and heading of the vehicle, $\kappa$ is the curvature of the path and $\kappa_{max}$ is the maximum curvature allowed. 
The analytic solution to this problem is at most a sequence of 3 arcs, either left of right circular arcs at maximum curvature or straight lines.\cite{shkel2001classification}\\
%
This problem and its solution can be applied in vehicle moving slowly and where the curvature change is almost instantaneous. However, real vehicles have a physical limit in the maximum curvature and maximum curvature rate. For this reason, we are looking to an extension of the Dubins path to incorporate this limitation as we will see in the next section exploiting clothoids.\cite{bertolazzi2015g1,bertolazzi2018clothoids}
%
\section*{Problem}
%
The problem we want to solve is to minimize the time while connecting two points in the Cartesian space with a prescribed heading and curvature both at initial and final time. This problem can be formulated in the following way:
%
\begin{equation}
  \begin{split}
    \min_{J} \quad & v T \\ 
    \text{s.t.} \quad
      &\dot{x}(t)      = v \cos(\theta(t)) \\
      &\dot{y}(t)      = v \sin(\theta(t)) \\
      &\dot{\theta}(t) = v \kappa(t)       \\
      &\dot{\kappa}(t) = J(t)              \\
      &x(0)      = x_0,      \; x(T)      = x_T     \\
      &y(0)      = y_0,      \; y(T)      = y_T     \\
      &\theta(0) = \theta_0, \; \theta(T) = \theta_T\\
      &\kappa(0) = \kappa_0, \; \kappa(T) = \kappa_T\\
      &-\kappa_{max} \le \kappa(t) \le \kappa_{max} \\
      &-J_{max} \le J(t) \le J_{max}
  \end{split}
\end{equation}
%
Where $v$ is a fixed velocity, $x$, $y$ and $\theta$ are the position and heading of the vehicle, $\kappa$ is the curvature of the path, $\kappa_{max}$ is the maximum curvature allowed, $J$ is the controlled curvature rate (Jerk) and $J_{max}$ is the maximum curvature rate allowed.\\
%
This problem can be translated into a BVP (Boundary Value Problem) and solved in a semi-analitycal fashion. In fact, the solution is composed of several arcs either at maximum rate (positive or negative or at zero rate). 
%
\subsection*{Analytic solution}
%
The Hemiltonian function of the problem is:
%
\begin{equation}
  \begin{split}
    H = &\lambda_1(t) v \cos(\theta(t)) + \lambda_2(t) v \sin(\theta(t)) \\
        &+ \lambda_3(t) v \kappa(t) + \lambda_4(t) J(t)\\
        &+ \mu_1(t) (\kappa(t)-\kappa_{max}) \\ 
        &+ \mu_2(t) (-\kappa(t)+\kappa_{max})
  \end{split}
\end{equation}
%
The costate equations are:
%
\begin{equation}
  \begin{split}
    &\dot{\lambda_1}(t) = 0 \\
    &\dot{\lambda_2}(t) = 0 \\
    &\dot{\lambda_3}(t) = \lambda_1(t) v \sin(\theta(t)) - \lambda_2(t) v \cos(\theta(t)) \\
    &\dot{\lambda_4}(t) = -\lambda_3(t) v -\mu_1(t) + \mu_2(t) \\
  \end{split}
\end{equation}
%
Which yield that $\lambda_1$ and $\lambda_2$ are constant.\\
Moreover, the control is
%
\begin{equation}
  J(t) = \underset{ J \in [-J_{max},J_{max}]}{ \textrm{argmin} } \; H 
\end{equation}
%
However, $H$ is linear in $J$, thus the second derivative with respect to the control is null, and the problem became singular. In the case of a singular arc, the control is either at the maximum or minimum of the control set or at zero.
%
\begin{equation}
  J(t) = \begin{cases}
    +J_{max} & \text{if } \lambda_4(t) > 0 \\
    -J_{max} & \text{if } \lambda_4(t) < 0 \\
    0 & \text{if } \lambda_4(t) = 0
  \end{cases}
\end{equation}
%
\subsection*{Physical interpretation}
%
The physical interpretation is that the vehicle is either changing the curvature at maximum rate, or keeping the curvature constant.
The only case when the curvature is kept constant is when the vehicle is travelling straight or when it is travelling at maximum curvature.
Thus, the problem can be treated as a mixed integer optimization that in general is NP-hard.\\
%
Figure \ref{fig:possiblecombination} illustrates all the possible combination of maneuvers connecting point $P_0$ and $P_T$. All intermediate points (at most $7$) are switching point between clothoids and circular arcs or straight lines. From the starting point (and configuration) the vehicle can either steer toward maximum, minimum or zero curvature. If point $P_0$ already satisfy a bound the arc of clothoid $L_1$ is not necessary and will have zero length.
There are some useless connection such as the repetition of two straight line or two arcs with same curvature which are accounted as special case connecting directly the point to the final configuration.\\
%
Figure \ref{fig:possiblecombination} represents in fact a graph connecting the initial and final configuration with all the possible combination of maneuvers. The problem could be solved with a graph search algorithm. However, the number of possible path inside the graph is not high and know at priory. Thus, a naive exploration of all the possible combination is feasible.\\
%
From figure \ref{fig:possiblecombination}, we can count $12$ possible $7$-arcs connections with a strange familiar resemblance to the Dubins path. In addition, $3$ are the $3$-arcs connections and $6$ are the $5$-arcs connections. The total number of possible combination is $21$ as shown in table \ref{tab:possiblecombination}.\\
%
However, we are neglecting the possibility that, for some configuration of initial and final point the vehicle could perform maneuvers not reaching the maximum curvature values. This, will need a separate analysis when it comes to the naive exploration.  
%
\begin{figure}[ht]
  \centering
  \resizebox{1.1\linewidth}{!}{%
    
\tikzset{every picture/.style={line width=0.5pt}} %set default line width to 0.75pt 

% \begin{tikzpicture}[x=0.75pt,y=0.75pt,yscale=-1,xscale=1]
\begin{tikzpicture}[
    every node/.style={minimum size=0.05cm, font=\footnotesize, inner sep=0.01cm,scale = 1}
]

% Level 0
\node[draw,circle,font=\footnotesize, inner sep=0.01cm] (P0) at (0,0) {$P_0$};

% Level 1
\node[draw,circle] (P11) at ( 3.0,-1.) {$P_1$};
\node[draw,circle] (P12) at ( 0.0,-1.) {$P_1$};
\node[draw,circle] (P13) at (-3.0,-1.) {$P_1$};
\draw[red] (P0) -- node[midway, right, shift={(+0.2cm,0)}] {$L_1$} (P11);
\draw[red] (P0) -- (P12);
\draw[red] (P0) -- (P13);

% Level 2
\node[draw,circle] (P21) at ( 3.0,-2) {$P_2$};
\node[draw,circle] (P22) at ( 0.0,-2) {$P_2$};
\node[draw,circle] (P23) at (-3.0,-2) {$P_2$};

\draw[dodgerblue] (P11) -- node[midway, right, shift={(+0.2cm,0)}] {$L_2$} (P21);
\draw[springgreen] (P12) -- (P22);
\draw[cyan] (P13) -- (P23);

% Level 3
\node[draw,circle,dashed] (P31) at ( 4.0,-3) {$P_3$};
\node[draw,circle] (P32) at ( 3.0,-3) {$P_3$};
\node[draw,circle] (P33) at ( 2.0,-3) {$P_3$};
%
\node[draw,circle] (P34) at ( 1.0,-3) {$P_3$};
\node[draw,circle,dashed] (P35) at ( 0.0,-3) {$P_3$};
\node[draw,circle] (P36) at (-1.0,-3) {$P_3$};
%
\node[draw,circle] (P37) at (-2.0,-3) {$P_3$};
\node[draw,circle] (P38) at (-3.0,-3) {$P_3$};
\node[draw,circle,dashed] (P39) at (-4.0,-3) {$P_3$};




\draw[red,dashed] (P21) -- node[midway, right, shift={(+0.2cm,0)}] {$L_3$} (P31);
\draw[red] (P21) -- (P32);
\draw[red] (P21) -- (P33);
%
\draw[red] (P22) -- (P34);
\draw[red,dashed] (P22) -- (P35);
\draw[red] (P22) -- (P36);
%
\draw[red] (P23) -- (P37);
\draw[red] (P23) -- (P38);
\draw[red,dashed] (P23) -- (P39);
%

% layer 4

\node[draw,circle,dashed] (P41) at ( 4.0,-4) {$P_4$};
\node[draw,circle] (P42) at ( 3.0,-4) {$P_4$};
\node[draw,circle] (P43) at ( 2.0,-4) {$P_4$};
%
\node[draw,circle] (P44) at ( 1.0,-4) {$P_4$};
\node[draw,circle,dashed] (P45) at ( 0.0,-4) {$P_4$};
\node[draw,circle] (P46) at (-1.0,-4) {$P_4$};
%
\node[draw,circle] (P47) at (-2.0,-4) {$P_4$};
\node[draw,circle] (P48) at (-3.0,-4) {$P_4$};
\node[draw,circle,dashed] (P49) at (-4.0,-4) {$P_4$};
%

\draw[dodgerblue,dashed] (P31) -- node[midway, right, shift={(+0.2cm,0)}] {$L_4$} (P41);
\draw[springgreen] (P32) -- (P42);
\draw[cyan] (P33) -- (P43);
%
\draw[dodgerblue] (P34) -- (P44);
\draw[springgreen,dashed] (P35) -- (P45);
\draw[cyan] (P36) -- (P46);
%
\draw[dodgerblue] (P37) -- (P47);
\draw[springgreen] (P38) -- (P48);
\draw[cyan,dashed] (P39) -- (P49);
%

% layer 5

\node[draw,circle] (P51) at ( 4.5,-5) {$P_5$};
\node[draw,circle,dashed] (P52) at ( 4.0,-5) {$P_5$};
\node[draw,circle] (P53) at ( 3.5,-5) {$P_5$};
%
\node[draw,circle] (P54) at ( 3.0,-5) {$P_5$};
\node[draw,circle] (P55) at ( 2.5,-5) {$P_5$};
\node[draw,circle,dashed] (P56) at ( 2.0,-5) {$P_5$};
%
\node[draw,circle,dashed] (P57) at ( 1.5,-5) {$P_5$};
\node[draw,circle] (P58) at ( 1.0,-5) {$P_5$};
\node[draw,circle] (P59) at ( 0.5,-5) {$P_5$};
%
\node[draw,circle] (P510) at (-0.5,-5) {$P_5$};
\node[draw,circle] (P511) at (-1.0,-5) {$P_5$};
\node[draw,circle,dashed] (P512) at (-1.5,-5) {$P_5$};
%
\node[draw,circle,dashed] (P513) at (-2.0,-5) {$P_5$};
\node[draw,circle] (P514) at (-2.5,-5) {$P_5$};
\node[draw,circle] (P515) at (-3.0,-5) {$P_5$};
%
\node[draw,circle] (P516) at (-3.5,-5) {$P_5$};
\node[draw,circle,dashed] (P517) at (-4.0,-5) {$P_5$};
\node[draw,circle] (P518) at (-4.5,-5) {$P_5$};
%




%
\draw[red] (P42) -- node[midway, right, shift={(+0.2cm,0)}] {$L_5$} (P51);
\draw[red,dashed] (P42) -- (P52);
\draw[red] (P42) -- (P53);
%
\draw[red] (P43) -- (P54);
\draw[red] (P43) -- (P55);
\draw[red,dashed] (P43) -- (P56);
%
\draw[red,dashed] (P44) -- (P57);
\draw[red] (P44) -- (P58);
\draw[red] (P44) -- (P59);
%
\draw[red] (P46) -- (P510);
\draw[red] (P46) -- (P511);
\draw[red,dashed] (P46) -- (P512);
%
\draw[red,dashed] (P47) -- (P513);
\draw[red] (P47) -- (P514);
\draw[red] (P47) -- (P515);
%
\draw[red] (P48) -- (P516);
\draw[red,dashed] (P48) -- (P517);
\draw[red] (P48) -- (P518);

% layer 6
\node[draw,circle] (P61) at ( 4.5,-6) {$P_6$};
\node[draw,circle,dashed] (P62) at ( 4.0,-6) {$P_6$};
\node[draw,circle] (P63) at ( 3.5,-6) {$P_6$};
%
\node[draw,circle] (P64) at ( 3.0,-6) {$P_6$};
\node[draw,circle] (P65) at ( 2.5,-6) {$P_6$};
\node[draw,circle,dashed] (P66) at ( 2.0,-6) {$P_6$};
%
\node[draw,circle,dashed] (P67) at ( 1.5,-6) {$P_6$};
\node[draw,circle] (P68) at ( 1.0,-6) {$P_6$};
\node[draw,circle] (P69) at ( 0.5,-6) {$P_6$};
%
\node[draw,circle] (P610) at (-0.5,-6) {$P_6$};
\node[draw,circle] (P611) at (-1.0,-6) {$P_6$};
\node[draw,circle,dashed] (P612) at (-1.5,-6) {$P_6$};
%
\node[draw,circle,dashed] (P613) at (-2.0,-6) {$P_6$};
\node[draw,circle] (P614) at (-2.5,-6) {$P_6$};
\node[draw,circle] (P615) at (-3.0,-6) {$P_6$};
%
\node[draw,circle] (P616) at (-3.5,-6) {$P_6$};
\node[draw,circle,dashed] (P617) at (-4.0,-6) {$P_6$};
\node[draw,circle] (P618) at (-4.5,-6) {$P_6$};
%

%
\draw[dodgerblue] (P51) -- node[midway, right, shift={(+0.2cm,0)}] {$L_6$} (P61);
\draw[springgreen,dashed] (P52) -- (P62);
\draw[cyan] (P53) -- (P63);
%
\draw[dodgerblue] (P54) -- (P64);
\draw[springgreen] (P55) -- (P65);
\draw[cyan,dashed] (P56) -- (P66);
%
\draw[dodgerblue,dashed] (P57) -- (P67);
\draw[springgreen] (P58) -- (P68);
\draw[cyan] (P59) -- (P69);
%
\draw[dodgerblue] (P510) -- (P610);
\draw[springgreen] (P511) -- (P611);
\draw[cyan,dashed] (P512) -- (P612);
%
\draw[dodgerblue,dashed] (P513) -- (P613);
\draw[springgreen] (P514) -- (P614);
\draw[cyan] (P515) -- (P615);
%
\draw[dodgerblue] (P516) -- (P616);
\draw[springgreen,dashed] (P517) -- (P617);
\draw[cyan] (P518) -- (P618);
%

% % Level 7

\node[draw,circle,font=\footnotesize, inner sep=0.01cm] (P7) at (0,-9) {$P_T$};

\draw[red] (P61) -- node[midway, right, shift={(+0.2cm,0)}] {$L_7$} (P7);
\draw[red] (P63) -- (P7);
%   
\draw[red] (P64) -- (P7);
\draw[red] (P65) -- (P7);
%
\draw[red] (P68) -- (P7);
\draw[red] (P69) -- (P7);
%
\draw[red] (P610) -- (P7);
\draw[red] (P611) -- (P7);
%
\draw[red] (P614) -- (P7);
\draw[red] (P615) -- (P7);
%
\draw[red] (P616) -- (P7);
\draw[red] (P618) -- (P7);
%

% curved connection between P21 and P7
\draw[orange] (P21) to[out=0,in=0] (P7);

% curved large connction between P23 and P7
\draw[orange] (P23) to[out=180,in=180] (P7);

% connect P22 and P7
\draw[orange] (P22) to[out=-100,in=90] (P7);

% connect P42 and P7
\draw[orange] (P42) to[out=227,in=60] (P7);

% connect P43 and P7
\draw[orange] (P43) to[out=-105,in=60] (P7);

% connect P44 and P7
\draw[orange] (P44) to[out=-70,in=60] (P7);

% connect P46 and P7
\draw[orange] (P46) to[out=-110,in=120] (P7);

% connect P47 and P7
\draw[orange] (P47) to[out=-75,in=120] (P7);

% connect P48 and P7
\draw[orange] (P48) to[out=-47,in=120] (P7);



% draw a color legend at the corner of the tikz picture with line - name iside a transparent box with black borders
\draw[dashed] (-5,-7.5)  -- (-4.5,-7.5) node[right] {Dead branch};
\draw[red]    (-5,-8.0)  -- (-4.5,-8.0) node[right] {Clothoids};
\draw[springgreen]  (-5,-8.5)  -- (-4.5,-8.5) node[right] {Lines};
\draw[dodgerblue]   (-5,-9.0)  -- (-4.5,-9.0) node[right] {Arcs $k_{max}$};
\draw[cyan]   (-5,-9.5)  -- (-4.5,-9.5) node[right] {Arcs $k_{min}$};
\draw[orange] (-5,-10.0)  -- (-4.5,-10.0) node[right] {Special cases};




\end{tikzpicture}
%
  }%
  % 
\tikzset{every picture/.style={line width=0.5pt}} %set default line width to 0.75pt 

% \begin{tikzpicture}[x=0.75pt,y=0.75pt,yscale=-1,xscale=1]
\begin{tikzpicture}[
    every node/.style={minimum size=0.05cm, font=\footnotesize, inner sep=0.01cm,scale = 1}
]

% Level 0
\node[draw,circle,font=\footnotesize, inner sep=0.01cm] (P0) at (0,0) {$P_0$};

% Level 1
\node[draw,circle] (P11) at ( 3.0,-1.) {$P_1$};
\node[draw,circle] (P12) at ( 0.0,-1.) {$P_1$};
\node[draw,circle] (P13) at (-3.0,-1.) {$P_1$};
\draw[red] (P0) -- node[midway, right, shift={(+0.2cm,0)}] {$L_1$} (P11);
\draw[red] (P0) -- (P12);
\draw[red] (P0) -- (P13);

% Level 2
\node[draw,circle] (P21) at ( 3.0,-2) {$P_2$};
\node[draw,circle] (P22) at ( 0.0,-2) {$P_2$};
\node[draw,circle] (P23) at (-3.0,-2) {$P_2$};

\draw[dodgerblue] (P11) -- node[midway, right, shift={(+0.2cm,0)}] {$L_2$} (P21);
\draw[springgreen] (P12) -- (P22);
\draw[cyan] (P13) -- (P23);

% Level 3
\node[draw,circle,dashed] (P31) at ( 4.0,-3) {$P_3$};
\node[draw,circle] (P32) at ( 3.0,-3) {$P_3$};
\node[draw,circle] (P33) at ( 2.0,-3) {$P_3$};
%
\node[draw,circle] (P34) at ( 1.0,-3) {$P_3$};
\node[draw,circle,dashed] (P35) at ( 0.0,-3) {$P_3$};
\node[draw,circle] (P36) at (-1.0,-3) {$P_3$};
%
\node[draw,circle] (P37) at (-2.0,-3) {$P_3$};
\node[draw,circle] (P38) at (-3.0,-3) {$P_3$};
\node[draw,circle,dashed] (P39) at (-4.0,-3) {$P_3$};




\draw[red,dashed] (P21) -- node[midway, right, shift={(+0.2cm,0)}] {$L_3$} (P31);
\draw[red] (P21) -- (P32);
\draw[red] (P21) -- (P33);
%
\draw[red] (P22) -- (P34);
\draw[red,dashed] (P22) -- (P35);
\draw[red] (P22) -- (P36);
%
\draw[red] (P23) -- (P37);
\draw[red] (P23) -- (P38);
\draw[red,dashed] (P23) -- (P39);
%

% layer 4

\node[draw,circle,dashed] (P41) at ( 4.0,-4) {$P_4$};
\node[draw,circle] (P42) at ( 3.0,-4) {$P_4$};
\node[draw,circle] (P43) at ( 2.0,-4) {$P_4$};
%
\node[draw,circle] (P44) at ( 1.0,-4) {$P_4$};
\node[draw,circle,dashed] (P45) at ( 0.0,-4) {$P_4$};
\node[draw,circle] (P46) at (-1.0,-4) {$P_4$};
%
\node[draw,circle] (P47) at (-2.0,-4) {$P_4$};
\node[draw,circle] (P48) at (-3.0,-4) {$P_4$};
\node[draw,circle,dashed] (P49) at (-4.0,-4) {$P_4$};
%

\draw[dodgerblue,dashed] (P31) -- node[midway, right, shift={(+0.2cm,0)}] {$L_4$} (P41);
\draw[springgreen] (P32) -- (P42);
\draw[cyan] (P33) -- (P43);
%
\draw[dodgerblue] (P34) -- (P44);
\draw[springgreen,dashed] (P35) -- (P45);
\draw[cyan] (P36) -- (P46);
%
\draw[dodgerblue] (P37) -- (P47);
\draw[springgreen] (P38) -- (P48);
\draw[cyan,dashed] (P39) -- (P49);
%

% layer 5

\node[draw,circle] (P51) at ( 4.5,-5) {$P_5$};
\node[draw,circle,dashed] (P52) at ( 4.0,-5) {$P_5$};
\node[draw,circle] (P53) at ( 3.5,-5) {$P_5$};
%
\node[draw,circle] (P54) at ( 3.0,-5) {$P_5$};
\node[draw,circle] (P55) at ( 2.5,-5) {$P_5$};
\node[draw,circle,dashed] (P56) at ( 2.0,-5) {$P_5$};
%
\node[draw,circle,dashed] (P57) at ( 1.5,-5) {$P_5$};
\node[draw,circle] (P58) at ( 1.0,-5) {$P_5$};
\node[draw,circle] (P59) at ( 0.5,-5) {$P_5$};
%
\node[draw,circle] (P510) at (-0.5,-5) {$P_5$};
\node[draw,circle] (P511) at (-1.0,-5) {$P_5$};
\node[draw,circle,dashed] (P512) at (-1.5,-5) {$P_5$};
%
\node[draw,circle,dashed] (P513) at (-2.0,-5) {$P_5$};
\node[draw,circle] (P514) at (-2.5,-5) {$P_5$};
\node[draw,circle] (P515) at (-3.0,-5) {$P_5$};
%
\node[draw,circle] (P516) at (-3.5,-5) {$P_5$};
\node[draw,circle,dashed] (P517) at (-4.0,-5) {$P_5$};
\node[draw,circle] (P518) at (-4.5,-5) {$P_5$};
%




%
\draw[red] (P42) -- node[midway, right, shift={(+0.2cm,0)}] {$L_5$} (P51);
\draw[red,dashed] (P42) -- (P52);
\draw[red] (P42) -- (P53);
%
\draw[red] (P43) -- (P54);
\draw[red] (P43) -- (P55);
\draw[red,dashed] (P43) -- (P56);
%
\draw[red,dashed] (P44) -- (P57);
\draw[red] (P44) -- (P58);
\draw[red] (P44) -- (P59);
%
\draw[red] (P46) -- (P510);
\draw[red] (P46) -- (P511);
\draw[red,dashed] (P46) -- (P512);
%
\draw[red,dashed] (P47) -- (P513);
\draw[red] (P47) -- (P514);
\draw[red] (P47) -- (P515);
%
\draw[red] (P48) -- (P516);
\draw[red,dashed] (P48) -- (P517);
\draw[red] (P48) -- (P518);

% layer 6
\node[draw,circle] (P61) at ( 4.5,-6) {$P_6$};
\node[draw,circle,dashed] (P62) at ( 4.0,-6) {$P_6$};
\node[draw,circle] (P63) at ( 3.5,-6) {$P_6$};
%
\node[draw,circle] (P64) at ( 3.0,-6) {$P_6$};
\node[draw,circle] (P65) at ( 2.5,-6) {$P_6$};
\node[draw,circle,dashed] (P66) at ( 2.0,-6) {$P_6$};
%
\node[draw,circle,dashed] (P67) at ( 1.5,-6) {$P_6$};
\node[draw,circle] (P68) at ( 1.0,-6) {$P_6$};
\node[draw,circle] (P69) at ( 0.5,-6) {$P_6$};
%
\node[draw,circle] (P610) at (-0.5,-6) {$P_6$};
\node[draw,circle] (P611) at (-1.0,-6) {$P_6$};
\node[draw,circle,dashed] (P612) at (-1.5,-6) {$P_6$};
%
\node[draw,circle,dashed] (P613) at (-2.0,-6) {$P_6$};
\node[draw,circle] (P614) at (-2.5,-6) {$P_6$};
\node[draw,circle] (P615) at (-3.0,-6) {$P_6$};
%
\node[draw,circle] (P616) at (-3.5,-6) {$P_6$};
\node[draw,circle,dashed] (P617) at (-4.0,-6) {$P_6$};
\node[draw,circle] (P618) at (-4.5,-6) {$P_6$};
%

%
\draw[dodgerblue] (P51) -- node[midway, right, shift={(+0.2cm,0)}] {$L_6$} (P61);
\draw[springgreen,dashed] (P52) -- (P62);
\draw[cyan] (P53) -- (P63);
%
\draw[dodgerblue] (P54) -- (P64);
\draw[springgreen] (P55) -- (P65);
\draw[cyan,dashed] (P56) -- (P66);
%
\draw[dodgerblue,dashed] (P57) -- (P67);
\draw[springgreen] (P58) -- (P68);
\draw[cyan] (P59) -- (P69);
%
\draw[dodgerblue] (P510) -- (P610);
\draw[springgreen] (P511) -- (P611);
\draw[cyan,dashed] (P512) -- (P612);
%
\draw[dodgerblue,dashed] (P513) -- (P613);
\draw[springgreen] (P514) -- (P614);
\draw[cyan] (P515) -- (P615);
%
\draw[dodgerblue] (P516) -- (P616);
\draw[springgreen,dashed] (P517) -- (P617);
\draw[cyan] (P518) -- (P618);
%

% % Level 7

\node[draw,circle,font=\footnotesize, inner sep=0.01cm] (P7) at (0,-9) {$P_T$};

\draw[red] (P61) -- node[midway, right, shift={(+0.2cm,0)}] {$L_7$} (P7);
\draw[red] (P63) -- (P7);
%   
\draw[red] (P64) -- (P7);
\draw[red] (P65) -- (P7);
%
\draw[red] (P68) -- (P7);
\draw[red] (P69) -- (P7);
%
\draw[red] (P610) -- (P7);
\draw[red] (P611) -- (P7);
%
\draw[red] (P614) -- (P7);
\draw[red] (P615) -- (P7);
%
\draw[red] (P616) -- (P7);
\draw[red] (P618) -- (P7);
%

% curved connection between P21 and P7
\draw[orange] (P21) to[out=0,in=0] (P7);

% curved large connction between P23 and P7
\draw[orange] (P23) to[out=180,in=180] (P7);

% connect P22 and P7
\draw[orange] (P22) to[out=-100,in=90] (P7);

% connect P42 and P7
\draw[orange] (P42) to[out=227,in=60] (P7);

% connect P43 and P7
\draw[orange] (P43) to[out=-105,in=60] (P7);

% connect P44 and P7
\draw[orange] (P44) to[out=-70,in=60] (P7);

% connect P46 and P7
\draw[orange] (P46) to[out=-110,in=120] (P7);

% connect P47 and P7
\draw[orange] (P47) to[out=-75,in=120] (P7);

% connect P48 and P7
\draw[orange] (P48) to[out=-47,in=120] (P7);



% draw a color legend at the corner of the tikz picture with line - name iside a transparent box with black borders
\draw[dashed] (-5,-7.5)  -- (-4.5,-7.5) node[right] {Dead branch};
\draw[red]    (-5,-8.0)  -- (-4.5,-8.0) node[right] {Clothoids};
\draw[springgreen]  (-5,-8.5)  -- (-4.5,-8.5) node[right] {Lines};
\draw[dodgerblue]   (-5,-9.0)  -- (-4.5,-9.0) node[right] {Arcs $k_{max}$};
\draw[cyan]   (-5,-9.5)  -- (-4.5,-9.5) node[right] {Arcs $k_{min}$};
\draw[orange] (-5,-10.0)  -- (-4.5,-10.0) node[right] {Special cases};




\end{tikzpicture}

  \caption{Possible Combination}
  \label{fig:possiblecombination}
\end{figure}
%
\begin{table}[ht]
  \centering
  \begin{tabular}{c|c|c|c}
  \hline
  \textbf{\#} & \textbf{$L_2$} & \textbf{$L_4$} & \textbf{$L_6$} \\
  \hline
  1 & $L$ & $S$ & $L$  \\
  2 & $L$ & $S$ & $R$  \\
  3 & $L$ & $R$ & $L$  \\
  4 & $L$ & $R$ & $S$  \\
  5 & $S$ & $L$ & $S$  \\
  6 & $S$ & $L$ & $R$  \\
  7 & $S$ & $R$ & $L$  \\
  8 & $S$ & $R$ & $S$  \\
  9 & $R$ & $L$ & $S$  \\
  10 & $R$ & $L$ & $R$  \\
  11 & $R$ & $S$ & $L$  \\
  12 & $R$ & $S$ & $R$  \\
  13 & $L$ & $-$ & $-$  \\
  14 & $S$ & $-$ & $-$  \\
  15 & $R$ & $-$ & $-$  \\
  16 & $L$ & $S$ & $-$  \\
  17 & $L$ & $R$ & $-$  \\
  18 & $S$ & $L$ & $-$  \\
  19 & $S$ & $R$ & $-$  \\
  20 & $R$ & $L$ & $-$  \\
  21 & $R$ & $S$ & $-$  \\
  \hline
  \end{tabular}
  \caption{Possible combination.\\ $L$ = Left arc, $R$ = Right arc,\\$S$ = Straight line, $-$ = none}
  \label{tab:possiblecombination}
\end{table}
%
\section*{Numeric solution}
%
The problem was tackled with two numerical approaches. The first writing the problem as an Optimal Control Problem (OCP) and solving it with the indirect direct method and constraints formulated as penalization with PINS (PINS Is Not a Solver)\cite{biral2016notes}. The second approach was a naive exploration of all the possible maneuver combination to solve the problem.
%
\subsection*{PINS}
%
The problem was developed and solved using PINS to analyze the numerical solution of the optimal control problem and to have a reference solution to compare the naive exploration of all the possible combination of maneuvers.
\subsubsection*{Numerical Results}



%
%%%%%%%%%%%%%
%%%%%%%%%%%%
%
\subsection*{Duboids MATLAB implementation and exploration}
%
The problem can be solved with a naive exploration of all the possible combination of maneuvers. In fact, all possible connection are a combination of the simple elementary Dubins connected by segments with linearly varying curvature also known as clothoids.\cite{bertolazzi2015g1}\\
An analysis done by hand and the of the numerical solution using PINS suggest, that there are at most $7$ connection between two points. With $3$ arcs at constant curvature and $4$ arcs that connect all the arcs. Moreover, from figure \ref{fig:possiblecombination} and table \ref{tab:possiblecombination} we can see that there are $21$ possible combination of maneuvers.\\
As stated before, this approach do not account for combination of maneuvers not reaching the maximum curvature values. This is a limitation of the approach that can be overcome with a more complex exploration of the possible combination. However, the achieved algorithm yields a valid suboptimal solution.
%
\subsubsection*{Algorithm structure}
%
The algorithm was implemented in MATLAB. The implementation is divided in two main parts. The first part is a class that act as a collector for all the possible combination of maneuvers generated. This class will determine if a connection of a certain type is suitable/feasible for the specific initial and final point. The second part is a class for the single Duboid maneuver. Given the topology (shape) of the maneuver as in table \ref{tab:possiblecombination} the class will generate the maneuver matching if possible the boundary conditions and compute the length of the maneuver.\\
The first class will store a list of all feasible candidate-to-be-best maneuver and select the best one according to the minimum-time/minimum length criteria.
%
\subsubsection*{Arch length computation}
The Duboids lacks of a closed form solution at this stage of the development therefore the length of the maneuver is computed with a numerical optimization using \textit{fmincon} function of MATLAB.\\
The constraint on the final curvature is already satisfied by the definition of the problem. However, coordinate and orientation of the final point depends on the choice of the maneuver and length of segment $2$, $4$ and $6$.
In general, $P_7$ the "final" point do not coincide with $P_T$. Thus, we should compute a suitable triplet of length for the segment $L_2$, $L_4$ and $L_6$. To satisfy this constraint we can minimize a function cost such as:
%
\begin{equation}
  \label{eq:costfunction}
  \begin{split}
    \min_{L_2,L_4,L_6} \quad & (x_7-x_T)^2 + (y_7-y_T)^2 + (\theta_7-\theta_T)^2\\
    \text{s.t.} \quad & [x_7,y_7,\theta_7] = \mathrm{Duboid}(L_1,L_2,L_3)\\
    & 0 \leq L_2 \leq L_{2,max}\\
    & 0 \leq L_4 \leq L_{4,max}\\
    & 0 \leq L_6 \leq L_{6,max}
  \end{split}
\end{equation}
%
where $L_{2,max}$, $L_{4,max}$ and $L_{6,max}$ are the maximum length of the segments $L_2$, $L_4$ and $L_6$ respectively. For circular arcs the maximum length is a full circle with radius $R_{max}=1/\kappa_{max}$. For the straight line the maximum length is not trivial, however we can safely find a reasonable upper bound for our application (i.e. 10 times the distance between initial and final point).
%
\subsubsection*{Duboids results}
%
In this section we can see some of the results obtained with the Duboids algorithm. The results are shown in figures \ref{fig:DuboidsRes0}, \ref{fig:DuboidsRes1}, \ref{fig:DuboidsRes2}, \ref{fig:DuboidsRes3} and \ref{fig:DuboidsRes4}.\\
The first figure (Figure \ref{fig:DuboidsRes0}) shows results obtained with initial condition set to zero for position, heading and curvature. The final condition is set to $x = 40$, $y=20$, $\theta=0$ and $\kappa=0$.\\
%
\begin{figure}[ht]
  \centering
  % This file was created by matlab2tikz.
%
%The latest updates can be retrieved from
%  http://www.mathworks.com/matlabcentral/fileexchange/22022-matlab2tikz-matlab2tikz
%where you can also make suggestions and rate matlab2tikz.
%
\begin{tikzpicture}

\begin{axis}[%
width=\linewidth,
height=0.776\linewidth,
at={(0\linewidth,0\linewidth)},
scale only axis,
xmin=0,
xmax=40,
xlabel style={font=\color{white!15!black}},
xlabel={x(m)},
ymin=-5.7741935483871,
ymax=25.7741935483871,
ylabel style={font=\color{white!15!black}},
ylabel={y(m)},
axis background/.style={fill=white},
title style={font=\bfseries},
title={$L_{tot}$ = 45.1441, $k_{max}$ = 0.15, $J_{max}$ = 0.1, Type = [LSR]},
axis x line*=bottom,
axis y line*=left,
xmajorgrids,
xminorgrids,
ymajorgrids,
yminorgrids
]
\addplot [color=red, line width=2.0pt, forget plot]
  table[row sep=crcr]{%
0	0\\
0.0149999999998102	5.62499999994915e-08\\
0.029999999993925	4.49999999934911e-07\\
0.044999999953868	1.51874999888789e-06\\
0.0599999998056	3.59999999166857e-06\\
0.0749999994067383	7.03124996027265e-06\\
0.089999998523775	1.21499998576497e-05\\
0.104999996809296	1.92937495812201e-05\\
0.1199999937792	2.87999989335771e-05\\
0.134999988789917	4.10062475678123e-05\\
0.149999981015626	5.62499949148997e-05\\
0.164999969425477	7.48687400905784e-05\\
0.179999952760806	9.71999817791672e-05\\
0.194999929512356	0.000123581218091752\\
0.209999897897498	0.000154349946396183\\
0.224999855837445	0.000189843663116309\\
0.239999800934476	0.00023039986349791\\
0.254999730449155	0.000276356041338733\\
0.269999641277546	0.000328049688680096\\
0.284999529928439	0.00038581829545748\\
0.29999939250057	0.000449999349107562\\
0.314999224659834	0.000520930334129102\\
0.329999021616518	0.000598948731595155\\
0.344998778102512	0.000684392018614011\\
0.359998488348539	0.000777597667736327\\
0.374998146061372	0.000878903146305879\\
0.389997744401064	0.000988645915751365\\
0.404997275958169	0.00110716343081671\\
0.419996732730967	0.0012347931387273\\
0.434996106102694	0.00137187247828961\\
0.44999538681877	0.0015187388789216\\
0.464994564964025	0.0016757297596114\\
0.479993629939938	0.00184318252780169\\
0.494992570441861	0.00202143457819716\\
0.509991374436264	0.0022108232914926\\
0.524990029137964	0.00241168603301898\\
0.539988520987371	0.00262436015130494\\
0.554986835627727	0.00284918297655125\\
0.569984957882355	0.00308649181901552\\
0.584982871731902	0.00333662396730478\\
0.599980560291598	0.00359991668657317\\
0.614978005788507	0.00387670721662243\\
0.629975189538788	0.00416733276990235\\
0.64497209192496	0.00447213052940889\\
0.65996869237317	0.00479143764647726\\
0.674964969330465	0.00512559123846742\\
0.689960900242074	0.00547492838633954\\
0.704956461528692	0.00583978613211674\\
0.719951628563768	0.00622050147623264\\
0.734946375650808	0.00661741137476123\\
0.749940676000675	0.00703085273652633\\
0.764934501708899	0.00746116242008827\\
0.779927823733006	0.00790867723060517\\
0.794920611869835	0.00837373391656631\\
0.809912834732883	0.00885666916639496\\
0.824904459729649	0.00935781960491831\\
0.839895453038988	0.0098775217897017\\
0.854885779588485	0.0104161122072449\\
0.869875403031825	0.0109739272690379\\
0.88486428572619	0.0115513033074728\\
0.899852388709659	0.0121485765716116\\
0.914839671678623	0.0127660832228037\\
0.929826092965216	0.0134041593301545\\
0.944811609514758	0.0140631408658399\\
0.959796176863217	0.014743363700265\\
0.974779749114681	0.0154451635970648\\
0.989762278918851	0.016168876207944\\
1.00474371744855	0.0169148370673536\\
1.01972401437727	0.0176833815870013\\
1.03470311785666	0.0184748450501945\\
1.0496809744942	0.0192895626060114\\
1.06465752933066	0.0201278692632996\\
1.07963272581783	0.0209900998844985\\
1.09460650579607	0.0218765891792838\\
1.10957880947201	0.0227876716980311\\
1.12454957539624	0.0237236818250961\\
1.13951874044096	0.0246849537719092\\
1.15448623977781	0.0256718215698825\\
1.16945200685555	0.0266846190631253\\
1.18441597337789	0.0277236799009672\\
1.19937806928134	0.0287893375302854\\
1.21433822271301	0.0298819251876342\\
1.22929636000855	0.0310017758911742\\
1.24425240567004	0.0321492224323983\\
1.25920628234399	0.0333245973676532\\
1.27415791079928	0.0345282330094523\\
1.28910720990525	0.0357604614175797\\
1.30405409660975	0.0370216143899805\\
1.31899848591725	0.0383120234534375\\
1.33394029086702	0.0396320198540298\\
1.34887942251132	0.0409819345473722\\
1.36381578989365	0.0423620981886327\\
1.37874930002704	0.0437728411223257\\
1.3936798578724	0.0452144933718782\\
1.40860736631693	0.0466873846289674\\
1.42353172615255	0.048191844242627\\
1.43845283605441	0.0497282012081196\\
1.45337059255943	0.0512967841555736\\
1.46828489004495	0.0528979213383813\\
1.48319562070737	0.0545319406213571\\
1.49810267454089	0.0561991694686528\\
};
\addplot [color=green, line width=2.0pt, forget plot]
  table[row sep=crcr]{%
1.49810267454089	0.0561991694686528\\
1.51762787324192	0.0584342197618744\\
1.53714639839389	0.0607268187281459\\
1.5566580803779	0.0630769464444424\\
1.57616274963453	0.065484582487803\\
1.59566023666531	0.0679497059355137\\
1.61515037203418	0.0704722953652834\\
1.63463298636896	0.0730523288554365\\
1.65410791036283	0.0756897839850968\\
1.6735749747758	0.0783846378343869\\
1.6930340104362	0.0811368669846272\\
1.7124848482421	0.0839464475185351\\
1.73192731916283	0.0868133550204395\\
1.75136125424043	0.0897375645764868\\
1.77078648459111	0.0927190507748613\\
1.79020284140672	0.0957577877060037\\
1.80961015595625	0.0988537489628384\\
1.82900825958723	0.102006907641002\\
1.84839698372726	0.105217236339076\\
1.86777615988545	0.108484707158826\\
1.88714561965387	0.111809291705445\\
1.90650519470903	0.1151909610878\\
1.92585471681335	0.118629685918678\\
1.94519401781659	0.122125436315051\\
1.96452292965737	0.125678181898328\\
1.98384128436455	0.129287891794619\\
2.00314891405877	0.132954534635011\\
2.02244565095384	0.13667807855583\\
2.04173132735826	0.140458491198927\\
2.06100577567663	0.144295739711953\\
2.08026882841111	0.148189790748647\\
2.09952031816293	0.152140610469127\\
2.11876007763376	0.156148164540181\\
2.13798793962724	0.16021241813557\\
2.15720373705039	0.164333335936323\\
2.17640730291506	0.168510882131053\\
2.19559847033942	0.172745020416261\\
2.21477707254936	0.177035713996657\\
2.23394294287998	0.181382925585473\\
2.25309591477701	0.185786617404795\\
2.27223582179828	0.190246751185887\\
2.29136249761514	0.194763288169521\\
2.31047577601394	0.199336189106319\\
2.32957549089746	0.203965414257093\\
2.34866147628631	0.208650923393186\\
2.36773356632046	0.213392675796829\\
2.38679159526059	0.218190630261485\\
2.40583539748961	0.223044745092218\\
2.42486480751404	0.227954978106047\\
2.44387965996548	0.232921286632318\\
2.46287978960202	0.237943627513069\\
2.48186503130972	0.243021957103413\\
2.50083522010401	0.248156231271909\\
2.51979019113111	0.25334640540095\\
2.53872977966951	0.258592434387152\\
2.55765382113136	0.263894272641743\\
2.57656215106395	0.269251874090957\\
2.59545460515105	0.274665192176444\\
2.61433101921444	0.280134179855662\\
2.63319122921527	0.285658789602295\\
2.65203507125551	0.291238973406664\\
2.67086238157936	0.296874682776139\\
2.68967299657471	0.302565868735569\\
2.7084667527745	0.308312481827701\\
2.72724348685821	0.314114472113612\\
2.74600303565321	0.319971789173144\\
2.76474523613624	0.325884382105339\\
2.78346992543479	0.331852199528886\\
2.80217694082852	0.337875189582563\\
2.82086611975069	0.343953299925691\\
2.83953729978955	0.350086477738586\\
2.85819031868976	0.356274669723022\\
2.87682501435382	0.362517822102689\\
2.89544122484346	0.368815880623664\\
2.91403878838102	0.375168790554881\\
2.93261754335093	0.381576496688609\\
2.95117732830103	0.388038943340928\\
2.96971798194403	0.394556074352215\\
2.98823934315889	0.401127833087631\\
3.00674125099223	0.407754162437616\\
3.02522354465972	0.414435004818381\\
3.04368606354748	0.421170302172414\\
3.06212864721347	0.427959995968978\\
3.08055113538889	0.434804027204622\\
3.09895336797958	0.441702336403699\\
3.1173351850674	0.448654863618875\\
3.13569642691163	0.455661548431653\\
3.15403693395035	0.462722329952901\\
3.17235654680181	0.469837146823377\\
3.19065510626587	0.477005937214264\\
3.20893245332531	0.484228638827706\\
3.22718842914726	0.491505188897352\\
3.2454228750846	0.498835524188899\\
3.26363563267725	0.506219581000645\\
3.28182654365366	0.513657295164035\\
3.29999544993211	0.521148602044228\\
3.31814219362208	0.528693436540652\\
3.33626661702568	0.536291733087573\\
3.35436856263897	0.543943425654664\\
3.37244787315336	0.551648447747578\\
3.39050439145695	0.559406732408526\\
};
\addplot [color=red, line width=2.0pt, forget plot]
  table[row sep=crcr]{%
3.39050439145695	0.559406732408526\\
3.40427067880942	0.565364016903493\\
3.41802366222663	0.571351951023899\\
3.43176340888294	0.577370195916311\\
3.44548998806308	0.583418413473378\\
3.45920347113764	0.589496266333757\\
3.47290393153872	0.595603417881799\\
3.48659144473557	0.601739532246993\\
3.50026608821038	0.607904274303182\\
3.51392794143403	0.614097309667537\\
3.52757708584206	0.620318304699316\\
3.54121360481052	0.626566926498376\\
3.5548375836321	0.632842842903483\\
3.56844910949213	0.639145722490382\\
3.58204827144481	0.645475234569665\\
3.5956351603894	0.65183104918441\\
3.60920986904655	0.658212837107616\\
3.62277249193466	0.664620269839427\\
3.63632312534636	0.671053019604147\\
3.64986186732497	0.677510759347049\\
3.66338881764113	0.683993162730991\\
3.67690407776946	0.690499904132824\\
3.69040775086528	0.697030658639608\\
3.7038999417414	0.703585102044636\\
3.71738075684502	0.710162910843269\\
3.73085030423463	0.716763762228575\\
3.74430869355709	0.723387334086795\\
3.75775603602464	0.730033304992618\\
3.77119244439211	0.736701354204279\\
3.78461803293411	0.743391161658482\\
3.79803291742234	0.750102407965148\\
3.81143721510295	0.756834774401988\\
3.82483104467396	0.763587942908913\\
3.83821452626277	0.770361596082274\\
3.85158778140374	0.777155417168939\\
3.86495093301579	0.783969090060211\\
3.87830410538015	0.790802299285583\\
3.89164742411808	0.797654730006346\\
3.90498101616876	0.804526068009035\\
3.91830500976713	0.811415999698725\\
3.93161953442191	0.818324212092192\\
3.94492472089362	0.825250392810906\\
3.95822070117264	0.832194230073906\\
3.97150760845743	0.839155412690515\\
3.98478557713269	0.846133630052928\\
3.9980547427477	0.853128572128662\\
4.01131524199464	0.860139929452874\\
4.02456721268697	0.867167393120547\\
4.03781079373795	0.874210654778549\\
4.05104612513914	0.881269406617571\\
4.06427334793898	0.888343341363935\\
4.07749260422144	0.89543215227129\\
4.09070403708472	0.902535533112181\\
4.10390779062002	0.909653178169512\\
4.11710400989032	0.916784782227892\\
4.13029284090928	0.923930040564869\\
4.14347443062012	0.931088648942058\\
4.15664892687464	0.938260303596166\\
4.16981647841222	0.945444701229908\\
4.18297723483889	0.95264153900283\\
4.19613134660647	0.959850514522026\\
4.20927896499173	0.967071325832766\\
4.22242024207562	0.974303671409022\\
4.23555533072256	0.981547250143916\\
4.24868438455971	0.988801761340066\\
4.26180755795638	0.996066904699852\\
4.2749250060034	1.0033423803156\\
4.2880368844926	1.01062788865968\\
4.30114334989627	1.01792313057453\\
4.31424455934671	1.02522780726259\\
4.32734067061578	1.03254162027618\\
4.34043184209456	1.03986427150729\\
4.35351823277293	1.04719546317733\\
4.36660000221932	1.05453489782674\\
4.37967731056037	1.06188227830464\\
4.39275031846074	1.06923730775832\\
4.40581918710285	1.07659968962273\\
4.41888407816673	1.08396912760989\\
4.43194515380987	1.09134532569824\\
4.44500257664708	1.09872798812195\\
4.45805650973041	1.10611681936016\\
4.4711071165291	1.11351152412617\\
4.48415456090951	1.12091180735663\\
4.49719900711515	1.12831737420064\\
4.51024061974666	1.13572793000877\\
4.52327956374183	1.14314318032214\\
4.53631600435571	1.15056283086141\\
4.54935010714064	1.15798658751568\\
4.56238203792636	1.16541415633145\\
4.57541196280009	1.17284524350153\\
4.58844004808674	1.18027955535382\\
4.60146646032895	1.18771679834024\\
4.61449136626734	1.19515667902546\\
4.6275149328206	1.20259890407572\\
4.64053732706573	1.21004318024759\\
4.65355871621821	1.21748921437673\\
4.6665792676122	1.2249367133666\\
4.67959914868075	1.23238538417719\\
4.692618526936	1.23983493381372\\
4.70563756994944	1.24728506931536\\
4.71865644533208	1.2547354977439\\
};
\addplot [color=green, line width=2.0pt, forget plot]
  table[row sep=crcr]{%
4.71865644533208	1.2547354977439\\
5.02428322855391	1.42964079600883\\
5.32991001177575	1.60454609427375\\
5.63553679499759	1.77945139253868\\
5.94116357821943	1.9543566908036\\
6.24679036144126	2.12926198906853\\
6.5524171446631	2.30416728733345\\
6.85804392788494	2.47907258559838\\
7.16367071110678	2.65397788386331\\
7.46929749432861	2.82888318212823\\
7.77492427755045	3.00378848039316\\
8.08055106077229	3.17869377865808\\
8.38617784399412	3.35359907692301\\
8.69180462721596	3.52850437518793\\
8.9974314104378	3.70340967345286\\
9.30305819365964	3.87831497171778\\
9.60868497688147	4.05322026998271\\
9.91431176010331	4.22812556824763\\
10.2199385433251	4.40303086651256\\
10.525565326547	4.57793616477749\\
10.8311921097688	4.75284146304241\\
11.1368188929907	4.92774676130734\\
11.4424456762125	5.10265205957226\\
11.7480724594343	5.27755735783719\\
12.0536992426562	5.45246265610211\\
12.359326025878	5.62736795436704\\
12.6649528090998	5.80227325263196\\
12.9705795923217	5.97717855089689\\
13.2762063755435	6.15208384916182\\
13.5818331587654	6.32698914742674\\
13.8874599419872	6.50189444569167\\
14.193086725209	6.67679974395659\\
14.4987135084309	6.85170504222152\\
14.8043402916527	7.02661034048644\\
15.1099670748745	7.20151563875137\\
15.4155938580964	7.37642093701629\\
15.7212206413182	7.55132623528122\\
16.0268474245401	7.72623153354614\\
16.3324742077619	7.90113683181107\\
16.6381009909837	8.076042130076\\
16.9437277742056	8.25094742834092\\
17.2493545574274	8.42585272660585\\
17.5549813406492	8.60075802487077\\
17.8606081238711	8.7756633231357\\
18.1662349070929	8.95056862140062\\
18.4718616903148	9.12547391966555\\
18.7774884735366	9.30037921793047\\
19.0831152567584	9.4752845161954\\
19.3887420399803	9.65018981446032\\
19.6943688232021	9.82509511272525\\
19.9999956064239	10.0000004109902\\
20.3056223896458	10.1749057092551\\
20.6112491728676	10.34981100752\\
20.9168759560895	10.524716305785\\
21.2225027393113	10.6996216040499\\
21.5281295225331	10.8745269023148\\
21.833756305755	11.0494322005797\\
22.1393830889768	11.2243374988447\\
22.4450098721986	11.3992427971096\\
22.7506366554205	11.5741480953745\\
23.0562634386423	11.7490533936394\\
23.3618902218642	11.9239586919044\\
23.667517005086	12.0988639901693\\
23.9731437883078	12.2737692884342\\
24.2787705715297	12.4486745866991\\
24.5843973547515	12.6235798849641\\
24.8900241379733	12.798485183229\\
25.1956509211952	12.9733904814939\\
25.501277704417	13.1482957797588\\
25.8069044876389	13.3232010780238\\
26.1125312708607	13.4981063762887\\
26.4181580540825	13.6730116745536\\
26.7237848373044	13.8479169728185\\
27.0294116205262	14.0228222710835\\
27.335038403748	14.1977275693484\\
27.6406651869699	14.3726328676133\\
27.9462919701917	14.5475381658782\\
28.2519187534136	14.7224434641432\\
28.5575455366354	14.8973487624081\\
28.8631723198572	15.072254060673\\
29.1687991030791	15.2471593589379\\
29.4744258863009	15.4220646572029\\
29.7800526695227	15.5969699554678\\
30.0856794527446	15.7718752537327\\
30.3913062359664	15.9467805519976\\
30.6969330191882	16.1216858502626\\
31.0025598024101	16.2965911485275\\
31.3081865856319	16.4714964467924\\
31.6138133688538	16.6464017450573\\
31.9194401520756	16.8213070433223\\
32.2250669352974	16.9962123415872\\
32.5306937185193	17.1711176398521\\
32.8363205017411	17.346022938117\\
33.1419472849629	17.520928236382\\
33.4475740681848	17.6958335346469\\
33.7532008514066	17.8707388329118\\
34.0588276346285	18.0456441311767\\
34.3644544178503	18.2205494294417\\
34.6700812010721	18.3954547277066\\
34.975707984294	18.5703600259715\\
35.2813347675158	18.7452653242364\\
};
\addplot [color=red, line width=2.0pt, forget plot]
  table[row sep=crcr]{%
35.2813347675158	18.7452653242364\\
35.2943536428985	18.752715752665\\
35.3073726859119	18.7601658881666\\
35.3203920641671	18.7676154378032\\
35.3334119452357	18.7750641086137\\
35.3464324966297	18.7825116076036\\
35.3594538857822	18.7899576417328\\
35.3724762800273	18.7974019179046\\
35.3854998465805	18.8048441429549\\
35.3985247525189	18.8122840236401\\
35.4115511647611	18.8197212666265\\
35.4245792500478	18.8271555784788\\
35.4376091749215	18.8345866656489\\
35.4506411057072	18.8420142344647\\
35.4636752084922	18.8494379911189\\
35.4767116491061	18.8568576416582\\
35.4897505931012	18.8642728919716\\
35.5027922057327	18.8716834477797\\
35.5158366519384	18.8790890146237\\
35.5288840963188	18.8864892978542\\
35.5419347031175	18.8938840026202\\
35.5549886362008	18.9012728338584\\
35.568046059038	18.9086554962821\\
35.5811071346812	18.9160316943705\\
35.594172025745	18.9234011323576\\
35.6072408943871	18.930763514222\\
35.6203139022875	18.9381185436757\\
35.6333912106286	18.9454659241536\\
35.646472980075	18.952805358803\\
35.6595593707533	18.9601365504731\\
35.6726505422321	18.9674592017042\\
35.6857466535012	18.9747730147178\\
35.6988478629516	18.9820776914058\\
35.7119543283553	18.9893729333207\\
35.7250662068445	18.9966584416647\\
35.7381836548915	19.0039339172805\\
35.7513068282882	19.0111990606403\\
35.7644358821253	19.0184535718364\\
35.7775709707723	19.0256971505713\\
35.7907122478562	19.0329294961476\\
35.8038598662414	19.0401503074583\\
35.817013978009	19.0473592829775\\
35.8301747344357	19.0545561207504\\
35.8433422859733	19.0617405183842\\
35.8565167822278	19.0689121730383\\
35.8696983719386	19.0760707814155\\
35.8828872029576	19.0832160397525\\
35.8960834222279	19.0903476438108\\
35.9092871757632	19.0974652888682\\
35.9224986086264	19.1045686697091\\
35.9357178649089	19.1116574806164\\
35.9489450877088	19.1187314153628\\
35.9621804191099	19.1257901672018\\
35.9754240001609	19.1328334288598\\
35.9886759708533	19.1398608925275\\
36.0019364701002	19.1468722498517\\
36.0152056357152	19.1538671919274\\
36.0284836043905	19.1608454092898\\
36.0417705116752	19.1678065919064\\
36.0550664919543	19.1747504291694\\
36.068371678426	19.1816766098882\\
36.0816862030808	19.1885848222816\\
36.0950101966791	19.1954747539713\\
36.1083437887298	19.202346091974\\
36.1216871074677	19.2091985226948\\
36.1350402798321	19.2160317319201\\
36.1484034314442	19.2228454048114\\
36.1617766865851	19.2296392258981\\
36.1751601681739	19.2364128790714\\
36.1885539977449	19.2431660475784\\
36.2019582954255	19.2498984140152\\
36.2153731799138	19.2566096603219\\
36.2287987684558	19.2632994677761\\
36.2422351768232	19.2699675169877\\
36.2556825192908	19.2766134878936\\
36.2691409086133	19.2832370597518\\
36.2826104560029	19.2898379111371\\
36.2960912711065	19.2964157199357\\
36.3095834619826	19.3029701633407\\
36.3230871350784	19.3095009178475\\
36.3366023952068	19.3160076592494\\
36.3501293455229	19.3224900626333\\
36.3636680875015	19.3289478023762\\
36.3772187209132	19.3353805521409\\
36.3907813438013	19.3417879848727\\
36.4043560524585	19.3481697727959\\
36.4179429414031	19.3545255874107\\
36.4315421033558	19.36085509949\\
36.4451536292158	19.3671579790769\\
36.4587776080374	19.373433895482\\
36.4724141270058	19.379682517281\\
36.4860632714139	19.3859035123128\\
36.4997251246375	19.3920965476772\\
36.5133997681123	19.3982612897334\\
36.5270872813092	19.4043974040985\\
36.5407877417102	19.4105045556466\\
36.5545012247848	19.416582408507\\
36.5682278039649	19.422630626064\\
36.5819675506213	19.4286488709564\\
36.5957205340385	19.4346368050769\\
36.6094868213909	19.4405940895718\\
};
\addplot [color=green, line width=2.0pt, forget plot]
  table[row sep=crcr]{%
36.6094868213909	19.4405940895718\\
36.6275434185365	19.448352407971\\
36.6456228080911	19.4560574633369\\
36.6637248329407	19.4637091887111\\
36.6818493357748	19.4713075175982\\
36.6999961590873	19.4788523839671\\
36.7181651451782	19.4863437222511\\
36.736356136155	19.493781467349\\
36.7545689739339	19.5011655546249\\
36.7728035002412	19.5084959199096\\
36.7910595566148	19.5157724995005\\
36.8093369844055	19.5229952301625\\
36.8276356247783	19.5301640491284\\
36.8459553187139	19.5372788940997\\
36.8642959070101	19.5443397032466\\
36.882657230283	19.5513464152092\\
36.9010391289686	19.5582989690976\\
36.9194414433241	19.5651973044925\\
36.9378640134292	19.5720413614458\\
36.9563066791877	19.5788310804812\\
36.9747692803287	19.5855664025945\\
36.9932516564079	19.5922472692541\\
37.0117536468095	19.5988736224019\\
37.0302750907469	19.6054454044534\\
37.0488158272646	19.6119625582983\\
37.0673756952396	19.6184250273011\\
37.0859545333823	19.6248327553015\\
37.1045521802384	19.6311856866148\\
37.1231684741904	19.6374837660327\\
37.1418032534583	19.6437269388235\\
37.1604563561017	19.6499151507325\\
37.1791276200211	19.6560483479829\\
37.1978168829589	19.6621264772757\\
37.2165239825013	19.6681494857907\\
37.2352487560793	19.6741173211866\\
37.2539910409704	19.6800299316014\\
37.2727506743001	19.6858872656534\\
37.2915274930428	19.691689272441\\
37.3103213340237	19.6974359015434\\
37.3291320339203	19.7031271030211\\
37.3479594292633	19.7087628274163\\
37.3668033564383	19.7143430257534\\
37.3856636516875	19.719867649539\\
37.4045401511106	19.7253366507631\\
37.4234326906666	19.7307499818987\\
37.4423411061751	19.7361075959027\\
37.4612652333177	19.7414094462165\\
37.4802049076395	19.7466554867655\\
37.4991599645505	19.7518456719606\\
37.518130239327	19.756979956698\\
37.537115567113	19.7620582963594\\
37.5561157829218	19.7670806468131\\
37.5751307216372	19.7720469644136\\
37.5941602180151	19.7769572060025\\
37.6132041066849	19.7818113289088\\
37.6322622221511	19.786609290949\\
37.6513343987941	19.7913510504277\\
37.6704204708727	19.796036566138\\
37.6895202725244	19.8006657973617\\
37.7086336377678	19.8052387038698\\
37.7277604005034	19.8097552459226\\
37.7469003945154	19.8142153842705\\
37.7660534534728	19.8186190801537\\
37.7852194109314	19.8229662953032\\
37.8043981003345	19.8272569919407\\
37.8235893550152	19.8314911327791\\
37.8427930081971	19.8356686810228\\
37.8620088929962	19.8397896003679\\
37.881236842422	19.8438538550027\\
37.9004766893794	19.847861409608\\
37.9197282666698	19.8518122293571\\
37.9389914069927	19.8557062799166\\
37.9582659429471	19.8595435274463\\
37.9775517070328	19.8633239385996\\
37.9968485316523	19.8670474805239\\
38.0161562491119	19.8707141208609\\
38.0354746916231	19.8743238277466\\
38.0548036913043	19.8778765698117\\
38.0741430801823	19.8813723161823\\
38.0934926901933	19.8848110364794\\
38.1128523531848	19.8881927008199\\
38.1322219009172	19.8915172798162\\
38.1516011650646	19.8947847445771\\
38.1709899772169	19.8979950667076\\
38.190388168881	19.9011482183092\\
38.2097955714822	19.9042441719803\\
38.2292120163658	19.9072829008164\\
38.2486373347986	19.9102643784104\\
38.2680713579702	19.9131885788523\\
38.2875139169946	19.9160554767304\\
38.3069648429115	19.9188650471306\\
38.3264239666881	19.9216172656371\\
38.3458911192201	19.9243121083326\\
38.3653661313337	19.9269495517981\\
38.3848488337866	19.9295295731139\\
38.4043390572698	19.9320521498588\\
38.4238366324089	19.9345172601111\\
38.4433413897655	19.9369248824485\\
38.4628531598389	19.9392749959483\\
38.4823717730674	19.9415675801873\\
38.50189705983	19.9438026152425\\
};
\addplot [color=red, line width=2.0pt, forget plot]
  table[row sep=crcr]{%
38.50189705983	19.9438026152425\\
38.5168041158082	19.9454698249139\\
38.5317148485725	19.9471038250163\\
38.5466291481177	19.948704943014\\
38.5615469066404	19.9502735067718\\
38.5764680185186	19.9518098445434\\
38.5913923802895	19.9533142849589\\
38.6063198906287	19.9547871570139\\
38.6212504503285	19.9562287900574\\
38.6361839622766	19.9576395137812\\
38.6511203314343	19.9590196582089\\
38.6660594648151	19.9603695536851\\
38.6810012714628	19.9616895308652\\
38.6959456624302	19.9629799207048\\
38.710892550757	19.9642410544501\\
38.7258418514481	19.965473263628\\
38.7407934814517	19.9666768800367\\
38.7557473596375	19.9678522357359\\
38.7707034067751	19.9689996630382\\
38.7856615455111	19.9701194945002\\
38.8006217003482	19.9712120629134\\
38.8155837976225	19.9722777012961\\
38.8305477654814	19.9733167428849\\
38.845513533862	19.9743295211268\\
38.8604810344683	19.9753163696711\\
38.8754502007495	19.9762776223622\\
38.8904209678778	19.9772136132315\\
38.9053932727257	19.9781246764905\\
38.9203670538443	19.9790111465236\\
38.9353422514406	19.9798733578813\\
38.9503188073554	19.9807116452733\\
38.9652966650409	19.9815263435622\\
38.9802757695384	19.9823177877569\\
38.9952560674557	19.9830863130065\\
39.010237506945	19.9838322545944\\
39.0252200376801	19.9845559479323\\
39.0402036108343	19.9852577285549\\
39.0551881790578	19.9859379321137\\
39.070173696455	19.9865968943726\\
39.0851601185624	19.9872349512021\\
39.1001474023257	19.9878524385742\\
39.1151355060774	19.9884496925583\\
39.1301243895145	19.9890270493156\\
39.1451140136754	19.9895848450953\\
39.1601043409177	19.9901234162299\\
39.1750953348955	19.9906430991309\\
39.1900869605369	19.9911442302848\\
39.2050791840212	19.9916271462492\\
39.2200719727563	19.992092183649\\
39.235065295356	19.9925396791728\\
39.2500591216178	19.9929699695689\\
39.2650534224995	19.9933833916426\\
39.280048170097	19.9937802822524\\
39.2950433376218	19.9941609783074\\
39.3100388993778	19.9945258167635\\
39.3250348307388	19.9948751346212\\
39.3400311081259	19.9952092689225\\
39.3550277089848	19.9955285567486\\
39.3700246117631	19.9958333352167\\
39.3850217958872	19.9961239414782\\
39.4000192417401	19.9964007127161\\
39.4150169306385	19.9966639861429\\
39.4300148448098	19.9969140989985\\
39.4450129673697	19.997151388548\\
39.4600112822992	19.99737619208\\
39.4750097744222	19.9975888469048\\
39.4900084293823	19.9977896903527\\
39.5050072336203	19.9979790597721\\
39.5200061743515	19.9981572925285\\
39.5350052395428	19.9983247260025\\
39.55000441789	19.9984816975888\\
39.565003698795	19.998628544695\\
39.580003072343	19.99876560474\\
39.59500252928	19.9988932151532\\
39.6100020609895	19.9990117133734\\
39.6250016594704	19.999121436848\\
39.6400013173135	19.9992227230316\\
39.6550010276794	19.9993159093857\\
39.6700007842753	19.9994013333776\\
39.6850005813324	19.99947933248\\
39.7000004135828	19.9995502441698\\
39.7150002762375	19.9996144059282\\
39.7300001649627	19.9996721552397\\
39.7450000758576	19.9997238295918\\
39.7600000054314	19.9997697664743\\
39.7749999505806	19.9998103033794\\
39.7899999085662	19.9998457778007\\
39.8049998769909	19.9998765272337\\
39.8199998537763	19.9999028891746\\
39.8349998371404	19.9999252011209\\
39.8499998255742	19.9999438005707\\
39.8649998178195	19.9999590250227\\
39.8799998128459	19.9999712119759\\
39.894999809828	19.9999806989299\\
39.9099998081227	19.9999878233842\\
39.9249998072463	19.9999929228387\\
39.9399998068518	19.9999963347933\\
39.9549998067062	19.9999983967479\\
39.9699998066675	19.9999994462025\\
39.9849998066622	19.9999998206571\\
39.999999806662	19.9999998576117\\
};
\addplot [color=red, line width=2.0pt, only marks, mark size=2.5pt, mark=*, mark options={solid, fill=red, red}, forget plot]
  table[row sep=crcr]{%
1.49810267454089	0.0561991694686528\\
};
\addplot [color=red, line width=2.0pt, only marks, mark size=2.5pt, mark=*, mark options={solid, fill=red, red}, forget plot]
  table[row sep=crcr]{%
3.39050439145695	0.559406732408526\\
};
\addplot [color=red, line width=2.0pt, only marks, mark size=2.5pt, mark=*, mark options={solid, fill=red, red}, forget plot]
  table[row sep=crcr]{%
4.71865644533208	1.2547354977439\\
};
\addplot [color=red, line width=2.0pt, only marks, mark size=2.5pt, mark=*, mark options={solid, fill=red, red}, forget plot]
  table[row sep=crcr]{%
35.2813347675158	18.7452653242364\\
};
\addplot [color=red, line width=2.0pt, only marks, mark size=2.5pt, mark=*, mark options={solid, fill=red, red}, forget plot]
  table[row sep=crcr]{%
36.6094868213909	19.4405940895718\\
};
\addplot [color=red, line width=2.0pt, only marks, mark size=2.5pt, mark=*, mark options={solid, fill=red, red}, forget plot]
  table[row sep=crcr]{%
38.50189705983	19.9438026152425\\
};
\addplot [color=red, line width=2.0pt, only marks, mark size=2.5pt, mark=*, mark options={solid, fill=red, red}, forget plot]
  table[row sep=crcr]{%
39.999999806662	19.9999998576117\\
};
\addplot [color=blue, line width=2.0pt, only marks, mark size=2.5pt, mark=*, mark options={solid, fill=blue, blue}, forget plot]
  table[row sep=crcr]{%
0	0\\
};
\addplot [color=blue, line width=2.0pt, only marks, mark size=2.5pt, mark=*, mark options={solid, fill=blue, blue}, forget plot]
  table[row sep=crcr]{%
40	20\\
};
\end{axis}
\end{tikzpicture}%%
  \caption{Duboids solution example 1}
  \label{fig:DuboidsRes0}
\end{figure}
%
The second example is a trivial connection with a straight line (Figure \ref{fig:DuboidsRes1}).\\
%
\begin{figure}[ht]
  \centering
  % This file was created by matlab2tikz.
%
%The latest updates can be retrieved from
%  http://www.mathworks.com/matlabcentral/fileexchange/22022-matlab2tikz-matlab2tikz
%where you can also make suggestions and rate matlab2tikz.
%
\begin{tikzpicture}

\begin{axis}[%
width=\linewidth,
height=0.776\linewidth,
at={(0\linewidth,0\linewidth)},
scale only axis,
xmin=0,
xmax=40,
xlabel style={font=\color{white!15!black}},
xlabel={x(m)},
ymin=-15.7741935483871,
ymax=15.7741935483871,
ylabel style={font=\color{white!15!black}},
ylabel={y(m)},
axis background/.style={fill=white},
title style={font=\bfseries},
title={$L_{tot}$ = 40, $k_{max}$ = 0.15, $J_{max}$ = 0.08, Type = [S00]},
axis x line*=bottom,
axis y line*=left,
xmajorgrids,
xminorgrids,
ymajorgrids,
yminorgrids
]
\addplot [color=red, line width=2.0pt, forget plot]
  table[row sep=crcr]{%
0	0\\
0	0\\
0	0\\
0	0\\
0	0\\
0	0\\
0	0\\
0	0\\
0	0\\
0	0\\
0	0\\
0	0\\
0	0\\
0	0\\
0	0\\
0	0\\
0	0\\
0	0\\
0	0\\
0	0\\
0	0\\
0	0\\
0	0\\
0	0\\
0	0\\
0	0\\
0	0\\
0	0\\
0	0\\
0	0\\
0	0\\
0	0\\
0	0\\
0	0\\
0	0\\
0	0\\
0	0\\
0	0\\
0	0\\
0	0\\
0	0\\
0	0\\
0	0\\
0	0\\
0	0\\
0	0\\
0	0\\
0	0\\
0	0\\
0	0\\
0	0\\
0	0\\
0	0\\
0	0\\
0	0\\
0	0\\
0	0\\
0	0\\
0	0\\
0	0\\
0	0\\
0	0\\
0	0\\
0	0\\
0	0\\
0	0\\
0	0\\
0	0\\
0	0\\
0	0\\
0	0\\
0	0\\
0	0\\
0	0\\
0	0\\
0	0\\
0	0\\
0	0\\
0	0\\
0	0\\
0	0\\
0	0\\
0	0\\
0	0\\
0	0\\
0	0\\
0	0\\
0	0\\
0	0\\
0	0\\
0	0\\
0	0\\
0	0\\
0	0\\
0	0\\
0	0\\
0	0\\
0	0\\
0	0\\
0	0\\
0	0\\
};
\addplot [color=green, line width=2.0pt, forget plot]
  table[row sep=crcr]{%
0	0\\
0.399999997019739	0\\
0.799999994039479	0\\
1.19999999105922	0\\
1.59999998807896	0\\
1.9999999850987	0\\
2.39999998211844	0\\
2.79999997913818	0\\
3.19999997615791	0\\
3.59999997317765	0\\
3.99999997019739	0\\
4.39999996721713	0\\
4.79999996423687	0\\
5.19999996125661	0\\
5.59999995827635	0\\
5.99999995529609	0\\
6.39999995231583	0\\
6.79999994933557	0\\
7.19999994635531	0\\
7.59999994337505	0\\
7.99999994039479	0\\
8.39999993741453	0\\
8.79999993443426	0\\
9.19999993145401	0\\
9.59999992847374	0\\
9.99999992549348	0\\
10.3999999225132	0\\
10.799999919533	0\\
11.1999999165527	0\\
11.5999999135724	0\\
11.9999999105922	0\\
12.3999999076119	0\\
12.7999999046317	0\\
13.1999999016514	0\\
13.5999998986711	0\\
13.9999998956909	0\\
14.3999998927106	0\\
14.7999998897304	0\\
15.1999998867501	0\\
15.5999998837698	0\\
15.9999998807896	0\\
16.3999998778093	0\\
16.7999998748291	0\\
17.1999998718488	0\\
17.5999998688685	0\\
17.9999998658883	0\\
18.399999862908	0\\
18.7999998599278	0\\
19.1999998569475	0\\
19.5999998539672	0\\
19.999999850987	0\\
20.3999998480067	0\\
20.7999998450264	0\\
21.1999998420462	0\\
21.5999998390659	0\\
21.9999998360857	0\\
22.3999998331054	0\\
22.7999998301251	0\\
23.1999998271449	0\\
23.5999998241646	0\\
23.9999998211844	0\\
24.3999998182041	0\\
24.7999998152238	0\\
25.1999998122436	0\\
25.5999998092633	0\\
25.9999998062831	0\\
26.3999998033028	0\\
26.7999998003225	0\\
27.1999997973423	0\\
27.599999794362	0\\
27.9999997913818	0\\
28.3999997884015	0\\
28.7999997854212	0\\
29.199999782441	0\\
29.5999997794607	0\\
29.9999997764805	0\\
30.3999997735002	0\\
30.7999997705199	0\\
31.1999997675397	0\\
31.5999997645594	0\\
31.9999997615791	0\\
32.3999997585989	0\\
32.7999997556186	0\\
33.1999997526384	0\\
33.5999997496581	0\\
33.9999997466778	0\\
34.3999997436976	0\\
34.7999997407173	0\\
35.1999997377371	0\\
35.5999997347568	0\\
35.9999997317765	0\\
36.3999997287963	0\\
36.799999725816	0\\
37.1999997228358	0\\
37.5999997198555	0\\
37.9999997168752	0\\
38.399999713895	0\\
38.7999997109147	0\\
39.1999997079345	0\\
39.5999997049542	0\\
39.9999997019739	0\\
};
\addplot [color=red, line width=2.0pt, forget plot]
  table[row sep=crcr]{%
39.9999997019739	0\\
39.9999997019739	0\\
39.9999997019739	0\\
39.9999997019739	0\\
39.9999997019739	0\\
39.9999997019739	0\\
39.9999997019739	0\\
39.9999997019739	0\\
39.9999997019739	0\\
39.9999997019739	0\\
39.9999997019739	0\\
39.9999997019739	0\\
39.9999997019739	0\\
39.9999997019739	0\\
39.9999997019739	0\\
39.9999997019739	0\\
39.9999997019739	0\\
39.9999997019739	0\\
39.9999997019739	0\\
39.9999997019739	0\\
39.9999997019739	0\\
39.9999997019739	0\\
39.9999997019739	0\\
39.9999997019739	0\\
39.9999997019739	0\\
39.9999997019739	0\\
39.9999997019739	0\\
39.9999997019739	0\\
39.9999997019739	0\\
39.9999997019739	0\\
39.9999997019739	0\\
39.9999997019739	0\\
39.9999997019739	0\\
39.9999997019739	0\\
39.9999997019739	0\\
39.9999997019739	0\\
39.9999997019739	0\\
39.9999997019739	0\\
39.9999997019739	0\\
39.9999997019739	0\\
39.9999997019739	0\\
39.9999997019739	0\\
39.9999997019739	0\\
39.9999997019739	0\\
39.9999997019739	0\\
39.9999997019739	0\\
39.9999997019739	0\\
39.9999997019739	0\\
39.9999997019739	0\\
39.9999997019739	0\\
39.9999997019739	0\\
39.9999997019739	0\\
39.9999997019739	0\\
39.9999997019739	0\\
39.9999997019739	0\\
39.9999997019739	0\\
39.9999997019739	0\\
39.9999997019739	0\\
39.9999997019739	0\\
39.9999997019739	0\\
39.9999997019739	0\\
39.9999997019739	0\\
39.9999997019739	0\\
39.9999997019739	0\\
39.9999997019739	0\\
39.9999997019739	0\\
39.9999997019739	0\\
39.9999997019739	0\\
39.9999997019739	0\\
39.9999997019739	0\\
39.9999997019739	0\\
39.9999997019739	0\\
39.9999997019739	0\\
39.9999997019739	0\\
39.9999997019739	0\\
39.9999997019739	0\\
39.9999997019739	0\\
39.9999997019739	0\\
39.9999997019739	0\\
39.9999997019739	0\\
39.9999997019739	0\\
39.9999997019739	0\\
39.9999997019739	0\\
39.9999997019739	0\\
39.9999997019739	0\\
39.9999997019739	0\\
39.9999997019739	0\\
39.9999997019739	0\\
39.9999997019739	0\\
39.9999997019739	0\\
39.9999997019739	0\\
39.9999997019739	0\\
39.9999997019739	0\\
39.9999997019739	0\\
39.9999997019739	0\\
39.9999997019739	0\\
39.9999997019739	0\\
39.9999997019739	0\\
39.9999997019739	0\\
39.9999997019739	0\\
39.9999997019739	0\\
};
\addplot [color=green, line width=2.0pt, forget plot]
  table[row sep=crcr]{%
39.9999997019739	0\\
39.9999997019739	0\\
39.9999997019739	0\\
39.9999997019739	0\\
39.9999997019739	0\\
39.9999997019739	0\\
39.9999997019739	0\\
39.9999997019739	0\\
39.9999997019739	0\\
39.9999997019739	0\\
39.9999997019739	0\\
39.9999997019739	0\\
39.9999997019739	0\\
39.9999997019739	0\\
39.9999997019739	0\\
39.9999997019739	0\\
39.9999997019739	0\\
39.9999997019739	0\\
39.9999997019739	0\\
39.9999997019739	0\\
39.9999997019739	0\\
39.9999997019739	0\\
39.9999997019739	0\\
39.9999997019739	0\\
39.9999997019739	0\\
39.9999997019739	0\\
39.9999997019739	0\\
39.9999997019739	0\\
39.9999997019739	0\\
39.9999997019739	0\\
39.9999997019739	0\\
39.9999997019739	0\\
39.9999997019739	0\\
39.9999997019739	0\\
39.9999997019739	0\\
39.9999997019739	0\\
39.9999997019739	0\\
39.9999997019739	0\\
39.9999997019739	0\\
39.9999997019739	0\\
39.9999997019739	0\\
39.9999997019739	0\\
39.9999997019739	0\\
39.9999997019739	0\\
39.9999997019739	0\\
39.9999997019739	0\\
39.9999997019739	0\\
39.9999997019739	0\\
39.9999997019739	0\\
39.9999997019739	0\\
39.9999997019739	0\\
39.9999997019739	0\\
39.9999997019739	0\\
39.9999997019739	0\\
39.9999997019739	0\\
39.9999997019739	0\\
39.9999997019739	0\\
39.9999997019739	0\\
39.9999997019739	0\\
39.9999997019739	0\\
39.9999997019739	0\\
39.9999997019739	0\\
39.9999997019739	0\\
39.9999997019739	0\\
39.9999997019739	0\\
39.9999997019739	0\\
39.9999997019739	0\\
39.9999997019739	0\\
39.9999997019739	0\\
39.9999997019739	0\\
39.9999997019739	0\\
39.9999997019739	0\\
39.9999997019739	0\\
39.9999997019739	0\\
39.9999997019739	0\\
39.9999997019739	0\\
39.9999997019739	0\\
39.9999997019739	0\\
39.9999997019739	0\\
39.9999997019739	0\\
39.9999997019739	0\\
39.9999997019739	0\\
39.9999997019739	0\\
39.9999997019739	0\\
39.9999997019739	0\\
39.9999997019739	0\\
39.9999997019739	0\\
39.9999997019739	0\\
39.9999997019739	0\\
39.9999997019739	0\\
39.9999997019739	0\\
39.9999997019739	0\\
39.9999997019739	0\\
39.9999997019739	0\\
39.9999997019739	0\\
39.9999997019739	0\\
39.9999997019739	0\\
39.9999997019739	0\\
39.9999997019739	0\\
39.9999997019739	0\\
39.9999997019739	0\\
};
\addplot [color=red, line width=2.0pt, forget plot]
  table[row sep=crcr]{%
39.9999997019739	0\\
39.9999997019739	0\\
39.9999997019739	0\\
39.9999997019739	0\\
39.9999997019739	0\\
39.9999997019739	0\\
39.9999997019739	0\\
39.9999997019739	0\\
39.9999997019739	0\\
39.9999997019739	0\\
39.9999997019739	0\\
39.9999997019739	0\\
39.9999997019739	0\\
39.9999997019739	0\\
39.9999997019739	0\\
39.9999997019739	0\\
39.9999997019739	0\\
39.9999997019739	0\\
39.9999997019739	0\\
39.9999997019739	0\\
39.9999997019739	0\\
39.9999997019739	0\\
39.9999997019739	0\\
39.9999997019739	0\\
39.9999997019739	0\\
39.9999997019739	0\\
39.9999997019739	0\\
39.9999997019739	0\\
39.9999997019739	0\\
39.9999997019739	0\\
39.9999997019739	0\\
39.9999997019739	0\\
39.9999997019739	0\\
39.9999997019739	0\\
39.9999997019739	0\\
39.9999997019739	0\\
39.9999997019739	0\\
39.9999997019739	0\\
39.9999997019739	0\\
39.9999997019739	0\\
39.9999997019739	0\\
39.9999997019739	0\\
39.9999997019739	0\\
39.9999997019739	0\\
39.9999997019739	0\\
39.9999997019739	0\\
39.9999997019739	0\\
39.9999997019739	0\\
39.9999997019739	0\\
39.9999997019739	0\\
39.9999997019739	0\\
39.9999997019739	0\\
39.9999997019739	0\\
39.9999997019739	0\\
39.9999997019739	0\\
39.9999997019739	0\\
39.9999997019739	0\\
39.9999997019739	0\\
39.9999997019739	0\\
39.9999997019739	0\\
39.9999997019739	0\\
39.9999997019739	0\\
39.9999997019739	0\\
39.9999997019739	0\\
39.9999997019739	0\\
39.9999997019739	0\\
39.9999997019739	0\\
39.9999997019739	0\\
39.9999997019739	0\\
39.9999997019739	0\\
39.9999997019739	0\\
39.9999997019739	0\\
39.9999997019739	0\\
39.9999997019739	0\\
39.9999997019739	0\\
39.9999997019739	0\\
39.9999997019739	0\\
39.9999997019739	0\\
39.9999997019739	0\\
39.9999997019739	0\\
39.9999997019739	0\\
39.9999997019739	0\\
39.9999997019739	0\\
39.9999997019739	0\\
39.9999997019739	0\\
39.9999997019739	0\\
39.9999997019739	0\\
39.9999997019739	0\\
39.9999997019739	0\\
39.9999997019739	0\\
39.9999997019739	0\\
39.9999997019739	0\\
39.9999997019739	0\\
39.9999997019739	0\\
39.9999997019739	0\\
39.9999997019739	0\\
39.9999997019739	0\\
39.9999997019739	0\\
39.9999997019739	0\\
39.9999997019739	0\\
39.9999997019739	0\\
};
\addplot [color=green, line width=2.0pt, forget plot]
  table[row sep=crcr]{%
39.9999997019739	0\\
39.9999997019739	0\\
39.9999997019739	0\\
39.9999997019739	0\\
39.9999997019739	0\\
39.9999997019739	0\\
39.9999997019739	0\\
39.9999997019739	0\\
39.9999997019739	0\\
39.9999997019739	0\\
39.9999997019739	0\\
39.9999997019739	0\\
39.9999997019739	0\\
39.9999997019739	0\\
39.9999997019739	0\\
39.9999997019739	0\\
39.9999997019739	0\\
39.9999997019739	0\\
39.9999997019739	0\\
39.9999997019739	0\\
39.9999997019739	0\\
39.9999997019739	0\\
39.9999997019739	0\\
39.9999997019739	0\\
39.9999997019739	0\\
39.9999997019739	0\\
39.9999997019739	0\\
39.9999997019739	0\\
39.9999997019739	0\\
39.9999997019739	0\\
39.9999997019739	0\\
39.9999997019739	0\\
39.9999997019739	0\\
39.9999997019739	0\\
39.9999997019739	0\\
39.9999997019739	0\\
39.9999997019739	0\\
39.9999997019739	0\\
39.9999997019739	0\\
39.9999997019739	0\\
39.9999997019739	0\\
39.9999997019739	0\\
39.9999997019739	0\\
39.9999997019739	0\\
39.9999997019739	0\\
39.9999997019739	0\\
39.9999997019739	0\\
39.9999997019739	0\\
39.9999997019739	0\\
39.9999997019739	0\\
39.9999997019739	0\\
39.9999997019739	0\\
39.9999997019739	0\\
39.9999997019739	0\\
39.9999997019739	0\\
39.9999997019739	0\\
39.9999997019739	0\\
39.9999997019739	0\\
39.9999997019739	0\\
39.9999997019739	0\\
39.9999997019739	0\\
39.9999997019739	0\\
39.9999997019739	0\\
39.9999997019739	0\\
39.9999997019739	0\\
39.9999997019739	0\\
39.9999997019739	0\\
39.9999997019739	0\\
39.9999997019739	0\\
39.9999997019739	0\\
39.9999997019739	0\\
39.9999997019739	0\\
39.9999997019739	0\\
39.9999997019739	0\\
39.9999997019739	0\\
39.9999997019739	0\\
39.9999997019739	0\\
39.9999997019739	0\\
39.9999997019739	0\\
39.9999997019739	0\\
39.9999997019739	0\\
39.9999997019739	0\\
39.9999997019739	0\\
39.9999997019739	0\\
39.9999997019739	0\\
39.9999997019739	0\\
39.9999997019739	0\\
39.9999997019739	0\\
39.9999997019739	0\\
39.9999997019739	0\\
39.9999997019739	0\\
39.9999997019739	0\\
39.9999997019739	0\\
39.9999997019739	0\\
39.9999997019739	0\\
39.9999997019739	0\\
39.9999997019739	0\\
39.9999997019739	0\\
39.9999997019739	0\\
39.9999997019739	0\\
39.9999997019739	0\\
};
\addplot [color=red, line width=2.0pt, forget plot]
  table[row sep=crcr]{%
39.9999997019739	0\\
39.9999997019739	0\\
39.9999997019739	0\\
39.9999997019739	0\\
39.9999997019739	0\\
39.9999997019739	0\\
39.9999997019739	0\\
39.9999997019739	0\\
39.9999997019739	0\\
39.9999997019739	0\\
39.9999997019739	0\\
39.9999997019739	0\\
39.9999997019739	0\\
39.9999997019739	0\\
39.9999997019739	0\\
39.9999997019739	0\\
39.9999997019739	0\\
39.9999997019739	0\\
39.9999997019739	0\\
39.9999997019739	0\\
39.9999997019739	0\\
39.9999997019739	0\\
39.9999997019739	0\\
39.9999997019739	0\\
39.9999997019739	0\\
39.9999997019739	0\\
39.9999997019739	0\\
39.9999997019739	0\\
39.9999997019739	0\\
39.9999997019739	0\\
39.9999997019739	0\\
39.9999997019739	0\\
39.9999997019739	0\\
39.9999997019739	0\\
39.9999997019739	0\\
39.9999997019739	0\\
39.9999997019739	0\\
39.9999997019739	0\\
39.9999997019739	0\\
39.9999997019739	0\\
39.9999997019739	0\\
39.9999997019739	0\\
39.9999997019739	0\\
39.9999997019739	0\\
39.9999997019739	0\\
39.9999997019739	0\\
39.9999997019739	0\\
39.9999997019739	0\\
39.9999997019739	0\\
39.9999997019739	0\\
39.9999997019739	0\\
39.9999997019739	0\\
39.9999997019739	0\\
39.9999997019739	0\\
39.9999997019739	0\\
39.9999997019739	0\\
39.9999997019739	0\\
39.9999997019739	0\\
39.9999997019739	0\\
39.9999997019739	0\\
39.9999997019739	0\\
39.9999997019739	0\\
39.9999997019739	0\\
39.9999997019739	0\\
39.9999997019739	0\\
39.9999997019739	0\\
39.9999997019739	0\\
39.9999997019739	0\\
39.9999997019739	0\\
39.9999997019739	0\\
39.9999997019739	0\\
39.9999997019739	0\\
39.9999997019739	0\\
39.9999997019739	0\\
39.9999997019739	0\\
39.9999997019739	0\\
39.9999997019739	0\\
39.9999997019739	0\\
39.9999997019739	0\\
39.9999997019739	0\\
39.9999997019739	0\\
39.9999997019739	0\\
39.9999997019739	0\\
39.9999997019739	0\\
39.9999997019739	0\\
39.9999997019739	0\\
39.9999997019739	0\\
39.9999997019739	0\\
39.9999997019739	0\\
39.9999997019739	0\\
39.9999997019739	0\\
39.9999997019739	0\\
39.9999997019739	0\\
39.9999997019739	0\\
39.9999997019739	0\\
39.9999997019739	0\\
39.9999997019739	0\\
39.9999997019739	0\\
39.9999997019739	0\\
39.9999997019739	0\\
39.9999997019739	0\\
};
\addplot [color=red, line width=2.0pt, only marks, mark size=2.5pt, mark=*, mark options={solid, fill=red, red}, forget plot]
  table[row sep=crcr]{%
0	0\\
};
\addplot [color=red, line width=2.0pt, only marks, mark size=2.5pt, mark=*, mark options={solid, fill=red, red}, forget plot]
  table[row sep=crcr]{%
39.9999997019739	0\\
};
\addplot [color=red, line width=2.0pt, only marks, mark size=2.5pt, mark=*, mark options={solid, fill=red, red}, forget plot]
  table[row sep=crcr]{%
39.9999997019739	0\\
};
\addplot [color=red, line width=2.0pt, only marks, mark size=2.5pt, mark=*, mark options={solid, fill=red, red}, forget plot]
  table[row sep=crcr]{%
39.9999997019739	0\\
};
\addplot [color=red, line width=2.0pt, only marks, mark size=2.5pt, mark=*, mark options={solid, fill=red, red}, forget plot]
  table[row sep=crcr]{%
39.9999997019739	0\\
};
\addplot [color=red, line width=2.0pt, only marks, mark size=2.5pt, mark=*, mark options={solid, fill=red, red}, forget plot]
  table[row sep=crcr]{%
39.9999997019739	0\\
};
\addplot [color=red, line width=2.0pt, only marks, mark size=2.5pt, mark=*, mark options={solid, fill=red, red}, forget plot]
  table[row sep=crcr]{%
39.9999997019739	0\\
};
\addplot [color=blue, line width=2.0pt, only marks, mark size=2.5pt, mark=*, mark options={solid, fill=blue, blue}, forget plot]
  table[row sep=crcr]{%
0	0\\
};
\addplot [color=blue, line width=2.0pt, only marks, mark size=2.5pt, mark=*, mark options={solid, fill=blue, blue}, forget plot]
  table[row sep=crcr]{%
40	0\\
};
\end{axis}
\end{tikzpicture}%%
  \caption{Duboids solution example 2}
  \label{fig:DuboidsRes1}
\end{figure}
%
The third example (Figure \ref{fig:DuboidsRes2}) shows a connection from $x=0$, $y=0$, $\theta=0$ and $\kappa=0$ to $x=-1$, $y=0$, $\theta=0$ and $\kappa=0$ obtaining a fancy shape given the constraints on curvature and jerk.\\
%
\begin{figure}[ht]
  \centering
  % This file was created by matlab2tikz.
%
%The latest updates can be retrieved from
%  http://www.mathworks.com/matlabcentral/fileexchange/22022-matlab2tikz-matlab2tikz
%where you can also make suggestions and rate matlab2tikz.
%
\begin{tikzpicture}

\begin{axis}[%
width=\linewidth,
height=0.776\linewidth,
at={(0\linewidth,0\linewidth)},
scale only axis,
xmin=-13.1872964510542,
xmax=12.1872931956209,
xlabel style={font=\color{white!15!black}},
xlabel={x(m)},
ymin=-10.006591850697,
ymax=10.0065925641807,
ylabel style={font=\color{white!15!black}},
ylabel={y(m)},
axis background/.style={fill=white},
title style={font=\bfseries},
title={$L_{tot}$ = 61.891, $k_{max}$ = 0.2, $J_{max}$ = 0.2, Type = [RSL]},
axis x line*=bottom,
axis y line*=left,
xmajorgrids,
xminorgrids,
ymajorgrids,
yminorgrids
]
\addplot [color=red, line width=2.0pt, forget plot]
  table[row sep=crcr]{%
0	0\\
0.0099999999999	-3.33333333330952e-08\\
0.0199999999968	-2.6666666663619e-07\\
0.0299999999757	-8.99999999479286e-07\\
0.0399999998976	-2.13333332943238e-06\\
0.0499999996875	-4.16666664806548e-06\\
0.0599999992224	-7.19999993334857e-06\\
0.0699999983193	-1.14333331372517e-05\\
0.0799999967232001	-1.70666661673448e-05\\
0.0899999940951002	-2.42999988611979e-05\\
0.0999999900000005	-3.3333330952381e-05\\
0.109999983894901	-4.43666620268643e-05\\
0.119999975116802	-5.75999914686177e-05\\
0.129999962870705	-7.32333183932116e-05\\
0.13999994621761	-9.14666415682164e-05\\
0.149999924062518	-0.000112499959319203\\
0.159999895142432	-0.000136533269420143\\
0.169999858014355	-0.000163766568967009\\
0.179999811043292	-0.000194399854233374\\
0.189999752390249	-0.000228633120506817\\
0.199999680000237	-0.000266666361904917\\
0.209999591590268	-0.00030869957116966\\
0.219999484637359	-0.000354932739439041\\
0.229999356366534	-0.000405565855994663\\
0.239999203738823	-0.000460798907984147\\
0.249999023439266	-0.000520831880117134\\
0.259998811864914	-0.000585864754333691\\
0.26999856511283	-0.000656097509443925\\
0.279998278968097	-0.00073173012073759\\
0.289997948891816	-0.000812962559562502\\
0.299997570009112	-0.000899994792870563\\
0.309997137097141	-0.000993026782730174\\
0.319996644573089	-0.00109225848580387\\
0.329996086482187	-0.00119788985278996\\
0.33999545648571	-0.00131012082782695\\
0.349994747848989	-0.0014291513478596\\
0.359993953429418	-0.0015551813419654\\
0.369993065664467	-0.00168841073064023\\
0.379992076559689	-0.00182903942504202\\
0.389990977676733	-0.0019772673261913\\
0.399989760121362	-0.00213329432412727\\
0.409988414531465	-0.00229732029701837\\
0.419986931065072	-0.00246954511022602\\
0.42998529938838	-0.00265016861532041\\
0.439983508663765	-0.00283939064904713\\
0.449981547537812	-0.0030374110322434\\
0.459979404129336	-0.00324442956870276\\
0.46997706601741	-0.00346064604398701\\
0.479974520229399	-0.00368626022418416\\
0.489971753228985	-0.00392147185461128\\
0.499968750904212	-0.00416648065846096\\
0.509965498555515	-0.0044214863353903\\
0.519961980883772	-0.00468668856005106\\
0.529958181978341	-0.00496228698056003\\
0.539954085305113	-0.00524848121690818\\
0.549949673694569	-0.00554547085930752\\
0.559944929329829	-0.00585345546647454\\
0.56993983373472	-0.00617263456384888\\
0.579934367761841	-0.00650320764174615\\
0.589928511580634	-0.00684537415344374\\
0.59992224466546	-0.00719933351319829\\
0.609915545783684	-0.00756528509419381\\
0.619908392983756	-0.00794342822641914\\
0.629900763583314	-0.00833396219447361\\
0.639892634157276	-0.0087370862352997\\
0.649883980525953	-0.00915299953584159\\
0.659874777743159	-0.00958190123062827\\
0.669865000084332	-0.0100239903992802\\
0.679854621034666	-0.0104794660639382\\
0.689843613277245	-0.0109485271866136\\
0.69983194868119	-0.0114313726664579\\
0.709819598289813	-0.0119282013369519\\
0.719806532308778	-0.012439211963012\\
0.729792720094278	-0.0129646032380125\\
0.739778130141214	-0.0135045737807243\\
0.749762730071391	-0.0140593221321665\\
0.75974648662172	-0.0146290467523714\\
0.769729365632437	-0.0152139460170618\\
0.779711332035326	-0.0158142182142379\\
0.789692349841963	-0.0164300615406744\\
0.799672382131964	-0.0170616740983251\\
0.809651391041255	-0.0177092538906361\\
0.819629337750348	-0.0183729988187629\\
0.829606182472637	-0.0190531066776944\\
0.839581884442705	-0.0197497751522788\\
0.849556401904651	-0.0204632018131534\\
0.859529692100426	-0.0211935841125752\\
0.869501711258192	-0.0219411193801515\\
0.879472414580694	-0.02270600481847\\
0.889441756233656	-0.023488437498627\\
0.899409689334183	-0.0242886143556521\\
0.909376165939199	-0.0251067321838288\\
0.919341137033889	-0.0259429876319097\\
0.92930455252017	-0.0267975771982251\\
0.939266361205181	-0.0276706972256841\\
0.949226510789796	-0.0285625438966671\\
0.959184947857156	-0.0294733132278078\\
0.969141617861228	-0.0304032010646653\\
0.979096465115384	-0.0313524030762829\\
0.989049432781012	-0.032321114749635\\
0.99900046285614	-0.0333095313839588\\
};
\addplot [color=green, line width=2.0pt, forget plot]
  table[row sep=crcr]{%
0.99900046285614	-0.0333095313839588\\
1.28107399012502	-0.0697411757808274\\
1.56061964966624	-0.122152660505431\\
1.83673291291116	-0.190374397214547\\
2.10852035752876	-0.274185640181141\\
2.37510255828077	-0.373315200563437\\
2.63561693258636	-0.487442323893249\\
2.88922053158886	-0.616197727944283\\
3.13509276769348	-0.759164797622177\\
3.3724380697504	-0.915880933009952\\
3.60048845729199	-1.085839046207\\
3.81850602549457	-1.26848920211825\\
4.02578533282409	-1.46324039788445\\
4.22165568364011	-1.66946247519573\\
4.40548329837206	-1.88648815930094\\
4.57667336424587	-2.11361521811491\\
4.73467195992542	-2.35010873443759\\
4.87896784784103	-2.59520348393258\\
5.00909412840582	-2.84810641117096\\
5.12462975076717	-3.10799919572826\\
5.22520087520498	-3.37404090003171\\
5.31048208276842	-3.64537069039003\\
5.3801974282372	-3.92111062240123\\
5.43412133300011	-4.20036848172571\\
5.47207931496193	-4.48224067103277\\
5.49394855311677	-4.76581513377915\\
5.49965828496124	-5.0501743053591\\
5.4891900354613	-5.33439808207698\\
5.46257767683215	-5.61756679833561\\
5.41990731893756	-5.89876420240705\\
5.36131703066334	-6.17708042115708\\
5.28699639316655	-6.45161490413042\\
5.19718588644594	-6.72147933747049\\
5.09217611121848	-6.98580051824503\\
4.97230684861986	-7.24372317987723\\
4.83796596077142	-7.49441275953993\\
4.68958813577096	-7.73705809855863\\
4.52765348116829	-7.97087406708534\\
4.35268597047662	-8.19510410455073\\
4.16525174774649	-8.40902266767447\\
3.96595729568801	-8.61193757811251\\
3.75544747326893	-8.80319226214524\\
3.53440342913834	-8.98216787515931\\
3.30354039762743	-9.14828530404912\\
3.06360538445891	-9.3010070410586\\
2.81537474965347	-9.43983892300024\\
2.55965169545419	-9.56433173022364\\
2.29726366739737	-9.67408264015983\\
2.02905967693905	-9.76873653073811\\
1.75590755430056	-9.84798712945796\\
1.4786911404219	-9.91157800439788\\
1.19830742710918	-9.95930339395464\\
0.915663654629652	-9.99100887262803\\
0.631674376145634	-10.006591850697\\
0.347258498486133	-10.00600190617\\
0.0633363088311863	-9.9892409479364\\
-0.219173503070174	-9.95636320958917\\
-0.499356817513137	-9.90747507394129\\
-0.776307042663017	-9.84273472880163\\
-1.04912804802311	-9.76235165512517\\
-1.31693706405276	-9.66658594919346\\
-1.5788675385532	-9.55574748101862\\
-1.8340719405788	-9.43019489169396\\
-2.08172450280107	-9.29033443293554\\
-2.32102389345193	-9.13661865256964\\
-2.55119580920055	-8.96954493021936\\
-2.77149548057412	-8.78965386792869\\
-2.98121008181551	-8.59752754093123\\
-3.17966103738041	-8.39378761422381\\
-3.36620621761076	-8.17909333103909\\
-3.54024201647978	-7.95413937972596\\
-3.70120530468581	-7.71965364593972\\
-3.84857525177518	-7.47639485741555\\
-3.9818750113983	-7.2251501289459\\
-4.10067326424594	-6.96673241550559\\
-4.20458561367327	-6.70197788176563\\
-4.29327582949579	-6.4317431965072\\
-4.3664569359325	-6.15690276069022\\
-4.42389214017623	-5.87834587814583\\
-4.46539559858638	-5.59697387804743\\
-4.49083301802497	-5.31369719847131\\
-4.50012209039024	-5.02943244048334\\
-4.49323275894174	-4.74509940228418\\
-4.47018731555525	-4.46161810300945\\
-4.4310603285927	-4.17990580581495\\
-4.37597840162064	-3.90087404987951\\
-4.30511976375786	-3.62542570092893\\
-4.21871369297771	-3.35445202982469\\
-4.11703977423119	-3.08882982867031\\
-4.00042699479123	-2.82941857376673\\
-3.86925267974538	-2.57705764459657\\
-3.72394127108152	-2.33256360783597\\
-3.56496295431671	-2.09672757518193\\
-3.39283213711354	-1.8703126435446\\
-3.20810578480622	-1.65405142588733\\
-3.0113816182226	-1.44864368070388\\
-2.80329617963311	-1.25475404780313\\
-2.58452277308483	-1.07300989772773\\
-2.35576928578519	-0.903999301765161\\
-2.11777589758448	-0.748269129119909\\
-1.87131268596889	-0.606323277403495\\
};
\addplot [color=red, line width=2.0pt, forget plot]
  table[row sep=crcr]{%
-1.87131268596889	-0.606323277403495\\
-1.86250395115063	-0.601589763074723\\
-1.8536858612179	-0.596873699292566\\
-1.84485854471502	-0.592174927998872\\
-1.83602212918422	-0.587493290515461\\
-1.82717674117394	-0.582828627553709\\
-1.81832250624725	-0.578180779224062\\
-1.80945954899028	-0.573549585045433\\
-1.80058799302073	-0.568934883954531\\
-1.79170796099651	-0.564336514315062\\
-1.7828195746243	-0.559754313926874\\
-1.77392295466834	-0.555188120034982\\
-1.76501822095915	-0.55063776933851\\
-1.75610549240238	-0.546103097999556\\
-1.74718488698767	-0.541583941651943\\
-1.73825652179764	-0.537080135409902\\
-1.72932051301684	-0.532591513876653\\
-1.72037697594087	-0.528117911152909\\
-1.71142602498543	-0.523659160845286\\
-1.70246777369552	-0.519215096074628\\
-1.69350233475463	-0.514785549484254\\
-1.68452981999405	-0.510370353248108\\
-1.67555034040217	-0.505969339078844\\
-1.66656400613383	-0.501582338235807\\
-1.65757092651978	-0.497209181532955\\
-1.64857121007609	-0.492849699346676\\
-1.63956496451375	-0.488503721623552\\
-1.63055229674813	-0.484171077888023\\
-1.62153331290866	-0.479851597249976\\
-1.61250811834844	-0.475545108412271\\
-1.60347681765395	-0.471251439678173\\
-1.59443951465478	-0.466970418958725\\
-1.58539631243342	-0.462701873780025\\
-1.57634731333504	-0.458445631290458\\
-1.5672926189774	-0.454201518267829\\
-1.55823233026071	-0.44996936112644\\
-1.5491665473776	-0.445748985924096\\
-1.54009536982303	-0.441540218369023\\
-1.53101889640438	-0.437342883826752\\
-1.52193722525146	-0.433156807326893\\
-1.51285045382658	-0.428981813569882\\
-1.50375867893469	-0.424817726933629\\
-1.4946619967335	-0.420664371480128\\
-1.48556050274373	-0.416521570961983\\
-1.47645429185923	-0.412389148828877\\
-1.46734345835732	-0.408266928233992\\
-1.45822809590901	-0.40415473204034\\
-1.44910829758934	-0.400052382827068\\
-1.4399841558877	-0.395959702895672\\
-1.43085576271821	-0.39187651427618\\
-1.42172320943012	-0.387802638733255\\
-1.41258658681821	-0.383737897772258\\
-1.40344598513327	-0.379682112645254\\
-1.39430149409257	-0.375635104356948\\
-1.38515320289034	-0.371596693670597\\
-1.37600120020832	-0.367566701113839\\
-1.36684557422633	-0.363544946984499\\
-1.35768641263277	-0.359531251356318\\
-1.34852380263534	-0.355525434084661\\
-1.33935783097152	-0.35152731481215\\
-1.33018858391933	-0.347536712974275\\
-1.32101614730795	-0.343553447804942\\
-1.31184060652837	-0.339577338341979\\
-1.30266204654417	-0.335608203432612\\
-1.29348055190216	-0.331645861738873\\
-1.2842962067432	-0.327690131743\\
-1.27510909481292	-0.323740831752757\\
-1.2659192994725	-0.319797779906755\\
-1.25672690370947	-0.315860794179705\\
-1.24753199014855	-0.311929692387642\\
-1.23833464106243	-0.308004292193123\\
-1.22913493838265	-0.304084411110377\\
-1.21993296371047	-0.300169866510426\\
-1.21072879832771	-0.296260475626173\\
-1.20152252320766	-0.292356055557462\\
-1.19231421902596	-0.288456423276094\\
-1.18310396617157	-0.284561395630833\\
-1.17389184475764	-0.280670789352369\\
-1.16467793463246	-0.276784421058251\\
-1.15546231539043	-0.272902107257811\\
-1.14624506638302	-0.269023664357044\\
-1.13702626672973	-0.265148908663475\\
-1.12780599532906	-0.261277656390994\\
-1.11858433086955	-0.257409723664683\\
-1.10936135184072	-0.253544926525603\\
-1.10013713654412	-0.24968308093558\\
-1.09091176310435	-0.245824002781959\\
-1.08168530948003	-0.241967507882342\\
-1.07245785347491	-0.238113411989322\\
-1.06322947274883	-0.234261530795177\\
-1.05400024482885	-0.230411679936581\\
-1.04477024712021	-0.22656367499927\\
-1.03553955691744	-0.222717331522721\\
-1.0263082514154	-0.218872465004807\\
-1.01707640772037	-0.215028890906441\\
-1.00784410286107	-0.211186424656222\\
-0.998611413799773	-0.207344881655058\\
-0.989378417443348	-0.203504077280801\\
-0.980145190654354	-0.199663826892853\\
-0.970911810262118	-0.195823945836794\\
-0.961678353073797	-0.191984249448981\\
};
\addplot [color=green, line width=2.0pt, forget plot]
  table[row sep=crcr]{%
-0.961678353073797	-0.191984249448981\\
-0.952444818367418	-0.188144556926445\\
-0.943211283661039	-0.184304864403908\\
-0.93397774895466	-0.180465171881372\\
-0.924744214248281	-0.176625479358835\\
-0.915510679541902	-0.172785786836299\\
-0.906277144835522	-0.168946094313762\\
-0.897043610129143	-0.165106401791226\\
-0.887810075422768	-0.161266709268691\\
-0.878576540716389	-0.157427016746154\\
-0.86934300601001	-0.153587324223618\\
-0.860109471303631	-0.149747631701081\\
-0.850875936597251	-0.145907939178545\\
-0.841642401890872	-0.142068246656009\\
-0.832408867184493	-0.138228554133472\\
-0.823175332478114	-0.134388861610936\\
-0.813941797771735	-0.130549169088399\\
-0.804708263065356	-0.126709476565863\\
-0.795474728358977	-0.122869784043326\\
-0.786241193652598	-0.11903009152079\\
-0.777007658946219	-0.115190398998253\\
-0.76777412423984	-0.111350706475717\\
-0.758540589533464	-0.107511013953182\\
-0.749307054827085	-0.103671321430645\\
-0.740073520120706	-0.099831628908109\\
-0.730839985414327	-0.0959919363855725\\
-0.721606450707948	-0.0921522438630361\\
-0.712372916001569	-0.0883125513404996\\
-0.70313938129519	-0.0844728588179632\\
-0.693905846588811	-0.0806331662954267\\
-0.684672311882432	-0.0767934737728902\\
-0.675438777176053	-0.0729537812503538\\
-0.666205242469674	-0.0691140887278173\\
-0.656971707763295	-0.0652743962052809\\
-0.647738173056916	-0.0614347036827444\\
-0.638504638350537	-0.057595011160208\\
-0.629271103644161	-0.0537553186376729\\
-0.620037568937782	-0.0499156261151364\\
-0.610804034231403	-0.0460759335926\\
-0.601570499525024	-0.0422362410700635\\
-0.592336964818645	-0.0383965485475271\\
-0.583103430112266	-0.0345568560249906\\
-0.573869895405887	-0.0307171635024542\\
-0.564636360699508	-0.0268774709799177\\
-0.555402825993129	-0.0230377784573813\\
-0.54616929128675	-0.0191980859348448\\
-0.536935756580371	-0.0153583934123083\\
-0.527702221873992	-0.0115187008897719\\
-0.518468687167612	-0.00767900836723544\\
-0.509235152461233	-0.00383931584469899\\
-0.500001617754858	3.76677836102584e-07\\
-0.490768083048479	0.00384006920037256\\
-0.4815345483421	0.00767976172290901\\
-0.47230101363572	0.0115194542454455\\
-0.463067478929341	0.0153591467679819\\
-0.453833944222962	0.0191988392905184\\
-0.444600409516583	0.0230385318130548\\
-0.435366874810204	0.0268782243355913\\
-0.426133340103825	0.0307179168581277\\
-0.416899805397446	0.0345576093806642\\
-0.407666270691067	0.0383973019032006\\
-0.398432735984688	0.0422369944257371\\
-0.389199201278309	0.0460766869482735\\
-0.37996566657193	0.04991637947081\\
-0.370732131865551	0.0537560719933465\\
-0.361498597159175	0.0575957645158815\\
-0.352265062452796	0.061435457038418\\
-0.343031527746417	0.0652751495609545\\
-0.333797993040038	0.0691148420834909\\
-0.324564458333659	0.0729545346060274\\
-0.31533092362728	0.0767942271285638\\
-0.306097388920901	0.0806339196511003\\
-0.296863854214522	0.0844736121736367\\
-0.287630319508143	0.0883133046961732\\
-0.278396784801764	0.0921529972187096\\
-0.269163250095385	0.0959926897412461\\
-0.259929715389006	0.0998323822637825\\
-0.250696180682627	0.103672074786319\\
-0.241462645976248	0.107511767308855\\
-0.232229111269872	0.111351459831391\\
-0.222995576563493	0.115191152353927\\
-0.213762041857114	0.119030844876463\\
-0.204528507150735	0.122870537399\\
-0.195294972444356	0.126710229921536\\
-0.186061437737977	0.130549922444073\\
-0.176827903031598	0.134389614966609\\
-0.167594368325219	0.138229307489146\\
-0.158360833618839	0.142069000011682\\
-0.14912729891246	0.145908692534219\\
-0.139893764206081	0.149748385056755\\
-0.130660229499702	0.153588077579292\\
-0.121426694793323	0.157427770101828\\
-0.112193160086944	0.161267462624364\\
-0.102959625380568	0.1651071551469\\
-0.0937260906741894	0.168946847669436\\
-0.0844925559678104	0.172786540191972\\
-0.0752590212614313	0.176626232714509\\
-0.0660254865550523	0.180465925237045\\
-0.0567919518486732	0.184305617759582\\
-0.0475584171422942	0.188145310282118\\
-0.0383248824359152	0.191985002804655\\
};
\addplot [color=red, line width=2.0pt, forget plot]
  table[row sep=crcr]{%
-0.0383248824359152	0.191985002804655\\
-0.029091425247594	0.195824699192467\\
-0.0198580448553583	0.199664580248527\\
-0.0106248180663638	0.203504830636475\\
-0.00139182170993877	0.207345635010732\\
0.00784086735136327	0.211187178011896\\
0.0170731722106604	0.215029644262115\\
0.0263050159056929	0.218873218360481\\
0.0355363214077274	0.222718084878396\\
0.0447670116104961	0.226564428354943\\
0.0539970093191399	0.230412433292255\\
0.0632262372391217	0.234262284150851\\
0.0724546179651937	0.238114165344995\\
0.0816820739703163	0.241968261238016\\
0.0909085275946352	0.245824756137633\\
0.100133901034414	0.249683834291255\\
0.109358116331007	0.253545679881276\\
0.118581095359837	0.257410477020356\\
0.127802759819348	0.261278409746667\\
0.137023031220014	0.265149662019148\\
0.146241830873308	0.269024417712718\\
0.155459079880719	0.272902860613485\\
0.164674699122748	0.276785174413926\\
0.173888609247928	0.280671542708043\\
0.183100730661864	0.284562148986508\\
0.19231098351625	0.288457176631768\\
0.201519287697944	0.292356808913136\\
0.210725562817997	0.296261228981847\\
0.219929728200761	0.3001706198661\\
0.229131702872942	0.304085164466052\\
0.238331405552714	0.308005045548797\\
0.247528754638837	0.311930445743317\\
0.25672366819976	0.315861547535379\\
0.265916063962788	0.31979853326243\\
0.275105859303209	0.323741585108431\\
0.284292971233492	0.327690885098674\\
0.29347731639245	0.331646615094549\\
0.302658811034454	0.335608956788285\\
0.311837371018662	0.339578091697654\\
0.321012911798236	0.343554201160615\\
0.330185348409623	0.34753746632995\\
0.339354595461806	0.351528068167824\\
0.348520567125624	0.355526187440334\\
0.357683177123066	0.359532004711993\\
0.366842338716615	0.363545700340173\\
0.375997964698614	0.367567454469514\\
0.385149967380627	0.37159744702627\\
0.394298258582858	0.375635857712623\\
0.40344274962356	0.379682866000927\\
0.412583351308498	0.383738651127934\\
0.421719973920408	0.38780339208893\\
0.4308525272085	0.391877267631854\\
0.43998092037799	0.395960456251347\\
0.449105062079628	0.400053136182741\\
0.4582248603993	0.404155485396015\\
0.467340222847607	0.408267681589665\\
0.476451056349516	0.412389902184551\\
0.485557267234017	0.416522324317658\\
0.494658761223793	0.420665124835803\\
0.503755443424975	0.424818480289303\\
0.512847218316875	0.428982566925557\\
0.521933989741754	0.433157560682568\\
0.531015660894672	0.437343637182426\\
0.540092134313314	0.441540971724697\\
0.549163311867886	0.445749739279771\\
0.558229094751004	0.449970114482116\\
0.567289383467689	0.454202271623502\\
0.576344077825333	0.458446384646134\\
0.585393076923709	0.4627026271357\\
0.594436279145073	0.466971172314399\\
0.603473582144238	0.471252193033847\\
0.612504882838732	0.475545861767947\\
0.621530077398951	0.479852350605651\\
0.630549061238421	0.484171831243697\\
0.639561729004044	0.488504474979229\\
0.648567974566385	0.492850452702352\\
0.657567691010064	0.497209934888628\\
0.666560770624117	0.501583091591481\\
0.67554710489246	0.505970092434519\\
0.684526584484341	0.510371106603783\\
0.693499099244916	0.514786302839928\\
0.70246453818581	0.519215849430305\\
0.711422789475726	0.523659914200962\\
0.720373740431165	0.528118664508584\\
0.729317277507132	0.532592267232327\\
0.738253286287928	0.537080888765578\\
0.747181651477959	0.541584695007618\\
0.756102256892667	0.54610385135523\\
0.765014985449448	0.550638522694187\\
0.773919719158636	0.555188873390657\\
0.782816339114592	0.559755067282549\\
0.791704725486795	0.564337267670736\\
0.800584757511025	0.568935637310207\\
0.809456313480566	0.573550338401109\\
0.818319270737539	0.578181532579735\\
0.827173505664235	0.582829380909386\\
0.836018893674511	0.587494043871137\\
0.844855309205313	0.592175681354548\\
0.853682625708191	0.59687445264824\\
0.862500715640927	0.601590516430401\\
0.871309450459186	0.606324030759171\\
};
\addplot [color=green, line width=2.0pt, forget plot]
  table[row sep=crcr]{%
0.871309450459186	0.606324030759171\\
1.11777268948622	0.748269899318639\\
1.35576610309427	0.904000091869693\\
1.5845196135408	1.07301071062244\\
1.80329304073323	1.25475488617501\\
2.01137849723725	1.44864454702446\\
2.20810267879554	1.65405232239585\\
2.39282904294527	1.87031357223229\\
2.56495986868511	2.09672853777769\\
2.72393819052725	2.33256460579328\\
2.86924960067637	2.57705867908193\\
3.00042391350425	2.82941964564953\\
3.11703668693422	3.0888309385143\\
3.21871059581284	3.35445317788093\\
3.30511665282469	3.62542688713049\\
3.37597527300013	3.90087527383776\\
3.43105717837117	4.17990706681751\\
3.47018413984862	4.46161940001991\\
3.49322955391976	4.74510073394349\\
3.50011885230063	5.02943380511288\\
3.49082974321743	5.31369859407775\\
3.46539228353621	5.59697530232918\\
3.42388878150756	5.87834732850123\\
3.36645353044096	6.15690423422736\\
3.29327237417056	6.43174469005532\\
3.2045821057183	6.70197939188814\\
3.10066970110038	6.96673393851449\\
2.98187139075593	7.22515166091778\\
2.84857157160275	7.4763963942089\\
2.70120156324022	7.71965518321359\\
2.54023821232404	7.95414091295985\\
2.36620234962856	8.17909485555398\\
2.17965710478934	8.39378912520428\\
1.98120608417868	8.59752903344856\\
1.77149141781036	8.78965533696483\\
1.55119168159285	8.96954637069162\\
1.32101970165431	9.13662005935616\\
1.08172024784368	9.29033580090118\\
0.834067623871253	9.43019621571671\\
0.578863161886137	9.55574875601632\\
0.316932629597567	9.66658717015073\\
0.0491235583298282	9.76235281712035\\
-0.223697499343607	9.84273582703381\\
-0.500647773513186	9.90747610375702\\
-0.78083113332093	9.95636416650901\\
-1.06334098658035	9.98924182768103\\
-1.34726321324609	10.0060027046858\\
-1.63167912324161	10.0065925641807\\
-1.91566842907411	9.99100949755137\\
-2.1983122236183	9.95930392708714\\
-2.47869595343363	9.91157844282931\\
-2.75591237799431	9.84798747061943\\
-3.02906450525679	9.768736772422\\
-3.29726849406624	9.6740827805384\\
-3.55965651401042	9.56433176786621\\
-3.81537955346754	9.43983885688884\\
-4.06361016676186	9.3010068706021\\
-4.30354515153828	9.14828502909558\\
-4.53440814769242	8.9821674960067\\
-4.75545214944701	8.80319177955021\\
-4.96596192244626	8.61193699329741\\
-5.16525631804694	8.40902198233232\\
-5.35269047731815	8.19510332084834\\
-5.52765791761796	7.97087318766431\\
-5.68959249499558	7.73705712653434\\
-5.83797023606917	7.4944116984985\\
-5.97231103345191	7.24372203387036\\
-6.09218019924042	6.98579929178279\\
-6.19718987153891	6.72147803551201\\
-6.28700026946789	6.45161353207252\\
-6.36132079259667	6.17707898482088\\
-6.41991096124221	5.89876270802255\\
-6.46258119459165	5.61756525252426\\
-6.48919342413094	5.33439649183243\\
-6.49966154039453	5.05017267802597\\
-6.4939516715907	4.76581347703004\\
-6.47208229320081	4.48223899284353\\
-6.43412416819803	4.20036679034908\\
-6.38020011807883	3.92110892633888\\
-6.31048462544824	3.64536899836311\\
-6.22520326944466	3.37403922095011\\
-6.1246319958311	3.1079975386585\\
-6.00909622411465	2.84810478530304\\
-5.87896979458323	2.59520189854568\\
-5.73467375866669	2.35010719886498\\
-5.5766750165364	2.11361374170803\\
-5.4054848063515	1.88648675139281\\
-5.22165705004035	1.66946114506395\\
-5.02578656096963	1.46323915471376\\
-4.81850711930057	1.26848805496291\\
-4.60048942126	1.08583800395308\\
-4.37243890896161	0.915880004337761\\
-4.13509348779962	0.759163990968867\\
-3.88922113880073	0.616197051466866\\
-3.63561743365996	0.487441785432138\\
-3.37510296050112	0.373314807606681\\
-3.10852066869136	0.274185399829483\\
-2.83673314130121	0.190374316147488\\
-2.56061980403565	0.122152744948402\\
-2.28107407966726	0.0697414314736332\\
-1.99900049717896	0.0333099635506408\\
};
\addplot [color=red, line width=2.0pt, forget plot]
  table[row sep=crcr]{%
-1.99900049717896	0.0333099635506408\\
-1.98904946646784	0.0323215533193271\\
-1.97909649817889	0.0313528480502308\\
-1.96914165031397	0.0304036524440793\\
-1.95918497971157	0.0294737710138607\\
-1.94922654205817	0.0285630080904961\\
-1.9392663918997	0.0276711678283905\\
-1.92930458265288	0.0267980542108766\\
-1.91934116661671	0.0259434710555407\\
-1.90937619498393	0.0251072220194393\\
-1.8994097178525	0.0242891106042117\\
-1.8894417842371	0.0234889401610726\\
-1.87947244208069	0.0227065138957079\\
-1.86950173826601	0.0219416348730575\\
-1.85952971862725	0.0211941060219967\\
-1.84955642796151	0.0204637301399081\\
-1.83958191004051	0.0197503098971562\\
-1.82960620762217	0.0190536478414563\\
-1.81962936246227	0.0183735464021451\\
-1.80965141532609	0.0177098078943481\\
-1.79967240600011	0.0170622345230493\\
-1.7896923733037	0.0164306283870698\\
-1.7797113551008	0.0158147914829388\\
-1.76972938831167	0.0152145257086787\\
-1.7597465089246	0.0146296328674909\\
-1.74976275200768	0.0140599146713536\\
-1.73977815172056	0.0135051727445218\\
-1.72979274132617	0.0129652086269413\\
-1.71980655320261	0.0124398237775728\\
-1.70981961885484	0.011928819577625\\
-1.69983196892654	0.0114319973337036\\
-1.6898436332119	0.0109491582808728\\
-1.67985464066751	0.010480103585634\\
-1.6698650194241	0.0100246343488172\\
-1.65987479679846	0.00958255160839297\\
-1.64988399930528	0.00915365634220425\\
-1.63989265266899	0.00873774947061368\\
-1.62990078183564	0.00833463185907619\\
-1.61990841098479	0.00794410432063179\\
-1.6099155635414	0.00756596761832346\\
-1.59992226218771	0.00720002246753704\\
-1.58992852887513	0.0068460695382695\\
-1.57993438483617	0.00650390945732298\\
-1.56993985059634	0.00617334281042809\\
-1.55994494598608	0.00585417014429445\\
-1.54994969015265	0.00554619196859391\\
-1.53995410157209	0.00524920875787537\\
-1.52995819806117	0.00496302095341067\\
-1.51996199678927	0.00468742896497674\\
-1.50996551429037	0.00442223317257177\\
-1.49996876647499	0.00416723392806916\\
-1.48997176864211	0.00392223155680708\\
-1.47997453549118	0.00368702635911879\\
-1.46997708113402	0.00346141861180297\\
-1.45997941910682	0.00324520856953369\\
-1.4499815623821	0.00303819646621456\\
-1.43998352338063	0.00284018251627568\\
-1.4299853139835	0.00265096691591614\\
-1.41998694554397	0.00247034984429129\\
-1.40998842889954	0.00229813146464837\\
-1.3999897743839	0.00213411192541084\\
-1.38999099183888	0.00197809136121084\\
-1.37999209062646	0.00182986989387391\\
-1.36999307964075	0.00168924763335497\\
-1.35999396731998	0.00155602467862829\\
-1.34999476165846	0.00143000111853073\\
-1.33999547021859	0.0013109770325615\\
-1.32999610014285	0.00119875249163843\\
-1.31999665816579	0.00109312755881263\\
-1.30999715062599	0.000993902289941395\\
-1.2999975834781	0.000900876734322534\\
-1.28999796230481	0.000813850935290047\\
-1.27999829232882	0.000732624930772135\\
-1.26999857842489	0.000656998753813726\\
-1.25999882513178	0.000586772433064021\\
-1.24999903666429	0.000521745993230698\\
-1.23999921692523	0.000461719455501083\\
-1.22999936951739	0.000406492837932722\\
-1.21999949775564	0.000355866155813918\\
-1.20999960467881	0.000309639421995075\\
-1.19999969306173	0.00026761264719283\\
-1.18999976542727	0.000229585840267476\\
-1.17999982405829	0.000195359008475629\\
-1.16999987100965	0.000164732157698412\\
-1.1599999081202	0.000137505292646998\\
-1.14999993702482	0.000113478417046796\\
-1.13999995916638	9.24515338008909e-05\\
-1.12999997580775	7.42246451345496e-05\\
-1.11999998804379	5.85977527213973e-05\\
-1.10999999681337	4.53708577933397e-05\\
-1.10000000291138	3.43439612342752e-05\\
-1.09000000700067	2.53170636598184e-05\\
-1.08000000962411	1.80901654835962e-05\\
-1.07000001121659	1.24632669718376e-05\\
-1.06000001211697	8.2363682866842e-06\\
-1.05000001258012	5.20946952047082e-06\\
-1.04000001278891	3.18257072102824e-06\\
-1.03000001286622	1.95567191036559e-06\\
-1.02000001288692	1.32877309686248e-06\\
-1.01000001288987	1.10187428293394e-06\\
-1.00000001288995	1.07497546891072e-06\\
};
\addplot [color=red, line width=2.0pt, only marks, mark size=2.5pt, mark=*, mark options={solid, fill=red, red}, forget plot]
  table[row sep=crcr]{%
0.99900046285614	-0.0333095313839588\\
};
\addplot [color=red, line width=2.0pt, only marks, mark size=2.5pt, mark=*, mark options={solid, fill=red, red}, forget plot]
  table[row sep=crcr]{%
-1.87131268596889	-0.606323277403495\\
};
\addplot [color=red, line width=2.0pt, only marks, mark size=2.5pt, mark=*, mark options={solid, fill=red, red}, forget plot]
  table[row sep=crcr]{%
-0.961678353073797	-0.191984249448981\\
};
\addplot [color=red, line width=2.0pt, only marks, mark size=2.5pt, mark=*, mark options={solid, fill=red, red}, forget plot]
  table[row sep=crcr]{%
-0.0383248824359152	0.191985002804655\\
};
\addplot [color=red, line width=2.0pt, only marks, mark size=2.5pt, mark=*, mark options={solid, fill=red, red}, forget plot]
  table[row sep=crcr]{%
0.871309450459182	0.606324030759169\\
};
\addplot [color=red, line width=2.0pt, only marks, mark size=2.5pt, mark=*, mark options={solid, fill=red, red}, forget plot]
  table[row sep=crcr]{%
-1.99900049717896	0.0333099635506408\\
};
\addplot [color=red, line width=2.0pt, only marks, mark size=2.5pt, mark=*, mark options={solid, fill=red, red}, forget plot]
  table[row sep=crcr]{%
-1.00000001288995	1.07497546893116e-06\\
};
\addplot [color=blue, line width=2.0pt, only marks, mark size=2.5pt, mark=*, mark options={solid, fill=blue, blue}, forget plot]
  table[row sep=crcr]{%
0	0\\
};
\addplot [color=blue, line width=2.0pt, only marks, mark size=2.5pt, mark=*, mark options={solid, fill=blue, blue}, forget plot]
  table[row sep=crcr]{%
-1	0\\
};
\end{axis}
\end{tikzpicture}%%
  \caption{Duboids solution example 3}
  \label{fig:DuboidsRes2}
\end{figure}
%
The fourth example (Figure \ref{fig:DuboidsRes3}) shows a connection from $x=0$, $y=0$, $\theta=0$ and $\kappa=0$ to $x=0$, $y=0$, $\theta=\pi$ and $\kappa=0$ obtaining a water drop like shape.\\
%
\begin{figure}[ht]
  \centering
  % This file was created by matlab2tikz.
%
%The latest updates can be retrieved from
%  http://www.mathworks.com/matlabcentral/fileexchange/22022-matlab2tikz-matlab2tikz
%where you can also make suggestions and rate matlab2tikz.
%
\begin{tikzpicture}[scale = 0.7]

\begin{axis}[%
width=\linewidth,
height=0.776\linewidth,
at={(0\linewidth,0\linewidth)},
scale only axis,
xmin=0,
xmax=13.8717811499801,
xlabel style={font=\color{white!15!black}},
xlabel={x(m)},
ymin=-5.47040528366376,
ymax=5.47040275236895,
ylabel style={font=\color{white!15!black}},
ylabel={y(m)},
axis background/.style={fill=white},
title style={font=\bfseries},
title={$L_{tot}$ = 37.0618, $k_{max}$ = 0.2, $J_{max}$ = 0.5, Type = [LRL]},
axis x line*=bottom,
axis y line*=left,
xmajorgrids,
xminorgrids,
ymajorgrids,
yminorgrids
]
\addplot [color=red, line width=2.0pt, forget plot]
  table[row sep=crcr]{%
0	0\\
0.0039999999999936	5.33333333332724e-09\\
0.0079999999997952	4.26666666658865e-08\\
0.0119999999984448	1.4399999998667e-07\\
0.0159999999934464	3.41333333233469e-07\\
0.01999999998	6.66666666190477e-07\\
0.0239999999502336	1.15199999829372e-06\\
0.0279999998924352	1.82933332831364e-06\\
0.0319999997902848	2.73066665388403e-06\\
0.0359999996220864	3.88799997084666e-06\\
0.03999999936	5.33333327238096e-06\\
0.0439999989692736	7.09866654788772e-06\\
0.0479999984074752	9.2159997815966e-06\\
0.0519999976237249	1.17173329508662e-05\\
0.0559999965579265	1.46346660241463e-05\\
0.0599999951400002	1.79999989585715e-05\\
0.0639999932891139	2.18453316971554e-05\\
0.0679999909129158	2.62026641655549e-05\\
0.0719999879067657	3.11039962683733e-05\\
0.0759999841529679	3.65813278849726e-05\\
0.0799999795200024	4.26666588647626e-05\\
0.0839999738617574	4.93919890219376e-05\\
0.0879999670167609	5.67893181296299e-05\\
0.0919999588074134	6.48906459134479e-05\\
0.0959999490392189	7.37279720443694e-05\\
0.0999999375000181	8.33332961309598e-05\\
0.103999923959219	9.37386177108827e-05\\
0.107999908167031	0.000104975936241674\\
0.111999889853695	0.000117077251090747\\
0.115999868728715	0.000130074561524601\\
0.119999844480093	0.000143999866697198\\
0.123999816773559	0.000158885165637479\\
0.127999785251802	0.000174762457235992\\
0.131999749533705	0.0001916637402306\\
0.135999709213574	0.000209621013191226\\
0.139999663860374	0.000228666274503633\\
0.143999613016955	0.00024883152235217\\
0.147999556199291	0.000270148754701491\\
0.151999492895708	0.000292649969277189\\
0.155999422566116	0.000316367163545326\\
0.159999344641243	0.000341332334690826\\
0.163999258521866	0.000367577479594705\\
0.167999163578043	0.0003951345948101\\
0.171999059148347	0.000424035676537064\\
0.175998944539097	0.000454312720596112\\
0.179998819023587	0.000485997722400469\\
0.183998681841325	0.000519122676927002\\
0.187998532197261	0.000553719578685794\\
0.191998369261017	0.000589820421688346\\
0.195998192166126	0.000627457199414355\\
0.199998000009259	0.000666661904777056\\
0.203997791849459	0.000707466530087088\\
0.207997566707374	0.000749903067014852\\
0.211997323564488	0.000794003506551339\\
0.215997061362356	0.000839799838967381\\
0.219996779001833	0.000887324053771324\\
0.22399647534231	0.000936608139665051\\
0.227996149200946	0.000987684084498361\\
0.231995799351897	0.00104058387522166\\
0.235995424525555	0.0010953394978369\\
0.239995023407776	0.00115198293734686\\
0.243994594639112	0.0012105461777025\\
0.247994136814051	0.00127106120174863\\
0.25199364848024	0.0013335599911677\\
0.255993128137727	0.00139807452642166\\
0.259992574238189	0.00146463678669201\\
0.263991985184165	0.00153327874981787\\
0.267991359328293	0.00160403239223204\\
0.27199069497254	0.00167692968889515\\
0.275989990367434	0.00175200261322773\\
0.279989243711304	0.00182928313704023\\
0.283988453149507	0.00190880323046097\\
0.287987616773665	0.00199059486186195\\
0.291986732620897	0.00207468999778252\\
0.295985798673054	0.0021611206028509\\
0.299984812855953	0.00224991863970341\\
0.303983773038611	0.00234111606890158\\
0.307982677032479	0.00243474484884685\\
0.311981522590676	0.00253083693569309\\
0.315980307407224	0.00262942428325673\\
0.319979029116282	0.0027305388429245\\
0.323977685291382	0.00283421256355887\\
0.327976273444663	0.00294047739140096\\
0.331974791026105	0.00304936526997105\\
0.335973235422765	0.00316090813996663\\
0.339971603958015	0.00327513793915787\\
0.343969893890772	0.00339208660228061\\
0.347968102414737	0.00351178606092669\\
0.351966226657634	0.00363426824343178\\
0.355964263680437	0.0037595650747605\\
0.359962210476616	0.00388770847638889\\
0.363960063971367	0.00401873036618416\\
0.367957821020851	0.00415266265828177\\
0.371955478411431	0.00428953726295971\\
0.375953032858908	0.00442938608650997\\
0.379950481007759	0.00457224103110726\\
0.383947819430375	0.00471813399467477\\
0.387945044626296	0.00486709687074723\\
0.391942153021452	0.00501916154833079\\
0.395939140967401	0.00517435991176029\\
0.399936004740566	0.00533272384055325\\
};
\addplot [color=green, line width=2.0pt, forget plot]
  table[row sep=crcr]{%
0.399936004740566	0.00533272384055325\\
0.44627425662624	0.00740256711858648\\
0.492591313055718	0.0099021916352755\\
0.538883187950072	0.0128313822711515\\
0.585145897397518	0.0161898869379777\\
0.631375459996268	0.0199774166004455\\
0.677567897197179	0.0241936453010502\\
0.723719233646143	0.0288382101881415\\
0.769825497526213	0.0339107115471506\\
0.815882720899422	0.0394107128349907\\
0.861886940048259	0.0453377407176271\\
0.9078341958168	0.0516912851108105\\
0.953720533951426	0.0584707992239771\\
0.999542005441135	0.0656756996073041\\
1.04529466685739	0.0733053662019235\\
1.09097458069351	0.0813591423932847\\
1.13657781570351	0.0898363350676616\\
1.18210044724045	0.098736214671803\\
1.22753855759417	0.10805801527572\\
1.27288823632847	0.1178009346386\\
1.31814558061764	0.127964134277853\\
1.36330669558234	0.138546739541263\\
1.40836769462478	0.149547839682272\\
1.45332469976325	0.16096648793835\\
1.49817384196579	0.172801701612482\\
1.54291126148322	0.185052462157733\\
1.5875331081813	0.197717715264909\\
1.63203554187206	0.21079637095329\\
1.67641473264431	0.224287303664435\\
1.72066686119321	0.238189352359049\\
1.764788119149	0.252501320616901\\
1.80877470940476	0.26722197673979\\
1.85262284644312	0.282350053857546\\
1.89632875666213	0.297884250037057\\
1.93988867869997	0.313823228394317\\
1.98329886375867	0.330165617209474\\
2.02655557592671	0.346910010044884\\
2.0696550925006	0.364054965866153\\
2.11259370430517	0.38159900916615\\
2.15536771601286	0.399540630091988\\
2.19797344646169	0.417878284574967\\
2.24040722897211	0.43661039446346\\
2.2826654116625	0.455735347658721\\
2.32474435776353	0.475251498253633\\
2.36664044593104	0.495157166674348\\
2.40835007055781	0.515450639824839\\
2.44986964208378	0.536130171234327\\
2.49119558730499	0.557193981207581\\
2.53232434968111	0.578640256978085\\
2.5732523896415	0.600467152864042\\
2.61397618488984	0.622672790427219\\
2.65449223070725	0.645255258634602\\
2.69479704025392	0.668212614022863\\
2.73488714486917	0.691542880865615\\
2.77475909437	0.715244051343446\\
2.81440945734798	0.739314085716713\\
2.85383482146458	0.763750912501082\\
2.89303179374483	0.788552428645804\\
2.93199700086934	0.8137164997147\\
2.97072708946456	0.839240960069859\\
3.00921872639144	0.86512361305801\\
3.04746859903224	0.891362231199568\\
3.08547341557562	0.917954556380332\\
3.12322990529996	0.944898300045822\\
3.16073481885481	0.972191143398233\\
3.19798492854054	0.999830737595989\\
3.23497702858614	1.02781470395589\\
3.27170793542508	1.05614063415783\\
3.30817448796931	1.08480609045204\\
3.34437354788129	1.11380860586889\\
3.3803019998441	1.14314568443121\\
3.41595675182952	1.1728148013691\\
3.45133473536416	1.20281340333716\\
3.48643290579351	1.23313890863433\\
3.52124824254398	1.26378870742595\\
3.55577774938285	1.2947601619685\\
3.59001845467611	1.32605060683647\\
3.62396741164423	1.35765734915187\\
3.65762169861577	1.3895776688159\\
3.69097841927875	1.42180881874307\\
3.72403470293001	1.45434802509761\\
3.75678770472218	1.48719248753221\\
3.78923460590856	1.52033937942899\\
3.82137261408568	1.55378584814279\\
3.85319896343362	1.58752901524665\\
3.88471091495405	1.62156597677952\\
3.91590575670593	1.6558938034962\\
3.94678080403893	1.69050954111944\\
3.97733339982444	1.72541021059416\\
4.00756091468426	1.76059280834383\\
4.03746074721689	1.79605430652899\\
4.06703032422142	1.8317916533078\\
4.09626710091895	1.8678017730987\\
4.12516856117161	1.90408156684508\\
4.15373221769911	1.940627912282\\
4.18195561229278	1.9774376642049\\
4.20983631602716	2.01450765474027\\
4.23737192946898	2.05183469361825\\
4.26456008288369	2.08941556844726\\
4.29139843643942	2.12724704499039\\
4.31788468040832	2.16532586744377\\
};
\addplot [color=red, line width=2.0pt, forget plot]
  table[row sep=crcr]{%
4.31788468040832	2.16532586744377\\
4.32241705455914	2.17191810311729\\
4.32693908653588	2.17851743747424\\
4.3314509766231	2.18512370991175\\
4.3359529257954	2.19173676078417\\
4.34044513570058	2.19835643138484\\
4.34492780864309	2.2049825639277\\
4.34940114756749	2.21161500152886\\
4.35386535604215	2.21825358818794\\
4.35832063824307	2.22489816876946\\
4.36276719893779	2.23154858898401\\
4.36720524346948	2.2382046953694\\
4.37163497774116	2.24486633527164\\
4.37605660820004	2.25153335682591\\
4.38047034182193	2.25820560893738\\
4.38487638609588	2.26488294126199\\
4.38927494900885	2.27156520418708\\
4.39366623903056	2.27825224881205\\
4.39805046509835	2.28494392692877\\
4.40242783660231	2.29164009100212\\
4.40679856337035	2.29834059415029\\
4.4111628556535	2.30504529012508\\
4.41552092411123	2.31175403329215\\
4.41987297979693	2.31846667861113\\
4.42421923414341	2.32518308161574\\
4.42855989894857	2.33190309839384\\
4.43289518636112	2.33862658556735\\
4.43722530886637	2.3453534002722\\
4.44155047927213	2.35208340013819\\
4.44587091069467	2.35881644326879\\
4.45018681654478	2.3655523882209\\
4.45449841051388	2.37229109398453\\
4.45880590656019	2.37903241996254\\
4.463109518895	2.38577622595019\\
4.46740946196898	2.39252237211475\\
4.47170595045853	2.39927071897505\\
4.47599919925226	2.40602112738098\\
4.48028942343741	2.41277345849296\\
4.48457683828643	2.41952757376137\\
4.48886165924353	2.42628333490599\\
4.49314410191133	2.43304060389537\\
4.49742438203748	2.43979924292617\\
4.50170271550139	2.44655911440253\\
4.50597931830095	2.45332008091536\\
4.51025440653929	2.46008200522162\\
4.51452819641155	2.46684475022364\\
4.51880090419174	2.47360817894837\\
4.52307274621951	2.48037215452659\\
4.52734393888704	2.48713654017221\\
4.53161469862586	2.49390119916146\\
4.53588524189378	2.50066599481214\\
4.54015578516169	2.50743079046281\\
4.54442654490052	2.51419544945206\\
4.54869773756804	2.52095983509768\\
4.55296957959581	2.52772381067591\\
4.557242287376	2.53448723940063\\
4.56151607724826	2.54124998440266\\
4.5657911654866	2.54801190870892\\
4.57006776828616	2.55477287522174\\
4.57434610175007	2.5615327466981\\
4.57862638187622	2.56829138572891\\
4.58290882454402	2.57504865471828\\
4.58719364550112	2.58180441586291\\
4.59148106035014	2.58855853113132\\
4.59577128453529	2.59531086224329\\
4.60006453332902	2.60206127064922\\
4.60436102181857	2.60880961750952\\
4.60866096489255	2.61555576367408\\
4.61296457722736	2.62229956966173\\
4.61727207327367	2.62904089563974\\
4.62158366724277	2.63577960140338\\
4.62589957309288	2.64251554635548\\
4.63022000451542	2.64924858948608\\
4.63454517492118	2.65597858935207\\
4.63887529742643	2.66270540405693\\
4.64321058483898	2.66942889123043\\
4.64755124964414	2.67614890800853\\
4.65189750399062	2.68286531101315\\
4.65624955967632	2.68957795633212\\
4.66060762813405	2.69628669949919\\
4.6649719204172	2.70299139547398\\
4.66934264718524	2.70969189862215\\
4.6737200186892	2.7163880626955\\
4.67810424475699	2.72307974081223\\
4.6824955347787	2.72976678543719\\
4.68689409769167	2.73644904836229\\
4.69130014196563	2.74312638068689\\
4.69571387558751	2.74979863279837\\
4.70013550604639	2.75646565435264\\
4.70456524031807	2.76312729425487\\
4.70900328484977	2.76978340064026\\
4.71344984554448	2.77643382085481\\
4.7179051277454	2.78307840143634\\
4.72236933622006	2.78971698809542\\
4.72684267514446	2.79634942569657\\
4.73132534808697	2.80297555823943\\
4.73581755799215	2.8095952288401\\
4.74031950716445	2.81620827971252\\
4.74483139725167	2.82281455215003\\
4.74935342922841	2.82941388650698\\
4.75388580337923	2.8360061221805\\
};
\addplot [color=green, line width=2.0pt, forget plot]
  table[row sep=crcr]{%
4.75388580337923	2.8360061221805\\
4.90311336577669	3.04132673495169\\
5.06256825163365	3.23880979443824\\
5.2318395430342	3.4279463834828\\
5.41049102502064	3.60824909392241\\
5.59806230972679	3.77925328264745\\
5.79407002280521	3.94051826899474\\
5.998009049091	4.09162847038904\\
6.20935383429209	4.23219447330628\\
6.42755973935141	4.36185403679877\\
6.65206444399095	4.48027302599619\\
6.88228939582049	4.58714627317672\\
7.11764130127683	4.68219836418945\\
7.35751365455127	4.76518434820122\\
7.60128830056527	4.83589036893902\\
7.84833702796639	4.89413421580124\\
8.09802318803956	4.93976579341739\\
8.34970333536146	4.97266750844642\\
8.60272888597017	4.99275457261673\\
8.85644778877693	4.99997522122697\\
9.11020620591276	4.99431084654454\\
9.36335019767956	4.97577604575802\\
9.61522740776371	4.94441858335999\\
9.86518874436927	4.90031926805709\\
10.1125900529384	4.8435917445247\\
10.3567937761484	4.77438220054264\\
10.5971705969077	4.69286899026683\\
10.8331010601166	4.59926217460763\\
11.0639771690133	4.49380297989939\\
11.2892039519921	4.37676317625608\\
11.5082009958546	4.24844437721518\\
11.7204039415449	4.10917726247455\\
11.925265938512	3.95932072572525\\
12.1222590539528	3.79926094977653\\
12.310875633304	3.62941041135622\\
12.4906296084764	3.4502068181513\\
12.6610577504602	3.26211198082783\\
12.8217208630746	3.06561062293714\\
12.9722049147829	2.86120913177506\\
13.1121221056593	2.64943425341344\\
13.2411118667556	2.43083173526662\\
13.3588417892936	2.20596491969119\\
13.4650084812877	1.97541329224334\\
13.5593383493916	1.7397709883348\\
13.6415883039527	1.49964526213596\\
13.7115463854582	1.25565492167165\\
13.7690323107585	1.00842873414251\\
13.8138979376596	0.758603805581344\\
13.8460276466884	0.506823939020199\\
13.8653386390457	0.253737975399139\\
13.8717811499801	-1.87850773743139e-06\\
13.865338577033	-0.253741730840089\\
13.8460275228229	-0.50682768974164\\
13.8138977522605	-0.75860754845045\\
13.7690320643035	-1.00843246604669\\
13.7115460785824	-1.25565863952657\\
13.641587937447	-1.49964896289349\\
13.5593379242005	-1.73977466899088\\
13.4650079985068	-1.9754169498457\\
13.3588412501672	-2.20596855134697\\
13.241111272673	-2.43083533814983\\
13.1121214581514	-2.64943782477225\\
12.9722042155183	-2.86121266893885\\
12.8217201138554	-3.06561412332343\\
12.6610569532172	-3.26211544194891\\
12.4906287652639	-3.45021023762065\\
12.3108747462952	-3.62941378689466\\
12.1222581254334	-3.79926427921809\\
11.9252649708749	-3.95932400702274\\
11.7204029372837	-4.10918049370487\\
11.5081999575573	-4.24844755658424\\
11.2892028823343	-4.37676630210343\\
11.0639760707517	-4.49380605070252\\
10.8330999360812	-4.59926518898588\\
10.5971694499953	-4.69287194698494\\
10.3567926093147	-4.77438509851395\\
10.1125888691902	-4.84359458281393\\
9.86518754675724	-4.90032204588277\\
9.61522619937405	-4.94442130009645\\
9.36334898162631	-4.97577870093705\\
9.11020498532971	-4.99431343985654\\
8.85644656680955	-4.99997775252178\\
8.60272766576748	-4.992757041904\\
8.34970212006795	-4.9726699158956\\
8.09802198078706	-4.93976813935729\\
7.84833583186601	-4.89413650071919\\
7.60128711869938	-4.8358925934796\\
7.35751248996557	-4.76518651316458\\
7.11764015697245	-4.6822004705293\\
6.88228827474634	-4.58714832199783\\
6.65206334903606	-4.48027501855156\\
6.42755867333749	-4.36185597448641\\
6.20935279996627	-4.23219635766557\\
5.99800804911877	-4.0916303030968\\
5.79406905976351	-3.9405200518609\\
5.59806138609739	-3.77925501761039\\
5.41049014318375	-3.60825078304395\\
5.23183870526232	-3.42794802894288\\
5.06256746008575	-3.23881139852932\\
4.90311262249259	-3.04132830007286\\
4.75388511027439	-2.83600765083127\\
};
\addplot [color=red, line width=2.0pt, forget plot]
  table[row sep=crcr]{%
4.75388511027439	-2.83600765083127\\
4.74935273773467	-2.82941541405006\\
4.74483070737078	-2.82281607858794\\
4.74031881889809	-2.81620980504775\\
4.73581687134199	-2.80959675307508\\
4.73132466305462	-2.80297708137655\\
4.7268419917315	-2.79635094773814\\
4.72236865442804	-2.78971850904373\\
4.71790444757581	-2.78307992129362\\
4.71344916699879	-2.77643533962325\\
4.7090026079294	-2.76978491832198\\
4.70456456502443	-2.76312881085196\\
4.70013483238081	-2.75646716986712\\
4.69571320355132	-2.74980014723223\\
4.69129947156009	-2.74312789404206\\
4.68689342891805	-2.73645056064064\\
4.68249486763818	-2.72976829664056\\
4.67810357925076	-2.72308125094239\\
4.67371935481837	-2.71638957175418\\
4.66934198495092	-2.70969340661103\\
4.66497125982045	-2.70299290239468\\
4.66060696917589	-2.69628820535327\\
4.65624890235774	-2.68957946112112\\
4.65189684831258	-2.68286681473852\\
4.64755059560755	-2.67615041067171\\
4.64320993244472	-2.66943039283278\\
4.63887464667536	-2.66270690459975\\
4.63454452581411	-2.65598008883664\\
4.63021935705312	-2.64925008791359\\
4.6258989272761	-2.6425170437271\\
4.62158302307222	-2.63578109772022\\
4.61727143075002	-2.62904239090285\\
4.61296393635125	-2.62230106387211\\
4.60866032566459	-2.61555725683269\\
4.60436038423934	-2.60881110961724\\
4.60006389739905	-2.6020627617069\\
4.59577065025508	-2.59531235225173\\
4.59148042772017	-2.58856002009124\\
4.58719301452182	-2.58180590377501\\
4.58290819521578	-2.5750501415832\\
4.57862575419942	-2.56829287154722\\
4.57434547572505	-2.56153423147034\\
4.57006714391321	-2.55477435894837\\
4.565790542766	-2.54801339139037\\
4.56151545618024	-2.5412514660393\\
4.55724166796076	-2.53448871999279\\
4.55296896183352	-2.52772529022384\\
4.54869712145882	-2.5209613136016\\
4.54442593044448	-2.51419692691212\\
4.5401551723589	-2.50743226687912\\
4.53588463074427	-2.50066747018474\\
4.53161408912963	-2.49390267349037\\
4.52734333104405	-2.48713801345737\\
4.52307214002971	-2.4803736267679\\
4.51880029965502	-2.47360965014566\\
4.51452759352777	-2.46684622037671\\
4.51025380530829	-2.46008347433019\\
4.50597871872253	-2.45332154897912\\
4.50170211757532	-2.44656058142112\\
4.49742378576348	-2.43980070889915\\
4.49314350728911	-2.43304206882227\\
4.48886106627275	-2.42628479878629\\
4.48457624696672	-2.41952903659448\\
4.48028883376836	-2.41277492027825\\
4.47599861123345	-2.40602258811777\\
4.47170536408949	-2.39927217866259\\
4.46740887724919	-2.39252383075225\\
4.46310893582394	-2.38577768353681\\
4.45880532513728	-2.37903387649738\\
4.45449783073851	-2.37229254946664\\
4.45018623841631	-2.36555384264928\\
4.44587033421243	-2.35881789664239\\
4.44154990443541	-2.3520848524559\\
4.43722473567443	-2.34535485153286\\
4.43289461481317	-2.33862803576974\\
4.42855932904381	-2.33190454753672\\
4.42421866588098	-2.32518452969779\\
4.41987241317596	-2.31846812563097\\
4.4155203591308	-2.31175547924837\\
4.41116229231264	-2.30504673501622\\
4.40679800166809	-2.29834203797482\\
4.40242727653761	-2.29164153375846\\
4.39804990667016	-2.2849453686153\\
4.39366568223778	-2.2782536894271\\
4.38927439385035	-2.27156664372893\\
4.38487583257048	-2.26488437972885\\
4.38046978992844	-2.25820704632743\\
4.37605605793721	-2.25153479313726\\
4.37163442910772	-2.24486777050237\\
4.3672046964641	-2.23820612951753\\
4.36276665355913	-2.23155002204751\\
4.35832009448974	-2.22489960074624\\
4.35386481391272	-2.21825501907587\\
4.34940060706049	-2.21161643132577\\
4.34492726975703	-2.20498399263135\\
4.34044459843391	-2.19835785899295\\
4.33595239014654	-2.19173818729441\\
4.33145044259044	-2.18512513532174\\
4.32693855411775	-2.17851886178155\\
4.32241652375386	-2.17191952631943\\
4.31788415121414	-2.16532728953823\\
};
\addplot [color=green, line width=2.0pt, forget plot]
  table[row sep=crcr]{%
4.31788415121414	-2.16532728953823\\
4.29139791655541	-2.12724846061731\\
4.26455957224949	-2.08941697752051\\
4.23737142802345	-2.05183609605227\\
4.20983582370837	-2.01450905045006\\
4.181955129038	-1.97743905310606\\
4.15373174344484	-1.94062929429069\\
4.12516809585358	-1.90408294187806\\
4.09626664447211	-1.86780314107334\\
4.06702987657995	-1.83179301414206\\
4.03746030831423	-1.79605566014146\\
4.00756048445308	-1.76059415465371\\
3.97733297819666	-1.72541154952127\\
3.94678039094576	-1.69051087258425\\
3.91590535207782	-1.65589512741982\\
3.88471051872071	-1.62156729308369\\
3.85319857552406	-1.58753032385379\\
3.82137223442819	-1.55378714897598\\
3.7892342344307	-1.52034067241195\\
3.75678734135083	-1.48719377258935\\
3.72403434759133	-1.45434930215404\\
3.69097807189823	-1.42181008772456\\
3.65762135911818	-1.38957892964894\\
3.62396707995371	-1.35765860176363\\
3.59001813071608	-1.32605185115483\\
3.55577743307608	-1.29476139792205\\
3.52124793381261	-1.26378993494402\\
3.48643260455899	-1.23314012764694\\
3.45133444154729	-1.20281461377511\\
3.41595646535049	-1.17281600316387\\
3.38030172062245	-1.14314687751508\\
3.34437327583594	-1.11380979017483\\
3.30817422301855	-1.08480726591383\\
3.27170767748662	-1.05614180070998\\
3.23497677757704	-1.02781586153368\\
3.1979846843773	-0.999831886135469\\
3.16073458145333	-0.972192282836237\\
3.12322967457555	-0.944899430319965\\
3.08547319144301	-0.917955677429011\\
3.0474683814056	-0.891363342961982\\
3.00921851518439	-0.865124715474152\\
2.97072688459014	-0.839242053080521\\
2.93199680224007	-0.813717583261491\\
2.89303160127269	-0.788553502671144\\
2.85383463506099	-0.763751976948206\\
2.81440927692387	-0.739315140529686\\
2.77475891983577	-0.715245096467159\\
2.73488697613471	-0.691543916245787\\
2.69479687722862	-0.66821363960606\\
2.65449207330002	-0.645256274368229\\
2.61397603300909	-0.622673796259522\\
2.57325224319518	-0.600468148744128\\
2.53232420857668	-0.578641242855912\\
2.49119545144947	-0.557194957033968\\
2.44986951138371	-0.536131136960955\\
2.40834994491932	-0.515451595404263\\
2.36664032525981	-0.495158112059992\\
2.32474424196478	-0.475252433399796\\
2.28266530064109	-0.455736272520589\\
2.24040712263244	-0.436611308997099\\
2.19797334470777	-0.41787918873733\\
2.1553676187483	-0.399541523840925\\
2.11259361143321	-0.381599892460404\\
2.06965500392408	-0.364055838665364\\
2.02655549154812	-0.346910872309599\\
1.98329878348011	-0.330166468901145\\
1.93988860242322	-0.313824069475305\\
1.89632868428859	-0.297885080470633\\
1.85262277787388	-0.282350873607901\\
1.80877464454057	-0.267222785772033\\
1.76478805789029	-0.252502118897061\\
1.72066680344009	-0.238190139854083\\
1.6764146782966	-0.224288080342227\\
1.63203549082929	-0.210797136782651\\
1.5875330603427	-0.197718470215587\\
1.54291121674776	-0.185053206200413\\
1.49817380023214	-0.172802434718784\\
1.45332466092987	-0.160967210080837\\
1.40836765858984	-0.14954855083445\\
1.36330666224378	-0.138547439677583\\
1.31814554987316	-0.127964823373713\\
1.27288820807556	-0.11780161267035\\
1.22753853173009	-0.10805868222066\\
1.18210042366226	-0.0987368705081873\\
1.13657779430809	-0.0898369797747016\\
1.09097456137752	-0.081359775951149\\
1.04529464951731	-0.073305988591738\\
0.9995419899733	-0.0656763108111574\\
0.953720520251996	-0.0584713992249201\\
0.907834183781774	-0.0516918738928572\\
0.861886929573506	-0.0453383182657586\\
0.815882711880666	-0.0394112791351537\\
0.769825489859057	-0.0339112665862587\\
0.72371922722606	-0.0288387539540782\\
0.67756789191955	-0.0241941777826703\\
0.631375455756369	-0.019977937787572\\
0.585145894090524	-0.0161903968214064\\
0.538883185471093	-0.0128318808426537\\
0.492591311299785	-0.00990267888759281\\
0.446274255488311	-0.00740304304543541\\
0.39993600411556	-0.00533318843662593\\
};
\addplot [color=red, line width=2.0pt, forget plot]
  table[row sep=crcr]{%
0.39993600411556	-0.00533318843662593\\
0.395939140381123	-0.00517482353047837\\
0.39194215247312	-0.0050196241896639\\
0.38794504411515	-0.00486755853466557\\
0.383947818955657	-0.00471859468115007\\
0.379950480568712	-0.00457270074011162\\
0.375953032454796	-0.00442984481801698\\
0.371955478041518	-0.00428999501694311\\
0.36795782068441	-0.00415311943471652\\
0.363960063667672	-0.00401918616504569\\
0.359962210204962	-0.00388816329765356\\
0.355964263440121	-0.00376001891840599\\
0.351966226447952	-0.00363472110943548\\
0.347968102235008	-0.00351223794926726\\
0.343969893740315	-0.00339253751293776\\
0.339971603836151	-0.00327558787211185\\
0.335973235328836	-0.00316135709519803\\
0.331974790959454	-0.0030498132474611\\
0.327976273404634	-0.0029409243911315\\
0.32397768527734	-0.00283465858551325\\
0.319979029127594	-0.0027309838870861\\
0.315980307443256	-0.00262986834960889\\
0.311981522650818	-0.00253128002422104\\
0.307982677116121	-0.00243518695953558\\
0.303983773145151	-0.00234155720173657\\
0.299984812984789	-0.00225035879467128\\
0.295985798823606	-0.00216155977993842\\
0.291986732792586	-0.00207512819697662\\
0.287987616965913	-0.00199103208315052\\
0.283988453361758	-0.00190923947383243\\
0.279989243943003	-0.00182971840248333\\
0.275989990618025	-0.00175243690073175\\
0.271990695241491	-0.00167736299844988\\
0.267991359615072	-0.00160446472382805\\
0.263991985488241	-0.00153371010344491\\
0.259992574559052	-0.00146506716234164\\
0.255993128474869	-0.00139850392408526\\
0.25199364883316	-0.00133398841083698\\
0.247994137182249	-0.00127148864341565\\
0.24399459502211	-0.0012109726413599\\
0.239995023805097	-0.00115240842298755\\
0.235995424936723	-0.00109576400545401\\
0.231995799776456	-0.00104100740480884\\
0.227996149638443	-0.000988106636049526\\
0.223996475792292	-0.000937029713174306\\
0.219996779463868	-0.00088774464923326\\
0.215997061836015	-0.00084021945637679\\
0.211997324049341	-0.000794422145903208\\
0.207997567203013	-0.000750320728304604\\
0.203997792355478	-0.000707883213310317\\
0.199998000525258	-0.000667077609929607\\
0.195998192691707	-0.000627871926492247\\
0.191998369795804	-0.000590234170687964\\
0.187998532740877	-0.000554132349603673\\
0.183998682393397	-0.000519534469759833\\
0.179998819583761	-0.000486408537145264\\
0.175998945107021	-0.000454722557250015\\
0.171999059723671	-0.000424444535097356\\
0.167999164160436	-0.000395542475274345\\
0.163999259110999	-0.000367984381960582\\
0.159999345236789	-0.000341738258956127\\
0.15599942316777	-0.000316772109708099\\
0.151999493503164	-0.000293053937335573\\
0.147999556812244	-0.00027055174465373\\
0.143999613635123	-0.000249233534196711\\
0.139999664483475	-0.000229067308239018\\
0.135999709841335	-0.000210021068816108\\
0.131999750165852	-0.000192062817743722\\
0.127999785888085	-0.000175160556636257\\
0.123999817413726	-0.000159282286923856\\
0.119999845123896	-0.000144396009868737\\
0.115999869375925	-0.000130469726580475\\
0.111999890504086	-0.000117471438030191\\
0.107999908820376	-0.00010536914506398\\
0.103999924615315	-9.41308484154604e-05\\
0.09999993815866	-8.37245487172509e-05\\
0.0959999497002049	-7.41182465118738e-05\\
0.0919999594705627	-6.52799422617453e-05\\
0.0879999676818936	-5.7177636358339e-05\\
0.0839999745287012	-4.97793291307189e-05\\
0.0799999801885861	-4.30530208533207e-05\\
0.075999984823042	-3.69667117530689e-05\\
0.0719999885781815	-3.14884020157871e-05\\
0.0679999915855253	-2.65860917920986e-05\\
0.0639999939627913	-2.22277812026814e-05\\
0.0599999958146202	-1.83814703429509e-05\\
0.0559999972333646	-1.50151592872668e-05\\
0.0519999982998787	-1.20968480926506e-05\\
0.0479999990842431	-9.59453680197076e-06\\
0.0439999996465545	-7.4762254467949e-06\\
0.0400000000377148	-5.70991404978691e-06\\
0.036000000300157	-4.26360262671701e-06\\
0.0320000004686336	-3.10529118819289e-06\\
0.0280000005710068	-2.20297974104452e-06\\
0.0240000006289731	-1.52466828944284e-06\\
0.0200000006588606	-1.03835683574875e-06\\
0.0160000006723817	-7.12045381191291e-07\\
0.0120000006774307	-5.1373392635359e-07\\
0.00800000067880826	-4.11422471427183e-07\\
0.00400000067901098	-3.73111016492806e-07\\
6.79021061372976e-10	-3.66799561558748e-07\\
};
\addplot [color=red, line width=2.0pt, only marks, mark size=2.5pt, mark=*, mark options={solid, fill=red, red}, forget plot]
  table[row sep=crcr]{%
0.399936004740566	0.00533272384055325\\
};
\addplot [color=red, line width=2.0pt, only marks, mark size=2.5pt, mark=*, mark options={solid, fill=red, red}, forget plot]
  table[row sep=crcr]{%
4.31788468040832	2.16532586744377\\
};
\addplot [color=red, line width=2.0pt, only marks, mark size=2.5pt, mark=*, mark options={solid, fill=red, red}, forget plot]
  table[row sep=crcr]{%
4.75388580337923	2.8360061221805\\
};
\addplot [color=red, line width=2.0pt, only marks, mark size=2.5pt, mark=*, mark options={solid, fill=red, red}, forget plot]
  table[row sep=crcr]{%
4.75388511027439	-2.83600765083127\\
};
\addplot [color=red, line width=2.0pt, only marks, mark size=2.5pt, mark=*, mark options={solid, fill=red, red}, forget plot]
  table[row sep=crcr]{%
4.31788415121414	-2.16532728953823\\
};
\addplot [color=red, line width=2.0pt, only marks, mark size=2.5pt, mark=*, mark options={solid, fill=red, red}, forget plot]
  table[row sep=crcr]{%
0.39993600411556	-0.00533318843662593\\
};
\addplot [color=red, line width=2.0pt, only marks, mark size=2.5pt, mark=*, mark options={solid, fill=red, red}, forget plot]
  table[row sep=crcr]{%
6.7901873027764e-10	-3.66799561558196e-07\\
};
\addplot [color=blue, line width=2.0pt, only marks, mark size=2.5pt, mark=*, mark options={solid, fill=blue, blue}, forget plot]
  table[row sep=crcr]{%
0	0\\
};
\addplot [color=blue, line width=2.0pt, only marks, mark size=2.5pt, mark=*, mark options={solid, fill=blue, blue}, forget plot]
  table[row sep=crcr]{%
0	0\\
};
\end{axis}
\end{tikzpicture}%%
  \caption{Duboids solution example 4}
  \label{fig:DuboidsRes3}
\end{figure}
%
The fifth example (Figure \ref{fig:DuboidsRes4}) shows a connection from $x=0$, $y=0$, $\theta=0$ and $\kappa=0$ to $x=0$, $y=0$, $\theta=\pi/2$ and $\kappa=0$ obtaining a convoluted form.\\
%
\begin{figure}[ht]
  \centering
  % This file was created by matlab2tikz.
%
%The latest updates can be retrieved from
%  http://www.mathworks.com/matlabcentral/fileexchange/22022-matlab2tikz-matlab2tikz
%where you can also make suggestions and rate matlab2tikz.
%
\begin{tikzpicture}

\begin{axis}[%
width=\linewidth,
height=0.776\linewidth,
at={(0\linewidth,0\linewidth)},
scale only axis,
xmin=-4.60136435657085,
xmax=5.03889624988018,
xlabel style={font=\color{white!15!black}},
xlabel={x(m)},
ymin=-3.58291739226538,
ymax=4.02044944088712,
ylabel style={font=\color{white!15!black}},
ylabel={y(m)},
axis background/.style={fill=white},
title style={font=\bfseries},
title={$L_{tot}$ = 37.7126, $k_{max}$ = 0.5, $J_{max}$ = 0.5, Type = [LRL]},
axis x line*=bottom,
axis y line*=left,
xmajorgrids,
xminorgrids,
ymajorgrids,
yminorgrids
]
\addplot [color=red, line width=2.0pt, forget plot]
  table[row sep=crcr]{%
0	0\\
0.009999999999375	8.33333333296131e-08\\
0.01999999998	6.66666666190476e-07\\
0.029999999848125	2.24999999186384e-06\\
0.03999999936	5.33333327238095e-06\\
0.049999998046875	1.04166663760231e-05\\
0.0599999951400002	1.79999989585715e-05\\
0.0699999894956257	2.85833302695574e-05\\
0.0799999795200024	4.26666588647625e-05\\
0.089999963094382	6.07499822062188e-05\\
0.0999999375000181	8.33332961309598e-05\\
0.109999899343168	0.000110916594169772\\
0.119999844480093	0.000143999866697198\\
0.129999767942067	0.000183083099894043\\
0.139999663860374	0.000228666274503633\\
0.14999952539132	0.000281249364363084\\
0.159999344641243	0.000341332334690825\\
0.16999911259152	0.000409415140111639\\
0.179998819023587	0.000485997722400469\\
0.189998452443961	0.000571580007926256\\
0.199998000009259	0.000666661904777056\\
0.209997447451239	0.000771743299547711\\
0.219996779001833	0.000887324053771324\\
0.229995977318198	0.00101390399997582\\
0.239995023407776	0.00115198293734686\\
0.249993896553361	0.00130206062697838\\
0.259992574238189	0.00146463678669201\\
0.269991032071029	0.00164021108540679\\
0.279989243711304	0.00182928313704023\\
0.289987180794227	0.00203235249392229\\
0.299984812855953	0.00224991863970341\\
0.309982107258766	0.00248248098173797\\
0.319979029116282	0.0027305388429245\\
0.329975541218689	0.00299459145298403\\
0.339971603958015	0.00327513793915787\\
0.349967175253435	0.0035726773163062\\
0.359962210476616	0.00388770847638889\\
0.369956662377103	0.00422073017731001\\
0.379950481007759	0.00457224103110726\\
0.389943613650245	0.00494273949146801\\
0.399936004740566	0.00533272384055325\\
0.409927595794669	0.00574269217511096\\
0.419918325334115	0.00617314239186055\\
0.429908128811808	0.00662457217212975\\
0.439896938537814	0.00709747896572564\\
0.44988468360525	0.00759235997402141\\
0.459871289816262	0.00810971213224044\\
0.469856679608105	0.00865003209091956\\
0.479840771979308	0.00921381619653296\\
0.489823482415957	0.00980156047125886\\
0.499804722818087	0.0104137605918704\\
0.509784401426195	0.0110509118677329\\
0.519762422747881	0.0117135092178894\\
0.529738687484628	0.0124020471472162\\
0.539713092458725	0.0131170197216309\\
0.549685530540343	0.0138589205423346\\
0.559655890574774	0.0146282427190708\\
0.569624057309844	0.015425478842383\\
0.579589911323503	0.0162511209548533\\
0.589553328951615	0.0171056605213053\\
0.599514182215937	0.0179895883979516\\
0.609472338752324	0.0189033948004718\\
0.619427661739149	0.0198475692710004\\
0.629380009825963	0.0208226006440093\\
0.639329237062402	0.0218289770110674\\
0.64927519282735	0.0228671856844605\\
0.659217721758378	0.0239377131596537\\
0.669156663681469	0.0250410450765807\\
0.679091853541037	0.0261776661797427\\
0.689023121330264	0.0273480602771007\\
0.698950292021764	0.0285527101977446\\
0.708873185498583	0.0297920977483229\\
0.718791616485562	0.0310667036682178\\
0.72870539448107	0.0323770075834479\\
0.738614323689129	0.033723487959285\\
0.748518202951936	0.0351066220515678\\
0.758416825682818	0.0365268858566984\\
0.768309979799618	0.0379847540603047\\
0.77819744765854	0.0394806999845558\\
0.788079005988475	0.0410151955341141\\
0.79795442582581	0.0425887111407103\\
0.807823472449766	0.0442017157063264\\
0.817685905318254	0.045854676544973\\
0.827541478004292	0.0475480593230471\\
0.837389938132998	0.0492823279982562\\
0.847231027319174	0.0510579447570965\\
0.857064481105505	0.0528753699508699\\
0.866890028901404	0.0547350620302302\\
0.876707393922514	0.0566374774782424\\
0.886516293130892	0.0585830707419458\\
0.896316437175907	0.0605722941624074\\
0.906107530335867	0.0626055979032542\\
0.915889270460404	0.0646834298776731\\
0.925661348913641	0.0668062356738681\\
0.935423450518165	0.0689744584789635\\
0.945175253499836	0.0711885390013423\\
0.95491642943346	0.0734489153914118\\
0.964646643189338	0.0757560231607851\\
0.974365552880748	0.0781102950998698\\
0.984072809812352	0.0805121611938562\\
0.993768058429589	0.082962048537095\\
};
\addplot [color=green, line width=2.0pt, forget plot]
  table[row sep=crcr]{%
0.993768058429589	0.082962048537095\\
1.09032874389115	0.110215465856275\\
1.18540114459326	0.142277177556056\\
1.27874599385169	0.179066494610061\\
1.37012837267417	0.220490830128294\\
1.45931830097588	0.266445932368596\\
1.54609131636552	0.316816147104778\\
1.630229039045	0.37147470869115\\
1.71151972140121	0.43028405909092\\
1.78975878090672	0.493096194065571\\
1.8647493149883	0.559753035653971\\
1.93630259656738	0.630086830003813\\
2.0042385490255	0.703920569554155\\
2.06838619939931	0.781068438506577\\
2.12858410866454	0.861336280463842\\
2.18468077802617	0.944522087059157\\
2.23653503019222	1.03041650634633\\
2.28401636467164	1.11880336967134\\
2.32700528620211	1.20946023569939\\
2.3653936054812	1.30215895022827\\
2.39908471144409	1.39666622037919\\
2.4279938144026	1.49274420171994\\
2.45204815943357	1.59015109684299\\
2.47118720947963	1.68864176389178\\
2.4853627977015	1.78796833350406\\
2.49453924869849	1.88788083261944\\
2.49869346829199	1.98812781358121\\
2.49781500164611	2.08845698694935\\
2.49190605957911	2.18861585643193\\
2.48098151299955	2.28835235433722\\
2.46506885548093	2.38741547594701\\
2.4442081340692	2.48555591121486\\
2.41845184849727	2.58252667219939\\
2.38786481905998	2.67808371465353\\
2.35252402348231	2.77198655220553\\
2.31251840319112	2.86399886158584\\
2.2679486394782	2.95388907737686\\
2.21892690011781	3.04143097478869\\
2.1655765570764	3.12640423899423\\
2.10803187602504	3.20859501959082\\
2.04643767843585	3.287796468793\\
1.98094897711291	3.36380926200187\\
1.91173058607482	3.43644209944114\\
1.83895670577079	3.50551218759713\\
1.76281048467406	3.57084569925132\\
1.68348355835607	3.63227821094766\\
1.60117556720128	3.68965511679351\\
1.51609365397647	3.74283201755302\\
1.42845194251897	3.79167508405362\\
1.33847099885577	3.83606139399102\\
1.24637727610965	3.8758792412852\\
1.15240254458951	3.91102841720865\\
1.0567833084989	3.9414204625796\\
0.959760210730999	3.96697889038533\\
0.861577427247719	3.98763937827541\\
0.762482052567302	4.00334993044039\\
0.662723477906797	4.01407100846852\\
0.56255276354451	4.01977563085128\\
0.462222006981947	4.02044944088712\\
0.361983708495408	4.01609074281269\\
0.262090135673922	4.00671050607052\\
0.162792688542787	3.99233233770245\\
0.0643412668704865	3.97299242293833\\
-0.0330163587487385	3.94873943412937\\
-0.129035170453855	3.91963440825551\\
-0.223473519748267	3.88575059331494\\
-0.316093735651443	3.84717326398241\\
-0.406662722840391	3.80399950700032\\
-0.494952548275706	3.75633797684252\\
-0.580741014835778	3.70430862226598\\
-0.663812220515563	3.64804238443824\\
-0.743957101782546	3.58768086740062\\
-0.820973959722484	3.52337598169634\\
-0.894668967650758	3.45528956206047\\
-0.964856658911868	3.38359296013391\\
-1.0313603936394	3.30846661322638\\
-1.09401280330182	3.23009959021367\\
-1.15265621191528	3.14868911571208\\
-1.20714303286343	3.06444007372734\\
-1.25733614032547	2.97756449202751\\
-1.30310921437781	2.88828100853711\\
-1.34434705890073	2.79681432109582\\
-1.38094589148999	2.70339462196611\\
-1.41281360464377	2.60825701851338\\
-1.4398699975677	2.51164094151619\\
-1.46204697801448	2.41378954259598\\
-1.47928873365015	2.3149490822825\\
-1.49155187251585	2.21536831025525\\
-1.4988055322314	2.11529783932046\\
-1.50103145766603	2.01498951469912\\
-1.49822404688065	1.91469578021355\\
-1.49039036522613	1.81466904296736\\
-1.4775501275621	1.71516103811794\\
-1.45973564864098	1.61642219533989\\
-1.43699176178211	1.51870100857401\\
-1.40937570604074	1.42224341064793\\
-1.37695698215564	1.32729215434216\\
-1.33981717763817	1.2340862014594\\
-1.29804976144263	1.14286012143438\\
-1.25175984873503	1.05384350099807\\
-1.2010639363519	0.967260366381544\\
};
\addplot [color=red, line width=2.0pt, forget plot]
  table[row sep=crcr]{%
-1.2010639363519	0.967260366381544\\
-1.19044441022212	0.950312729443986\\
-1.17965930809285	0.933469973264685\\
-1.1687130132101	0.91673152590558\\
-1.15760986110355	0.900096709096101\\
-1.14635413899746	0.883564741290592\\
-1.13495008533281	0.867134740692135\\
-1.12340188939738	0.850805728241255\\
-1.11171369106024	0.834576630568183\\
-1.09988958060768	0.818446282907478\\
-1.08793359867698	0.802413431973994\\
-1.07584973628515	0.786476738799303\\
-1.06364193494926	0.770634781527816\\
-1.05131408689522	0.754886058172003\\
-1.03887003535216	0.73922898932623\\
-1.02631357492907	0.723661920838859\\
-1.01364845207091	0.708183126442392\\
-1.00087836559127	0.692790810341517\\
-0.988006967278539	0.677483109759086\\
-0.975037862572854	0.662258097440124\\
-0.961974611311084	0.647113784114066\\
-0.948820728536989	0.632048120915548\\
-0.935579685373988	0.617059001764162\\
-0.922254909957835	0.602144265703654\\
-0.90884978842668	0.587301699201183\\
-0.895367665965934	0.572529038407272\\
-0.881811847905518	0.557823971377232\\
-0.868185600867033	0.543184140254875\\
-0.854492153958497	0.528607143419402\\
-0.840734700014319	0.514090537596462\\
-0.826916396878226	0.499631839934369\\
-0.813040368726926	0.485228530046627\\
-0.799109707432308	0.470878052021859\\
-0.78512747396005	0.456577816402406\\
-0.771096699802522	0.442325202132794\\
-0.75702038844394	0.428117558479437\\
-0.742901516855746	0.413952206922887\\
-0.728743037020237	0.399826443024063\\
-0.714547877480485	0.385737538265869\\
-0.700318944914661	0.371682741871695\\
-0.686059125732857	0.35765928260231\\
-0.671771287694584	0.343664370532685\\
-0.657458281545093	0.329695198810315\\
-0.643122942668753	0.315748945396647\\
-0.628768092757696	0.301822774793228\\
-0.614396541493989	0.28791383975423\\
-0.600011088243579	0.27401928298699\\
-0.585614523760334	0.260136238842272\\
-0.571209631898434	0.246261834995928\\
-0.556799191331467	0.23239319412367\\
-0.542385977276509	0.218527435570659\\
-0.52797276322155	0.204661677017648\\
-0.513562322654583	0.19079303614539\\
-0.499157430792683	0.176918632299046\\
-0.484760866309437	0.163035588154328\\
-0.470375413059028	0.149141031387088\\
-0.45600386179532	0.135232096348089\\
-0.441649011884264	0.121305925744671\\
-0.427313673007924	0.107359672331003\\
-0.413000666858433	0.0933905006086325\\
-0.39871282882016	0.0793955885390077\\
-0.384453009638356	0.0653721292696225\\
-0.370224077072532	0.0513173328754485\\
-0.35602891753278	0.0372284281172545\\
-0.341870437697271	0.0231026642184313\\
-0.327751566109077	0.00893731266188101\\
-0.313675254750495	-0.00527033099147642\\
-0.299644480592967	-0.0195229452610881\\
-0.285662247120709	-0.0338231808805411\\
-0.271731585826092	-0.0481736589053074\\
-0.257855557674791	-0.0625769687930512\\
-0.244037254538698	-0.0770356664551436\\
-0.23027980059452	-0.0915522722780841\\
-0.216586353685984	-0.106129269113557\\
-0.202960106647499	-0.120769100235914\\
-0.189404288587083	-0.135474167265954\\
-0.175922166126337	-0.150246828059865\\
-0.162517044595182	-0.165089394562336\\
-0.149192269179029	-0.180004130622844\\
-0.135951226016028	-0.19499324977423\\
-0.122797343241934	-0.210058912972748\\
-0.109734091980162	-0.225203226298807\\
-0.0967649872744782	-0.240428238617768\\
-0.0838935889617432	-0.255735939200199\\
-0.0711235024821073	-0.271128255301074\\
-0.0584583796239513	-0.286607049697541\\
-0.0459019192008529	-0.302174118184912\\
-0.0334578676577944	-0.317831187030685\\
-0.0211300196037606	-0.333579910386498\\
-0.00892221826786305	-0.349421867657985\\
0.00316164412396199	-0.365358560832676\\
0.0151176260546627	-0.38139141176616\\
0.0269417365072282	-0.397521759426866\\
0.0386299348443596	-0.413750857099937\\
0.050178130779794	-0.430079869550816\\
0.0615821844444396	-0.446509870149274\\
0.0728379065505356	-0.463041837954784\\
0.0839410586570836	-0.479676654764262\\
0.0948873535398293	-0.496415102123367\\
0.105672455669105	-0.513257858302668\\
0.116291981798888	-0.530205495240226\\
};
\addplot [color=green, line width=2.0pt, forget plot]
  table[row sep=crcr]{%
0.116291981798888	-0.530205495240226\\
0.174712960895231	-0.630909378742977\\
0.227175324532961	-0.734841969168952\\
0.273501300281821	-0.841651083583531\\
0.31353390929599	-0.950974791747085\\
0.347137498248005	-1.06244264254118\\
0.374198198999658	-1.17567691926842\\
0.394624314452239	-1.29029391957216\\
0.40834662926864	-1.40590525563914\\
0.415318644414418	-1.52211917027898\\
0.415516734723067	-1.63854186442124\\
0.408940228951565	-1.75477883153131\\
0.395611412054924	-1.87043619442381\\
0.375575449672051	-1.98512203994322\\
0.348900235078778	-2.09844774698939\\
0.315676159126702	-2.21002930338757\\
0.276015803947382	-2.31948860714089\\
0.230053561459828	-2.42645474765567\\
0.177945177973979	-2.53056526259821\\
0.119867226433306	-2.63146736612411\\
0.0560165080849091	-2.72881914431796\\
-0.0133906143954297	-2.82229071379291\\
-0.0881189500634783	-2.91156533952382\\
-0.167915276667711	-2.99634050812631\\
-0.252509198710815	-3.07632895294478\\
-0.341614063703961	-3.15125962747573\\
-0.434927933509036	-3.22087862382811\\
-0.532134607477395	-3.28495003310807\\
-0.63290469391811	-3.343256744813\\
-0.736896726265025	-3.3956011825258\\
-0.843758320160366	-3.44180597341655\\
-0.953127367534065	-3.4817145492829\\
-1.06463326363258	-3.51519167709248\\
-1.17789816283929	-3.5421239172296\\
-1.29253825903114	-3.56242000789344\\
-1.40816508613279	-3.57601117434507\\
-1.52438683446142	-3.58285136195558\\
-1.64080967840156	-3.58291739226538\\
-1.75703911091101	-3.5762090415261\\
-1.87268128033591	-3.56274904145875\\
-1.9873443250049	-3.54258300222569\\
-2.1006397010801	-3.51577925787736\\
-2.21218349916541	-3.48242863479752\\
-2.32159774521067	-3.44264414393163\\
-2.42851168130343	-3.39656059784126\\
-2.53256302200839	-3.34433415388217\\
-2.63339918199721	-3.28614178505411\\
-2.73067847080883	-3.22218068031525\\
-2.82407125069181	-3.15266757639346\\
-2.9132610536051	-3.07783802335859\\
-2.99794565359245	-2.99794558644437\\
-3.0778380908964	-2.91326098682469\\
-3.15266764434174	-2.82407118425577\\
-3.22218074869334	-2.73067840469271\\
-3.28614185387989	-2.63339911617544\\
-3.34433422317201	-2.53256295645443\\
-3.39656066760996	-2.42851161598983\\
-3.44264421419236	-2.32159768010915\\
-3.48242870556179	-2.21218343424698\\
-3.51577932915497	-2.10063963631515\\
-3.5425830740247	-1.98734426036331\\
-3.56274911378546	-1.87268121578713\\
-3.57620911438501	-1.75703904642417\\
-3.5829174656592	-1.64080961394559\\
-3.58285143588519	-1.52438677000515\\
-3.57601124880955	-1.40816502164505\\
-3.56242008289004	-1.29253819448085\\
-3.5421239927538	-1.17789809819559\\
-3.51519175313794	-1.06463319886493\\
-3.48171462584153	-0.953127302612351\\
-3.44180605047851	-0.843758255054984\\
-3.39560126007955	-0.736896660947005\\
-3.34325682284533	-0.632904628359194\\
-3.28495011160416	-0.532134541650146\\
-3.22087870277156	-0.434927867386921\\
-3.15125970684863	-0.341613997261447\\
-3.07632903272774	-0.25250913192346\\
-2.99634058829859	-0.167915209512238\\
-2.91156542006333	-0.0881188825178569\\
-2.82229079467633	-0.0133905464389531\\
-2.7288192255208	0.056016576471553\\
-2.6314674476208	0.119867295267976\\
-2.53056534436219	0.177945247273014\\
-2.42645482965945	0.230053631237995\\
-2.3194886893562	0.276015874217821\\
-2.21002938578541	0.315676229900888\\
-2.09844782954012	0.348900306366479\\
-1.98512212261672	0.375575521481291\\
-1.87043627718951	0.395611484391965\\
-1.75477891435836	0.408940301820877\\
-1.63854194727855	0.415516808127318\\
-1.52211925313539	0.415318718354463\\
-1.40590533846345	0.408346703743517\\
-1.29029400233333	0.394624389459176\\
-1.17567700193558	0.374198274534077\\
-1.06244272508381	0.347137574303544\\
-0.95097487413506	0.31353398586452\\
-0.841651165787274	0.273501377353476\\
-0.734842051159495	0.227175402096166\\
-0.63090946049208	0.174713038936747\\
-0.530205576720467	0.116292060303857\\
};
\addplot [color=red, line width=2.0pt, forget plot]
  table[row sep=crcr]{%
-0.530205576720467	0.116292060303857\\
-0.513257939734036	0.105672534252069\\
-0.496415183505101	0.0948874322003065\\
-0.479676736095621	0.0839411373945944\\
-0.463041919235044	0.072837985364602\\
-0.446509951377731	0.0615822633345866\\
-0.430079950726791	0.050178209745555\\
-0.413750938222766	0.0386300138852692\\
-0.397521840495904	0.0269418156228262\\
-0.381391492780782	0.0151177052444948\\
-0.365358641792276	0.00316172338757965\\
-0.349421948561974	-0.00892213893090153\\
-0.333579991234305	-0.0211299401938921\\
-0.317831267821755	-0.0334577881754504\\
-0.302174198918713	-0.0459018396464527\\
-0.286607130373557	-0.0584582999979082\\
-0.271128335918803	-0.0711234227848285\\
-0.255736019759158	-0.0838935091936271\\
-0.240428319117491	-0.0967649074359139\\
-0.225203306738845	-0.10973401207153\\
-0.210058993352668	-0.122797263263604\\
-0.194993330093611	-0.135951145968366\\
-0.180004210881288	-0.149192189062386\\
-0.165089474759459	-0.162516964409897\\
-0.150246908195296	-0.175922085872745\\
-0.135474247339338	-0.189404208265505\\
-0.120769180246912	-0.202960026258247\\
-0.106129349061845	-0.216586273229357\\
-0.0915523521633546	-0.230279720070807\\
-0.0770357462771005	-0.244037173948178\\
-0.0625770485514119	-0.257855477017732\\
-0.0481737385998097	-0.271731505102746\\
-0.0338232605109327	-0.28566216633132\\
-0.0195230248271314	-0.299644399737766\\
-0.00527041049294841	-0.313675173829701\\
0.00893723322519029	-0.327751485122898\\
0.0231025848467164	-0.3418703566459\\
0.0372283488106988	-0.3560288364164\\
0.0513172536342236	-0.370223995891315\\
0.065372050093881	-0.384452928392458\\
0.0793955094288906	-0.398712747509722\\
0.0933904215642699	-0.413000585483589\\
0.10735959335251	-0.427313591568792\\
0.121305846832152	-0.441648930380949\\
0.135232017501633	-0.456003780227914\\
0.149140952606771	-0.470375331427611\\
0.163035509440217	-0.484760784614078\\
0.17691855365119	-0.499157349033432\\
0.190792957563827	-0.513562240831479\\
0.204661598502403	-0.52797268133462\\
0.218527357121746	-0.542385895325767\\
0.232393115741088	-0.556799109316912\\
0.246261756679665	-0.571209549820054\\
0.260136160592301	-0.585614441618101\\
0.274019204803274	-0.600011006037455\\
0.28791376163672	-0.614396459223922\\
0.301822696741858	-0.628768010423619\\
0.315748867411339	-0.643122860270584\\
0.329695120890981	-0.657458199082741\\
0.343664292679221	-0.671771205167944\\
0.357659204814601	-0.686059043141811\\
0.37168266414961	-0.700318862259075\\
0.385737460609268	-0.714547794760217\\
0.399826365432792	-0.728742954235133\\
0.413952129396775	-0.742901434005633\\
0.428117481018301	-0.757020305528635\\
0.44232512473644	-0.771096616821832\\
0.456577739070623	-0.785127390913767\\
0.470877974754424	-0.799109624320213\\
0.485228452843301	-0.813040285548787\\
0.499631762794903	-0.826916313633801\\
0.514090460520592	-0.840734616703355\\
0.528607066406846	-0.854492070580726\\
0.543184063305337	-0.868185517422176\\
0.557823894490404	-0.881811764393286\\
0.572528961582829	-0.895367582386027\\
0.587301622438787	-0.908849704778788\\
0.60214418900295	-0.922254826241635\\
0.617058925124779	-0.935579601589147\\
0.632048044337102	-0.948820644683166\\
0.647113707596159	-0.961974527387929\\
0.662258020982336	-0.975037778580003\\
0.677483033360983	-0.988006883215619\\
0.692790734002649	-1.00087828145791\\
0.708183050162294	-1.0136483678667\\
0.723661844617048	-1.02631349065362\\
0.739228913162204	-1.03886995100508\\
0.754885982065246	-1.05131400247608\\
0.770634705477797	-1.06364185045764\\
0.786476662805465	-1.07584965172063\\
0.802413356035767	-1.08793351403911\\
0.818446207024273	-1.09988949589603\\
0.834576554739395	-1.11171360627436\\
0.850805652466258	-1.1234018045368\\
0.867134664970283	-1.13495000039709\\
0.883564665621222	-1.14635405398612\\
0.900096633478536	-1.15760977601614\\
0.916731450339112	-1.16871292804613\\
0.933469897748592	-1.17965922285184\\
0.950312653977528	-1.1904443249036\\
0.967260290963958	-1.20106385095539\\
};
\addplot [color=green, line width=2.0pt, forget plot]
  table[row sep=crcr]{%
0.967260290963958	-1.20106385095539\\
1.05384342712276	-1.251759763663\\
1.14286004915029	-1.29804967655079\\
1.23408613080478	-1.33981709277827\\
1.32729208534435	-1.37695689717624\\
1.42224334332308	-1.4093756207879\\
1.51870094292695	-1.43699167610017\\
1.61642213136399	-1.45973556237326\\
1.71516097579506	-1.47755004055171\\
1.81466898226775	-1.4903902773167\\
1.91469572109594	-1.49822395791712\\
2.01498945711077	-1.5010313674954\\
2.11529778319723	-1.4988054407035\\
2.21536825552181	-1.49155177948409\\
2.31494902885251	-1.47928863897226\\
2.41378949037235	-1.46204688155327\\
2.51164089039141	-1.43986989919176\\
2.60825696836979	-1.41281350422815\\
2.70339457267636	-1.38094578891693\\
2.79681427252327	-1.34434695406028\\
2.88828096053633	-1.30310910716845\\
2.97756444444474	-1.25733603065476\\
3.0644400264011	-1.20714292064855\\
3.1486890684737	-1.15265609708358\\
3.23009954288798	-1.09401268579131\\
3.30846656563228	-1.03136027339924\\
3.38359291208513	-0.964856535902783\\
3.45528951336625	-0.894668841845432\\
3.52337593216215	-0.820973831105894\\
3.58768081682895	-0.74395697035228\\
3.64804233262935	-0.663812086282061\\
3.70430856901874	-0.580740877822547\\
3.75633792195521	-0.494952408519471\\
3.80399945027141	-0.406662580391225\\
3.84717320521146	-0.316093590572807\\
3.88575053230339	-0.223473372117042\\
3.91963434480757	-0.129035020360284\\
3.94873936805282	-0.0330162062963445\\
3.97299235404538	0.0643414215650395\\
3.9923322658106	0.162792845349893\\
4.00671043100335	0.262090294451278\\
4.0160906644007	0.361983869088301\\
4.02044935896853	0.462222169223607\\
4.01977554527279	0.562552927256499\\
4.0140709190861	0.662723642899457\\
4.00334983712002	0.762482218640231\\
3.98763928089378	0.861577594190325\\
3.96697878883052	0.959760378323061\\
3.94142035675176	1.05678347651121\\
3.91102830702058	1.15240271278451\\
3.87587912666296	1.24637744424217\\
3.83606127487448	1.33847116667373\\
3.79167496039692	1.4284521097642\\
3.74283188932508	1.51609382038547\\
3.6896549839784	1.6011757325061\\
3.63227807354499	1.68348372228515\\
3.57084555727654	1.76281064695313\\
3.50551204108181	1.83895686612378\\
3.43644194843319	1.9117307442248\\
3.36380910656567	1.98094913278303\\
3.28779630900951	2.04643783135033\\
3.20859485555768	2.10803202591011\\
3.12640407082566	2.1655767036613\\
3.04143080261553	2.21892704313576\\
2.95388890134639	2.26794877866741\\
2.86399868186159	2.31251853829573\\
2.77198636896709	2.35252415425344\\
2.67808352809622	2.38786494525663\\
2.58252648253393	2.4184519698873\\
2.48555571866696	2.4442082504303\\
2.38741528075691	2.46506896660147\\
2.28835215675916	2.48098161867957\\
2.1886156567336	2.49190615963112\\
2.08845678541124	2.49781509589596\\
1.98812761049596	2.49869355657972\\
1.88788062829111	2.49453933087906\\
1.78796812824736	2.48536287364554\\
1.68864155803123	2.47118727907417\\
1.5901508907121	2.45204822258268\\
1.49274399566029	2.427993871028\\
1.39666601473945	2.39908476148572\\
1.3021587453633	2.36539364889775\\
1.20946003196916	2.32700532297147\\
1.11880316743993	2.2840163947913\\
1.03041630598084	2.23653505367963\\
0.944521888928621	2.18468079491902\\
0.861336084938114	2.12858411902103\\
0.781068245955191	2.06838620329828\\
0.703920380345178	2.00423854656656\\
0.630086644502688	1.93630258787094\\
0.559752854222345	1.86474930019559\\
0.493096017060124	1.78975876017966\\
0.430283886862188	1.71151969492227\\
0.371474541582365	1.63022900701701\\
0.316815985450694	1.54609127901136\\
0.266445776494338	1.45931825853815\\
0.22049068034822	1.37012832541478\\
0.179066351226632	1.27874594205136\\
0.14227704085875	1.18540108855098\\
0.110215336120496	1.09032868392362\\
0.0829619260231352	0.993767994870484\\
};
\addplot [color=red, line width=2.0pt, forget plot]
  table[row sep=crcr]{%
0.0829619260231352	0.993767994870484\\
0.0805120394494035	0.984072746058801\\
0.0781101741258756	0.974365488936558\\
0.0757559029581759	0.964646579058293\\
0.0734487959610849	0.954916365119303\\
0.0711884203441675	0.945175189006276\\
0.0689743405957829	0.935423385848869\\
0.0668061185655005	0.925661284072257\\
0.0646833135449103	0.915889205450537\\
0.0626054823468618	0.906107465161078\\
0.0605721793831293	0.896316371839738\\
0.0585829567405	0.886516227636841\\
0.056637364255324	0.876707328274044\\
0.0547349495865097	0.866889963101936\\
0.0528752582869981	0.857064415158436\\
0.0510578338737009	0.847230961227858\\
0.0492822178959428	0.837389871900755\\
0.0475479500023997	0.827541411634396\\
0.0458545680065574	0.817685838813957\\
0.044201607950687	0.807823405814277\\
0.0425886041683709	0.797954359062293\\
0.0410150893455816	0.78807893910007\\
0.0394805945803175	0.778197380648346\\
0.0379846494408295	0.768309912670693\\
0.0365267820224367	0.758416758438178\\
0.0351065190029548	0.748518135594572\\
0.0337233856967376	0.738614256221988\\
0.0323769061073661	0.728705326907055\\
0.0310666029789874	0.718791548807551\\
0.0297919978463133	0.708873117719409\\
0.0285526110833097	0.698950224144223\\
0.0273479619505794	0.689023053357106\\
0.0261775686414609	0.679091785474986\\
0.0250409483268497	0.669156595525208\\
0.0239376171987705	0.659217653514541\\
0.0228670905127106	0.649275124498548\\
0.0218288826287227	0.6393291686512\\
0.0208225070513295	0.629379941334887\\
0.0198474764682326	0.61942759317068\\
0.0189033027878527	0.609472270108919\\
0.017989497175706	0.599514113500005\\
0.0171055700896473	0.589553260165529\\
0.016251031313986	0.579589842469588\\
0.0154253899925001	0.569623988390399\\
0.0146281546603561	0.559655821592055\\
0.0138588332749615	0.549685461496559\\
0.0131169332457649	0.539713023356059\\
0.0124019614630133	0.529738618325217\\
0.0117134243254973	0.519762353533822\\
0.0110508277672906	0.509784332159541\\
0.0104136772835099	0.499804653500865\\
0.00980147795510396	0.489823413050147\\
0.00921373447270001	0.479840702566844\\
0.00864995115951871	0.469856610150896\\
0.00810963199337461	0.45987122031617\\
0.00759228062778715	0.449884614064098\\
0.00709740041221289	0.439896868957379\\
0.00662449441142351	0.42990805919384\\
0.00617306542403963	0.419918255680319\\
0.00574261600024852	0.409927526106704\\
0.00533264845871796	0.399935935020063\\
0.00494266490272339	0.389943543898792\\
0.00457216723551227	0.379950411226902\\
0.00422065717491871	0.369956592568342\\
0.00388763626725179	0.359962140641424\\
0.0035726059004696	0.349967105393242\\
0.00327506731666439	0.339971534074208\\
0.00299452162387236	0.32997547131261\\
0.00273046980723061	0.319978959189248\\
0.00248241273949429	0.309982037312045\\
0.00224985119093942	0.299984742890769\\
0.00203228583866486	0.289987110811777\\
0.00182921727531368	0.279989173712739\\
0.00164014601723285	0.269990962057459\\
0.00146457251209006	0.259992504210679\\
0.00130199714596597	0.24999382651295\\
0.00115192024993965	0.239994953355454\\
0.00101384210618726	0.229995907254912\\
0.00088726295361375	0.219996708928503\\
0.000771682993031639	0.209997377368737\\
0.000666602391911615	0.199997929918419\\
0.000571521288719364	0.189998382345569\\
0.000485939796859026	0.179998748918405\\
0.000409358008241411	0.169999042480261\\
0.000341275996496673	0.159999274524576\\
0.00028119381984914	0.149999455269886\\
0.000228611523673179	0.139999593734768\\
0.00018302914274991	0.129999697812845\\
0.000143946703241396	0.119999774347765\\
0.000110864224404184	0.109999829208215\\
8.32817200568052e-05	0.0999998673628785\\
6.0699199824574e-05	0.0899998929554519\\
4.26166701764025e-05	0.0799999093796322\\
2.85341352748724e-05	0.0699999193541396\\
1.79515976578911e-05	0.059999924997676\\
1.0369058769739e-05	0.0499999279039441\\
5.28651936043693e-06	0.0399999292166675\\
2.20397977437651e-06	0.0299999297045498\\
6.21440143263748e-07	0.0199999298363011\\
3.89005049651121e-08	0.00999992985562482\\
-4.36391338898279e-08	-7.01437548045192e-08\\
};
\addplot [color=red, line width=2.0pt, only marks, mark size=2.5pt, mark=*, mark options={solid, fill=red, red}, forget plot]
  table[row sep=crcr]{%
0.993768058429589	0.082962048537095\\
};
\addplot [color=red, line width=2.0pt, only marks, mark size=2.5pt, mark=*, mark options={solid, fill=red, red}, forget plot]
  table[row sep=crcr]{%
-1.2010639363519	0.967260366381544\\
};
\addplot [color=red, line width=2.0pt, only marks, mark size=2.5pt, mark=*, mark options={solid, fill=red, red}, forget plot]
  table[row sep=crcr]{%
0.116291981798888	-0.530205495240226\\
};
\addplot [color=red, line width=2.0pt, only marks, mark size=2.5pt, mark=*, mark options={solid, fill=red, red}, forget plot]
  table[row sep=crcr]{%
-0.530205576720467	0.116292060303857\\
};
\addplot [color=red, line width=2.0pt, only marks, mark size=2.5pt, mark=*, mark options={solid, fill=red, red}, forget plot]
  table[row sep=crcr]{%
0.967260290963958	-1.20106385095539\\
};
\addplot [color=red, line width=2.0pt, only marks, mark size=2.5pt, mark=*, mark options={solid, fill=red, red}, forget plot]
  table[row sep=crcr]{%
0.0829619260231352	0.993767994870484\\
};
\addplot [color=red, line width=2.0pt, only marks, mark size=2.5pt, mark=*, mark options={solid, fill=red, red}, forget plot]
  table[row sep=crcr]{%
-4.36391338898279e-08	-7.01437548045192e-08\\
};
\addplot [color=blue, line width=2.0pt, only marks, mark size=2.5pt, mark=*, mark options={solid, fill=blue, blue}, forget plot]
  table[row sep=crcr]{%
0	0\\
};
\addplot [color=blue, line width=2.0pt, only marks, mark size=2.5pt, mark=*, mark options={solid, fill=blue, blue}, forget plot]
  table[row sep=crcr]{%
0	0\\
};
\end{axis}
\end{tikzpicture}%%
  \caption{Duboids solution example 5}
  \label{fig:DuboidsRes4}
\end{figure}
%

\section*{Compare PINS and Duboids}

\section*{Conclusions}





%%%% References
\bibliographystyle{ieeetr}
%\bibliographystyle{plain}
%\footnotesize
\bibliography{Bibliography}
%



% \end{multicols}
 
\end{document}

